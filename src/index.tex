% Save this as tutorial.tex for the lwarp package tutorial.
\documentclass{book}
\usepackage{iftex}

% --- LOAD FONT SELECTION AND ENCODING BEFORE LOADING LWARP ---
\ifPDFTeX
	\usepackage{lmodern} % pdflatex or dvi latex
	\usepackage[T1]{fontenc}
	\usepackage[utf8]{inputenc}
\else
	\usepackage{fontspec} % XeLaTeX or LuaLaTeX
\fi

% --- LWARP IS LOADED NEXT ---

\usepackage[
% HomeHTMLFilename=index, % Filename of the homepage.
% HTMLFilename={node-}, % Filename prefix of other pages.
% IndexLanguage=english, % Language for xindy index, glossary.
% latexmk, % Use latexmk to compile.
% OSWindows, % Force Windows. (Usually automatic.)
mathjax, % Use MathJax to display math.
]{lwarp}

% \boolfalse{FileSectionNames} % If false, numbers the files.

% --- LOAD PDFLATEX MATH FONTS HERE ---

% --- OTHER PACKAGES ARE LOADED AFTER LWARP ---

\usepackage{standalone}

\usepackage{tikz}
\usepackage{amsthm}
\usepackage{amsmath}
\usepackage{amssymb}
\usepackage{algorithm}
\usepackage{algpseudocode}

\usetikzlibrary{graphs, positioning, shapes.geometric}

\usepackage{makeidx} \makeindex
\usepackage{xcolor}               % (Demonstration purposes only.)
\usepackage{hyperref,cleveref}    % LOAD THESE LAST!

% Declare theorem environments for amsthm
\newtheorem{theorem}{Theorem}
\newtheorem{lemma}{Lemma}

% --- LATEX AND HTML CUSTOMIZATION ---
\title{Notes on Everything}
\author{Anthony}
\date{\today}

\setcounter{tocdepth}{1}
\setcounter{secnumdepth}{-1}
\setcounter{FileDepth}{1}
\booltrue{CombineHigherDepths}
\setcounter{SideTOCDepth}{2}

% Renew commands

\renewcommand*\contentsname{Subjects}
\renewcommand\sidetocname{Subjects}

% HTML Directives
\HTMLTitle{Notes on Everything}
\HTMLAuthor{Anthony}
\HTMLLanguage{en-US}
\HTMLDescription{Personal notes on Mathematics and Computer Science}
\HTMLPageBottom{\LinkHome}

% Styling
\CSSFilename{lwarp.css}

\begin{document}

\maketitle % Or titlepage/titlingpage environment.
% An article abstract would go here.

Notes on Mathematics and Computer Science.

\tableofcontents % MUST BE BEFORE THE FIRST SECTION BREAK!

\listoffigures

% Save this as tutorial.tex for the lwarp package tutorial.
\documentclass{book}
\usepackage{iftex}

% --- LOAD FONT SELECTION AND ENCODING BEFORE LOADING LWARP ---
\ifPDFTeX
	\usepackage{lmodern} % pdflatex or dvi latex
	\usepackage[T1]{fontenc}
	\usepackage[utf8]{inputenc}
\else
	\usepackage{fontspec} % XeLaTeX or LuaLaTeX
\fi

% --- LWARP IS LOADED NEXT ---

\usepackage[
% HomeHTMLFilename=index, % Filename of the homepage.
% HTMLFilename={node-}, % Filename prefix of other pages.
% IndexLanguage=english, % Language for xindy index, glossary.
% latexmk, % Use latexmk to compile.
% OSWindows, % Force Windows. (Usually automatic.)
mathjax, % Use MathJax to display math.
]{lwarp}

% \boolfalse{FileSectionNames} % If false, numbers the files.

% --- LOAD PDFLATEX MATH FONTS HERE ---

% --- OTHER PACKAGES ARE LOADED AFTER LWARP ---

\usepackage{standalone}

\usepackage{tikz}
\usepackage{amsthm}
\usepackage{amsmath}
\usepackage{amssymb}
\usepackage{algorithm}
\usepackage{algpseudocode}

\usetikzlibrary{graphs, positioning, shapes.geometric}

\usepackage{makeidx} \makeindex
\usepackage{xcolor}               % (Demonstration purposes only.)
\usepackage{hyperref,cleveref}    % LOAD THESE LAST!

% Declare theorem environments for amsthm
\newtheorem{axiom}{axiom}
\newtheorem{lemma}{Lemma}
\newtheorem{theorem}{Theorem}
\newtheorem{example}{Example}
\newtheorem{definition}{Definition}

% --- LATEX AND HTML CUSTOMIZATION ---
\title{Notes on Everything}
\author{Anthony}
\date{\today}

\setcounter{tocdepth}{1}
\setcounter{secnumdepth}{-1}
\setcounter{FileDepth}{1}
\booltrue{CombineHigherDepths}
\setcounter{SideTOCDepth}{2}

% Renew commands

\renewcommand*\contentsname{Subjects}
\renewcommand\sidetocname{Subjects}

% HTML Directives
\HTMLTitle{Notes on Everything}
\HTMLAuthor{Anthony}
\HTMLLanguage{en-US}
\HTMLDescription{Personal notes on Mathematics and Computer Science}
\HTMLPageBottom{\LinkHome}

% Styling
\CSSFilename{lwarp.css}

\begin{document}

\maketitle % Or titlepage/titlingpage environment.
% An article abstract would go here.

Notes on Mathematics and Computer Science.

\tableofcontents % MUST BE BEFORE THE FIRST SECTION BREAK!

\listoffigures

% Save this as tutorial.tex for the lwarp package tutorial.
\documentclass{book}
\usepackage{iftex}

% --- LOAD FONT SELECTION AND ENCODING BEFORE LOADING LWARP ---
\ifPDFTeX
	\usepackage{lmodern} % pdflatex or dvi latex
	\usepackage[T1]{fontenc}
	\usepackage[utf8]{inputenc}
\else
	\usepackage{fontspec} % XeLaTeX or LuaLaTeX
\fi

% --- LWARP IS LOADED NEXT ---

\usepackage[
% HomeHTMLFilename=index, % Filename of the homepage.
% HTMLFilename={node-}, % Filename prefix of other pages.
% IndexLanguage=english, % Language for xindy index, glossary.
% latexmk, % Use latexmk to compile.
% OSWindows, % Force Windows. (Usually automatic.)
mathjax, % Use MathJax to display math.
]{lwarp}

% \boolfalse{FileSectionNames} % If false, numbers the files.

% --- LOAD PDFLATEX MATH FONTS HERE ---

% --- OTHER PACKAGES ARE LOADED AFTER LWARP ---

\usepackage{standalone}

\usepackage{tikz}
\usepackage{amsthm}
\usepackage{amsmath}
\usepackage{amssymb}
\usepackage{algorithm}
\usepackage{algpseudocode}

\usetikzlibrary{graphs, positioning, shapes.geometric}

\usepackage{makeidx} \makeindex
\usepackage{xcolor}               % (Demonstration purposes only.)
\usepackage{hyperref,cleveref}    % LOAD THESE LAST!

% Declare theorem environments for amsthm
\newtheorem{axiom}{axiom}
\newtheorem{lemma}{Lemma}
\newtheorem{theorem}{Theorem}
\newtheorem{example}{Example}
\newtheorem{definition}{Definition}

% --- LATEX AND HTML CUSTOMIZATION ---
\title{Notes on Everything}
\author{Anthony}
\date{\today}

\setcounter{tocdepth}{1}
\setcounter{secnumdepth}{-1}
\setcounter{FileDepth}{1}
\booltrue{CombineHigherDepths}
\setcounter{SideTOCDepth}{2}

% Renew commands

\renewcommand*\contentsname{Subjects}
\renewcommand\sidetocname{Subjects}

% HTML Directives
\HTMLTitle{Notes on Everything}
\HTMLAuthor{Anthony}
\HTMLLanguage{en-US}
\HTMLDescription{Personal notes on Mathematics and Computer Science}
\HTMLPageBottom{\LinkHome}

% Styling
\CSSFilename{lwarp.css}

\begin{document}

\maketitle % Or titlepage/titlingpage environment.
% An article abstract would go here.

Notes on Mathematics and Computer Science.

\tableofcontents % MUST BE BEFORE THE FIRST SECTION BREAK!

\listoffigures

% Save this as tutorial.tex for the lwarp package tutorial.
\documentclass{book}
\usepackage{iftex}

% --- LOAD FONT SELECTION AND ENCODING BEFORE LOADING LWARP ---
\ifPDFTeX
	\usepackage{lmodern} % pdflatex or dvi latex
	\usepackage[T1]{fontenc}
	\usepackage[utf8]{inputenc}
\else
	\usepackage{fontspec} % XeLaTeX or LuaLaTeX
\fi

% --- LWARP IS LOADED NEXT ---

\usepackage[
% HomeHTMLFilename=index, % Filename of the homepage.
% HTMLFilename={node-}, % Filename prefix of other pages.
% IndexLanguage=english, % Language for xindy index, glossary.
% latexmk, % Use latexmk to compile.
% OSWindows, % Force Windows. (Usually automatic.)
mathjax, % Use MathJax to display math.
]{lwarp}

% \boolfalse{FileSectionNames} % If false, numbers the files.

% --- LOAD PDFLATEX MATH FONTS HERE ---

% --- OTHER PACKAGES ARE LOADED AFTER LWARP ---

\usepackage{standalone}

\usepackage{tikz}
\usepackage{amsthm}
\usepackage{amsmath}
\usepackage{amssymb}
\usepackage{algorithm}
\usepackage{algpseudocode}

\usetikzlibrary{graphs, positioning, shapes.geometric}

\usepackage{makeidx} \makeindex
\usepackage{xcolor}               % (Demonstration purposes only.)
\usepackage{hyperref,cleveref}    % LOAD THESE LAST!

% Declare theorem environments for amsthm
\newtheorem{axiom}{axiom}
\newtheorem{lemma}{Lemma}
\newtheorem{theorem}{Theorem}
\newtheorem{example}{Example}
\newtheorem{definition}{Definition}

% --- LATEX AND HTML CUSTOMIZATION ---
\title{Notes on Everything}
\author{Anthony}
\date{\today}

\setcounter{tocdepth}{1}
\setcounter{secnumdepth}{-1}
\setcounter{FileDepth}{1}
\booltrue{CombineHigherDepths}
\setcounter{SideTOCDepth}{2}

% Renew commands

\renewcommand*\contentsname{Subjects}
\renewcommand\sidetocname{Subjects}

% HTML Directives
\HTMLTitle{Notes on Everything}
\HTMLAuthor{Anthony}
\HTMLLanguage{en-US}
\HTMLDescription{Personal notes on Mathematics and Computer Science}
\HTMLPageBottom{\LinkHome}

% Styling
\CSSFilename{lwarp.css}

\begin{document}

\maketitle % Or titlepage/titlingpage environment.
% An article abstract would go here.

Notes on Mathematics and Computer Science.

\tableofcontents % MUST BE BEFORE THE FIRST SECTION BREAK!

\listoffigures

\input{./sums/index}
\input{./algorithms/index}
\input{./combinatorics/index}
\input{./probability/index}
\input{./foundations/index}

\ForceHTMLPage 	% HTML index will be on its own page.
\ForceHTMLTOC 	% HTML index will have its own toc entry.
\printindex
\end{document}

% Save this as tutorial.tex for the lwarp package tutorial.
\documentclass{book}
\usepackage{iftex}

% --- LOAD FONT SELECTION AND ENCODING BEFORE LOADING LWARP ---
\ifPDFTeX
	\usepackage{lmodern} % pdflatex or dvi latex
	\usepackage[T1]{fontenc}
	\usepackage[utf8]{inputenc}
\else
	\usepackage{fontspec} % XeLaTeX or LuaLaTeX
\fi

% --- LWARP IS LOADED NEXT ---

\usepackage[
% HomeHTMLFilename=index, % Filename of the homepage.
% HTMLFilename={node-}, % Filename prefix of other pages.
% IndexLanguage=english, % Language for xindy index, glossary.
% latexmk, % Use latexmk to compile.
% OSWindows, % Force Windows. (Usually automatic.)
mathjax, % Use MathJax to display math.
]{lwarp}

% \boolfalse{FileSectionNames} % If false, numbers the files.

% --- LOAD PDFLATEX MATH FONTS HERE ---

% --- OTHER PACKAGES ARE LOADED AFTER LWARP ---

\usepackage{standalone}

\usepackage{tikz}
\usepackage{amsthm}
\usepackage{amsmath}
\usepackage{amssymb}
\usepackage{algorithm}
\usepackage{algpseudocode}

\usetikzlibrary{graphs, positioning, shapes.geometric}

\usepackage{makeidx} \makeindex
\usepackage{xcolor}               % (Demonstration purposes only.)
\usepackage{hyperref,cleveref}    % LOAD THESE LAST!

% Declare theorem environments for amsthm
\newtheorem{axiom}{axiom}
\newtheorem{lemma}{Lemma}
\newtheorem{theorem}{Theorem}
\newtheorem{example}{Example}
\newtheorem{definition}{Definition}

% --- LATEX AND HTML CUSTOMIZATION ---
\title{Notes on Everything}
\author{Anthony}
\date{\today}

\setcounter{tocdepth}{1}
\setcounter{secnumdepth}{-1}
\setcounter{FileDepth}{1}
\booltrue{CombineHigherDepths}
\setcounter{SideTOCDepth}{2}

% Renew commands

\renewcommand*\contentsname{Subjects}
\renewcommand\sidetocname{Subjects}

% HTML Directives
\HTMLTitle{Notes on Everything}
\HTMLAuthor{Anthony}
\HTMLLanguage{en-US}
\HTMLDescription{Personal notes on Mathematics and Computer Science}
\HTMLPageBottom{\LinkHome}

% Styling
\CSSFilename{lwarp.css}

\begin{document}

\maketitle % Or titlepage/titlingpage environment.
% An article abstract would go here.

Notes on Mathematics and Computer Science.

\tableofcontents % MUST BE BEFORE THE FIRST SECTION BREAK!

\listoffigures

\input{./sums/index}
\input{./algorithms/index}
\input{./combinatorics/index}
\input{./probability/index}
\input{./foundations/index}

\ForceHTMLPage 	% HTML index will be on its own page.
\ForceHTMLTOC 	% HTML index will have its own toc entry.
\printindex
\end{document}

% Save this as tutorial.tex for the lwarp package tutorial.
\documentclass{book}
\usepackage{iftex}

% --- LOAD FONT SELECTION AND ENCODING BEFORE LOADING LWARP ---
\ifPDFTeX
	\usepackage{lmodern} % pdflatex or dvi latex
	\usepackage[T1]{fontenc}
	\usepackage[utf8]{inputenc}
\else
	\usepackage{fontspec} % XeLaTeX or LuaLaTeX
\fi

% --- LWARP IS LOADED NEXT ---

\usepackage[
% HomeHTMLFilename=index, % Filename of the homepage.
% HTMLFilename={node-}, % Filename prefix of other pages.
% IndexLanguage=english, % Language for xindy index, glossary.
% latexmk, % Use latexmk to compile.
% OSWindows, % Force Windows. (Usually automatic.)
mathjax, % Use MathJax to display math.
]{lwarp}

% \boolfalse{FileSectionNames} % If false, numbers the files.

% --- LOAD PDFLATEX MATH FONTS HERE ---

% --- OTHER PACKAGES ARE LOADED AFTER LWARP ---

\usepackage{standalone}

\usepackage{tikz}
\usepackage{amsthm}
\usepackage{amsmath}
\usepackage{amssymb}
\usepackage{algorithm}
\usepackage{algpseudocode}

\usetikzlibrary{graphs, positioning, shapes.geometric}

\usepackage{makeidx} \makeindex
\usepackage{xcolor}               % (Demonstration purposes only.)
\usepackage{hyperref,cleveref}    % LOAD THESE LAST!

% Declare theorem environments for amsthm
\newtheorem{axiom}{axiom}
\newtheorem{lemma}{Lemma}
\newtheorem{theorem}{Theorem}
\newtheorem{example}{Example}
\newtheorem{definition}{Definition}

% --- LATEX AND HTML CUSTOMIZATION ---
\title{Notes on Everything}
\author{Anthony}
\date{\today}

\setcounter{tocdepth}{1}
\setcounter{secnumdepth}{-1}
\setcounter{FileDepth}{1}
\booltrue{CombineHigherDepths}
\setcounter{SideTOCDepth}{2}

% Renew commands

\renewcommand*\contentsname{Subjects}
\renewcommand\sidetocname{Subjects}

% HTML Directives
\HTMLTitle{Notes on Everything}
\HTMLAuthor{Anthony}
\HTMLLanguage{en-US}
\HTMLDescription{Personal notes on Mathematics and Computer Science}
\HTMLPageBottom{\LinkHome}

% Styling
\CSSFilename{lwarp.css}

\begin{document}

\maketitle % Or titlepage/titlingpage environment.
% An article abstract would go here.

Notes on Mathematics and Computer Science.

\tableofcontents % MUST BE BEFORE THE FIRST SECTION BREAK!

\listoffigures

\input{./sums/index}
\input{./algorithms/index}
\input{./combinatorics/index}
\input{./probability/index}
\input{./foundations/index}

\ForceHTMLPage 	% HTML index will be on its own page.
\ForceHTMLTOC 	% HTML index will have its own toc entry.
\printindex
\end{document}

% Save this as tutorial.tex for the lwarp package tutorial.
\documentclass{book}
\usepackage{iftex}

% --- LOAD FONT SELECTION AND ENCODING BEFORE LOADING LWARP ---
\ifPDFTeX
	\usepackage{lmodern} % pdflatex or dvi latex
	\usepackage[T1]{fontenc}
	\usepackage[utf8]{inputenc}
\else
	\usepackage{fontspec} % XeLaTeX or LuaLaTeX
\fi

% --- LWARP IS LOADED NEXT ---

\usepackage[
% HomeHTMLFilename=index, % Filename of the homepage.
% HTMLFilename={node-}, % Filename prefix of other pages.
% IndexLanguage=english, % Language for xindy index, glossary.
% latexmk, % Use latexmk to compile.
% OSWindows, % Force Windows. (Usually automatic.)
mathjax, % Use MathJax to display math.
]{lwarp}

% \boolfalse{FileSectionNames} % If false, numbers the files.

% --- LOAD PDFLATEX MATH FONTS HERE ---

% --- OTHER PACKAGES ARE LOADED AFTER LWARP ---

\usepackage{standalone}

\usepackage{tikz}
\usepackage{amsthm}
\usepackage{amsmath}
\usepackage{amssymb}
\usepackage{algorithm}
\usepackage{algpseudocode}

\usetikzlibrary{graphs, positioning, shapes.geometric}

\usepackage{makeidx} \makeindex
\usepackage{xcolor}               % (Demonstration purposes only.)
\usepackage{hyperref,cleveref}    % LOAD THESE LAST!

% Declare theorem environments for amsthm
\newtheorem{axiom}{axiom}
\newtheorem{lemma}{Lemma}
\newtheorem{theorem}{Theorem}
\newtheorem{example}{Example}
\newtheorem{definition}{Definition}

% --- LATEX AND HTML CUSTOMIZATION ---
\title{Notes on Everything}
\author{Anthony}
\date{\today}

\setcounter{tocdepth}{1}
\setcounter{secnumdepth}{-1}
\setcounter{FileDepth}{1}
\booltrue{CombineHigherDepths}
\setcounter{SideTOCDepth}{2}

% Renew commands

\renewcommand*\contentsname{Subjects}
\renewcommand\sidetocname{Subjects}

% HTML Directives
\HTMLTitle{Notes on Everything}
\HTMLAuthor{Anthony}
\HTMLLanguage{en-US}
\HTMLDescription{Personal notes on Mathematics and Computer Science}
\HTMLPageBottom{\LinkHome}

% Styling
\CSSFilename{lwarp.css}

\begin{document}

\maketitle % Or titlepage/titlingpage environment.
% An article abstract would go here.

Notes on Mathematics and Computer Science.

\tableofcontents % MUST BE BEFORE THE FIRST SECTION BREAK!

\listoffigures

\input{./sums/index}
\input{./algorithms/index}
\input{./combinatorics/index}
\input{./probability/index}
\input{./foundations/index}

\ForceHTMLPage 	% HTML index will be on its own page.
\ForceHTMLTOC 	% HTML index will have its own toc entry.
\printindex
\end{document}

% Save this as tutorial.tex for the lwarp package tutorial.
\documentclass{book}
\usepackage{iftex}

% --- LOAD FONT SELECTION AND ENCODING BEFORE LOADING LWARP ---
\ifPDFTeX
	\usepackage{lmodern} % pdflatex or dvi latex
	\usepackage[T1]{fontenc}
	\usepackage[utf8]{inputenc}
\else
	\usepackage{fontspec} % XeLaTeX or LuaLaTeX
\fi

% --- LWARP IS LOADED NEXT ---

\usepackage[
% HomeHTMLFilename=index, % Filename of the homepage.
% HTMLFilename={node-}, % Filename prefix of other pages.
% IndexLanguage=english, % Language for xindy index, glossary.
% latexmk, % Use latexmk to compile.
% OSWindows, % Force Windows. (Usually automatic.)
mathjax, % Use MathJax to display math.
]{lwarp}

% \boolfalse{FileSectionNames} % If false, numbers the files.

% --- LOAD PDFLATEX MATH FONTS HERE ---

% --- OTHER PACKAGES ARE LOADED AFTER LWARP ---

\usepackage{standalone}

\usepackage{tikz}
\usepackage{amsthm}
\usepackage{amsmath}
\usepackage{amssymb}
\usepackage{algorithm}
\usepackage{algpseudocode}

\usetikzlibrary{graphs, positioning, shapes.geometric}

\usepackage{makeidx} \makeindex
\usepackage{xcolor}               % (Demonstration purposes only.)
\usepackage{hyperref,cleveref}    % LOAD THESE LAST!

% Declare theorem environments for amsthm
\newtheorem{axiom}{axiom}
\newtheorem{lemma}{Lemma}
\newtheorem{theorem}{Theorem}
\newtheorem{example}{Example}
\newtheorem{definition}{Definition}

% --- LATEX AND HTML CUSTOMIZATION ---
\title{Notes on Everything}
\author{Anthony}
\date{\today}

\setcounter{tocdepth}{1}
\setcounter{secnumdepth}{-1}
\setcounter{FileDepth}{1}
\booltrue{CombineHigherDepths}
\setcounter{SideTOCDepth}{2}

% Renew commands

\renewcommand*\contentsname{Subjects}
\renewcommand\sidetocname{Subjects}

% HTML Directives
\HTMLTitle{Notes on Everything}
\HTMLAuthor{Anthony}
\HTMLLanguage{en-US}
\HTMLDescription{Personal notes on Mathematics and Computer Science}
\HTMLPageBottom{\LinkHome}

% Styling
\CSSFilename{lwarp.css}

\begin{document}

\maketitle % Or titlepage/titlingpage environment.
% An article abstract would go here.

Notes on Mathematics and Computer Science.

\tableofcontents % MUST BE BEFORE THE FIRST SECTION BREAK!

\listoffigures

\input{./sums/index}
\input{./algorithms/index}
\input{./combinatorics/index}
\input{./probability/index}
\input{./foundations/index}

\ForceHTMLPage 	% HTML index will be on its own page.
\ForceHTMLTOC 	% HTML index will have its own toc entry.
\printindex
\end{document}


\ForceHTMLPage 	% HTML index will be on its own page.
\ForceHTMLTOC 	% HTML index will have its own toc entry.
\printindex
\end{document}

% Save this as tutorial.tex for the lwarp package tutorial.
\documentclass{book}
\usepackage{iftex}

% --- LOAD FONT SELECTION AND ENCODING BEFORE LOADING LWARP ---
\ifPDFTeX
	\usepackage{lmodern} % pdflatex or dvi latex
	\usepackage[T1]{fontenc}
	\usepackage[utf8]{inputenc}
\else
	\usepackage{fontspec} % XeLaTeX or LuaLaTeX
\fi

% --- LWARP IS LOADED NEXT ---

\usepackage[
% HomeHTMLFilename=index, % Filename of the homepage.
% HTMLFilename={node-}, % Filename prefix of other pages.
% IndexLanguage=english, % Language for xindy index, glossary.
% latexmk, % Use latexmk to compile.
% OSWindows, % Force Windows. (Usually automatic.)
mathjax, % Use MathJax to display math.
]{lwarp}

% \boolfalse{FileSectionNames} % If false, numbers the files.

% --- LOAD PDFLATEX MATH FONTS HERE ---

% --- OTHER PACKAGES ARE LOADED AFTER LWARP ---

\usepackage{standalone}

\usepackage{tikz}
\usepackage{amsthm}
\usepackage{amsmath}
\usepackage{amssymb}
\usepackage{algorithm}
\usepackage{algpseudocode}

\usetikzlibrary{graphs, positioning, shapes.geometric}

\usepackage{makeidx} \makeindex
\usepackage{xcolor}               % (Demonstration purposes only.)
\usepackage{hyperref,cleveref}    % LOAD THESE LAST!

% Declare theorem environments for amsthm
\newtheorem{axiom}{axiom}
\newtheorem{lemma}{Lemma}
\newtheorem{theorem}{Theorem}
\newtheorem{example}{Example}
\newtheorem{definition}{Definition}

% --- LATEX AND HTML CUSTOMIZATION ---
\title{Notes on Everything}
\author{Anthony}
\date{\today}

\setcounter{tocdepth}{1}
\setcounter{secnumdepth}{-1}
\setcounter{FileDepth}{1}
\booltrue{CombineHigherDepths}
\setcounter{SideTOCDepth}{2}

% Renew commands

\renewcommand*\contentsname{Subjects}
\renewcommand\sidetocname{Subjects}

% HTML Directives
\HTMLTitle{Notes on Everything}
\HTMLAuthor{Anthony}
\HTMLLanguage{en-US}
\HTMLDescription{Personal notes on Mathematics and Computer Science}
\HTMLPageBottom{\LinkHome}

% Styling
\CSSFilename{lwarp.css}

\begin{document}

\maketitle % Or titlepage/titlingpage environment.
% An article abstract would go here.

Notes on Mathematics and Computer Science.

\tableofcontents % MUST BE BEFORE THE FIRST SECTION BREAK!

\listoffigures

% Save this as tutorial.tex for the lwarp package tutorial.
\documentclass{book}
\usepackage{iftex}

% --- LOAD FONT SELECTION AND ENCODING BEFORE LOADING LWARP ---
\ifPDFTeX
	\usepackage{lmodern} % pdflatex or dvi latex
	\usepackage[T1]{fontenc}
	\usepackage[utf8]{inputenc}
\else
	\usepackage{fontspec} % XeLaTeX or LuaLaTeX
\fi

% --- LWARP IS LOADED NEXT ---

\usepackage[
% HomeHTMLFilename=index, % Filename of the homepage.
% HTMLFilename={node-}, % Filename prefix of other pages.
% IndexLanguage=english, % Language for xindy index, glossary.
% latexmk, % Use latexmk to compile.
% OSWindows, % Force Windows. (Usually automatic.)
mathjax, % Use MathJax to display math.
]{lwarp}

% \boolfalse{FileSectionNames} % If false, numbers the files.

% --- LOAD PDFLATEX MATH FONTS HERE ---

% --- OTHER PACKAGES ARE LOADED AFTER LWARP ---

\usepackage{standalone}

\usepackage{tikz}
\usepackage{amsthm}
\usepackage{amsmath}
\usepackage{amssymb}
\usepackage{algorithm}
\usepackage{algpseudocode}

\usetikzlibrary{graphs, positioning, shapes.geometric}

\usepackage{makeidx} \makeindex
\usepackage{xcolor}               % (Demonstration purposes only.)
\usepackage{hyperref,cleveref}    % LOAD THESE LAST!

% Declare theorem environments for amsthm
\newtheorem{axiom}{axiom}
\newtheorem{lemma}{Lemma}
\newtheorem{theorem}{Theorem}
\newtheorem{example}{Example}
\newtheorem{definition}{Definition}

% --- LATEX AND HTML CUSTOMIZATION ---
\title{Notes on Everything}
\author{Anthony}
\date{\today}

\setcounter{tocdepth}{1}
\setcounter{secnumdepth}{-1}
\setcounter{FileDepth}{1}
\booltrue{CombineHigherDepths}
\setcounter{SideTOCDepth}{2}

% Renew commands

\renewcommand*\contentsname{Subjects}
\renewcommand\sidetocname{Subjects}

% HTML Directives
\HTMLTitle{Notes on Everything}
\HTMLAuthor{Anthony}
\HTMLLanguage{en-US}
\HTMLDescription{Personal notes on Mathematics and Computer Science}
\HTMLPageBottom{\LinkHome}

% Styling
\CSSFilename{lwarp.css}

\begin{document}

\maketitle % Or titlepage/titlingpage environment.
% An article abstract would go here.

Notes on Mathematics and Computer Science.

\tableofcontents % MUST BE BEFORE THE FIRST SECTION BREAK!

\listoffigures

\input{./sums/index}
\input{./algorithms/index}
\input{./combinatorics/index}
\input{./probability/index}
\input{./foundations/index}

\ForceHTMLPage 	% HTML index will be on its own page.
\ForceHTMLTOC 	% HTML index will have its own toc entry.
\printindex
\end{document}

% Save this as tutorial.tex for the lwarp package tutorial.
\documentclass{book}
\usepackage{iftex}

% --- LOAD FONT SELECTION AND ENCODING BEFORE LOADING LWARP ---
\ifPDFTeX
	\usepackage{lmodern} % pdflatex or dvi latex
	\usepackage[T1]{fontenc}
	\usepackage[utf8]{inputenc}
\else
	\usepackage{fontspec} % XeLaTeX or LuaLaTeX
\fi

% --- LWARP IS LOADED NEXT ---

\usepackage[
% HomeHTMLFilename=index, % Filename of the homepage.
% HTMLFilename={node-}, % Filename prefix of other pages.
% IndexLanguage=english, % Language for xindy index, glossary.
% latexmk, % Use latexmk to compile.
% OSWindows, % Force Windows. (Usually automatic.)
mathjax, % Use MathJax to display math.
]{lwarp}

% \boolfalse{FileSectionNames} % If false, numbers the files.

% --- LOAD PDFLATEX MATH FONTS HERE ---

% --- OTHER PACKAGES ARE LOADED AFTER LWARP ---

\usepackage{standalone}

\usepackage{tikz}
\usepackage{amsthm}
\usepackage{amsmath}
\usepackage{amssymb}
\usepackage{algorithm}
\usepackage{algpseudocode}

\usetikzlibrary{graphs, positioning, shapes.geometric}

\usepackage{makeidx} \makeindex
\usepackage{xcolor}               % (Demonstration purposes only.)
\usepackage{hyperref,cleveref}    % LOAD THESE LAST!

% Declare theorem environments for amsthm
\newtheorem{axiom}{axiom}
\newtheorem{lemma}{Lemma}
\newtheorem{theorem}{Theorem}
\newtheorem{example}{Example}
\newtheorem{definition}{Definition}

% --- LATEX AND HTML CUSTOMIZATION ---
\title{Notes on Everything}
\author{Anthony}
\date{\today}

\setcounter{tocdepth}{1}
\setcounter{secnumdepth}{-1}
\setcounter{FileDepth}{1}
\booltrue{CombineHigherDepths}
\setcounter{SideTOCDepth}{2}

% Renew commands

\renewcommand*\contentsname{Subjects}
\renewcommand\sidetocname{Subjects}

% HTML Directives
\HTMLTitle{Notes on Everything}
\HTMLAuthor{Anthony}
\HTMLLanguage{en-US}
\HTMLDescription{Personal notes on Mathematics and Computer Science}
\HTMLPageBottom{\LinkHome}

% Styling
\CSSFilename{lwarp.css}

\begin{document}

\maketitle % Or titlepage/titlingpage environment.
% An article abstract would go here.

Notes on Mathematics and Computer Science.

\tableofcontents % MUST BE BEFORE THE FIRST SECTION BREAK!

\listoffigures

\input{./sums/index}
\input{./algorithms/index}
\input{./combinatorics/index}
\input{./probability/index}
\input{./foundations/index}

\ForceHTMLPage 	% HTML index will be on its own page.
\ForceHTMLTOC 	% HTML index will have its own toc entry.
\printindex
\end{document}

% Save this as tutorial.tex for the lwarp package tutorial.
\documentclass{book}
\usepackage{iftex}

% --- LOAD FONT SELECTION AND ENCODING BEFORE LOADING LWARP ---
\ifPDFTeX
	\usepackage{lmodern} % pdflatex or dvi latex
	\usepackage[T1]{fontenc}
	\usepackage[utf8]{inputenc}
\else
	\usepackage{fontspec} % XeLaTeX or LuaLaTeX
\fi

% --- LWARP IS LOADED NEXT ---

\usepackage[
% HomeHTMLFilename=index, % Filename of the homepage.
% HTMLFilename={node-}, % Filename prefix of other pages.
% IndexLanguage=english, % Language for xindy index, glossary.
% latexmk, % Use latexmk to compile.
% OSWindows, % Force Windows. (Usually automatic.)
mathjax, % Use MathJax to display math.
]{lwarp}

% \boolfalse{FileSectionNames} % If false, numbers the files.

% --- LOAD PDFLATEX MATH FONTS HERE ---

% --- OTHER PACKAGES ARE LOADED AFTER LWARP ---

\usepackage{standalone}

\usepackage{tikz}
\usepackage{amsthm}
\usepackage{amsmath}
\usepackage{amssymb}
\usepackage{algorithm}
\usepackage{algpseudocode}

\usetikzlibrary{graphs, positioning, shapes.geometric}

\usepackage{makeidx} \makeindex
\usepackage{xcolor}               % (Demonstration purposes only.)
\usepackage{hyperref,cleveref}    % LOAD THESE LAST!

% Declare theorem environments for amsthm
\newtheorem{axiom}{axiom}
\newtheorem{lemma}{Lemma}
\newtheorem{theorem}{Theorem}
\newtheorem{example}{Example}
\newtheorem{definition}{Definition}

% --- LATEX AND HTML CUSTOMIZATION ---
\title{Notes on Everything}
\author{Anthony}
\date{\today}

\setcounter{tocdepth}{1}
\setcounter{secnumdepth}{-1}
\setcounter{FileDepth}{1}
\booltrue{CombineHigherDepths}
\setcounter{SideTOCDepth}{2}

% Renew commands

\renewcommand*\contentsname{Subjects}
\renewcommand\sidetocname{Subjects}

% HTML Directives
\HTMLTitle{Notes on Everything}
\HTMLAuthor{Anthony}
\HTMLLanguage{en-US}
\HTMLDescription{Personal notes on Mathematics and Computer Science}
\HTMLPageBottom{\LinkHome}

% Styling
\CSSFilename{lwarp.css}

\begin{document}

\maketitle % Or titlepage/titlingpage environment.
% An article abstract would go here.

Notes on Mathematics and Computer Science.

\tableofcontents % MUST BE BEFORE THE FIRST SECTION BREAK!

\listoffigures

\input{./sums/index}
\input{./algorithms/index}
\input{./combinatorics/index}
\input{./probability/index}
\input{./foundations/index}

\ForceHTMLPage 	% HTML index will be on its own page.
\ForceHTMLTOC 	% HTML index will have its own toc entry.
\printindex
\end{document}

% Save this as tutorial.tex for the lwarp package tutorial.
\documentclass{book}
\usepackage{iftex}

% --- LOAD FONT SELECTION AND ENCODING BEFORE LOADING LWARP ---
\ifPDFTeX
	\usepackage{lmodern} % pdflatex or dvi latex
	\usepackage[T1]{fontenc}
	\usepackage[utf8]{inputenc}
\else
	\usepackage{fontspec} % XeLaTeX or LuaLaTeX
\fi

% --- LWARP IS LOADED NEXT ---

\usepackage[
% HomeHTMLFilename=index, % Filename of the homepage.
% HTMLFilename={node-}, % Filename prefix of other pages.
% IndexLanguage=english, % Language for xindy index, glossary.
% latexmk, % Use latexmk to compile.
% OSWindows, % Force Windows. (Usually automatic.)
mathjax, % Use MathJax to display math.
]{lwarp}

% \boolfalse{FileSectionNames} % If false, numbers the files.

% --- LOAD PDFLATEX MATH FONTS HERE ---

% --- OTHER PACKAGES ARE LOADED AFTER LWARP ---

\usepackage{standalone}

\usepackage{tikz}
\usepackage{amsthm}
\usepackage{amsmath}
\usepackage{amssymb}
\usepackage{algorithm}
\usepackage{algpseudocode}

\usetikzlibrary{graphs, positioning, shapes.geometric}

\usepackage{makeidx} \makeindex
\usepackage{xcolor}               % (Demonstration purposes only.)
\usepackage{hyperref,cleveref}    % LOAD THESE LAST!

% Declare theorem environments for amsthm
\newtheorem{axiom}{axiom}
\newtheorem{lemma}{Lemma}
\newtheorem{theorem}{Theorem}
\newtheorem{example}{Example}
\newtheorem{definition}{Definition}

% --- LATEX AND HTML CUSTOMIZATION ---
\title{Notes on Everything}
\author{Anthony}
\date{\today}

\setcounter{tocdepth}{1}
\setcounter{secnumdepth}{-1}
\setcounter{FileDepth}{1}
\booltrue{CombineHigherDepths}
\setcounter{SideTOCDepth}{2}

% Renew commands

\renewcommand*\contentsname{Subjects}
\renewcommand\sidetocname{Subjects}

% HTML Directives
\HTMLTitle{Notes on Everything}
\HTMLAuthor{Anthony}
\HTMLLanguage{en-US}
\HTMLDescription{Personal notes on Mathematics and Computer Science}
\HTMLPageBottom{\LinkHome}

% Styling
\CSSFilename{lwarp.css}

\begin{document}

\maketitle % Or titlepage/titlingpage environment.
% An article abstract would go here.

Notes on Mathematics and Computer Science.

\tableofcontents % MUST BE BEFORE THE FIRST SECTION BREAK!

\listoffigures

\input{./sums/index}
\input{./algorithms/index}
\input{./combinatorics/index}
\input{./probability/index}
\input{./foundations/index}

\ForceHTMLPage 	% HTML index will be on its own page.
\ForceHTMLTOC 	% HTML index will have its own toc entry.
\printindex
\end{document}

% Save this as tutorial.tex for the lwarp package tutorial.
\documentclass{book}
\usepackage{iftex}

% --- LOAD FONT SELECTION AND ENCODING BEFORE LOADING LWARP ---
\ifPDFTeX
	\usepackage{lmodern} % pdflatex or dvi latex
	\usepackage[T1]{fontenc}
	\usepackage[utf8]{inputenc}
\else
	\usepackage{fontspec} % XeLaTeX or LuaLaTeX
\fi

% --- LWARP IS LOADED NEXT ---

\usepackage[
% HomeHTMLFilename=index, % Filename of the homepage.
% HTMLFilename={node-}, % Filename prefix of other pages.
% IndexLanguage=english, % Language for xindy index, glossary.
% latexmk, % Use latexmk to compile.
% OSWindows, % Force Windows. (Usually automatic.)
mathjax, % Use MathJax to display math.
]{lwarp}

% \boolfalse{FileSectionNames} % If false, numbers the files.

% --- LOAD PDFLATEX MATH FONTS HERE ---

% --- OTHER PACKAGES ARE LOADED AFTER LWARP ---

\usepackage{standalone}

\usepackage{tikz}
\usepackage{amsthm}
\usepackage{amsmath}
\usepackage{amssymb}
\usepackage{algorithm}
\usepackage{algpseudocode}

\usetikzlibrary{graphs, positioning, shapes.geometric}

\usepackage{makeidx} \makeindex
\usepackage{xcolor}               % (Demonstration purposes only.)
\usepackage{hyperref,cleveref}    % LOAD THESE LAST!

% Declare theorem environments for amsthm
\newtheorem{axiom}{axiom}
\newtheorem{lemma}{Lemma}
\newtheorem{theorem}{Theorem}
\newtheorem{example}{Example}
\newtheorem{definition}{Definition}

% --- LATEX AND HTML CUSTOMIZATION ---
\title{Notes on Everything}
\author{Anthony}
\date{\today}

\setcounter{tocdepth}{1}
\setcounter{secnumdepth}{-1}
\setcounter{FileDepth}{1}
\booltrue{CombineHigherDepths}
\setcounter{SideTOCDepth}{2}

% Renew commands

\renewcommand*\contentsname{Subjects}
\renewcommand\sidetocname{Subjects}

% HTML Directives
\HTMLTitle{Notes on Everything}
\HTMLAuthor{Anthony}
\HTMLLanguage{en-US}
\HTMLDescription{Personal notes on Mathematics and Computer Science}
\HTMLPageBottom{\LinkHome}

% Styling
\CSSFilename{lwarp.css}

\begin{document}

\maketitle % Or titlepage/titlingpage environment.
% An article abstract would go here.

Notes on Mathematics and Computer Science.

\tableofcontents % MUST BE BEFORE THE FIRST SECTION BREAK!

\listoffigures

\input{./sums/index}
\input{./algorithms/index}
\input{./combinatorics/index}
\input{./probability/index}
\input{./foundations/index}

\ForceHTMLPage 	% HTML index will be on its own page.
\ForceHTMLTOC 	% HTML index will have its own toc entry.
\printindex
\end{document}


\ForceHTMLPage 	% HTML index will be on its own page.
\ForceHTMLTOC 	% HTML index will have its own toc entry.
\printindex
\end{document}

% Save this as tutorial.tex for the lwarp package tutorial.
\documentclass{book}
\usepackage{iftex}

% --- LOAD FONT SELECTION AND ENCODING BEFORE LOADING LWARP ---
\ifPDFTeX
	\usepackage{lmodern} % pdflatex or dvi latex
	\usepackage[T1]{fontenc}
	\usepackage[utf8]{inputenc}
\else
	\usepackage{fontspec} % XeLaTeX or LuaLaTeX
\fi

% --- LWARP IS LOADED NEXT ---

\usepackage[
% HomeHTMLFilename=index, % Filename of the homepage.
% HTMLFilename={node-}, % Filename prefix of other pages.
% IndexLanguage=english, % Language for xindy index, glossary.
% latexmk, % Use latexmk to compile.
% OSWindows, % Force Windows. (Usually automatic.)
mathjax, % Use MathJax to display math.
]{lwarp}

% \boolfalse{FileSectionNames} % If false, numbers the files.

% --- LOAD PDFLATEX MATH FONTS HERE ---

% --- OTHER PACKAGES ARE LOADED AFTER LWARP ---

\usepackage{standalone}

\usepackage{tikz}
\usepackage{amsthm}
\usepackage{amsmath}
\usepackage{amssymb}
\usepackage{algorithm}
\usepackage{algpseudocode}

\usetikzlibrary{graphs, positioning, shapes.geometric}

\usepackage{makeidx} \makeindex
\usepackage{xcolor}               % (Demonstration purposes only.)
\usepackage{hyperref,cleveref}    % LOAD THESE LAST!

% Declare theorem environments for amsthm
\newtheorem{axiom}{axiom}
\newtheorem{lemma}{Lemma}
\newtheorem{theorem}{Theorem}
\newtheorem{example}{Example}
\newtheorem{definition}{Definition}

% --- LATEX AND HTML CUSTOMIZATION ---
\title{Notes on Everything}
\author{Anthony}
\date{\today}

\setcounter{tocdepth}{1}
\setcounter{secnumdepth}{-1}
\setcounter{FileDepth}{1}
\booltrue{CombineHigherDepths}
\setcounter{SideTOCDepth}{2}

% Renew commands

\renewcommand*\contentsname{Subjects}
\renewcommand\sidetocname{Subjects}

% HTML Directives
\HTMLTitle{Notes on Everything}
\HTMLAuthor{Anthony}
\HTMLLanguage{en-US}
\HTMLDescription{Personal notes on Mathematics and Computer Science}
\HTMLPageBottom{\LinkHome}

% Styling
\CSSFilename{lwarp.css}

\begin{document}

\maketitle % Or titlepage/titlingpage environment.
% An article abstract would go here.

Notes on Mathematics and Computer Science.

\tableofcontents % MUST BE BEFORE THE FIRST SECTION BREAK!

\listoffigures

% Save this as tutorial.tex for the lwarp package tutorial.
\documentclass{book}
\usepackage{iftex}

% --- LOAD FONT SELECTION AND ENCODING BEFORE LOADING LWARP ---
\ifPDFTeX
	\usepackage{lmodern} % pdflatex or dvi latex
	\usepackage[T1]{fontenc}
	\usepackage[utf8]{inputenc}
\else
	\usepackage{fontspec} % XeLaTeX or LuaLaTeX
\fi

% --- LWARP IS LOADED NEXT ---

\usepackage[
% HomeHTMLFilename=index, % Filename of the homepage.
% HTMLFilename={node-}, % Filename prefix of other pages.
% IndexLanguage=english, % Language for xindy index, glossary.
% latexmk, % Use latexmk to compile.
% OSWindows, % Force Windows. (Usually automatic.)
mathjax, % Use MathJax to display math.
]{lwarp}

% \boolfalse{FileSectionNames} % If false, numbers the files.

% --- LOAD PDFLATEX MATH FONTS HERE ---

% --- OTHER PACKAGES ARE LOADED AFTER LWARP ---

\usepackage{standalone}

\usepackage{tikz}
\usepackage{amsthm}
\usepackage{amsmath}
\usepackage{amssymb}
\usepackage{algorithm}
\usepackage{algpseudocode}

\usetikzlibrary{graphs, positioning, shapes.geometric}

\usepackage{makeidx} \makeindex
\usepackage{xcolor}               % (Demonstration purposes only.)
\usepackage{hyperref,cleveref}    % LOAD THESE LAST!

% Declare theorem environments for amsthm
\newtheorem{axiom}{axiom}
\newtheorem{lemma}{Lemma}
\newtheorem{theorem}{Theorem}
\newtheorem{example}{Example}
\newtheorem{definition}{Definition}

% --- LATEX AND HTML CUSTOMIZATION ---
\title{Notes on Everything}
\author{Anthony}
\date{\today}

\setcounter{tocdepth}{1}
\setcounter{secnumdepth}{-1}
\setcounter{FileDepth}{1}
\booltrue{CombineHigherDepths}
\setcounter{SideTOCDepth}{2}

% Renew commands

\renewcommand*\contentsname{Subjects}
\renewcommand\sidetocname{Subjects}

% HTML Directives
\HTMLTitle{Notes on Everything}
\HTMLAuthor{Anthony}
\HTMLLanguage{en-US}
\HTMLDescription{Personal notes on Mathematics and Computer Science}
\HTMLPageBottom{\LinkHome}

% Styling
\CSSFilename{lwarp.css}

\begin{document}

\maketitle % Or titlepage/titlingpage environment.
% An article abstract would go here.

Notes on Mathematics and Computer Science.

\tableofcontents % MUST BE BEFORE THE FIRST SECTION BREAK!

\listoffigures

\input{./sums/index}
\input{./algorithms/index}
\input{./combinatorics/index}
\input{./probability/index}
\input{./foundations/index}

\ForceHTMLPage 	% HTML index will be on its own page.
\ForceHTMLTOC 	% HTML index will have its own toc entry.
\printindex
\end{document}

% Save this as tutorial.tex for the lwarp package tutorial.
\documentclass{book}
\usepackage{iftex}

% --- LOAD FONT SELECTION AND ENCODING BEFORE LOADING LWARP ---
\ifPDFTeX
	\usepackage{lmodern} % pdflatex or dvi latex
	\usepackage[T1]{fontenc}
	\usepackage[utf8]{inputenc}
\else
	\usepackage{fontspec} % XeLaTeX or LuaLaTeX
\fi

% --- LWARP IS LOADED NEXT ---

\usepackage[
% HomeHTMLFilename=index, % Filename of the homepage.
% HTMLFilename={node-}, % Filename prefix of other pages.
% IndexLanguage=english, % Language for xindy index, glossary.
% latexmk, % Use latexmk to compile.
% OSWindows, % Force Windows. (Usually automatic.)
mathjax, % Use MathJax to display math.
]{lwarp}

% \boolfalse{FileSectionNames} % If false, numbers the files.

% --- LOAD PDFLATEX MATH FONTS HERE ---

% --- OTHER PACKAGES ARE LOADED AFTER LWARP ---

\usepackage{standalone}

\usepackage{tikz}
\usepackage{amsthm}
\usepackage{amsmath}
\usepackage{amssymb}
\usepackage{algorithm}
\usepackage{algpseudocode}

\usetikzlibrary{graphs, positioning, shapes.geometric}

\usepackage{makeidx} \makeindex
\usepackage{xcolor}               % (Demonstration purposes only.)
\usepackage{hyperref,cleveref}    % LOAD THESE LAST!

% Declare theorem environments for amsthm
\newtheorem{axiom}{axiom}
\newtheorem{lemma}{Lemma}
\newtheorem{theorem}{Theorem}
\newtheorem{example}{Example}
\newtheorem{definition}{Definition}

% --- LATEX AND HTML CUSTOMIZATION ---
\title{Notes on Everything}
\author{Anthony}
\date{\today}

\setcounter{tocdepth}{1}
\setcounter{secnumdepth}{-1}
\setcounter{FileDepth}{1}
\booltrue{CombineHigherDepths}
\setcounter{SideTOCDepth}{2}

% Renew commands

\renewcommand*\contentsname{Subjects}
\renewcommand\sidetocname{Subjects}

% HTML Directives
\HTMLTitle{Notes on Everything}
\HTMLAuthor{Anthony}
\HTMLLanguage{en-US}
\HTMLDescription{Personal notes on Mathematics and Computer Science}
\HTMLPageBottom{\LinkHome}

% Styling
\CSSFilename{lwarp.css}

\begin{document}

\maketitle % Or titlepage/titlingpage environment.
% An article abstract would go here.

Notes on Mathematics and Computer Science.

\tableofcontents % MUST BE BEFORE THE FIRST SECTION BREAK!

\listoffigures

\input{./sums/index}
\input{./algorithms/index}
\input{./combinatorics/index}
\input{./probability/index}
\input{./foundations/index}

\ForceHTMLPage 	% HTML index will be on its own page.
\ForceHTMLTOC 	% HTML index will have its own toc entry.
\printindex
\end{document}

% Save this as tutorial.tex for the lwarp package tutorial.
\documentclass{book}
\usepackage{iftex}

% --- LOAD FONT SELECTION AND ENCODING BEFORE LOADING LWARP ---
\ifPDFTeX
	\usepackage{lmodern} % pdflatex or dvi latex
	\usepackage[T1]{fontenc}
	\usepackage[utf8]{inputenc}
\else
	\usepackage{fontspec} % XeLaTeX or LuaLaTeX
\fi

% --- LWARP IS LOADED NEXT ---

\usepackage[
% HomeHTMLFilename=index, % Filename of the homepage.
% HTMLFilename={node-}, % Filename prefix of other pages.
% IndexLanguage=english, % Language for xindy index, glossary.
% latexmk, % Use latexmk to compile.
% OSWindows, % Force Windows. (Usually automatic.)
mathjax, % Use MathJax to display math.
]{lwarp}

% \boolfalse{FileSectionNames} % If false, numbers the files.

% --- LOAD PDFLATEX MATH FONTS HERE ---

% --- OTHER PACKAGES ARE LOADED AFTER LWARP ---

\usepackage{standalone}

\usepackage{tikz}
\usepackage{amsthm}
\usepackage{amsmath}
\usepackage{amssymb}
\usepackage{algorithm}
\usepackage{algpseudocode}

\usetikzlibrary{graphs, positioning, shapes.geometric}

\usepackage{makeidx} \makeindex
\usepackage{xcolor}               % (Demonstration purposes only.)
\usepackage{hyperref,cleveref}    % LOAD THESE LAST!

% Declare theorem environments for amsthm
\newtheorem{axiom}{axiom}
\newtheorem{lemma}{Lemma}
\newtheorem{theorem}{Theorem}
\newtheorem{example}{Example}
\newtheorem{definition}{Definition}

% --- LATEX AND HTML CUSTOMIZATION ---
\title{Notes on Everything}
\author{Anthony}
\date{\today}

\setcounter{tocdepth}{1}
\setcounter{secnumdepth}{-1}
\setcounter{FileDepth}{1}
\booltrue{CombineHigherDepths}
\setcounter{SideTOCDepth}{2}

% Renew commands

\renewcommand*\contentsname{Subjects}
\renewcommand\sidetocname{Subjects}

% HTML Directives
\HTMLTitle{Notes on Everything}
\HTMLAuthor{Anthony}
\HTMLLanguage{en-US}
\HTMLDescription{Personal notes on Mathematics and Computer Science}
\HTMLPageBottom{\LinkHome}

% Styling
\CSSFilename{lwarp.css}

\begin{document}

\maketitle % Or titlepage/titlingpage environment.
% An article abstract would go here.

Notes on Mathematics and Computer Science.

\tableofcontents % MUST BE BEFORE THE FIRST SECTION BREAK!

\listoffigures

\input{./sums/index}
\input{./algorithms/index}
\input{./combinatorics/index}
\input{./probability/index}
\input{./foundations/index}

\ForceHTMLPage 	% HTML index will be on its own page.
\ForceHTMLTOC 	% HTML index will have its own toc entry.
\printindex
\end{document}

% Save this as tutorial.tex for the lwarp package tutorial.
\documentclass{book}
\usepackage{iftex}

% --- LOAD FONT SELECTION AND ENCODING BEFORE LOADING LWARP ---
\ifPDFTeX
	\usepackage{lmodern} % pdflatex or dvi latex
	\usepackage[T1]{fontenc}
	\usepackage[utf8]{inputenc}
\else
	\usepackage{fontspec} % XeLaTeX or LuaLaTeX
\fi

% --- LWARP IS LOADED NEXT ---

\usepackage[
% HomeHTMLFilename=index, % Filename of the homepage.
% HTMLFilename={node-}, % Filename prefix of other pages.
% IndexLanguage=english, % Language for xindy index, glossary.
% latexmk, % Use latexmk to compile.
% OSWindows, % Force Windows. (Usually automatic.)
mathjax, % Use MathJax to display math.
]{lwarp}

% \boolfalse{FileSectionNames} % If false, numbers the files.

% --- LOAD PDFLATEX MATH FONTS HERE ---

% --- OTHER PACKAGES ARE LOADED AFTER LWARP ---

\usepackage{standalone}

\usepackage{tikz}
\usepackage{amsthm}
\usepackage{amsmath}
\usepackage{amssymb}
\usepackage{algorithm}
\usepackage{algpseudocode}

\usetikzlibrary{graphs, positioning, shapes.geometric}

\usepackage{makeidx} \makeindex
\usepackage{xcolor}               % (Demonstration purposes only.)
\usepackage{hyperref,cleveref}    % LOAD THESE LAST!

% Declare theorem environments for amsthm
\newtheorem{axiom}{axiom}
\newtheorem{lemma}{Lemma}
\newtheorem{theorem}{Theorem}
\newtheorem{example}{Example}
\newtheorem{definition}{Definition}

% --- LATEX AND HTML CUSTOMIZATION ---
\title{Notes on Everything}
\author{Anthony}
\date{\today}

\setcounter{tocdepth}{1}
\setcounter{secnumdepth}{-1}
\setcounter{FileDepth}{1}
\booltrue{CombineHigherDepths}
\setcounter{SideTOCDepth}{2}

% Renew commands

\renewcommand*\contentsname{Subjects}
\renewcommand\sidetocname{Subjects}

% HTML Directives
\HTMLTitle{Notes on Everything}
\HTMLAuthor{Anthony}
\HTMLLanguage{en-US}
\HTMLDescription{Personal notes on Mathematics and Computer Science}
\HTMLPageBottom{\LinkHome}

% Styling
\CSSFilename{lwarp.css}

\begin{document}

\maketitle % Or titlepage/titlingpage environment.
% An article abstract would go here.

Notes on Mathematics and Computer Science.

\tableofcontents % MUST BE BEFORE THE FIRST SECTION BREAK!

\listoffigures

\input{./sums/index}
\input{./algorithms/index}
\input{./combinatorics/index}
\input{./probability/index}
\input{./foundations/index}

\ForceHTMLPage 	% HTML index will be on its own page.
\ForceHTMLTOC 	% HTML index will have its own toc entry.
\printindex
\end{document}

% Save this as tutorial.tex for the lwarp package tutorial.
\documentclass{book}
\usepackage{iftex}

% --- LOAD FONT SELECTION AND ENCODING BEFORE LOADING LWARP ---
\ifPDFTeX
	\usepackage{lmodern} % pdflatex or dvi latex
	\usepackage[T1]{fontenc}
	\usepackage[utf8]{inputenc}
\else
	\usepackage{fontspec} % XeLaTeX or LuaLaTeX
\fi

% --- LWARP IS LOADED NEXT ---

\usepackage[
% HomeHTMLFilename=index, % Filename of the homepage.
% HTMLFilename={node-}, % Filename prefix of other pages.
% IndexLanguage=english, % Language for xindy index, glossary.
% latexmk, % Use latexmk to compile.
% OSWindows, % Force Windows. (Usually automatic.)
mathjax, % Use MathJax to display math.
]{lwarp}

% \boolfalse{FileSectionNames} % If false, numbers the files.

% --- LOAD PDFLATEX MATH FONTS HERE ---

% --- OTHER PACKAGES ARE LOADED AFTER LWARP ---

\usepackage{standalone}

\usepackage{tikz}
\usepackage{amsthm}
\usepackage{amsmath}
\usepackage{amssymb}
\usepackage{algorithm}
\usepackage{algpseudocode}

\usetikzlibrary{graphs, positioning, shapes.geometric}

\usepackage{makeidx} \makeindex
\usepackage{xcolor}               % (Demonstration purposes only.)
\usepackage{hyperref,cleveref}    % LOAD THESE LAST!

% Declare theorem environments for amsthm
\newtheorem{axiom}{axiom}
\newtheorem{lemma}{Lemma}
\newtheorem{theorem}{Theorem}
\newtheorem{example}{Example}
\newtheorem{definition}{Definition}

% --- LATEX AND HTML CUSTOMIZATION ---
\title{Notes on Everything}
\author{Anthony}
\date{\today}

\setcounter{tocdepth}{1}
\setcounter{secnumdepth}{-1}
\setcounter{FileDepth}{1}
\booltrue{CombineHigherDepths}
\setcounter{SideTOCDepth}{2}

% Renew commands

\renewcommand*\contentsname{Subjects}
\renewcommand\sidetocname{Subjects}

% HTML Directives
\HTMLTitle{Notes on Everything}
\HTMLAuthor{Anthony}
\HTMLLanguage{en-US}
\HTMLDescription{Personal notes on Mathematics and Computer Science}
\HTMLPageBottom{\LinkHome}

% Styling
\CSSFilename{lwarp.css}

\begin{document}

\maketitle % Or titlepage/titlingpage environment.
% An article abstract would go here.

Notes on Mathematics and Computer Science.

\tableofcontents % MUST BE BEFORE THE FIRST SECTION BREAK!

\listoffigures

\input{./sums/index}
\input{./algorithms/index}
\input{./combinatorics/index}
\input{./probability/index}
\input{./foundations/index}

\ForceHTMLPage 	% HTML index will be on its own page.
\ForceHTMLTOC 	% HTML index will have its own toc entry.
\printindex
\end{document}


\ForceHTMLPage 	% HTML index will be on its own page.
\ForceHTMLTOC 	% HTML index will have its own toc entry.
\printindex
\end{document}

% Save this as tutorial.tex for the lwarp package tutorial.
\documentclass{book}
\usepackage{iftex}

% --- LOAD FONT SELECTION AND ENCODING BEFORE LOADING LWARP ---
\ifPDFTeX
	\usepackage{lmodern} % pdflatex or dvi latex
	\usepackage[T1]{fontenc}
	\usepackage[utf8]{inputenc}
\else
	\usepackage{fontspec} % XeLaTeX or LuaLaTeX
\fi

% --- LWARP IS LOADED NEXT ---

\usepackage[
% HomeHTMLFilename=index, % Filename of the homepage.
% HTMLFilename={node-}, % Filename prefix of other pages.
% IndexLanguage=english, % Language for xindy index, glossary.
% latexmk, % Use latexmk to compile.
% OSWindows, % Force Windows. (Usually automatic.)
mathjax, % Use MathJax to display math.
]{lwarp}

% \boolfalse{FileSectionNames} % If false, numbers the files.

% --- LOAD PDFLATEX MATH FONTS HERE ---

% --- OTHER PACKAGES ARE LOADED AFTER LWARP ---

\usepackage{standalone}

\usepackage{tikz}
\usepackage{amsthm}
\usepackage{amsmath}
\usepackage{amssymb}
\usepackage{algorithm}
\usepackage{algpseudocode}

\usetikzlibrary{graphs, positioning, shapes.geometric}

\usepackage{makeidx} \makeindex
\usepackage{xcolor}               % (Demonstration purposes only.)
\usepackage{hyperref,cleveref}    % LOAD THESE LAST!

% Declare theorem environments for amsthm
\newtheorem{axiom}{axiom}
\newtheorem{lemma}{Lemma}
\newtheorem{theorem}{Theorem}
\newtheorem{example}{Example}
\newtheorem{definition}{Definition}

% --- LATEX AND HTML CUSTOMIZATION ---
\title{Notes on Everything}
\author{Anthony}
\date{\today}

\setcounter{tocdepth}{1}
\setcounter{secnumdepth}{-1}
\setcounter{FileDepth}{1}
\booltrue{CombineHigherDepths}
\setcounter{SideTOCDepth}{2}

% Renew commands

\renewcommand*\contentsname{Subjects}
\renewcommand\sidetocname{Subjects}

% HTML Directives
\HTMLTitle{Notes on Everything}
\HTMLAuthor{Anthony}
\HTMLLanguage{en-US}
\HTMLDescription{Personal notes on Mathematics and Computer Science}
\HTMLPageBottom{\LinkHome}

% Styling
\CSSFilename{lwarp.css}

\begin{document}

\maketitle % Or titlepage/titlingpage environment.
% An article abstract would go here.

Notes on Mathematics and Computer Science.

\tableofcontents % MUST BE BEFORE THE FIRST SECTION BREAK!

\listoffigures

% Save this as tutorial.tex for the lwarp package tutorial.
\documentclass{book}
\usepackage{iftex}

% --- LOAD FONT SELECTION AND ENCODING BEFORE LOADING LWARP ---
\ifPDFTeX
	\usepackage{lmodern} % pdflatex or dvi latex
	\usepackage[T1]{fontenc}
	\usepackage[utf8]{inputenc}
\else
	\usepackage{fontspec} % XeLaTeX or LuaLaTeX
\fi

% --- LWARP IS LOADED NEXT ---

\usepackage[
% HomeHTMLFilename=index, % Filename of the homepage.
% HTMLFilename={node-}, % Filename prefix of other pages.
% IndexLanguage=english, % Language for xindy index, glossary.
% latexmk, % Use latexmk to compile.
% OSWindows, % Force Windows. (Usually automatic.)
mathjax, % Use MathJax to display math.
]{lwarp}

% \boolfalse{FileSectionNames} % If false, numbers the files.

% --- LOAD PDFLATEX MATH FONTS HERE ---

% --- OTHER PACKAGES ARE LOADED AFTER LWARP ---

\usepackage{standalone}

\usepackage{tikz}
\usepackage{amsthm}
\usepackage{amsmath}
\usepackage{amssymb}
\usepackage{algorithm}
\usepackage{algpseudocode}

\usetikzlibrary{graphs, positioning, shapes.geometric}

\usepackage{makeidx} \makeindex
\usepackage{xcolor}               % (Demonstration purposes only.)
\usepackage{hyperref,cleveref}    % LOAD THESE LAST!

% Declare theorem environments for amsthm
\newtheorem{axiom}{axiom}
\newtheorem{lemma}{Lemma}
\newtheorem{theorem}{Theorem}
\newtheorem{example}{Example}
\newtheorem{definition}{Definition}

% --- LATEX AND HTML CUSTOMIZATION ---
\title{Notes on Everything}
\author{Anthony}
\date{\today}

\setcounter{tocdepth}{1}
\setcounter{secnumdepth}{-1}
\setcounter{FileDepth}{1}
\booltrue{CombineHigherDepths}
\setcounter{SideTOCDepth}{2}

% Renew commands

\renewcommand*\contentsname{Subjects}
\renewcommand\sidetocname{Subjects}

% HTML Directives
\HTMLTitle{Notes on Everything}
\HTMLAuthor{Anthony}
\HTMLLanguage{en-US}
\HTMLDescription{Personal notes on Mathematics and Computer Science}
\HTMLPageBottom{\LinkHome}

% Styling
\CSSFilename{lwarp.css}

\begin{document}

\maketitle % Or titlepage/titlingpage environment.
% An article abstract would go here.

Notes on Mathematics and Computer Science.

\tableofcontents % MUST BE BEFORE THE FIRST SECTION BREAK!

\listoffigures

\input{./sums/index}
\input{./algorithms/index}
\input{./combinatorics/index}
\input{./probability/index}
\input{./foundations/index}

\ForceHTMLPage 	% HTML index will be on its own page.
\ForceHTMLTOC 	% HTML index will have its own toc entry.
\printindex
\end{document}

% Save this as tutorial.tex for the lwarp package tutorial.
\documentclass{book}
\usepackage{iftex}

% --- LOAD FONT SELECTION AND ENCODING BEFORE LOADING LWARP ---
\ifPDFTeX
	\usepackage{lmodern} % pdflatex or dvi latex
	\usepackage[T1]{fontenc}
	\usepackage[utf8]{inputenc}
\else
	\usepackage{fontspec} % XeLaTeX or LuaLaTeX
\fi

% --- LWARP IS LOADED NEXT ---

\usepackage[
% HomeHTMLFilename=index, % Filename of the homepage.
% HTMLFilename={node-}, % Filename prefix of other pages.
% IndexLanguage=english, % Language for xindy index, glossary.
% latexmk, % Use latexmk to compile.
% OSWindows, % Force Windows. (Usually automatic.)
mathjax, % Use MathJax to display math.
]{lwarp}

% \boolfalse{FileSectionNames} % If false, numbers the files.

% --- LOAD PDFLATEX MATH FONTS HERE ---

% --- OTHER PACKAGES ARE LOADED AFTER LWARP ---

\usepackage{standalone}

\usepackage{tikz}
\usepackage{amsthm}
\usepackage{amsmath}
\usepackage{amssymb}
\usepackage{algorithm}
\usepackage{algpseudocode}

\usetikzlibrary{graphs, positioning, shapes.geometric}

\usepackage{makeidx} \makeindex
\usepackage{xcolor}               % (Demonstration purposes only.)
\usepackage{hyperref,cleveref}    % LOAD THESE LAST!

% Declare theorem environments for amsthm
\newtheorem{axiom}{axiom}
\newtheorem{lemma}{Lemma}
\newtheorem{theorem}{Theorem}
\newtheorem{example}{Example}
\newtheorem{definition}{Definition}

% --- LATEX AND HTML CUSTOMIZATION ---
\title{Notes on Everything}
\author{Anthony}
\date{\today}

\setcounter{tocdepth}{1}
\setcounter{secnumdepth}{-1}
\setcounter{FileDepth}{1}
\booltrue{CombineHigherDepths}
\setcounter{SideTOCDepth}{2}

% Renew commands

\renewcommand*\contentsname{Subjects}
\renewcommand\sidetocname{Subjects}

% HTML Directives
\HTMLTitle{Notes on Everything}
\HTMLAuthor{Anthony}
\HTMLLanguage{en-US}
\HTMLDescription{Personal notes on Mathematics and Computer Science}
\HTMLPageBottom{\LinkHome}

% Styling
\CSSFilename{lwarp.css}

\begin{document}

\maketitle % Or titlepage/titlingpage environment.
% An article abstract would go here.

Notes on Mathematics and Computer Science.

\tableofcontents % MUST BE BEFORE THE FIRST SECTION BREAK!

\listoffigures

\input{./sums/index}
\input{./algorithms/index}
\input{./combinatorics/index}
\input{./probability/index}
\input{./foundations/index}

\ForceHTMLPage 	% HTML index will be on its own page.
\ForceHTMLTOC 	% HTML index will have its own toc entry.
\printindex
\end{document}

% Save this as tutorial.tex for the lwarp package tutorial.
\documentclass{book}
\usepackage{iftex}

% --- LOAD FONT SELECTION AND ENCODING BEFORE LOADING LWARP ---
\ifPDFTeX
	\usepackage{lmodern} % pdflatex or dvi latex
	\usepackage[T1]{fontenc}
	\usepackage[utf8]{inputenc}
\else
	\usepackage{fontspec} % XeLaTeX or LuaLaTeX
\fi

% --- LWARP IS LOADED NEXT ---

\usepackage[
% HomeHTMLFilename=index, % Filename of the homepage.
% HTMLFilename={node-}, % Filename prefix of other pages.
% IndexLanguage=english, % Language for xindy index, glossary.
% latexmk, % Use latexmk to compile.
% OSWindows, % Force Windows. (Usually automatic.)
mathjax, % Use MathJax to display math.
]{lwarp}

% \boolfalse{FileSectionNames} % If false, numbers the files.

% --- LOAD PDFLATEX MATH FONTS HERE ---

% --- OTHER PACKAGES ARE LOADED AFTER LWARP ---

\usepackage{standalone}

\usepackage{tikz}
\usepackage{amsthm}
\usepackage{amsmath}
\usepackage{amssymb}
\usepackage{algorithm}
\usepackage{algpseudocode}

\usetikzlibrary{graphs, positioning, shapes.geometric}

\usepackage{makeidx} \makeindex
\usepackage{xcolor}               % (Demonstration purposes only.)
\usepackage{hyperref,cleveref}    % LOAD THESE LAST!

% Declare theorem environments for amsthm
\newtheorem{axiom}{axiom}
\newtheorem{lemma}{Lemma}
\newtheorem{theorem}{Theorem}
\newtheorem{example}{Example}
\newtheorem{definition}{Definition}

% --- LATEX AND HTML CUSTOMIZATION ---
\title{Notes on Everything}
\author{Anthony}
\date{\today}

\setcounter{tocdepth}{1}
\setcounter{secnumdepth}{-1}
\setcounter{FileDepth}{1}
\booltrue{CombineHigherDepths}
\setcounter{SideTOCDepth}{2}

% Renew commands

\renewcommand*\contentsname{Subjects}
\renewcommand\sidetocname{Subjects}

% HTML Directives
\HTMLTitle{Notes on Everything}
\HTMLAuthor{Anthony}
\HTMLLanguage{en-US}
\HTMLDescription{Personal notes on Mathematics and Computer Science}
\HTMLPageBottom{\LinkHome}

% Styling
\CSSFilename{lwarp.css}

\begin{document}

\maketitle % Or titlepage/titlingpage environment.
% An article abstract would go here.

Notes on Mathematics and Computer Science.

\tableofcontents % MUST BE BEFORE THE FIRST SECTION BREAK!

\listoffigures

\input{./sums/index}
\input{./algorithms/index}
\input{./combinatorics/index}
\input{./probability/index}
\input{./foundations/index}

\ForceHTMLPage 	% HTML index will be on its own page.
\ForceHTMLTOC 	% HTML index will have its own toc entry.
\printindex
\end{document}

% Save this as tutorial.tex for the lwarp package tutorial.
\documentclass{book}
\usepackage{iftex}

% --- LOAD FONT SELECTION AND ENCODING BEFORE LOADING LWARP ---
\ifPDFTeX
	\usepackage{lmodern} % pdflatex or dvi latex
	\usepackage[T1]{fontenc}
	\usepackage[utf8]{inputenc}
\else
	\usepackage{fontspec} % XeLaTeX or LuaLaTeX
\fi

% --- LWARP IS LOADED NEXT ---

\usepackage[
% HomeHTMLFilename=index, % Filename of the homepage.
% HTMLFilename={node-}, % Filename prefix of other pages.
% IndexLanguage=english, % Language for xindy index, glossary.
% latexmk, % Use latexmk to compile.
% OSWindows, % Force Windows. (Usually automatic.)
mathjax, % Use MathJax to display math.
]{lwarp}

% \boolfalse{FileSectionNames} % If false, numbers the files.

% --- LOAD PDFLATEX MATH FONTS HERE ---

% --- OTHER PACKAGES ARE LOADED AFTER LWARP ---

\usepackage{standalone}

\usepackage{tikz}
\usepackage{amsthm}
\usepackage{amsmath}
\usepackage{amssymb}
\usepackage{algorithm}
\usepackage{algpseudocode}

\usetikzlibrary{graphs, positioning, shapes.geometric}

\usepackage{makeidx} \makeindex
\usepackage{xcolor}               % (Demonstration purposes only.)
\usepackage{hyperref,cleveref}    % LOAD THESE LAST!

% Declare theorem environments for amsthm
\newtheorem{axiom}{axiom}
\newtheorem{lemma}{Lemma}
\newtheorem{theorem}{Theorem}
\newtheorem{example}{Example}
\newtheorem{definition}{Definition}

% --- LATEX AND HTML CUSTOMIZATION ---
\title{Notes on Everything}
\author{Anthony}
\date{\today}

\setcounter{tocdepth}{1}
\setcounter{secnumdepth}{-1}
\setcounter{FileDepth}{1}
\booltrue{CombineHigherDepths}
\setcounter{SideTOCDepth}{2}

% Renew commands

\renewcommand*\contentsname{Subjects}
\renewcommand\sidetocname{Subjects}

% HTML Directives
\HTMLTitle{Notes on Everything}
\HTMLAuthor{Anthony}
\HTMLLanguage{en-US}
\HTMLDescription{Personal notes on Mathematics and Computer Science}
\HTMLPageBottom{\LinkHome}

% Styling
\CSSFilename{lwarp.css}

\begin{document}

\maketitle % Or titlepage/titlingpage environment.
% An article abstract would go here.

Notes on Mathematics and Computer Science.

\tableofcontents % MUST BE BEFORE THE FIRST SECTION BREAK!

\listoffigures

\input{./sums/index}
\input{./algorithms/index}
\input{./combinatorics/index}
\input{./probability/index}
\input{./foundations/index}

\ForceHTMLPage 	% HTML index will be on its own page.
\ForceHTMLTOC 	% HTML index will have its own toc entry.
\printindex
\end{document}

% Save this as tutorial.tex for the lwarp package tutorial.
\documentclass{book}
\usepackage{iftex}

% --- LOAD FONT SELECTION AND ENCODING BEFORE LOADING LWARP ---
\ifPDFTeX
	\usepackage{lmodern} % pdflatex or dvi latex
	\usepackage[T1]{fontenc}
	\usepackage[utf8]{inputenc}
\else
	\usepackage{fontspec} % XeLaTeX or LuaLaTeX
\fi

% --- LWARP IS LOADED NEXT ---

\usepackage[
% HomeHTMLFilename=index, % Filename of the homepage.
% HTMLFilename={node-}, % Filename prefix of other pages.
% IndexLanguage=english, % Language for xindy index, glossary.
% latexmk, % Use latexmk to compile.
% OSWindows, % Force Windows. (Usually automatic.)
mathjax, % Use MathJax to display math.
]{lwarp}

% \boolfalse{FileSectionNames} % If false, numbers the files.

% --- LOAD PDFLATEX MATH FONTS HERE ---

% --- OTHER PACKAGES ARE LOADED AFTER LWARP ---

\usepackage{standalone}

\usepackage{tikz}
\usepackage{amsthm}
\usepackage{amsmath}
\usepackage{amssymb}
\usepackage{algorithm}
\usepackage{algpseudocode}

\usetikzlibrary{graphs, positioning, shapes.geometric}

\usepackage{makeidx} \makeindex
\usepackage{xcolor}               % (Demonstration purposes only.)
\usepackage{hyperref,cleveref}    % LOAD THESE LAST!

% Declare theorem environments for amsthm
\newtheorem{axiom}{axiom}
\newtheorem{lemma}{Lemma}
\newtheorem{theorem}{Theorem}
\newtheorem{example}{Example}
\newtheorem{definition}{Definition}

% --- LATEX AND HTML CUSTOMIZATION ---
\title{Notes on Everything}
\author{Anthony}
\date{\today}

\setcounter{tocdepth}{1}
\setcounter{secnumdepth}{-1}
\setcounter{FileDepth}{1}
\booltrue{CombineHigherDepths}
\setcounter{SideTOCDepth}{2}

% Renew commands

\renewcommand*\contentsname{Subjects}
\renewcommand\sidetocname{Subjects}

% HTML Directives
\HTMLTitle{Notes on Everything}
\HTMLAuthor{Anthony}
\HTMLLanguage{en-US}
\HTMLDescription{Personal notes on Mathematics and Computer Science}
\HTMLPageBottom{\LinkHome}

% Styling
\CSSFilename{lwarp.css}

\begin{document}

\maketitle % Or titlepage/titlingpage environment.
% An article abstract would go here.

Notes on Mathematics and Computer Science.

\tableofcontents % MUST BE BEFORE THE FIRST SECTION BREAK!

\listoffigures

\input{./sums/index}
\input{./algorithms/index}
\input{./combinatorics/index}
\input{./probability/index}
\input{./foundations/index}

\ForceHTMLPage 	% HTML index will be on its own page.
\ForceHTMLTOC 	% HTML index will have its own toc entry.
\printindex
\end{document}


\ForceHTMLPage 	% HTML index will be on its own page.
\ForceHTMLTOC 	% HTML index will have its own toc entry.
\printindex
\end{document}

% Save this as tutorial.tex for the lwarp package tutorial.
\documentclass{book}
\usepackage{iftex}

% --- LOAD FONT SELECTION AND ENCODING BEFORE LOADING LWARP ---
\ifPDFTeX
	\usepackage{lmodern} % pdflatex or dvi latex
	\usepackage[T1]{fontenc}
	\usepackage[utf8]{inputenc}
\else
	\usepackage{fontspec} % XeLaTeX or LuaLaTeX
\fi

% --- LWARP IS LOADED NEXT ---

\usepackage[
% HomeHTMLFilename=index, % Filename of the homepage.
% HTMLFilename={node-}, % Filename prefix of other pages.
% IndexLanguage=english, % Language for xindy index, glossary.
% latexmk, % Use latexmk to compile.
% OSWindows, % Force Windows. (Usually automatic.)
mathjax, % Use MathJax to display math.
]{lwarp}

% \boolfalse{FileSectionNames} % If false, numbers the files.

% --- LOAD PDFLATEX MATH FONTS HERE ---

% --- OTHER PACKAGES ARE LOADED AFTER LWARP ---

\usepackage{standalone}

\usepackage{tikz}
\usepackage{amsthm}
\usepackage{amsmath}
\usepackage{amssymb}
\usepackage{algorithm}
\usepackage{algpseudocode}

\usetikzlibrary{graphs, positioning, shapes.geometric}

\usepackage{makeidx} \makeindex
\usepackage{xcolor}               % (Demonstration purposes only.)
\usepackage{hyperref,cleveref}    % LOAD THESE LAST!

% Declare theorem environments for amsthm
\newtheorem{axiom}{axiom}
\newtheorem{lemma}{Lemma}
\newtheorem{theorem}{Theorem}
\newtheorem{example}{Example}
\newtheorem{definition}{Definition}

% --- LATEX AND HTML CUSTOMIZATION ---
\title{Notes on Everything}
\author{Anthony}
\date{\today}

\setcounter{tocdepth}{1}
\setcounter{secnumdepth}{-1}
\setcounter{FileDepth}{1}
\booltrue{CombineHigherDepths}
\setcounter{SideTOCDepth}{2}

% Renew commands

\renewcommand*\contentsname{Subjects}
\renewcommand\sidetocname{Subjects}

% HTML Directives
\HTMLTitle{Notes on Everything}
\HTMLAuthor{Anthony}
\HTMLLanguage{en-US}
\HTMLDescription{Personal notes on Mathematics and Computer Science}
\HTMLPageBottom{\LinkHome}

% Styling
\CSSFilename{lwarp.css}

\begin{document}

\maketitle % Or titlepage/titlingpage environment.
% An article abstract would go here.

Notes on Mathematics and Computer Science.

\tableofcontents % MUST BE BEFORE THE FIRST SECTION BREAK!

\listoffigures

% Save this as tutorial.tex for the lwarp package tutorial.
\documentclass{book}
\usepackage{iftex}

% --- LOAD FONT SELECTION AND ENCODING BEFORE LOADING LWARP ---
\ifPDFTeX
	\usepackage{lmodern} % pdflatex or dvi latex
	\usepackage[T1]{fontenc}
	\usepackage[utf8]{inputenc}
\else
	\usepackage{fontspec} % XeLaTeX or LuaLaTeX
\fi

% --- LWARP IS LOADED NEXT ---

\usepackage[
% HomeHTMLFilename=index, % Filename of the homepage.
% HTMLFilename={node-}, % Filename prefix of other pages.
% IndexLanguage=english, % Language for xindy index, glossary.
% latexmk, % Use latexmk to compile.
% OSWindows, % Force Windows. (Usually automatic.)
mathjax, % Use MathJax to display math.
]{lwarp}

% \boolfalse{FileSectionNames} % If false, numbers the files.

% --- LOAD PDFLATEX MATH FONTS HERE ---

% --- OTHER PACKAGES ARE LOADED AFTER LWARP ---

\usepackage{standalone}

\usepackage{tikz}
\usepackage{amsthm}
\usepackage{amsmath}
\usepackage{amssymb}
\usepackage{algorithm}
\usepackage{algpseudocode}

\usetikzlibrary{graphs, positioning, shapes.geometric}

\usepackage{makeidx} \makeindex
\usepackage{xcolor}               % (Demonstration purposes only.)
\usepackage{hyperref,cleveref}    % LOAD THESE LAST!

% Declare theorem environments for amsthm
\newtheorem{axiom}{axiom}
\newtheorem{lemma}{Lemma}
\newtheorem{theorem}{Theorem}
\newtheorem{example}{Example}
\newtheorem{definition}{Definition}

% --- LATEX AND HTML CUSTOMIZATION ---
\title{Notes on Everything}
\author{Anthony}
\date{\today}

\setcounter{tocdepth}{1}
\setcounter{secnumdepth}{-1}
\setcounter{FileDepth}{1}
\booltrue{CombineHigherDepths}
\setcounter{SideTOCDepth}{2}

% Renew commands

\renewcommand*\contentsname{Subjects}
\renewcommand\sidetocname{Subjects}

% HTML Directives
\HTMLTitle{Notes on Everything}
\HTMLAuthor{Anthony}
\HTMLLanguage{en-US}
\HTMLDescription{Personal notes on Mathematics and Computer Science}
\HTMLPageBottom{\LinkHome}

% Styling
\CSSFilename{lwarp.css}

\begin{document}

\maketitle % Or titlepage/titlingpage environment.
% An article abstract would go here.

Notes on Mathematics and Computer Science.

\tableofcontents % MUST BE BEFORE THE FIRST SECTION BREAK!

\listoffigures

\input{./sums/index}
\input{./algorithms/index}
\input{./combinatorics/index}
\input{./probability/index}
\input{./foundations/index}

\ForceHTMLPage 	% HTML index will be on its own page.
\ForceHTMLTOC 	% HTML index will have its own toc entry.
\printindex
\end{document}

% Save this as tutorial.tex for the lwarp package tutorial.
\documentclass{book}
\usepackage{iftex}

% --- LOAD FONT SELECTION AND ENCODING BEFORE LOADING LWARP ---
\ifPDFTeX
	\usepackage{lmodern} % pdflatex or dvi latex
	\usepackage[T1]{fontenc}
	\usepackage[utf8]{inputenc}
\else
	\usepackage{fontspec} % XeLaTeX or LuaLaTeX
\fi

% --- LWARP IS LOADED NEXT ---

\usepackage[
% HomeHTMLFilename=index, % Filename of the homepage.
% HTMLFilename={node-}, % Filename prefix of other pages.
% IndexLanguage=english, % Language for xindy index, glossary.
% latexmk, % Use latexmk to compile.
% OSWindows, % Force Windows. (Usually automatic.)
mathjax, % Use MathJax to display math.
]{lwarp}

% \boolfalse{FileSectionNames} % If false, numbers the files.

% --- LOAD PDFLATEX MATH FONTS HERE ---

% --- OTHER PACKAGES ARE LOADED AFTER LWARP ---

\usepackage{standalone}

\usepackage{tikz}
\usepackage{amsthm}
\usepackage{amsmath}
\usepackage{amssymb}
\usepackage{algorithm}
\usepackage{algpseudocode}

\usetikzlibrary{graphs, positioning, shapes.geometric}

\usepackage{makeidx} \makeindex
\usepackage{xcolor}               % (Demonstration purposes only.)
\usepackage{hyperref,cleveref}    % LOAD THESE LAST!

% Declare theorem environments for amsthm
\newtheorem{axiom}{axiom}
\newtheorem{lemma}{Lemma}
\newtheorem{theorem}{Theorem}
\newtheorem{example}{Example}
\newtheorem{definition}{Definition}

% --- LATEX AND HTML CUSTOMIZATION ---
\title{Notes on Everything}
\author{Anthony}
\date{\today}

\setcounter{tocdepth}{1}
\setcounter{secnumdepth}{-1}
\setcounter{FileDepth}{1}
\booltrue{CombineHigherDepths}
\setcounter{SideTOCDepth}{2}

% Renew commands

\renewcommand*\contentsname{Subjects}
\renewcommand\sidetocname{Subjects}

% HTML Directives
\HTMLTitle{Notes on Everything}
\HTMLAuthor{Anthony}
\HTMLLanguage{en-US}
\HTMLDescription{Personal notes on Mathematics and Computer Science}
\HTMLPageBottom{\LinkHome}

% Styling
\CSSFilename{lwarp.css}

\begin{document}

\maketitle % Or titlepage/titlingpage environment.
% An article abstract would go here.

Notes on Mathematics and Computer Science.

\tableofcontents % MUST BE BEFORE THE FIRST SECTION BREAK!

\listoffigures

\input{./sums/index}
\input{./algorithms/index}
\input{./combinatorics/index}
\input{./probability/index}
\input{./foundations/index}

\ForceHTMLPage 	% HTML index will be on its own page.
\ForceHTMLTOC 	% HTML index will have its own toc entry.
\printindex
\end{document}

% Save this as tutorial.tex for the lwarp package tutorial.
\documentclass{book}
\usepackage{iftex}

% --- LOAD FONT SELECTION AND ENCODING BEFORE LOADING LWARP ---
\ifPDFTeX
	\usepackage{lmodern} % pdflatex or dvi latex
	\usepackage[T1]{fontenc}
	\usepackage[utf8]{inputenc}
\else
	\usepackage{fontspec} % XeLaTeX or LuaLaTeX
\fi

% --- LWARP IS LOADED NEXT ---

\usepackage[
% HomeHTMLFilename=index, % Filename of the homepage.
% HTMLFilename={node-}, % Filename prefix of other pages.
% IndexLanguage=english, % Language for xindy index, glossary.
% latexmk, % Use latexmk to compile.
% OSWindows, % Force Windows. (Usually automatic.)
mathjax, % Use MathJax to display math.
]{lwarp}

% \boolfalse{FileSectionNames} % If false, numbers the files.

% --- LOAD PDFLATEX MATH FONTS HERE ---

% --- OTHER PACKAGES ARE LOADED AFTER LWARP ---

\usepackage{standalone}

\usepackage{tikz}
\usepackage{amsthm}
\usepackage{amsmath}
\usepackage{amssymb}
\usepackage{algorithm}
\usepackage{algpseudocode}

\usetikzlibrary{graphs, positioning, shapes.geometric}

\usepackage{makeidx} \makeindex
\usepackage{xcolor}               % (Demonstration purposes only.)
\usepackage{hyperref,cleveref}    % LOAD THESE LAST!

% Declare theorem environments for amsthm
\newtheorem{axiom}{axiom}
\newtheorem{lemma}{Lemma}
\newtheorem{theorem}{Theorem}
\newtheorem{example}{Example}
\newtheorem{definition}{Definition}

% --- LATEX AND HTML CUSTOMIZATION ---
\title{Notes on Everything}
\author{Anthony}
\date{\today}

\setcounter{tocdepth}{1}
\setcounter{secnumdepth}{-1}
\setcounter{FileDepth}{1}
\booltrue{CombineHigherDepths}
\setcounter{SideTOCDepth}{2}

% Renew commands

\renewcommand*\contentsname{Subjects}
\renewcommand\sidetocname{Subjects}

% HTML Directives
\HTMLTitle{Notes on Everything}
\HTMLAuthor{Anthony}
\HTMLLanguage{en-US}
\HTMLDescription{Personal notes on Mathematics and Computer Science}
\HTMLPageBottom{\LinkHome}

% Styling
\CSSFilename{lwarp.css}

\begin{document}

\maketitle % Or titlepage/titlingpage environment.
% An article abstract would go here.

Notes on Mathematics and Computer Science.

\tableofcontents % MUST BE BEFORE THE FIRST SECTION BREAK!

\listoffigures

\input{./sums/index}
\input{./algorithms/index}
\input{./combinatorics/index}
\input{./probability/index}
\input{./foundations/index}

\ForceHTMLPage 	% HTML index will be on its own page.
\ForceHTMLTOC 	% HTML index will have its own toc entry.
\printindex
\end{document}

% Save this as tutorial.tex for the lwarp package tutorial.
\documentclass{book}
\usepackage{iftex}

% --- LOAD FONT SELECTION AND ENCODING BEFORE LOADING LWARP ---
\ifPDFTeX
	\usepackage{lmodern} % pdflatex or dvi latex
	\usepackage[T1]{fontenc}
	\usepackage[utf8]{inputenc}
\else
	\usepackage{fontspec} % XeLaTeX or LuaLaTeX
\fi

% --- LWARP IS LOADED NEXT ---

\usepackage[
% HomeHTMLFilename=index, % Filename of the homepage.
% HTMLFilename={node-}, % Filename prefix of other pages.
% IndexLanguage=english, % Language for xindy index, glossary.
% latexmk, % Use latexmk to compile.
% OSWindows, % Force Windows. (Usually automatic.)
mathjax, % Use MathJax to display math.
]{lwarp}

% \boolfalse{FileSectionNames} % If false, numbers the files.

% --- LOAD PDFLATEX MATH FONTS HERE ---

% --- OTHER PACKAGES ARE LOADED AFTER LWARP ---

\usepackage{standalone}

\usepackage{tikz}
\usepackage{amsthm}
\usepackage{amsmath}
\usepackage{amssymb}
\usepackage{algorithm}
\usepackage{algpseudocode}

\usetikzlibrary{graphs, positioning, shapes.geometric}

\usepackage{makeidx} \makeindex
\usepackage{xcolor}               % (Demonstration purposes only.)
\usepackage{hyperref,cleveref}    % LOAD THESE LAST!

% Declare theorem environments for amsthm
\newtheorem{axiom}{axiom}
\newtheorem{lemma}{Lemma}
\newtheorem{theorem}{Theorem}
\newtheorem{example}{Example}
\newtheorem{definition}{Definition}

% --- LATEX AND HTML CUSTOMIZATION ---
\title{Notes on Everything}
\author{Anthony}
\date{\today}

\setcounter{tocdepth}{1}
\setcounter{secnumdepth}{-1}
\setcounter{FileDepth}{1}
\booltrue{CombineHigherDepths}
\setcounter{SideTOCDepth}{2}

% Renew commands

\renewcommand*\contentsname{Subjects}
\renewcommand\sidetocname{Subjects}

% HTML Directives
\HTMLTitle{Notes on Everything}
\HTMLAuthor{Anthony}
\HTMLLanguage{en-US}
\HTMLDescription{Personal notes on Mathematics and Computer Science}
\HTMLPageBottom{\LinkHome}

% Styling
\CSSFilename{lwarp.css}

\begin{document}

\maketitle % Or titlepage/titlingpage environment.
% An article abstract would go here.

Notes on Mathematics and Computer Science.

\tableofcontents % MUST BE BEFORE THE FIRST SECTION BREAK!

\listoffigures

\input{./sums/index}
\input{./algorithms/index}
\input{./combinatorics/index}
\input{./probability/index}
\input{./foundations/index}

\ForceHTMLPage 	% HTML index will be on its own page.
\ForceHTMLTOC 	% HTML index will have its own toc entry.
\printindex
\end{document}

% Save this as tutorial.tex for the lwarp package tutorial.
\documentclass{book}
\usepackage{iftex}

% --- LOAD FONT SELECTION AND ENCODING BEFORE LOADING LWARP ---
\ifPDFTeX
	\usepackage{lmodern} % pdflatex or dvi latex
	\usepackage[T1]{fontenc}
	\usepackage[utf8]{inputenc}
\else
	\usepackage{fontspec} % XeLaTeX or LuaLaTeX
\fi

% --- LWARP IS LOADED NEXT ---

\usepackage[
% HomeHTMLFilename=index, % Filename of the homepage.
% HTMLFilename={node-}, % Filename prefix of other pages.
% IndexLanguage=english, % Language for xindy index, glossary.
% latexmk, % Use latexmk to compile.
% OSWindows, % Force Windows. (Usually automatic.)
mathjax, % Use MathJax to display math.
]{lwarp}

% \boolfalse{FileSectionNames} % If false, numbers the files.

% --- LOAD PDFLATEX MATH FONTS HERE ---

% --- OTHER PACKAGES ARE LOADED AFTER LWARP ---

\usepackage{standalone}

\usepackage{tikz}
\usepackage{amsthm}
\usepackage{amsmath}
\usepackage{amssymb}
\usepackage{algorithm}
\usepackage{algpseudocode}

\usetikzlibrary{graphs, positioning, shapes.geometric}

\usepackage{makeidx} \makeindex
\usepackage{xcolor}               % (Demonstration purposes only.)
\usepackage{hyperref,cleveref}    % LOAD THESE LAST!

% Declare theorem environments for amsthm
\newtheorem{axiom}{axiom}
\newtheorem{lemma}{Lemma}
\newtheorem{theorem}{Theorem}
\newtheorem{example}{Example}
\newtheorem{definition}{Definition}

% --- LATEX AND HTML CUSTOMIZATION ---
\title{Notes on Everything}
\author{Anthony}
\date{\today}

\setcounter{tocdepth}{1}
\setcounter{secnumdepth}{-1}
\setcounter{FileDepth}{1}
\booltrue{CombineHigherDepths}
\setcounter{SideTOCDepth}{2}

% Renew commands

\renewcommand*\contentsname{Subjects}
\renewcommand\sidetocname{Subjects}

% HTML Directives
\HTMLTitle{Notes on Everything}
\HTMLAuthor{Anthony}
\HTMLLanguage{en-US}
\HTMLDescription{Personal notes on Mathematics and Computer Science}
\HTMLPageBottom{\LinkHome}

% Styling
\CSSFilename{lwarp.css}

\begin{document}

\maketitle % Or titlepage/titlingpage environment.
% An article abstract would go here.

Notes on Mathematics and Computer Science.

\tableofcontents % MUST BE BEFORE THE FIRST SECTION BREAK!

\listoffigures

\input{./sums/index}
\input{./algorithms/index}
\input{./combinatorics/index}
\input{./probability/index}
\input{./foundations/index}

\ForceHTMLPage 	% HTML index will be on its own page.
\ForceHTMLTOC 	% HTML index will have its own toc entry.
\printindex
\end{document}


\ForceHTMLPage 	% HTML index will be on its own page.
\ForceHTMLTOC 	% HTML index will have its own toc entry.
\printindex
\end{document}


\ForceHTMLPage 	% HTML index will be on its own page.
\ForceHTMLTOC 	% HTML index will have its own toc entry.
\printindex
\end{document}

% Save this as tutorial.tex for the lwarp package tutorial.
\documentclass{book}
\usepackage{iftex}

% --- LOAD FONT SELECTION AND ENCODING BEFORE LOADING LWARP ---
\ifPDFTeX
	\usepackage{lmodern} % pdflatex or dvi latex
	\usepackage[T1]{fontenc}
	\usepackage[utf8]{inputenc}
\else
	\usepackage{fontspec} % XeLaTeX or LuaLaTeX
\fi

% --- LWARP IS LOADED NEXT ---

\usepackage[
% HomeHTMLFilename=index, % Filename of the homepage.
% HTMLFilename={node-}, % Filename prefix of other pages.
% IndexLanguage=english, % Language for xindy index, glossary.
% latexmk, % Use latexmk to compile.
% OSWindows, % Force Windows. (Usually automatic.)
mathjax, % Use MathJax to display math.
]{lwarp}

% \boolfalse{FileSectionNames} % If false, numbers the files.

% --- LOAD PDFLATEX MATH FONTS HERE ---

% --- OTHER PACKAGES ARE LOADED AFTER LWARP ---

\usepackage{standalone}

\usepackage{tikz}
\usepackage{amsthm}
\usepackage{amsmath}
\usepackage{amssymb}
\usepackage{algorithm}
\usepackage{algpseudocode}

\usetikzlibrary{graphs, positioning, shapes.geometric}

\usepackage{makeidx} \makeindex
\usepackage{xcolor}               % (Demonstration purposes only.)
\usepackage{hyperref,cleveref}    % LOAD THESE LAST!

% Declare theorem environments for amsthm
\newtheorem{axiom}{axiom}
\newtheorem{lemma}{Lemma}
\newtheorem{theorem}{Theorem}
\newtheorem{example}{Example}
\newtheorem{definition}{Definition}

% --- LATEX AND HTML CUSTOMIZATION ---
\title{Notes on Everything}
\author{Anthony}
\date{\today}

\setcounter{tocdepth}{1}
\setcounter{secnumdepth}{-1}
\setcounter{FileDepth}{1}
\booltrue{CombineHigherDepths}
\setcounter{SideTOCDepth}{2}

% Renew commands

\renewcommand*\contentsname{Subjects}
\renewcommand\sidetocname{Subjects}

% HTML Directives
\HTMLTitle{Notes on Everything}
\HTMLAuthor{Anthony}
\HTMLLanguage{en-US}
\HTMLDescription{Personal notes on Mathematics and Computer Science}
\HTMLPageBottom{\LinkHome}

% Styling
\CSSFilename{lwarp.css}

\begin{document}

\maketitle % Or titlepage/titlingpage environment.
% An article abstract would go here.

Notes on Mathematics and Computer Science.

\tableofcontents % MUST BE BEFORE THE FIRST SECTION BREAK!

\listoffigures

% Save this as tutorial.tex for the lwarp package tutorial.
\documentclass{book}
\usepackage{iftex}

% --- LOAD FONT SELECTION AND ENCODING BEFORE LOADING LWARP ---
\ifPDFTeX
	\usepackage{lmodern} % pdflatex or dvi latex
	\usepackage[T1]{fontenc}
	\usepackage[utf8]{inputenc}
\else
	\usepackage{fontspec} % XeLaTeX or LuaLaTeX
\fi

% --- LWARP IS LOADED NEXT ---

\usepackage[
% HomeHTMLFilename=index, % Filename of the homepage.
% HTMLFilename={node-}, % Filename prefix of other pages.
% IndexLanguage=english, % Language for xindy index, glossary.
% latexmk, % Use latexmk to compile.
% OSWindows, % Force Windows. (Usually automatic.)
mathjax, % Use MathJax to display math.
]{lwarp}

% \boolfalse{FileSectionNames} % If false, numbers the files.

% --- LOAD PDFLATEX MATH FONTS HERE ---

% --- OTHER PACKAGES ARE LOADED AFTER LWARP ---

\usepackage{standalone}

\usepackage{tikz}
\usepackage{amsthm}
\usepackage{amsmath}
\usepackage{amssymb}
\usepackage{algorithm}
\usepackage{algpseudocode}

\usetikzlibrary{graphs, positioning, shapes.geometric}

\usepackage{makeidx} \makeindex
\usepackage{xcolor}               % (Demonstration purposes only.)
\usepackage{hyperref,cleveref}    % LOAD THESE LAST!

% Declare theorem environments for amsthm
\newtheorem{axiom}{axiom}
\newtheorem{lemma}{Lemma}
\newtheorem{theorem}{Theorem}
\newtheorem{example}{Example}
\newtheorem{definition}{Definition}

% --- LATEX AND HTML CUSTOMIZATION ---
\title{Notes on Everything}
\author{Anthony}
\date{\today}

\setcounter{tocdepth}{1}
\setcounter{secnumdepth}{-1}
\setcounter{FileDepth}{1}
\booltrue{CombineHigherDepths}
\setcounter{SideTOCDepth}{2}

% Renew commands

\renewcommand*\contentsname{Subjects}
\renewcommand\sidetocname{Subjects}

% HTML Directives
\HTMLTitle{Notes on Everything}
\HTMLAuthor{Anthony}
\HTMLLanguage{en-US}
\HTMLDescription{Personal notes on Mathematics and Computer Science}
\HTMLPageBottom{\LinkHome}

% Styling
\CSSFilename{lwarp.css}

\begin{document}

\maketitle % Or titlepage/titlingpage environment.
% An article abstract would go here.

Notes on Mathematics and Computer Science.

\tableofcontents % MUST BE BEFORE THE FIRST SECTION BREAK!

\listoffigures

% Save this as tutorial.tex for the lwarp package tutorial.
\documentclass{book}
\usepackage{iftex}

% --- LOAD FONT SELECTION AND ENCODING BEFORE LOADING LWARP ---
\ifPDFTeX
	\usepackage{lmodern} % pdflatex or dvi latex
	\usepackage[T1]{fontenc}
	\usepackage[utf8]{inputenc}
\else
	\usepackage{fontspec} % XeLaTeX or LuaLaTeX
\fi

% --- LWARP IS LOADED NEXT ---

\usepackage[
% HomeHTMLFilename=index, % Filename of the homepage.
% HTMLFilename={node-}, % Filename prefix of other pages.
% IndexLanguage=english, % Language for xindy index, glossary.
% latexmk, % Use latexmk to compile.
% OSWindows, % Force Windows. (Usually automatic.)
mathjax, % Use MathJax to display math.
]{lwarp}

% \boolfalse{FileSectionNames} % If false, numbers the files.

% --- LOAD PDFLATEX MATH FONTS HERE ---

% --- OTHER PACKAGES ARE LOADED AFTER LWARP ---

\usepackage{standalone}

\usepackage{tikz}
\usepackage{amsthm}
\usepackage{amsmath}
\usepackage{amssymb}
\usepackage{algorithm}
\usepackage{algpseudocode}

\usetikzlibrary{graphs, positioning, shapes.geometric}

\usepackage{makeidx} \makeindex
\usepackage{xcolor}               % (Demonstration purposes only.)
\usepackage{hyperref,cleveref}    % LOAD THESE LAST!

% Declare theorem environments for amsthm
\newtheorem{axiom}{axiom}
\newtheorem{lemma}{Lemma}
\newtheorem{theorem}{Theorem}
\newtheorem{example}{Example}
\newtheorem{definition}{Definition}

% --- LATEX AND HTML CUSTOMIZATION ---
\title{Notes on Everything}
\author{Anthony}
\date{\today}

\setcounter{tocdepth}{1}
\setcounter{secnumdepth}{-1}
\setcounter{FileDepth}{1}
\booltrue{CombineHigherDepths}
\setcounter{SideTOCDepth}{2}

% Renew commands

\renewcommand*\contentsname{Subjects}
\renewcommand\sidetocname{Subjects}

% HTML Directives
\HTMLTitle{Notes on Everything}
\HTMLAuthor{Anthony}
\HTMLLanguage{en-US}
\HTMLDescription{Personal notes on Mathematics and Computer Science}
\HTMLPageBottom{\LinkHome}

% Styling
\CSSFilename{lwarp.css}

\begin{document}

\maketitle % Or titlepage/titlingpage environment.
% An article abstract would go here.

Notes on Mathematics and Computer Science.

\tableofcontents % MUST BE BEFORE THE FIRST SECTION BREAK!

\listoffigures

\input{./sums/index}
\input{./algorithms/index}
\input{./combinatorics/index}
\input{./probability/index}
\input{./foundations/index}

\ForceHTMLPage 	% HTML index will be on its own page.
\ForceHTMLTOC 	% HTML index will have its own toc entry.
\printindex
\end{document}

% Save this as tutorial.tex for the lwarp package tutorial.
\documentclass{book}
\usepackage{iftex}

% --- LOAD FONT SELECTION AND ENCODING BEFORE LOADING LWARP ---
\ifPDFTeX
	\usepackage{lmodern} % pdflatex or dvi latex
	\usepackage[T1]{fontenc}
	\usepackage[utf8]{inputenc}
\else
	\usepackage{fontspec} % XeLaTeX or LuaLaTeX
\fi

% --- LWARP IS LOADED NEXT ---

\usepackage[
% HomeHTMLFilename=index, % Filename of the homepage.
% HTMLFilename={node-}, % Filename prefix of other pages.
% IndexLanguage=english, % Language for xindy index, glossary.
% latexmk, % Use latexmk to compile.
% OSWindows, % Force Windows. (Usually automatic.)
mathjax, % Use MathJax to display math.
]{lwarp}

% \boolfalse{FileSectionNames} % If false, numbers the files.

% --- LOAD PDFLATEX MATH FONTS HERE ---

% --- OTHER PACKAGES ARE LOADED AFTER LWARP ---

\usepackage{standalone}

\usepackage{tikz}
\usepackage{amsthm}
\usepackage{amsmath}
\usepackage{amssymb}
\usepackage{algorithm}
\usepackage{algpseudocode}

\usetikzlibrary{graphs, positioning, shapes.geometric}

\usepackage{makeidx} \makeindex
\usepackage{xcolor}               % (Demonstration purposes only.)
\usepackage{hyperref,cleveref}    % LOAD THESE LAST!

% Declare theorem environments for amsthm
\newtheorem{axiom}{axiom}
\newtheorem{lemma}{Lemma}
\newtheorem{theorem}{Theorem}
\newtheorem{example}{Example}
\newtheorem{definition}{Definition}

% --- LATEX AND HTML CUSTOMIZATION ---
\title{Notes on Everything}
\author{Anthony}
\date{\today}

\setcounter{tocdepth}{1}
\setcounter{secnumdepth}{-1}
\setcounter{FileDepth}{1}
\booltrue{CombineHigherDepths}
\setcounter{SideTOCDepth}{2}

% Renew commands

\renewcommand*\contentsname{Subjects}
\renewcommand\sidetocname{Subjects}

% HTML Directives
\HTMLTitle{Notes on Everything}
\HTMLAuthor{Anthony}
\HTMLLanguage{en-US}
\HTMLDescription{Personal notes on Mathematics and Computer Science}
\HTMLPageBottom{\LinkHome}

% Styling
\CSSFilename{lwarp.css}

\begin{document}

\maketitle % Or titlepage/titlingpage environment.
% An article abstract would go here.

Notes on Mathematics and Computer Science.

\tableofcontents % MUST BE BEFORE THE FIRST SECTION BREAK!

\listoffigures

\input{./sums/index}
\input{./algorithms/index}
\input{./combinatorics/index}
\input{./probability/index}
\input{./foundations/index}

\ForceHTMLPage 	% HTML index will be on its own page.
\ForceHTMLTOC 	% HTML index will have its own toc entry.
\printindex
\end{document}

% Save this as tutorial.tex for the lwarp package tutorial.
\documentclass{book}
\usepackage{iftex}

% --- LOAD FONT SELECTION AND ENCODING BEFORE LOADING LWARP ---
\ifPDFTeX
	\usepackage{lmodern} % pdflatex or dvi latex
	\usepackage[T1]{fontenc}
	\usepackage[utf8]{inputenc}
\else
	\usepackage{fontspec} % XeLaTeX or LuaLaTeX
\fi

% --- LWARP IS LOADED NEXT ---

\usepackage[
% HomeHTMLFilename=index, % Filename of the homepage.
% HTMLFilename={node-}, % Filename prefix of other pages.
% IndexLanguage=english, % Language for xindy index, glossary.
% latexmk, % Use latexmk to compile.
% OSWindows, % Force Windows. (Usually automatic.)
mathjax, % Use MathJax to display math.
]{lwarp}

% \boolfalse{FileSectionNames} % If false, numbers the files.

% --- LOAD PDFLATEX MATH FONTS HERE ---

% --- OTHER PACKAGES ARE LOADED AFTER LWARP ---

\usepackage{standalone}

\usepackage{tikz}
\usepackage{amsthm}
\usepackage{amsmath}
\usepackage{amssymb}
\usepackage{algorithm}
\usepackage{algpseudocode}

\usetikzlibrary{graphs, positioning, shapes.geometric}

\usepackage{makeidx} \makeindex
\usepackage{xcolor}               % (Demonstration purposes only.)
\usepackage{hyperref,cleveref}    % LOAD THESE LAST!

% Declare theorem environments for amsthm
\newtheorem{axiom}{axiom}
\newtheorem{lemma}{Lemma}
\newtheorem{theorem}{Theorem}
\newtheorem{example}{Example}
\newtheorem{definition}{Definition}

% --- LATEX AND HTML CUSTOMIZATION ---
\title{Notes on Everything}
\author{Anthony}
\date{\today}

\setcounter{tocdepth}{1}
\setcounter{secnumdepth}{-1}
\setcounter{FileDepth}{1}
\booltrue{CombineHigherDepths}
\setcounter{SideTOCDepth}{2}

% Renew commands

\renewcommand*\contentsname{Subjects}
\renewcommand\sidetocname{Subjects}

% HTML Directives
\HTMLTitle{Notes on Everything}
\HTMLAuthor{Anthony}
\HTMLLanguage{en-US}
\HTMLDescription{Personal notes on Mathematics and Computer Science}
\HTMLPageBottom{\LinkHome}

% Styling
\CSSFilename{lwarp.css}

\begin{document}

\maketitle % Or titlepage/titlingpage environment.
% An article abstract would go here.

Notes on Mathematics and Computer Science.

\tableofcontents % MUST BE BEFORE THE FIRST SECTION BREAK!

\listoffigures

\input{./sums/index}
\input{./algorithms/index}
\input{./combinatorics/index}
\input{./probability/index}
\input{./foundations/index}

\ForceHTMLPage 	% HTML index will be on its own page.
\ForceHTMLTOC 	% HTML index will have its own toc entry.
\printindex
\end{document}

% Save this as tutorial.tex for the lwarp package tutorial.
\documentclass{book}
\usepackage{iftex}

% --- LOAD FONT SELECTION AND ENCODING BEFORE LOADING LWARP ---
\ifPDFTeX
	\usepackage{lmodern} % pdflatex or dvi latex
	\usepackage[T1]{fontenc}
	\usepackage[utf8]{inputenc}
\else
	\usepackage{fontspec} % XeLaTeX or LuaLaTeX
\fi

% --- LWARP IS LOADED NEXT ---

\usepackage[
% HomeHTMLFilename=index, % Filename of the homepage.
% HTMLFilename={node-}, % Filename prefix of other pages.
% IndexLanguage=english, % Language for xindy index, glossary.
% latexmk, % Use latexmk to compile.
% OSWindows, % Force Windows. (Usually automatic.)
mathjax, % Use MathJax to display math.
]{lwarp}

% \boolfalse{FileSectionNames} % If false, numbers the files.

% --- LOAD PDFLATEX MATH FONTS HERE ---

% --- OTHER PACKAGES ARE LOADED AFTER LWARP ---

\usepackage{standalone}

\usepackage{tikz}
\usepackage{amsthm}
\usepackage{amsmath}
\usepackage{amssymb}
\usepackage{algorithm}
\usepackage{algpseudocode}

\usetikzlibrary{graphs, positioning, shapes.geometric}

\usepackage{makeidx} \makeindex
\usepackage{xcolor}               % (Demonstration purposes only.)
\usepackage{hyperref,cleveref}    % LOAD THESE LAST!

% Declare theorem environments for amsthm
\newtheorem{axiom}{axiom}
\newtheorem{lemma}{Lemma}
\newtheorem{theorem}{Theorem}
\newtheorem{example}{Example}
\newtheorem{definition}{Definition}

% --- LATEX AND HTML CUSTOMIZATION ---
\title{Notes on Everything}
\author{Anthony}
\date{\today}

\setcounter{tocdepth}{1}
\setcounter{secnumdepth}{-1}
\setcounter{FileDepth}{1}
\booltrue{CombineHigherDepths}
\setcounter{SideTOCDepth}{2}

% Renew commands

\renewcommand*\contentsname{Subjects}
\renewcommand\sidetocname{Subjects}

% HTML Directives
\HTMLTitle{Notes on Everything}
\HTMLAuthor{Anthony}
\HTMLLanguage{en-US}
\HTMLDescription{Personal notes on Mathematics and Computer Science}
\HTMLPageBottom{\LinkHome}

% Styling
\CSSFilename{lwarp.css}

\begin{document}

\maketitle % Or titlepage/titlingpage environment.
% An article abstract would go here.

Notes on Mathematics and Computer Science.

\tableofcontents % MUST BE BEFORE THE FIRST SECTION BREAK!

\listoffigures

\input{./sums/index}
\input{./algorithms/index}
\input{./combinatorics/index}
\input{./probability/index}
\input{./foundations/index}

\ForceHTMLPage 	% HTML index will be on its own page.
\ForceHTMLTOC 	% HTML index will have its own toc entry.
\printindex
\end{document}

% Save this as tutorial.tex for the lwarp package tutorial.
\documentclass{book}
\usepackage{iftex}

% --- LOAD FONT SELECTION AND ENCODING BEFORE LOADING LWARP ---
\ifPDFTeX
	\usepackage{lmodern} % pdflatex or dvi latex
	\usepackage[T1]{fontenc}
	\usepackage[utf8]{inputenc}
\else
	\usepackage{fontspec} % XeLaTeX or LuaLaTeX
\fi

% --- LWARP IS LOADED NEXT ---

\usepackage[
% HomeHTMLFilename=index, % Filename of the homepage.
% HTMLFilename={node-}, % Filename prefix of other pages.
% IndexLanguage=english, % Language for xindy index, glossary.
% latexmk, % Use latexmk to compile.
% OSWindows, % Force Windows. (Usually automatic.)
mathjax, % Use MathJax to display math.
]{lwarp}

% \boolfalse{FileSectionNames} % If false, numbers the files.

% --- LOAD PDFLATEX MATH FONTS HERE ---

% --- OTHER PACKAGES ARE LOADED AFTER LWARP ---

\usepackage{standalone}

\usepackage{tikz}
\usepackage{amsthm}
\usepackage{amsmath}
\usepackage{amssymb}
\usepackage{algorithm}
\usepackage{algpseudocode}

\usetikzlibrary{graphs, positioning, shapes.geometric}

\usepackage{makeidx} \makeindex
\usepackage{xcolor}               % (Demonstration purposes only.)
\usepackage{hyperref,cleveref}    % LOAD THESE LAST!

% Declare theorem environments for amsthm
\newtheorem{axiom}{axiom}
\newtheorem{lemma}{Lemma}
\newtheorem{theorem}{Theorem}
\newtheorem{example}{Example}
\newtheorem{definition}{Definition}

% --- LATEX AND HTML CUSTOMIZATION ---
\title{Notes on Everything}
\author{Anthony}
\date{\today}

\setcounter{tocdepth}{1}
\setcounter{secnumdepth}{-1}
\setcounter{FileDepth}{1}
\booltrue{CombineHigherDepths}
\setcounter{SideTOCDepth}{2}

% Renew commands

\renewcommand*\contentsname{Subjects}
\renewcommand\sidetocname{Subjects}

% HTML Directives
\HTMLTitle{Notes on Everything}
\HTMLAuthor{Anthony}
\HTMLLanguage{en-US}
\HTMLDescription{Personal notes on Mathematics and Computer Science}
\HTMLPageBottom{\LinkHome}

% Styling
\CSSFilename{lwarp.css}

\begin{document}

\maketitle % Or titlepage/titlingpage environment.
% An article abstract would go here.

Notes on Mathematics and Computer Science.

\tableofcontents % MUST BE BEFORE THE FIRST SECTION BREAK!

\listoffigures

\input{./sums/index}
\input{./algorithms/index}
\input{./combinatorics/index}
\input{./probability/index}
\input{./foundations/index}

\ForceHTMLPage 	% HTML index will be on its own page.
\ForceHTMLTOC 	% HTML index will have its own toc entry.
\printindex
\end{document}


\ForceHTMLPage 	% HTML index will be on its own page.
\ForceHTMLTOC 	% HTML index will have its own toc entry.
\printindex
\end{document}

% Save this as tutorial.tex for the lwarp package tutorial.
\documentclass{book}
\usepackage{iftex}

% --- LOAD FONT SELECTION AND ENCODING BEFORE LOADING LWARP ---
\ifPDFTeX
	\usepackage{lmodern} % pdflatex or dvi latex
	\usepackage[T1]{fontenc}
	\usepackage[utf8]{inputenc}
\else
	\usepackage{fontspec} % XeLaTeX or LuaLaTeX
\fi

% --- LWARP IS LOADED NEXT ---

\usepackage[
% HomeHTMLFilename=index, % Filename of the homepage.
% HTMLFilename={node-}, % Filename prefix of other pages.
% IndexLanguage=english, % Language for xindy index, glossary.
% latexmk, % Use latexmk to compile.
% OSWindows, % Force Windows. (Usually automatic.)
mathjax, % Use MathJax to display math.
]{lwarp}

% \boolfalse{FileSectionNames} % If false, numbers the files.

% --- LOAD PDFLATEX MATH FONTS HERE ---

% --- OTHER PACKAGES ARE LOADED AFTER LWARP ---

\usepackage{standalone}

\usepackage{tikz}
\usepackage{amsthm}
\usepackage{amsmath}
\usepackage{amssymb}
\usepackage{algorithm}
\usepackage{algpseudocode}

\usetikzlibrary{graphs, positioning, shapes.geometric}

\usepackage{makeidx} \makeindex
\usepackage{xcolor}               % (Demonstration purposes only.)
\usepackage{hyperref,cleveref}    % LOAD THESE LAST!

% Declare theorem environments for amsthm
\newtheorem{axiom}{axiom}
\newtheorem{lemma}{Lemma}
\newtheorem{theorem}{Theorem}
\newtheorem{example}{Example}
\newtheorem{definition}{Definition}

% --- LATEX AND HTML CUSTOMIZATION ---
\title{Notes on Everything}
\author{Anthony}
\date{\today}

\setcounter{tocdepth}{1}
\setcounter{secnumdepth}{-1}
\setcounter{FileDepth}{1}
\booltrue{CombineHigherDepths}
\setcounter{SideTOCDepth}{2}

% Renew commands

\renewcommand*\contentsname{Subjects}
\renewcommand\sidetocname{Subjects}

% HTML Directives
\HTMLTitle{Notes on Everything}
\HTMLAuthor{Anthony}
\HTMLLanguage{en-US}
\HTMLDescription{Personal notes on Mathematics and Computer Science}
\HTMLPageBottom{\LinkHome}

% Styling
\CSSFilename{lwarp.css}

\begin{document}

\maketitle % Or titlepage/titlingpage environment.
% An article abstract would go here.

Notes on Mathematics and Computer Science.

\tableofcontents % MUST BE BEFORE THE FIRST SECTION BREAK!

\listoffigures

% Save this as tutorial.tex for the lwarp package tutorial.
\documentclass{book}
\usepackage{iftex}

% --- LOAD FONT SELECTION AND ENCODING BEFORE LOADING LWARP ---
\ifPDFTeX
	\usepackage{lmodern} % pdflatex or dvi latex
	\usepackage[T1]{fontenc}
	\usepackage[utf8]{inputenc}
\else
	\usepackage{fontspec} % XeLaTeX or LuaLaTeX
\fi

% --- LWARP IS LOADED NEXT ---

\usepackage[
% HomeHTMLFilename=index, % Filename of the homepage.
% HTMLFilename={node-}, % Filename prefix of other pages.
% IndexLanguage=english, % Language for xindy index, glossary.
% latexmk, % Use latexmk to compile.
% OSWindows, % Force Windows. (Usually automatic.)
mathjax, % Use MathJax to display math.
]{lwarp}

% \boolfalse{FileSectionNames} % If false, numbers the files.

% --- LOAD PDFLATEX MATH FONTS HERE ---

% --- OTHER PACKAGES ARE LOADED AFTER LWARP ---

\usepackage{standalone}

\usepackage{tikz}
\usepackage{amsthm}
\usepackage{amsmath}
\usepackage{amssymb}
\usepackage{algorithm}
\usepackage{algpseudocode}

\usetikzlibrary{graphs, positioning, shapes.geometric}

\usepackage{makeidx} \makeindex
\usepackage{xcolor}               % (Demonstration purposes only.)
\usepackage{hyperref,cleveref}    % LOAD THESE LAST!

% Declare theorem environments for amsthm
\newtheorem{axiom}{axiom}
\newtheorem{lemma}{Lemma}
\newtheorem{theorem}{Theorem}
\newtheorem{example}{Example}
\newtheorem{definition}{Definition}

% --- LATEX AND HTML CUSTOMIZATION ---
\title{Notes on Everything}
\author{Anthony}
\date{\today}

\setcounter{tocdepth}{1}
\setcounter{secnumdepth}{-1}
\setcounter{FileDepth}{1}
\booltrue{CombineHigherDepths}
\setcounter{SideTOCDepth}{2}

% Renew commands

\renewcommand*\contentsname{Subjects}
\renewcommand\sidetocname{Subjects}

% HTML Directives
\HTMLTitle{Notes on Everything}
\HTMLAuthor{Anthony}
\HTMLLanguage{en-US}
\HTMLDescription{Personal notes on Mathematics and Computer Science}
\HTMLPageBottom{\LinkHome}

% Styling
\CSSFilename{lwarp.css}

\begin{document}

\maketitle % Or titlepage/titlingpage environment.
% An article abstract would go here.

Notes on Mathematics and Computer Science.

\tableofcontents % MUST BE BEFORE THE FIRST SECTION BREAK!

\listoffigures

\input{./sums/index}
\input{./algorithms/index}
\input{./combinatorics/index}
\input{./probability/index}
\input{./foundations/index}

\ForceHTMLPage 	% HTML index will be on its own page.
\ForceHTMLTOC 	% HTML index will have its own toc entry.
\printindex
\end{document}

% Save this as tutorial.tex for the lwarp package tutorial.
\documentclass{book}
\usepackage{iftex}

% --- LOAD FONT SELECTION AND ENCODING BEFORE LOADING LWARP ---
\ifPDFTeX
	\usepackage{lmodern} % pdflatex or dvi latex
	\usepackage[T1]{fontenc}
	\usepackage[utf8]{inputenc}
\else
	\usepackage{fontspec} % XeLaTeX or LuaLaTeX
\fi

% --- LWARP IS LOADED NEXT ---

\usepackage[
% HomeHTMLFilename=index, % Filename of the homepage.
% HTMLFilename={node-}, % Filename prefix of other pages.
% IndexLanguage=english, % Language for xindy index, glossary.
% latexmk, % Use latexmk to compile.
% OSWindows, % Force Windows. (Usually automatic.)
mathjax, % Use MathJax to display math.
]{lwarp}

% \boolfalse{FileSectionNames} % If false, numbers the files.

% --- LOAD PDFLATEX MATH FONTS HERE ---

% --- OTHER PACKAGES ARE LOADED AFTER LWARP ---

\usepackage{standalone}

\usepackage{tikz}
\usepackage{amsthm}
\usepackage{amsmath}
\usepackage{amssymb}
\usepackage{algorithm}
\usepackage{algpseudocode}

\usetikzlibrary{graphs, positioning, shapes.geometric}

\usepackage{makeidx} \makeindex
\usepackage{xcolor}               % (Demonstration purposes only.)
\usepackage{hyperref,cleveref}    % LOAD THESE LAST!

% Declare theorem environments for amsthm
\newtheorem{axiom}{axiom}
\newtheorem{lemma}{Lemma}
\newtheorem{theorem}{Theorem}
\newtheorem{example}{Example}
\newtheorem{definition}{Definition}

% --- LATEX AND HTML CUSTOMIZATION ---
\title{Notes on Everything}
\author{Anthony}
\date{\today}

\setcounter{tocdepth}{1}
\setcounter{secnumdepth}{-1}
\setcounter{FileDepth}{1}
\booltrue{CombineHigherDepths}
\setcounter{SideTOCDepth}{2}

% Renew commands

\renewcommand*\contentsname{Subjects}
\renewcommand\sidetocname{Subjects}

% HTML Directives
\HTMLTitle{Notes on Everything}
\HTMLAuthor{Anthony}
\HTMLLanguage{en-US}
\HTMLDescription{Personal notes on Mathematics and Computer Science}
\HTMLPageBottom{\LinkHome}

% Styling
\CSSFilename{lwarp.css}

\begin{document}

\maketitle % Or titlepage/titlingpage environment.
% An article abstract would go here.

Notes on Mathematics and Computer Science.

\tableofcontents % MUST BE BEFORE THE FIRST SECTION BREAK!

\listoffigures

\input{./sums/index}
\input{./algorithms/index}
\input{./combinatorics/index}
\input{./probability/index}
\input{./foundations/index}

\ForceHTMLPage 	% HTML index will be on its own page.
\ForceHTMLTOC 	% HTML index will have its own toc entry.
\printindex
\end{document}

% Save this as tutorial.tex for the lwarp package tutorial.
\documentclass{book}
\usepackage{iftex}

% --- LOAD FONT SELECTION AND ENCODING BEFORE LOADING LWARP ---
\ifPDFTeX
	\usepackage{lmodern} % pdflatex or dvi latex
	\usepackage[T1]{fontenc}
	\usepackage[utf8]{inputenc}
\else
	\usepackage{fontspec} % XeLaTeX or LuaLaTeX
\fi

% --- LWARP IS LOADED NEXT ---

\usepackage[
% HomeHTMLFilename=index, % Filename of the homepage.
% HTMLFilename={node-}, % Filename prefix of other pages.
% IndexLanguage=english, % Language for xindy index, glossary.
% latexmk, % Use latexmk to compile.
% OSWindows, % Force Windows. (Usually automatic.)
mathjax, % Use MathJax to display math.
]{lwarp}

% \boolfalse{FileSectionNames} % If false, numbers the files.

% --- LOAD PDFLATEX MATH FONTS HERE ---

% --- OTHER PACKAGES ARE LOADED AFTER LWARP ---

\usepackage{standalone}

\usepackage{tikz}
\usepackage{amsthm}
\usepackage{amsmath}
\usepackage{amssymb}
\usepackage{algorithm}
\usepackage{algpseudocode}

\usetikzlibrary{graphs, positioning, shapes.geometric}

\usepackage{makeidx} \makeindex
\usepackage{xcolor}               % (Demonstration purposes only.)
\usepackage{hyperref,cleveref}    % LOAD THESE LAST!

% Declare theorem environments for amsthm
\newtheorem{axiom}{axiom}
\newtheorem{lemma}{Lemma}
\newtheorem{theorem}{Theorem}
\newtheorem{example}{Example}
\newtheorem{definition}{Definition}

% --- LATEX AND HTML CUSTOMIZATION ---
\title{Notes on Everything}
\author{Anthony}
\date{\today}

\setcounter{tocdepth}{1}
\setcounter{secnumdepth}{-1}
\setcounter{FileDepth}{1}
\booltrue{CombineHigherDepths}
\setcounter{SideTOCDepth}{2}

% Renew commands

\renewcommand*\contentsname{Subjects}
\renewcommand\sidetocname{Subjects}

% HTML Directives
\HTMLTitle{Notes on Everything}
\HTMLAuthor{Anthony}
\HTMLLanguage{en-US}
\HTMLDescription{Personal notes on Mathematics and Computer Science}
\HTMLPageBottom{\LinkHome}

% Styling
\CSSFilename{lwarp.css}

\begin{document}

\maketitle % Or titlepage/titlingpage environment.
% An article abstract would go here.

Notes on Mathematics and Computer Science.

\tableofcontents % MUST BE BEFORE THE FIRST SECTION BREAK!

\listoffigures

\input{./sums/index}
\input{./algorithms/index}
\input{./combinatorics/index}
\input{./probability/index}
\input{./foundations/index}

\ForceHTMLPage 	% HTML index will be on its own page.
\ForceHTMLTOC 	% HTML index will have its own toc entry.
\printindex
\end{document}

% Save this as tutorial.tex for the lwarp package tutorial.
\documentclass{book}
\usepackage{iftex}

% --- LOAD FONT SELECTION AND ENCODING BEFORE LOADING LWARP ---
\ifPDFTeX
	\usepackage{lmodern} % pdflatex or dvi latex
	\usepackage[T1]{fontenc}
	\usepackage[utf8]{inputenc}
\else
	\usepackage{fontspec} % XeLaTeX or LuaLaTeX
\fi

% --- LWARP IS LOADED NEXT ---

\usepackage[
% HomeHTMLFilename=index, % Filename of the homepage.
% HTMLFilename={node-}, % Filename prefix of other pages.
% IndexLanguage=english, % Language for xindy index, glossary.
% latexmk, % Use latexmk to compile.
% OSWindows, % Force Windows. (Usually automatic.)
mathjax, % Use MathJax to display math.
]{lwarp}

% \boolfalse{FileSectionNames} % If false, numbers the files.

% --- LOAD PDFLATEX MATH FONTS HERE ---

% --- OTHER PACKAGES ARE LOADED AFTER LWARP ---

\usepackage{standalone}

\usepackage{tikz}
\usepackage{amsthm}
\usepackage{amsmath}
\usepackage{amssymb}
\usepackage{algorithm}
\usepackage{algpseudocode}

\usetikzlibrary{graphs, positioning, shapes.geometric}

\usepackage{makeidx} \makeindex
\usepackage{xcolor}               % (Demonstration purposes only.)
\usepackage{hyperref,cleveref}    % LOAD THESE LAST!

% Declare theorem environments for amsthm
\newtheorem{axiom}{axiom}
\newtheorem{lemma}{Lemma}
\newtheorem{theorem}{Theorem}
\newtheorem{example}{Example}
\newtheorem{definition}{Definition}

% --- LATEX AND HTML CUSTOMIZATION ---
\title{Notes on Everything}
\author{Anthony}
\date{\today}

\setcounter{tocdepth}{1}
\setcounter{secnumdepth}{-1}
\setcounter{FileDepth}{1}
\booltrue{CombineHigherDepths}
\setcounter{SideTOCDepth}{2}

% Renew commands

\renewcommand*\contentsname{Subjects}
\renewcommand\sidetocname{Subjects}

% HTML Directives
\HTMLTitle{Notes on Everything}
\HTMLAuthor{Anthony}
\HTMLLanguage{en-US}
\HTMLDescription{Personal notes on Mathematics and Computer Science}
\HTMLPageBottom{\LinkHome}

% Styling
\CSSFilename{lwarp.css}

\begin{document}

\maketitle % Or titlepage/titlingpage environment.
% An article abstract would go here.

Notes on Mathematics and Computer Science.

\tableofcontents % MUST BE BEFORE THE FIRST SECTION BREAK!

\listoffigures

\input{./sums/index}
\input{./algorithms/index}
\input{./combinatorics/index}
\input{./probability/index}
\input{./foundations/index}

\ForceHTMLPage 	% HTML index will be on its own page.
\ForceHTMLTOC 	% HTML index will have its own toc entry.
\printindex
\end{document}

% Save this as tutorial.tex for the lwarp package tutorial.
\documentclass{book}
\usepackage{iftex}

% --- LOAD FONT SELECTION AND ENCODING BEFORE LOADING LWARP ---
\ifPDFTeX
	\usepackage{lmodern} % pdflatex or dvi latex
	\usepackage[T1]{fontenc}
	\usepackage[utf8]{inputenc}
\else
	\usepackage{fontspec} % XeLaTeX or LuaLaTeX
\fi

% --- LWARP IS LOADED NEXT ---

\usepackage[
% HomeHTMLFilename=index, % Filename of the homepage.
% HTMLFilename={node-}, % Filename prefix of other pages.
% IndexLanguage=english, % Language for xindy index, glossary.
% latexmk, % Use latexmk to compile.
% OSWindows, % Force Windows. (Usually automatic.)
mathjax, % Use MathJax to display math.
]{lwarp}

% \boolfalse{FileSectionNames} % If false, numbers the files.

% --- LOAD PDFLATEX MATH FONTS HERE ---

% --- OTHER PACKAGES ARE LOADED AFTER LWARP ---

\usepackage{standalone}

\usepackage{tikz}
\usepackage{amsthm}
\usepackage{amsmath}
\usepackage{amssymb}
\usepackage{algorithm}
\usepackage{algpseudocode}

\usetikzlibrary{graphs, positioning, shapes.geometric}

\usepackage{makeidx} \makeindex
\usepackage{xcolor}               % (Demonstration purposes only.)
\usepackage{hyperref,cleveref}    % LOAD THESE LAST!

% Declare theorem environments for amsthm
\newtheorem{axiom}{axiom}
\newtheorem{lemma}{Lemma}
\newtheorem{theorem}{Theorem}
\newtheorem{example}{Example}
\newtheorem{definition}{Definition}

% --- LATEX AND HTML CUSTOMIZATION ---
\title{Notes on Everything}
\author{Anthony}
\date{\today}

\setcounter{tocdepth}{1}
\setcounter{secnumdepth}{-1}
\setcounter{FileDepth}{1}
\booltrue{CombineHigherDepths}
\setcounter{SideTOCDepth}{2}

% Renew commands

\renewcommand*\contentsname{Subjects}
\renewcommand\sidetocname{Subjects}

% HTML Directives
\HTMLTitle{Notes on Everything}
\HTMLAuthor{Anthony}
\HTMLLanguage{en-US}
\HTMLDescription{Personal notes on Mathematics and Computer Science}
\HTMLPageBottom{\LinkHome}

% Styling
\CSSFilename{lwarp.css}

\begin{document}

\maketitle % Or titlepage/titlingpage environment.
% An article abstract would go here.

Notes on Mathematics and Computer Science.

\tableofcontents % MUST BE BEFORE THE FIRST SECTION BREAK!

\listoffigures

\input{./sums/index}
\input{./algorithms/index}
\input{./combinatorics/index}
\input{./probability/index}
\input{./foundations/index}

\ForceHTMLPage 	% HTML index will be on its own page.
\ForceHTMLTOC 	% HTML index will have its own toc entry.
\printindex
\end{document}


\ForceHTMLPage 	% HTML index will be on its own page.
\ForceHTMLTOC 	% HTML index will have its own toc entry.
\printindex
\end{document}

% Save this as tutorial.tex for the lwarp package tutorial.
\documentclass{book}
\usepackage{iftex}

% --- LOAD FONT SELECTION AND ENCODING BEFORE LOADING LWARP ---
\ifPDFTeX
	\usepackage{lmodern} % pdflatex or dvi latex
	\usepackage[T1]{fontenc}
	\usepackage[utf8]{inputenc}
\else
	\usepackage{fontspec} % XeLaTeX or LuaLaTeX
\fi

% --- LWARP IS LOADED NEXT ---

\usepackage[
% HomeHTMLFilename=index, % Filename of the homepage.
% HTMLFilename={node-}, % Filename prefix of other pages.
% IndexLanguage=english, % Language for xindy index, glossary.
% latexmk, % Use latexmk to compile.
% OSWindows, % Force Windows. (Usually automatic.)
mathjax, % Use MathJax to display math.
]{lwarp}

% \boolfalse{FileSectionNames} % If false, numbers the files.

% --- LOAD PDFLATEX MATH FONTS HERE ---

% --- OTHER PACKAGES ARE LOADED AFTER LWARP ---

\usepackage{standalone}

\usepackage{tikz}
\usepackage{amsthm}
\usepackage{amsmath}
\usepackage{amssymb}
\usepackage{algorithm}
\usepackage{algpseudocode}

\usetikzlibrary{graphs, positioning, shapes.geometric}

\usepackage{makeidx} \makeindex
\usepackage{xcolor}               % (Demonstration purposes only.)
\usepackage{hyperref,cleveref}    % LOAD THESE LAST!

% Declare theorem environments for amsthm
\newtheorem{axiom}{axiom}
\newtheorem{lemma}{Lemma}
\newtheorem{theorem}{Theorem}
\newtheorem{example}{Example}
\newtheorem{definition}{Definition}

% --- LATEX AND HTML CUSTOMIZATION ---
\title{Notes on Everything}
\author{Anthony}
\date{\today}

\setcounter{tocdepth}{1}
\setcounter{secnumdepth}{-1}
\setcounter{FileDepth}{1}
\booltrue{CombineHigherDepths}
\setcounter{SideTOCDepth}{2}

% Renew commands

\renewcommand*\contentsname{Subjects}
\renewcommand\sidetocname{Subjects}

% HTML Directives
\HTMLTitle{Notes on Everything}
\HTMLAuthor{Anthony}
\HTMLLanguage{en-US}
\HTMLDescription{Personal notes on Mathematics and Computer Science}
\HTMLPageBottom{\LinkHome}

% Styling
\CSSFilename{lwarp.css}

\begin{document}

\maketitle % Or titlepage/titlingpage environment.
% An article abstract would go here.

Notes on Mathematics and Computer Science.

\tableofcontents % MUST BE BEFORE THE FIRST SECTION BREAK!

\listoffigures

% Save this as tutorial.tex for the lwarp package tutorial.
\documentclass{book}
\usepackage{iftex}

% --- LOAD FONT SELECTION AND ENCODING BEFORE LOADING LWARP ---
\ifPDFTeX
	\usepackage{lmodern} % pdflatex or dvi latex
	\usepackage[T1]{fontenc}
	\usepackage[utf8]{inputenc}
\else
	\usepackage{fontspec} % XeLaTeX or LuaLaTeX
\fi

% --- LWARP IS LOADED NEXT ---

\usepackage[
% HomeHTMLFilename=index, % Filename of the homepage.
% HTMLFilename={node-}, % Filename prefix of other pages.
% IndexLanguage=english, % Language for xindy index, glossary.
% latexmk, % Use latexmk to compile.
% OSWindows, % Force Windows. (Usually automatic.)
mathjax, % Use MathJax to display math.
]{lwarp}

% \boolfalse{FileSectionNames} % If false, numbers the files.

% --- LOAD PDFLATEX MATH FONTS HERE ---

% --- OTHER PACKAGES ARE LOADED AFTER LWARP ---

\usepackage{standalone}

\usepackage{tikz}
\usepackage{amsthm}
\usepackage{amsmath}
\usepackage{amssymb}
\usepackage{algorithm}
\usepackage{algpseudocode}

\usetikzlibrary{graphs, positioning, shapes.geometric}

\usepackage{makeidx} \makeindex
\usepackage{xcolor}               % (Demonstration purposes only.)
\usepackage{hyperref,cleveref}    % LOAD THESE LAST!

% Declare theorem environments for amsthm
\newtheorem{axiom}{axiom}
\newtheorem{lemma}{Lemma}
\newtheorem{theorem}{Theorem}
\newtheorem{example}{Example}
\newtheorem{definition}{Definition}

% --- LATEX AND HTML CUSTOMIZATION ---
\title{Notes on Everything}
\author{Anthony}
\date{\today}

\setcounter{tocdepth}{1}
\setcounter{secnumdepth}{-1}
\setcounter{FileDepth}{1}
\booltrue{CombineHigherDepths}
\setcounter{SideTOCDepth}{2}

% Renew commands

\renewcommand*\contentsname{Subjects}
\renewcommand\sidetocname{Subjects}

% HTML Directives
\HTMLTitle{Notes on Everything}
\HTMLAuthor{Anthony}
\HTMLLanguage{en-US}
\HTMLDescription{Personal notes on Mathematics and Computer Science}
\HTMLPageBottom{\LinkHome}

% Styling
\CSSFilename{lwarp.css}

\begin{document}

\maketitle % Or titlepage/titlingpage environment.
% An article abstract would go here.

Notes on Mathematics and Computer Science.

\tableofcontents % MUST BE BEFORE THE FIRST SECTION BREAK!

\listoffigures

\input{./sums/index}
\input{./algorithms/index}
\input{./combinatorics/index}
\input{./probability/index}
\input{./foundations/index}

\ForceHTMLPage 	% HTML index will be on its own page.
\ForceHTMLTOC 	% HTML index will have its own toc entry.
\printindex
\end{document}

% Save this as tutorial.tex for the lwarp package tutorial.
\documentclass{book}
\usepackage{iftex}

% --- LOAD FONT SELECTION AND ENCODING BEFORE LOADING LWARP ---
\ifPDFTeX
	\usepackage{lmodern} % pdflatex or dvi latex
	\usepackage[T1]{fontenc}
	\usepackage[utf8]{inputenc}
\else
	\usepackage{fontspec} % XeLaTeX or LuaLaTeX
\fi

% --- LWARP IS LOADED NEXT ---

\usepackage[
% HomeHTMLFilename=index, % Filename of the homepage.
% HTMLFilename={node-}, % Filename prefix of other pages.
% IndexLanguage=english, % Language for xindy index, glossary.
% latexmk, % Use latexmk to compile.
% OSWindows, % Force Windows. (Usually automatic.)
mathjax, % Use MathJax to display math.
]{lwarp}

% \boolfalse{FileSectionNames} % If false, numbers the files.

% --- LOAD PDFLATEX MATH FONTS HERE ---

% --- OTHER PACKAGES ARE LOADED AFTER LWARP ---

\usepackage{standalone}

\usepackage{tikz}
\usepackage{amsthm}
\usepackage{amsmath}
\usepackage{amssymb}
\usepackage{algorithm}
\usepackage{algpseudocode}

\usetikzlibrary{graphs, positioning, shapes.geometric}

\usepackage{makeidx} \makeindex
\usepackage{xcolor}               % (Demonstration purposes only.)
\usepackage{hyperref,cleveref}    % LOAD THESE LAST!

% Declare theorem environments for amsthm
\newtheorem{axiom}{axiom}
\newtheorem{lemma}{Lemma}
\newtheorem{theorem}{Theorem}
\newtheorem{example}{Example}
\newtheorem{definition}{Definition}

% --- LATEX AND HTML CUSTOMIZATION ---
\title{Notes on Everything}
\author{Anthony}
\date{\today}

\setcounter{tocdepth}{1}
\setcounter{secnumdepth}{-1}
\setcounter{FileDepth}{1}
\booltrue{CombineHigherDepths}
\setcounter{SideTOCDepth}{2}

% Renew commands

\renewcommand*\contentsname{Subjects}
\renewcommand\sidetocname{Subjects}

% HTML Directives
\HTMLTitle{Notes on Everything}
\HTMLAuthor{Anthony}
\HTMLLanguage{en-US}
\HTMLDescription{Personal notes on Mathematics and Computer Science}
\HTMLPageBottom{\LinkHome}

% Styling
\CSSFilename{lwarp.css}

\begin{document}

\maketitle % Or titlepage/titlingpage environment.
% An article abstract would go here.

Notes on Mathematics and Computer Science.

\tableofcontents % MUST BE BEFORE THE FIRST SECTION BREAK!

\listoffigures

\input{./sums/index}
\input{./algorithms/index}
\input{./combinatorics/index}
\input{./probability/index}
\input{./foundations/index}

\ForceHTMLPage 	% HTML index will be on its own page.
\ForceHTMLTOC 	% HTML index will have its own toc entry.
\printindex
\end{document}

% Save this as tutorial.tex for the lwarp package tutorial.
\documentclass{book}
\usepackage{iftex}

% --- LOAD FONT SELECTION AND ENCODING BEFORE LOADING LWARP ---
\ifPDFTeX
	\usepackage{lmodern} % pdflatex or dvi latex
	\usepackage[T1]{fontenc}
	\usepackage[utf8]{inputenc}
\else
	\usepackage{fontspec} % XeLaTeX or LuaLaTeX
\fi

% --- LWARP IS LOADED NEXT ---

\usepackage[
% HomeHTMLFilename=index, % Filename of the homepage.
% HTMLFilename={node-}, % Filename prefix of other pages.
% IndexLanguage=english, % Language for xindy index, glossary.
% latexmk, % Use latexmk to compile.
% OSWindows, % Force Windows. (Usually automatic.)
mathjax, % Use MathJax to display math.
]{lwarp}

% \boolfalse{FileSectionNames} % If false, numbers the files.

% --- LOAD PDFLATEX MATH FONTS HERE ---

% --- OTHER PACKAGES ARE LOADED AFTER LWARP ---

\usepackage{standalone}

\usepackage{tikz}
\usepackage{amsthm}
\usepackage{amsmath}
\usepackage{amssymb}
\usepackage{algorithm}
\usepackage{algpseudocode}

\usetikzlibrary{graphs, positioning, shapes.geometric}

\usepackage{makeidx} \makeindex
\usepackage{xcolor}               % (Demonstration purposes only.)
\usepackage{hyperref,cleveref}    % LOAD THESE LAST!

% Declare theorem environments for amsthm
\newtheorem{axiom}{axiom}
\newtheorem{lemma}{Lemma}
\newtheorem{theorem}{Theorem}
\newtheorem{example}{Example}
\newtheorem{definition}{Definition}

% --- LATEX AND HTML CUSTOMIZATION ---
\title{Notes on Everything}
\author{Anthony}
\date{\today}

\setcounter{tocdepth}{1}
\setcounter{secnumdepth}{-1}
\setcounter{FileDepth}{1}
\booltrue{CombineHigherDepths}
\setcounter{SideTOCDepth}{2}

% Renew commands

\renewcommand*\contentsname{Subjects}
\renewcommand\sidetocname{Subjects}

% HTML Directives
\HTMLTitle{Notes on Everything}
\HTMLAuthor{Anthony}
\HTMLLanguage{en-US}
\HTMLDescription{Personal notes on Mathematics and Computer Science}
\HTMLPageBottom{\LinkHome}

% Styling
\CSSFilename{lwarp.css}

\begin{document}

\maketitle % Or titlepage/titlingpage environment.
% An article abstract would go here.

Notes on Mathematics and Computer Science.

\tableofcontents % MUST BE BEFORE THE FIRST SECTION BREAK!

\listoffigures

\input{./sums/index}
\input{./algorithms/index}
\input{./combinatorics/index}
\input{./probability/index}
\input{./foundations/index}

\ForceHTMLPage 	% HTML index will be on its own page.
\ForceHTMLTOC 	% HTML index will have its own toc entry.
\printindex
\end{document}

% Save this as tutorial.tex for the lwarp package tutorial.
\documentclass{book}
\usepackage{iftex}

% --- LOAD FONT SELECTION AND ENCODING BEFORE LOADING LWARP ---
\ifPDFTeX
	\usepackage{lmodern} % pdflatex or dvi latex
	\usepackage[T1]{fontenc}
	\usepackage[utf8]{inputenc}
\else
	\usepackage{fontspec} % XeLaTeX or LuaLaTeX
\fi

% --- LWARP IS LOADED NEXT ---

\usepackage[
% HomeHTMLFilename=index, % Filename of the homepage.
% HTMLFilename={node-}, % Filename prefix of other pages.
% IndexLanguage=english, % Language for xindy index, glossary.
% latexmk, % Use latexmk to compile.
% OSWindows, % Force Windows. (Usually automatic.)
mathjax, % Use MathJax to display math.
]{lwarp}

% \boolfalse{FileSectionNames} % If false, numbers the files.

% --- LOAD PDFLATEX MATH FONTS HERE ---

% --- OTHER PACKAGES ARE LOADED AFTER LWARP ---

\usepackage{standalone}

\usepackage{tikz}
\usepackage{amsthm}
\usepackage{amsmath}
\usepackage{amssymb}
\usepackage{algorithm}
\usepackage{algpseudocode}

\usetikzlibrary{graphs, positioning, shapes.geometric}

\usepackage{makeidx} \makeindex
\usepackage{xcolor}               % (Demonstration purposes only.)
\usepackage{hyperref,cleveref}    % LOAD THESE LAST!

% Declare theorem environments for amsthm
\newtheorem{axiom}{axiom}
\newtheorem{lemma}{Lemma}
\newtheorem{theorem}{Theorem}
\newtheorem{example}{Example}
\newtheorem{definition}{Definition}

% --- LATEX AND HTML CUSTOMIZATION ---
\title{Notes on Everything}
\author{Anthony}
\date{\today}

\setcounter{tocdepth}{1}
\setcounter{secnumdepth}{-1}
\setcounter{FileDepth}{1}
\booltrue{CombineHigherDepths}
\setcounter{SideTOCDepth}{2}

% Renew commands

\renewcommand*\contentsname{Subjects}
\renewcommand\sidetocname{Subjects}

% HTML Directives
\HTMLTitle{Notes on Everything}
\HTMLAuthor{Anthony}
\HTMLLanguage{en-US}
\HTMLDescription{Personal notes on Mathematics and Computer Science}
\HTMLPageBottom{\LinkHome}

% Styling
\CSSFilename{lwarp.css}

\begin{document}

\maketitle % Or titlepage/titlingpage environment.
% An article abstract would go here.

Notes on Mathematics and Computer Science.

\tableofcontents % MUST BE BEFORE THE FIRST SECTION BREAK!

\listoffigures

\input{./sums/index}
\input{./algorithms/index}
\input{./combinatorics/index}
\input{./probability/index}
\input{./foundations/index}

\ForceHTMLPage 	% HTML index will be on its own page.
\ForceHTMLTOC 	% HTML index will have its own toc entry.
\printindex
\end{document}

% Save this as tutorial.tex for the lwarp package tutorial.
\documentclass{book}
\usepackage{iftex}

% --- LOAD FONT SELECTION AND ENCODING BEFORE LOADING LWARP ---
\ifPDFTeX
	\usepackage{lmodern} % pdflatex or dvi latex
	\usepackage[T1]{fontenc}
	\usepackage[utf8]{inputenc}
\else
	\usepackage{fontspec} % XeLaTeX or LuaLaTeX
\fi

% --- LWARP IS LOADED NEXT ---

\usepackage[
% HomeHTMLFilename=index, % Filename of the homepage.
% HTMLFilename={node-}, % Filename prefix of other pages.
% IndexLanguage=english, % Language for xindy index, glossary.
% latexmk, % Use latexmk to compile.
% OSWindows, % Force Windows. (Usually automatic.)
mathjax, % Use MathJax to display math.
]{lwarp}

% \boolfalse{FileSectionNames} % If false, numbers the files.

% --- LOAD PDFLATEX MATH FONTS HERE ---

% --- OTHER PACKAGES ARE LOADED AFTER LWARP ---

\usepackage{standalone}

\usepackage{tikz}
\usepackage{amsthm}
\usepackage{amsmath}
\usepackage{amssymb}
\usepackage{algorithm}
\usepackage{algpseudocode}

\usetikzlibrary{graphs, positioning, shapes.geometric}

\usepackage{makeidx} \makeindex
\usepackage{xcolor}               % (Demonstration purposes only.)
\usepackage{hyperref,cleveref}    % LOAD THESE LAST!

% Declare theorem environments for amsthm
\newtheorem{axiom}{axiom}
\newtheorem{lemma}{Lemma}
\newtheorem{theorem}{Theorem}
\newtheorem{example}{Example}
\newtheorem{definition}{Definition}

% --- LATEX AND HTML CUSTOMIZATION ---
\title{Notes on Everything}
\author{Anthony}
\date{\today}

\setcounter{tocdepth}{1}
\setcounter{secnumdepth}{-1}
\setcounter{FileDepth}{1}
\booltrue{CombineHigherDepths}
\setcounter{SideTOCDepth}{2}

% Renew commands

\renewcommand*\contentsname{Subjects}
\renewcommand\sidetocname{Subjects}

% HTML Directives
\HTMLTitle{Notes on Everything}
\HTMLAuthor{Anthony}
\HTMLLanguage{en-US}
\HTMLDescription{Personal notes on Mathematics and Computer Science}
\HTMLPageBottom{\LinkHome}

% Styling
\CSSFilename{lwarp.css}

\begin{document}

\maketitle % Or titlepage/titlingpage environment.
% An article abstract would go here.

Notes on Mathematics and Computer Science.

\tableofcontents % MUST BE BEFORE THE FIRST SECTION BREAK!

\listoffigures

\input{./sums/index}
\input{./algorithms/index}
\input{./combinatorics/index}
\input{./probability/index}
\input{./foundations/index}

\ForceHTMLPage 	% HTML index will be on its own page.
\ForceHTMLTOC 	% HTML index will have its own toc entry.
\printindex
\end{document}


\ForceHTMLPage 	% HTML index will be on its own page.
\ForceHTMLTOC 	% HTML index will have its own toc entry.
\printindex
\end{document}

% Save this as tutorial.tex for the lwarp package tutorial.
\documentclass{book}
\usepackage{iftex}

% --- LOAD FONT SELECTION AND ENCODING BEFORE LOADING LWARP ---
\ifPDFTeX
	\usepackage{lmodern} % pdflatex or dvi latex
	\usepackage[T1]{fontenc}
	\usepackage[utf8]{inputenc}
\else
	\usepackage{fontspec} % XeLaTeX or LuaLaTeX
\fi

% --- LWARP IS LOADED NEXT ---

\usepackage[
% HomeHTMLFilename=index, % Filename of the homepage.
% HTMLFilename={node-}, % Filename prefix of other pages.
% IndexLanguage=english, % Language for xindy index, glossary.
% latexmk, % Use latexmk to compile.
% OSWindows, % Force Windows. (Usually automatic.)
mathjax, % Use MathJax to display math.
]{lwarp}

% \boolfalse{FileSectionNames} % If false, numbers the files.

% --- LOAD PDFLATEX MATH FONTS HERE ---

% --- OTHER PACKAGES ARE LOADED AFTER LWARP ---

\usepackage{standalone}

\usepackage{tikz}
\usepackage{amsthm}
\usepackage{amsmath}
\usepackage{amssymb}
\usepackage{algorithm}
\usepackage{algpseudocode}

\usetikzlibrary{graphs, positioning, shapes.geometric}

\usepackage{makeidx} \makeindex
\usepackage{xcolor}               % (Demonstration purposes only.)
\usepackage{hyperref,cleveref}    % LOAD THESE LAST!

% Declare theorem environments for amsthm
\newtheorem{axiom}{axiom}
\newtheorem{lemma}{Lemma}
\newtheorem{theorem}{Theorem}
\newtheorem{example}{Example}
\newtheorem{definition}{Definition}

% --- LATEX AND HTML CUSTOMIZATION ---
\title{Notes on Everything}
\author{Anthony}
\date{\today}

\setcounter{tocdepth}{1}
\setcounter{secnumdepth}{-1}
\setcounter{FileDepth}{1}
\booltrue{CombineHigherDepths}
\setcounter{SideTOCDepth}{2}

% Renew commands

\renewcommand*\contentsname{Subjects}
\renewcommand\sidetocname{Subjects}

% HTML Directives
\HTMLTitle{Notes on Everything}
\HTMLAuthor{Anthony}
\HTMLLanguage{en-US}
\HTMLDescription{Personal notes on Mathematics and Computer Science}
\HTMLPageBottom{\LinkHome}

% Styling
\CSSFilename{lwarp.css}

\begin{document}

\maketitle % Or titlepage/titlingpage environment.
% An article abstract would go here.

Notes on Mathematics and Computer Science.

\tableofcontents % MUST BE BEFORE THE FIRST SECTION BREAK!

\listoffigures

% Save this as tutorial.tex for the lwarp package tutorial.
\documentclass{book}
\usepackage{iftex}

% --- LOAD FONT SELECTION AND ENCODING BEFORE LOADING LWARP ---
\ifPDFTeX
	\usepackage{lmodern} % pdflatex or dvi latex
	\usepackage[T1]{fontenc}
	\usepackage[utf8]{inputenc}
\else
	\usepackage{fontspec} % XeLaTeX or LuaLaTeX
\fi

% --- LWARP IS LOADED NEXT ---

\usepackage[
% HomeHTMLFilename=index, % Filename of the homepage.
% HTMLFilename={node-}, % Filename prefix of other pages.
% IndexLanguage=english, % Language for xindy index, glossary.
% latexmk, % Use latexmk to compile.
% OSWindows, % Force Windows. (Usually automatic.)
mathjax, % Use MathJax to display math.
]{lwarp}

% \boolfalse{FileSectionNames} % If false, numbers the files.

% --- LOAD PDFLATEX MATH FONTS HERE ---

% --- OTHER PACKAGES ARE LOADED AFTER LWARP ---

\usepackage{standalone}

\usepackage{tikz}
\usepackage{amsthm}
\usepackage{amsmath}
\usepackage{amssymb}
\usepackage{algorithm}
\usepackage{algpseudocode}

\usetikzlibrary{graphs, positioning, shapes.geometric}

\usepackage{makeidx} \makeindex
\usepackage{xcolor}               % (Demonstration purposes only.)
\usepackage{hyperref,cleveref}    % LOAD THESE LAST!

% Declare theorem environments for amsthm
\newtheorem{axiom}{axiom}
\newtheorem{lemma}{Lemma}
\newtheorem{theorem}{Theorem}
\newtheorem{example}{Example}
\newtheorem{definition}{Definition}

% --- LATEX AND HTML CUSTOMIZATION ---
\title{Notes on Everything}
\author{Anthony}
\date{\today}

\setcounter{tocdepth}{1}
\setcounter{secnumdepth}{-1}
\setcounter{FileDepth}{1}
\booltrue{CombineHigherDepths}
\setcounter{SideTOCDepth}{2}

% Renew commands

\renewcommand*\contentsname{Subjects}
\renewcommand\sidetocname{Subjects}

% HTML Directives
\HTMLTitle{Notes on Everything}
\HTMLAuthor{Anthony}
\HTMLLanguage{en-US}
\HTMLDescription{Personal notes on Mathematics and Computer Science}
\HTMLPageBottom{\LinkHome}

% Styling
\CSSFilename{lwarp.css}

\begin{document}

\maketitle % Or titlepage/titlingpage environment.
% An article abstract would go here.

Notes on Mathematics and Computer Science.

\tableofcontents % MUST BE BEFORE THE FIRST SECTION BREAK!

\listoffigures

\input{./sums/index}
\input{./algorithms/index}
\input{./combinatorics/index}
\input{./probability/index}
\input{./foundations/index}

\ForceHTMLPage 	% HTML index will be on its own page.
\ForceHTMLTOC 	% HTML index will have its own toc entry.
\printindex
\end{document}

% Save this as tutorial.tex for the lwarp package tutorial.
\documentclass{book}
\usepackage{iftex}

% --- LOAD FONT SELECTION AND ENCODING BEFORE LOADING LWARP ---
\ifPDFTeX
	\usepackage{lmodern} % pdflatex or dvi latex
	\usepackage[T1]{fontenc}
	\usepackage[utf8]{inputenc}
\else
	\usepackage{fontspec} % XeLaTeX or LuaLaTeX
\fi

% --- LWARP IS LOADED NEXT ---

\usepackage[
% HomeHTMLFilename=index, % Filename of the homepage.
% HTMLFilename={node-}, % Filename prefix of other pages.
% IndexLanguage=english, % Language for xindy index, glossary.
% latexmk, % Use latexmk to compile.
% OSWindows, % Force Windows. (Usually automatic.)
mathjax, % Use MathJax to display math.
]{lwarp}

% \boolfalse{FileSectionNames} % If false, numbers the files.

% --- LOAD PDFLATEX MATH FONTS HERE ---

% --- OTHER PACKAGES ARE LOADED AFTER LWARP ---

\usepackage{standalone}

\usepackage{tikz}
\usepackage{amsthm}
\usepackage{amsmath}
\usepackage{amssymb}
\usepackage{algorithm}
\usepackage{algpseudocode}

\usetikzlibrary{graphs, positioning, shapes.geometric}

\usepackage{makeidx} \makeindex
\usepackage{xcolor}               % (Demonstration purposes only.)
\usepackage{hyperref,cleveref}    % LOAD THESE LAST!

% Declare theorem environments for amsthm
\newtheorem{axiom}{axiom}
\newtheorem{lemma}{Lemma}
\newtheorem{theorem}{Theorem}
\newtheorem{example}{Example}
\newtheorem{definition}{Definition}

% --- LATEX AND HTML CUSTOMIZATION ---
\title{Notes on Everything}
\author{Anthony}
\date{\today}

\setcounter{tocdepth}{1}
\setcounter{secnumdepth}{-1}
\setcounter{FileDepth}{1}
\booltrue{CombineHigherDepths}
\setcounter{SideTOCDepth}{2}

% Renew commands

\renewcommand*\contentsname{Subjects}
\renewcommand\sidetocname{Subjects}

% HTML Directives
\HTMLTitle{Notes on Everything}
\HTMLAuthor{Anthony}
\HTMLLanguage{en-US}
\HTMLDescription{Personal notes on Mathematics and Computer Science}
\HTMLPageBottom{\LinkHome}

% Styling
\CSSFilename{lwarp.css}

\begin{document}

\maketitle % Or titlepage/titlingpage environment.
% An article abstract would go here.

Notes on Mathematics and Computer Science.

\tableofcontents % MUST BE BEFORE THE FIRST SECTION BREAK!

\listoffigures

\input{./sums/index}
\input{./algorithms/index}
\input{./combinatorics/index}
\input{./probability/index}
\input{./foundations/index}

\ForceHTMLPage 	% HTML index will be on its own page.
\ForceHTMLTOC 	% HTML index will have its own toc entry.
\printindex
\end{document}

% Save this as tutorial.tex for the lwarp package tutorial.
\documentclass{book}
\usepackage{iftex}

% --- LOAD FONT SELECTION AND ENCODING BEFORE LOADING LWARP ---
\ifPDFTeX
	\usepackage{lmodern} % pdflatex or dvi latex
	\usepackage[T1]{fontenc}
	\usepackage[utf8]{inputenc}
\else
	\usepackage{fontspec} % XeLaTeX or LuaLaTeX
\fi

% --- LWARP IS LOADED NEXT ---

\usepackage[
% HomeHTMLFilename=index, % Filename of the homepage.
% HTMLFilename={node-}, % Filename prefix of other pages.
% IndexLanguage=english, % Language for xindy index, glossary.
% latexmk, % Use latexmk to compile.
% OSWindows, % Force Windows. (Usually automatic.)
mathjax, % Use MathJax to display math.
]{lwarp}

% \boolfalse{FileSectionNames} % If false, numbers the files.

% --- LOAD PDFLATEX MATH FONTS HERE ---

% --- OTHER PACKAGES ARE LOADED AFTER LWARP ---

\usepackage{standalone}

\usepackage{tikz}
\usepackage{amsthm}
\usepackage{amsmath}
\usepackage{amssymb}
\usepackage{algorithm}
\usepackage{algpseudocode}

\usetikzlibrary{graphs, positioning, shapes.geometric}

\usepackage{makeidx} \makeindex
\usepackage{xcolor}               % (Demonstration purposes only.)
\usepackage{hyperref,cleveref}    % LOAD THESE LAST!

% Declare theorem environments for amsthm
\newtheorem{axiom}{axiom}
\newtheorem{lemma}{Lemma}
\newtheorem{theorem}{Theorem}
\newtheorem{example}{Example}
\newtheorem{definition}{Definition}

% --- LATEX AND HTML CUSTOMIZATION ---
\title{Notes on Everything}
\author{Anthony}
\date{\today}

\setcounter{tocdepth}{1}
\setcounter{secnumdepth}{-1}
\setcounter{FileDepth}{1}
\booltrue{CombineHigherDepths}
\setcounter{SideTOCDepth}{2}

% Renew commands

\renewcommand*\contentsname{Subjects}
\renewcommand\sidetocname{Subjects}

% HTML Directives
\HTMLTitle{Notes on Everything}
\HTMLAuthor{Anthony}
\HTMLLanguage{en-US}
\HTMLDescription{Personal notes on Mathematics and Computer Science}
\HTMLPageBottom{\LinkHome}

% Styling
\CSSFilename{lwarp.css}

\begin{document}

\maketitle % Or titlepage/titlingpage environment.
% An article abstract would go here.

Notes on Mathematics and Computer Science.

\tableofcontents % MUST BE BEFORE THE FIRST SECTION BREAK!

\listoffigures

\input{./sums/index}
\input{./algorithms/index}
\input{./combinatorics/index}
\input{./probability/index}
\input{./foundations/index}

\ForceHTMLPage 	% HTML index will be on its own page.
\ForceHTMLTOC 	% HTML index will have its own toc entry.
\printindex
\end{document}

% Save this as tutorial.tex for the lwarp package tutorial.
\documentclass{book}
\usepackage{iftex}

% --- LOAD FONT SELECTION AND ENCODING BEFORE LOADING LWARP ---
\ifPDFTeX
	\usepackage{lmodern} % pdflatex or dvi latex
	\usepackage[T1]{fontenc}
	\usepackage[utf8]{inputenc}
\else
	\usepackage{fontspec} % XeLaTeX or LuaLaTeX
\fi

% --- LWARP IS LOADED NEXT ---

\usepackage[
% HomeHTMLFilename=index, % Filename of the homepage.
% HTMLFilename={node-}, % Filename prefix of other pages.
% IndexLanguage=english, % Language for xindy index, glossary.
% latexmk, % Use latexmk to compile.
% OSWindows, % Force Windows. (Usually automatic.)
mathjax, % Use MathJax to display math.
]{lwarp}

% \boolfalse{FileSectionNames} % If false, numbers the files.

% --- LOAD PDFLATEX MATH FONTS HERE ---

% --- OTHER PACKAGES ARE LOADED AFTER LWARP ---

\usepackage{standalone}

\usepackage{tikz}
\usepackage{amsthm}
\usepackage{amsmath}
\usepackage{amssymb}
\usepackage{algorithm}
\usepackage{algpseudocode}

\usetikzlibrary{graphs, positioning, shapes.geometric}

\usepackage{makeidx} \makeindex
\usepackage{xcolor}               % (Demonstration purposes only.)
\usepackage{hyperref,cleveref}    % LOAD THESE LAST!

% Declare theorem environments for amsthm
\newtheorem{axiom}{axiom}
\newtheorem{lemma}{Lemma}
\newtheorem{theorem}{Theorem}
\newtheorem{example}{Example}
\newtheorem{definition}{Definition}

% --- LATEX AND HTML CUSTOMIZATION ---
\title{Notes on Everything}
\author{Anthony}
\date{\today}

\setcounter{tocdepth}{1}
\setcounter{secnumdepth}{-1}
\setcounter{FileDepth}{1}
\booltrue{CombineHigherDepths}
\setcounter{SideTOCDepth}{2}

% Renew commands

\renewcommand*\contentsname{Subjects}
\renewcommand\sidetocname{Subjects}

% HTML Directives
\HTMLTitle{Notes on Everything}
\HTMLAuthor{Anthony}
\HTMLLanguage{en-US}
\HTMLDescription{Personal notes on Mathematics and Computer Science}
\HTMLPageBottom{\LinkHome}

% Styling
\CSSFilename{lwarp.css}

\begin{document}

\maketitle % Or titlepage/titlingpage environment.
% An article abstract would go here.

Notes on Mathematics and Computer Science.

\tableofcontents % MUST BE BEFORE THE FIRST SECTION BREAK!

\listoffigures

\input{./sums/index}
\input{./algorithms/index}
\input{./combinatorics/index}
\input{./probability/index}
\input{./foundations/index}

\ForceHTMLPage 	% HTML index will be on its own page.
\ForceHTMLTOC 	% HTML index will have its own toc entry.
\printindex
\end{document}

% Save this as tutorial.tex for the lwarp package tutorial.
\documentclass{book}
\usepackage{iftex}

% --- LOAD FONT SELECTION AND ENCODING BEFORE LOADING LWARP ---
\ifPDFTeX
	\usepackage{lmodern} % pdflatex or dvi latex
	\usepackage[T1]{fontenc}
	\usepackage[utf8]{inputenc}
\else
	\usepackage{fontspec} % XeLaTeX or LuaLaTeX
\fi

% --- LWARP IS LOADED NEXT ---

\usepackage[
% HomeHTMLFilename=index, % Filename of the homepage.
% HTMLFilename={node-}, % Filename prefix of other pages.
% IndexLanguage=english, % Language for xindy index, glossary.
% latexmk, % Use latexmk to compile.
% OSWindows, % Force Windows. (Usually automatic.)
mathjax, % Use MathJax to display math.
]{lwarp}

% \boolfalse{FileSectionNames} % If false, numbers the files.

% --- LOAD PDFLATEX MATH FONTS HERE ---

% --- OTHER PACKAGES ARE LOADED AFTER LWARP ---

\usepackage{standalone}

\usepackage{tikz}
\usepackage{amsthm}
\usepackage{amsmath}
\usepackage{amssymb}
\usepackage{algorithm}
\usepackage{algpseudocode}

\usetikzlibrary{graphs, positioning, shapes.geometric}

\usepackage{makeidx} \makeindex
\usepackage{xcolor}               % (Demonstration purposes only.)
\usepackage{hyperref,cleveref}    % LOAD THESE LAST!

% Declare theorem environments for amsthm
\newtheorem{axiom}{axiom}
\newtheorem{lemma}{Lemma}
\newtheorem{theorem}{Theorem}
\newtheorem{example}{Example}
\newtheorem{definition}{Definition}

% --- LATEX AND HTML CUSTOMIZATION ---
\title{Notes on Everything}
\author{Anthony}
\date{\today}

\setcounter{tocdepth}{1}
\setcounter{secnumdepth}{-1}
\setcounter{FileDepth}{1}
\booltrue{CombineHigherDepths}
\setcounter{SideTOCDepth}{2}

% Renew commands

\renewcommand*\contentsname{Subjects}
\renewcommand\sidetocname{Subjects}

% HTML Directives
\HTMLTitle{Notes on Everything}
\HTMLAuthor{Anthony}
\HTMLLanguage{en-US}
\HTMLDescription{Personal notes on Mathematics and Computer Science}
\HTMLPageBottom{\LinkHome}

% Styling
\CSSFilename{lwarp.css}

\begin{document}

\maketitle % Or titlepage/titlingpage environment.
% An article abstract would go here.

Notes on Mathematics and Computer Science.

\tableofcontents % MUST BE BEFORE THE FIRST SECTION BREAK!

\listoffigures

\input{./sums/index}
\input{./algorithms/index}
\input{./combinatorics/index}
\input{./probability/index}
\input{./foundations/index}

\ForceHTMLPage 	% HTML index will be on its own page.
\ForceHTMLTOC 	% HTML index will have its own toc entry.
\printindex
\end{document}


\ForceHTMLPage 	% HTML index will be on its own page.
\ForceHTMLTOC 	% HTML index will have its own toc entry.
\printindex
\end{document}

% Save this as tutorial.tex for the lwarp package tutorial.
\documentclass{book}
\usepackage{iftex}

% --- LOAD FONT SELECTION AND ENCODING BEFORE LOADING LWARP ---
\ifPDFTeX
	\usepackage{lmodern} % pdflatex or dvi latex
	\usepackage[T1]{fontenc}
	\usepackage[utf8]{inputenc}
\else
	\usepackage{fontspec} % XeLaTeX or LuaLaTeX
\fi

% --- LWARP IS LOADED NEXT ---

\usepackage[
% HomeHTMLFilename=index, % Filename of the homepage.
% HTMLFilename={node-}, % Filename prefix of other pages.
% IndexLanguage=english, % Language for xindy index, glossary.
% latexmk, % Use latexmk to compile.
% OSWindows, % Force Windows. (Usually automatic.)
mathjax, % Use MathJax to display math.
]{lwarp}

% \boolfalse{FileSectionNames} % If false, numbers the files.

% --- LOAD PDFLATEX MATH FONTS HERE ---

% --- OTHER PACKAGES ARE LOADED AFTER LWARP ---

\usepackage{standalone}

\usepackage{tikz}
\usepackage{amsthm}
\usepackage{amsmath}
\usepackage{amssymb}
\usepackage{algorithm}
\usepackage{algpseudocode}

\usetikzlibrary{graphs, positioning, shapes.geometric}

\usepackage{makeidx} \makeindex
\usepackage{xcolor}               % (Demonstration purposes only.)
\usepackage{hyperref,cleveref}    % LOAD THESE LAST!

% Declare theorem environments for amsthm
\newtheorem{axiom}{axiom}
\newtheorem{lemma}{Lemma}
\newtheorem{theorem}{Theorem}
\newtheorem{example}{Example}
\newtheorem{definition}{Definition}

% --- LATEX AND HTML CUSTOMIZATION ---
\title{Notes on Everything}
\author{Anthony}
\date{\today}

\setcounter{tocdepth}{1}
\setcounter{secnumdepth}{-1}
\setcounter{FileDepth}{1}
\booltrue{CombineHigherDepths}
\setcounter{SideTOCDepth}{2}

% Renew commands

\renewcommand*\contentsname{Subjects}
\renewcommand\sidetocname{Subjects}

% HTML Directives
\HTMLTitle{Notes on Everything}
\HTMLAuthor{Anthony}
\HTMLLanguage{en-US}
\HTMLDescription{Personal notes on Mathematics and Computer Science}
\HTMLPageBottom{\LinkHome}

% Styling
\CSSFilename{lwarp.css}

\begin{document}

\maketitle % Or titlepage/titlingpage environment.
% An article abstract would go here.

Notes on Mathematics and Computer Science.

\tableofcontents % MUST BE BEFORE THE FIRST SECTION BREAK!

\listoffigures

% Save this as tutorial.tex for the lwarp package tutorial.
\documentclass{book}
\usepackage{iftex}

% --- LOAD FONT SELECTION AND ENCODING BEFORE LOADING LWARP ---
\ifPDFTeX
	\usepackage{lmodern} % pdflatex or dvi latex
	\usepackage[T1]{fontenc}
	\usepackage[utf8]{inputenc}
\else
	\usepackage{fontspec} % XeLaTeX or LuaLaTeX
\fi

% --- LWARP IS LOADED NEXT ---

\usepackage[
% HomeHTMLFilename=index, % Filename of the homepage.
% HTMLFilename={node-}, % Filename prefix of other pages.
% IndexLanguage=english, % Language for xindy index, glossary.
% latexmk, % Use latexmk to compile.
% OSWindows, % Force Windows. (Usually automatic.)
mathjax, % Use MathJax to display math.
]{lwarp}

% \boolfalse{FileSectionNames} % If false, numbers the files.

% --- LOAD PDFLATEX MATH FONTS HERE ---

% --- OTHER PACKAGES ARE LOADED AFTER LWARP ---

\usepackage{standalone}

\usepackage{tikz}
\usepackage{amsthm}
\usepackage{amsmath}
\usepackage{amssymb}
\usepackage{algorithm}
\usepackage{algpseudocode}

\usetikzlibrary{graphs, positioning, shapes.geometric}

\usepackage{makeidx} \makeindex
\usepackage{xcolor}               % (Demonstration purposes only.)
\usepackage{hyperref,cleveref}    % LOAD THESE LAST!

% Declare theorem environments for amsthm
\newtheorem{axiom}{axiom}
\newtheorem{lemma}{Lemma}
\newtheorem{theorem}{Theorem}
\newtheorem{example}{Example}
\newtheorem{definition}{Definition}

% --- LATEX AND HTML CUSTOMIZATION ---
\title{Notes on Everything}
\author{Anthony}
\date{\today}

\setcounter{tocdepth}{1}
\setcounter{secnumdepth}{-1}
\setcounter{FileDepth}{1}
\booltrue{CombineHigherDepths}
\setcounter{SideTOCDepth}{2}

% Renew commands

\renewcommand*\contentsname{Subjects}
\renewcommand\sidetocname{Subjects}

% HTML Directives
\HTMLTitle{Notes on Everything}
\HTMLAuthor{Anthony}
\HTMLLanguage{en-US}
\HTMLDescription{Personal notes on Mathematics and Computer Science}
\HTMLPageBottom{\LinkHome}

% Styling
\CSSFilename{lwarp.css}

\begin{document}

\maketitle % Or titlepage/titlingpage environment.
% An article abstract would go here.

Notes on Mathematics and Computer Science.

\tableofcontents % MUST BE BEFORE THE FIRST SECTION BREAK!

\listoffigures

\input{./sums/index}
\input{./algorithms/index}
\input{./combinatorics/index}
\input{./probability/index}
\input{./foundations/index}

\ForceHTMLPage 	% HTML index will be on its own page.
\ForceHTMLTOC 	% HTML index will have its own toc entry.
\printindex
\end{document}

% Save this as tutorial.tex for the lwarp package tutorial.
\documentclass{book}
\usepackage{iftex}

% --- LOAD FONT SELECTION AND ENCODING BEFORE LOADING LWARP ---
\ifPDFTeX
	\usepackage{lmodern} % pdflatex or dvi latex
	\usepackage[T1]{fontenc}
	\usepackage[utf8]{inputenc}
\else
	\usepackage{fontspec} % XeLaTeX or LuaLaTeX
\fi

% --- LWARP IS LOADED NEXT ---

\usepackage[
% HomeHTMLFilename=index, % Filename of the homepage.
% HTMLFilename={node-}, % Filename prefix of other pages.
% IndexLanguage=english, % Language for xindy index, glossary.
% latexmk, % Use latexmk to compile.
% OSWindows, % Force Windows. (Usually automatic.)
mathjax, % Use MathJax to display math.
]{lwarp}

% \boolfalse{FileSectionNames} % If false, numbers the files.

% --- LOAD PDFLATEX MATH FONTS HERE ---

% --- OTHER PACKAGES ARE LOADED AFTER LWARP ---

\usepackage{standalone}

\usepackage{tikz}
\usepackage{amsthm}
\usepackage{amsmath}
\usepackage{amssymb}
\usepackage{algorithm}
\usepackage{algpseudocode}

\usetikzlibrary{graphs, positioning, shapes.geometric}

\usepackage{makeidx} \makeindex
\usepackage{xcolor}               % (Demonstration purposes only.)
\usepackage{hyperref,cleveref}    % LOAD THESE LAST!

% Declare theorem environments for amsthm
\newtheorem{axiom}{axiom}
\newtheorem{lemma}{Lemma}
\newtheorem{theorem}{Theorem}
\newtheorem{example}{Example}
\newtheorem{definition}{Definition}

% --- LATEX AND HTML CUSTOMIZATION ---
\title{Notes on Everything}
\author{Anthony}
\date{\today}

\setcounter{tocdepth}{1}
\setcounter{secnumdepth}{-1}
\setcounter{FileDepth}{1}
\booltrue{CombineHigherDepths}
\setcounter{SideTOCDepth}{2}

% Renew commands

\renewcommand*\contentsname{Subjects}
\renewcommand\sidetocname{Subjects}

% HTML Directives
\HTMLTitle{Notes on Everything}
\HTMLAuthor{Anthony}
\HTMLLanguage{en-US}
\HTMLDescription{Personal notes on Mathematics and Computer Science}
\HTMLPageBottom{\LinkHome}

% Styling
\CSSFilename{lwarp.css}

\begin{document}

\maketitle % Or titlepage/titlingpage environment.
% An article abstract would go here.

Notes on Mathematics and Computer Science.

\tableofcontents % MUST BE BEFORE THE FIRST SECTION BREAK!

\listoffigures

\input{./sums/index}
\input{./algorithms/index}
\input{./combinatorics/index}
\input{./probability/index}
\input{./foundations/index}

\ForceHTMLPage 	% HTML index will be on its own page.
\ForceHTMLTOC 	% HTML index will have its own toc entry.
\printindex
\end{document}

% Save this as tutorial.tex for the lwarp package tutorial.
\documentclass{book}
\usepackage{iftex}

% --- LOAD FONT SELECTION AND ENCODING BEFORE LOADING LWARP ---
\ifPDFTeX
	\usepackage{lmodern} % pdflatex or dvi latex
	\usepackage[T1]{fontenc}
	\usepackage[utf8]{inputenc}
\else
	\usepackage{fontspec} % XeLaTeX or LuaLaTeX
\fi

% --- LWARP IS LOADED NEXT ---

\usepackage[
% HomeHTMLFilename=index, % Filename of the homepage.
% HTMLFilename={node-}, % Filename prefix of other pages.
% IndexLanguage=english, % Language for xindy index, glossary.
% latexmk, % Use latexmk to compile.
% OSWindows, % Force Windows. (Usually automatic.)
mathjax, % Use MathJax to display math.
]{lwarp}

% \boolfalse{FileSectionNames} % If false, numbers the files.

% --- LOAD PDFLATEX MATH FONTS HERE ---

% --- OTHER PACKAGES ARE LOADED AFTER LWARP ---

\usepackage{standalone}

\usepackage{tikz}
\usepackage{amsthm}
\usepackage{amsmath}
\usepackage{amssymb}
\usepackage{algorithm}
\usepackage{algpseudocode}

\usetikzlibrary{graphs, positioning, shapes.geometric}

\usepackage{makeidx} \makeindex
\usepackage{xcolor}               % (Demonstration purposes only.)
\usepackage{hyperref,cleveref}    % LOAD THESE LAST!

% Declare theorem environments for amsthm
\newtheorem{axiom}{axiom}
\newtheorem{lemma}{Lemma}
\newtheorem{theorem}{Theorem}
\newtheorem{example}{Example}
\newtheorem{definition}{Definition}

% --- LATEX AND HTML CUSTOMIZATION ---
\title{Notes on Everything}
\author{Anthony}
\date{\today}

\setcounter{tocdepth}{1}
\setcounter{secnumdepth}{-1}
\setcounter{FileDepth}{1}
\booltrue{CombineHigherDepths}
\setcounter{SideTOCDepth}{2}

% Renew commands

\renewcommand*\contentsname{Subjects}
\renewcommand\sidetocname{Subjects}

% HTML Directives
\HTMLTitle{Notes on Everything}
\HTMLAuthor{Anthony}
\HTMLLanguage{en-US}
\HTMLDescription{Personal notes on Mathematics and Computer Science}
\HTMLPageBottom{\LinkHome}

% Styling
\CSSFilename{lwarp.css}

\begin{document}

\maketitle % Or titlepage/titlingpage environment.
% An article abstract would go here.

Notes on Mathematics and Computer Science.

\tableofcontents % MUST BE BEFORE THE FIRST SECTION BREAK!

\listoffigures

\input{./sums/index}
\input{./algorithms/index}
\input{./combinatorics/index}
\input{./probability/index}
\input{./foundations/index}

\ForceHTMLPage 	% HTML index will be on its own page.
\ForceHTMLTOC 	% HTML index will have its own toc entry.
\printindex
\end{document}

% Save this as tutorial.tex for the lwarp package tutorial.
\documentclass{book}
\usepackage{iftex}

% --- LOAD FONT SELECTION AND ENCODING BEFORE LOADING LWARP ---
\ifPDFTeX
	\usepackage{lmodern} % pdflatex or dvi latex
	\usepackage[T1]{fontenc}
	\usepackage[utf8]{inputenc}
\else
	\usepackage{fontspec} % XeLaTeX or LuaLaTeX
\fi

% --- LWARP IS LOADED NEXT ---

\usepackage[
% HomeHTMLFilename=index, % Filename of the homepage.
% HTMLFilename={node-}, % Filename prefix of other pages.
% IndexLanguage=english, % Language for xindy index, glossary.
% latexmk, % Use latexmk to compile.
% OSWindows, % Force Windows. (Usually automatic.)
mathjax, % Use MathJax to display math.
]{lwarp}

% \boolfalse{FileSectionNames} % If false, numbers the files.

% --- LOAD PDFLATEX MATH FONTS HERE ---

% --- OTHER PACKAGES ARE LOADED AFTER LWARP ---

\usepackage{standalone}

\usepackage{tikz}
\usepackage{amsthm}
\usepackage{amsmath}
\usepackage{amssymb}
\usepackage{algorithm}
\usepackage{algpseudocode}

\usetikzlibrary{graphs, positioning, shapes.geometric}

\usepackage{makeidx} \makeindex
\usepackage{xcolor}               % (Demonstration purposes only.)
\usepackage{hyperref,cleveref}    % LOAD THESE LAST!

% Declare theorem environments for amsthm
\newtheorem{axiom}{axiom}
\newtheorem{lemma}{Lemma}
\newtheorem{theorem}{Theorem}
\newtheorem{example}{Example}
\newtheorem{definition}{Definition}

% --- LATEX AND HTML CUSTOMIZATION ---
\title{Notes on Everything}
\author{Anthony}
\date{\today}

\setcounter{tocdepth}{1}
\setcounter{secnumdepth}{-1}
\setcounter{FileDepth}{1}
\booltrue{CombineHigherDepths}
\setcounter{SideTOCDepth}{2}

% Renew commands

\renewcommand*\contentsname{Subjects}
\renewcommand\sidetocname{Subjects}

% HTML Directives
\HTMLTitle{Notes on Everything}
\HTMLAuthor{Anthony}
\HTMLLanguage{en-US}
\HTMLDescription{Personal notes on Mathematics and Computer Science}
\HTMLPageBottom{\LinkHome}

% Styling
\CSSFilename{lwarp.css}

\begin{document}

\maketitle % Or titlepage/titlingpage environment.
% An article abstract would go here.

Notes on Mathematics and Computer Science.

\tableofcontents % MUST BE BEFORE THE FIRST SECTION BREAK!

\listoffigures

\input{./sums/index}
\input{./algorithms/index}
\input{./combinatorics/index}
\input{./probability/index}
\input{./foundations/index}

\ForceHTMLPage 	% HTML index will be on its own page.
\ForceHTMLTOC 	% HTML index will have its own toc entry.
\printindex
\end{document}

% Save this as tutorial.tex for the lwarp package tutorial.
\documentclass{book}
\usepackage{iftex}

% --- LOAD FONT SELECTION AND ENCODING BEFORE LOADING LWARP ---
\ifPDFTeX
	\usepackage{lmodern} % pdflatex or dvi latex
	\usepackage[T1]{fontenc}
	\usepackage[utf8]{inputenc}
\else
	\usepackage{fontspec} % XeLaTeX or LuaLaTeX
\fi

% --- LWARP IS LOADED NEXT ---

\usepackage[
% HomeHTMLFilename=index, % Filename of the homepage.
% HTMLFilename={node-}, % Filename prefix of other pages.
% IndexLanguage=english, % Language for xindy index, glossary.
% latexmk, % Use latexmk to compile.
% OSWindows, % Force Windows. (Usually automatic.)
mathjax, % Use MathJax to display math.
]{lwarp}

% \boolfalse{FileSectionNames} % If false, numbers the files.

% --- LOAD PDFLATEX MATH FONTS HERE ---

% --- OTHER PACKAGES ARE LOADED AFTER LWARP ---

\usepackage{standalone}

\usepackage{tikz}
\usepackage{amsthm}
\usepackage{amsmath}
\usepackage{amssymb}
\usepackage{algorithm}
\usepackage{algpseudocode}

\usetikzlibrary{graphs, positioning, shapes.geometric}

\usepackage{makeidx} \makeindex
\usepackage{xcolor}               % (Demonstration purposes only.)
\usepackage{hyperref,cleveref}    % LOAD THESE LAST!

% Declare theorem environments for amsthm
\newtheorem{axiom}{axiom}
\newtheorem{lemma}{Lemma}
\newtheorem{theorem}{Theorem}
\newtheorem{example}{Example}
\newtheorem{definition}{Definition}

% --- LATEX AND HTML CUSTOMIZATION ---
\title{Notes on Everything}
\author{Anthony}
\date{\today}

\setcounter{tocdepth}{1}
\setcounter{secnumdepth}{-1}
\setcounter{FileDepth}{1}
\booltrue{CombineHigherDepths}
\setcounter{SideTOCDepth}{2}

% Renew commands

\renewcommand*\contentsname{Subjects}
\renewcommand\sidetocname{Subjects}

% HTML Directives
\HTMLTitle{Notes on Everything}
\HTMLAuthor{Anthony}
\HTMLLanguage{en-US}
\HTMLDescription{Personal notes on Mathematics and Computer Science}
\HTMLPageBottom{\LinkHome}

% Styling
\CSSFilename{lwarp.css}

\begin{document}

\maketitle % Or titlepage/titlingpage environment.
% An article abstract would go here.

Notes on Mathematics and Computer Science.

\tableofcontents % MUST BE BEFORE THE FIRST SECTION BREAK!

\listoffigures

\input{./sums/index}
\input{./algorithms/index}
\input{./combinatorics/index}
\input{./probability/index}
\input{./foundations/index}

\ForceHTMLPage 	% HTML index will be on its own page.
\ForceHTMLTOC 	% HTML index will have its own toc entry.
\printindex
\end{document}


\ForceHTMLPage 	% HTML index will be on its own page.
\ForceHTMLTOC 	% HTML index will have its own toc entry.
\printindex
\end{document}


\ForceHTMLPage 	% HTML index will be on its own page.
\ForceHTMLTOC 	% HTML index will have its own toc entry.
\printindex
\end{document}

% Save this as tutorial.tex for the lwarp package tutorial.
\documentclass{book}
\usepackage{iftex}

% --- LOAD FONT SELECTION AND ENCODING BEFORE LOADING LWARP ---
\ifPDFTeX
	\usepackage{lmodern} % pdflatex or dvi latex
	\usepackage[T1]{fontenc}
	\usepackage[utf8]{inputenc}
\else
	\usepackage{fontspec} % XeLaTeX or LuaLaTeX
\fi

% --- LWARP IS LOADED NEXT ---

\usepackage[
% HomeHTMLFilename=index, % Filename of the homepage.
% HTMLFilename={node-}, % Filename prefix of other pages.
% IndexLanguage=english, % Language for xindy index, glossary.
% latexmk, % Use latexmk to compile.
% OSWindows, % Force Windows. (Usually automatic.)
mathjax, % Use MathJax to display math.
]{lwarp}

% \boolfalse{FileSectionNames} % If false, numbers the files.

% --- LOAD PDFLATEX MATH FONTS HERE ---

% --- OTHER PACKAGES ARE LOADED AFTER LWARP ---

\usepackage{standalone}

\usepackage{tikz}
\usepackage{amsthm}
\usepackage{amsmath}
\usepackage{amssymb}
\usepackage{algorithm}
\usepackage{algpseudocode}

\usetikzlibrary{graphs, positioning, shapes.geometric}

\usepackage{makeidx} \makeindex
\usepackage{xcolor}               % (Demonstration purposes only.)
\usepackage{hyperref,cleveref}    % LOAD THESE LAST!

% Declare theorem environments for amsthm
\newtheorem{axiom}{axiom}
\newtheorem{lemma}{Lemma}
\newtheorem{theorem}{Theorem}
\newtheorem{example}{Example}
\newtheorem{definition}{Definition}

% --- LATEX AND HTML CUSTOMIZATION ---
\title{Notes on Everything}
\author{Anthony}
\date{\today}

\setcounter{tocdepth}{1}
\setcounter{secnumdepth}{-1}
\setcounter{FileDepth}{1}
\booltrue{CombineHigherDepths}
\setcounter{SideTOCDepth}{2}

% Renew commands

\renewcommand*\contentsname{Subjects}
\renewcommand\sidetocname{Subjects}

% HTML Directives
\HTMLTitle{Notes on Everything}
\HTMLAuthor{Anthony}
\HTMLLanguage{en-US}
\HTMLDescription{Personal notes on Mathematics and Computer Science}
\HTMLPageBottom{\LinkHome}

% Styling
\CSSFilename{lwarp.css}

\begin{document}

\maketitle % Or titlepage/titlingpage environment.
% An article abstract would go here.

Notes on Mathematics and Computer Science.

\tableofcontents % MUST BE BEFORE THE FIRST SECTION BREAK!

\listoffigures

% Save this as tutorial.tex for the lwarp package tutorial.
\documentclass{book}
\usepackage{iftex}

% --- LOAD FONT SELECTION AND ENCODING BEFORE LOADING LWARP ---
\ifPDFTeX
	\usepackage{lmodern} % pdflatex or dvi latex
	\usepackage[T1]{fontenc}
	\usepackage[utf8]{inputenc}
\else
	\usepackage{fontspec} % XeLaTeX or LuaLaTeX
\fi

% --- LWARP IS LOADED NEXT ---

\usepackage[
% HomeHTMLFilename=index, % Filename of the homepage.
% HTMLFilename={node-}, % Filename prefix of other pages.
% IndexLanguage=english, % Language for xindy index, glossary.
% latexmk, % Use latexmk to compile.
% OSWindows, % Force Windows. (Usually automatic.)
mathjax, % Use MathJax to display math.
]{lwarp}

% \boolfalse{FileSectionNames} % If false, numbers the files.

% --- LOAD PDFLATEX MATH FONTS HERE ---

% --- OTHER PACKAGES ARE LOADED AFTER LWARP ---

\usepackage{standalone}

\usepackage{tikz}
\usepackage{amsthm}
\usepackage{amsmath}
\usepackage{amssymb}
\usepackage{algorithm}
\usepackage{algpseudocode}

\usetikzlibrary{graphs, positioning, shapes.geometric}

\usepackage{makeidx} \makeindex
\usepackage{xcolor}               % (Demonstration purposes only.)
\usepackage{hyperref,cleveref}    % LOAD THESE LAST!

% Declare theorem environments for amsthm
\newtheorem{axiom}{axiom}
\newtheorem{lemma}{Lemma}
\newtheorem{theorem}{Theorem}
\newtheorem{example}{Example}
\newtheorem{definition}{Definition}

% --- LATEX AND HTML CUSTOMIZATION ---
\title{Notes on Everything}
\author{Anthony}
\date{\today}

\setcounter{tocdepth}{1}
\setcounter{secnumdepth}{-1}
\setcounter{FileDepth}{1}
\booltrue{CombineHigherDepths}
\setcounter{SideTOCDepth}{2}

% Renew commands

\renewcommand*\contentsname{Subjects}
\renewcommand\sidetocname{Subjects}

% HTML Directives
\HTMLTitle{Notes on Everything}
\HTMLAuthor{Anthony}
\HTMLLanguage{en-US}
\HTMLDescription{Personal notes on Mathematics and Computer Science}
\HTMLPageBottom{\LinkHome}

% Styling
\CSSFilename{lwarp.css}

\begin{document}

\maketitle % Or titlepage/titlingpage environment.
% An article abstract would go here.

Notes on Mathematics and Computer Science.

\tableofcontents % MUST BE BEFORE THE FIRST SECTION BREAK!

\listoffigures

% Save this as tutorial.tex for the lwarp package tutorial.
\documentclass{book}
\usepackage{iftex}

% --- LOAD FONT SELECTION AND ENCODING BEFORE LOADING LWARP ---
\ifPDFTeX
	\usepackage{lmodern} % pdflatex or dvi latex
	\usepackage[T1]{fontenc}
	\usepackage[utf8]{inputenc}
\else
	\usepackage{fontspec} % XeLaTeX or LuaLaTeX
\fi

% --- LWARP IS LOADED NEXT ---

\usepackage[
% HomeHTMLFilename=index, % Filename of the homepage.
% HTMLFilename={node-}, % Filename prefix of other pages.
% IndexLanguage=english, % Language for xindy index, glossary.
% latexmk, % Use latexmk to compile.
% OSWindows, % Force Windows. (Usually automatic.)
mathjax, % Use MathJax to display math.
]{lwarp}

% \boolfalse{FileSectionNames} % If false, numbers the files.

% --- LOAD PDFLATEX MATH FONTS HERE ---

% --- OTHER PACKAGES ARE LOADED AFTER LWARP ---

\usepackage{standalone}

\usepackage{tikz}
\usepackage{amsthm}
\usepackage{amsmath}
\usepackage{amssymb}
\usepackage{algorithm}
\usepackage{algpseudocode}

\usetikzlibrary{graphs, positioning, shapes.geometric}

\usepackage{makeidx} \makeindex
\usepackage{xcolor}               % (Demonstration purposes only.)
\usepackage{hyperref,cleveref}    % LOAD THESE LAST!

% Declare theorem environments for amsthm
\newtheorem{axiom}{axiom}
\newtheorem{lemma}{Lemma}
\newtheorem{theorem}{Theorem}
\newtheorem{example}{Example}
\newtheorem{definition}{Definition}

% --- LATEX AND HTML CUSTOMIZATION ---
\title{Notes on Everything}
\author{Anthony}
\date{\today}

\setcounter{tocdepth}{1}
\setcounter{secnumdepth}{-1}
\setcounter{FileDepth}{1}
\booltrue{CombineHigherDepths}
\setcounter{SideTOCDepth}{2}

% Renew commands

\renewcommand*\contentsname{Subjects}
\renewcommand\sidetocname{Subjects}

% HTML Directives
\HTMLTitle{Notes on Everything}
\HTMLAuthor{Anthony}
\HTMLLanguage{en-US}
\HTMLDescription{Personal notes on Mathematics and Computer Science}
\HTMLPageBottom{\LinkHome}

% Styling
\CSSFilename{lwarp.css}

\begin{document}

\maketitle % Or titlepage/titlingpage environment.
% An article abstract would go here.

Notes on Mathematics and Computer Science.

\tableofcontents % MUST BE BEFORE THE FIRST SECTION BREAK!

\listoffigures

\input{./sums/index}
\input{./algorithms/index}
\input{./combinatorics/index}
\input{./probability/index}
\input{./foundations/index}

\ForceHTMLPage 	% HTML index will be on its own page.
\ForceHTMLTOC 	% HTML index will have its own toc entry.
\printindex
\end{document}

% Save this as tutorial.tex for the lwarp package tutorial.
\documentclass{book}
\usepackage{iftex}

% --- LOAD FONT SELECTION AND ENCODING BEFORE LOADING LWARP ---
\ifPDFTeX
	\usepackage{lmodern} % pdflatex or dvi latex
	\usepackage[T1]{fontenc}
	\usepackage[utf8]{inputenc}
\else
	\usepackage{fontspec} % XeLaTeX or LuaLaTeX
\fi

% --- LWARP IS LOADED NEXT ---

\usepackage[
% HomeHTMLFilename=index, % Filename of the homepage.
% HTMLFilename={node-}, % Filename prefix of other pages.
% IndexLanguage=english, % Language for xindy index, glossary.
% latexmk, % Use latexmk to compile.
% OSWindows, % Force Windows. (Usually automatic.)
mathjax, % Use MathJax to display math.
]{lwarp}

% \boolfalse{FileSectionNames} % If false, numbers the files.

% --- LOAD PDFLATEX MATH FONTS HERE ---

% --- OTHER PACKAGES ARE LOADED AFTER LWARP ---

\usepackage{standalone}

\usepackage{tikz}
\usepackage{amsthm}
\usepackage{amsmath}
\usepackage{amssymb}
\usepackage{algorithm}
\usepackage{algpseudocode}

\usetikzlibrary{graphs, positioning, shapes.geometric}

\usepackage{makeidx} \makeindex
\usepackage{xcolor}               % (Demonstration purposes only.)
\usepackage{hyperref,cleveref}    % LOAD THESE LAST!

% Declare theorem environments for amsthm
\newtheorem{axiom}{axiom}
\newtheorem{lemma}{Lemma}
\newtheorem{theorem}{Theorem}
\newtheorem{example}{Example}
\newtheorem{definition}{Definition}

% --- LATEX AND HTML CUSTOMIZATION ---
\title{Notes on Everything}
\author{Anthony}
\date{\today}

\setcounter{tocdepth}{1}
\setcounter{secnumdepth}{-1}
\setcounter{FileDepth}{1}
\booltrue{CombineHigherDepths}
\setcounter{SideTOCDepth}{2}

% Renew commands

\renewcommand*\contentsname{Subjects}
\renewcommand\sidetocname{Subjects}

% HTML Directives
\HTMLTitle{Notes on Everything}
\HTMLAuthor{Anthony}
\HTMLLanguage{en-US}
\HTMLDescription{Personal notes on Mathematics and Computer Science}
\HTMLPageBottom{\LinkHome}

% Styling
\CSSFilename{lwarp.css}

\begin{document}

\maketitle % Or titlepage/titlingpage environment.
% An article abstract would go here.

Notes on Mathematics and Computer Science.

\tableofcontents % MUST BE BEFORE THE FIRST SECTION BREAK!

\listoffigures

\input{./sums/index}
\input{./algorithms/index}
\input{./combinatorics/index}
\input{./probability/index}
\input{./foundations/index}

\ForceHTMLPage 	% HTML index will be on its own page.
\ForceHTMLTOC 	% HTML index will have its own toc entry.
\printindex
\end{document}

% Save this as tutorial.tex for the lwarp package tutorial.
\documentclass{book}
\usepackage{iftex}

% --- LOAD FONT SELECTION AND ENCODING BEFORE LOADING LWARP ---
\ifPDFTeX
	\usepackage{lmodern} % pdflatex or dvi latex
	\usepackage[T1]{fontenc}
	\usepackage[utf8]{inputenc}
\else
	\usepackage{fontspec} % XeLaTeX or LuaLaTeX
\fi

% --- LWARP IS LOADED NEXT ---

\usepackage[
% HomeHTMLFilename=index, % Filename of the homepage.
% HTMLFilename={node-}, % Filename prefix of other pages.
% IndexLanguage=english, % Language for xindy index, glossary.
% latexmk, % Use latexmk to compile.
% OSWindows, % Force Windows. (Usually automatic.)
mathjax, % Use MathJax to display math.
]{lwarp}

% \boolfalse{FileSectionNames} % If false, numbers the files.

% --- LOAD PDFLATEX MATH FONTS HERE ---

% --- OTHER PACKAGES ARE LOADED AFTER LWARP ---

\usepackage{standalone}

\usepackage{tikz}
\usepackage{amsthm}
\usepackage{amsmath}
\usepackage{amssymb}
\usepackage{algorithm}
\usepackage{algpseudocode}

\usetikzlibrary{graphs, positioning, shapes.geometric}

\usepackage{makeidx} \makeindex
\usepackage{xcolor}               % (Demonstration purposes only.)
\usepackage{hyperref,cleveref}    % LOAD THESE LAST!

% Declare theorem environments for amsthm
\newtheorem{axiom}{axiom}
\newtheorem{lemma}{Lemma}
\newtheorem{theorem}{Theorem}
\newtheorem{example}{Example}
\newtheorem{definition}{Definition}

% --- LATEX AND HTML CUSTOMIZATION ---
\title{Notes on Everything}
\author{Anthony}
\date{\today}

\setcounter{tocdepth}{1}
\setcounter{secnumdepth}{-1}
\setcounter{FileDepth}{1}
\booltrue{CombineHigherDepths}
\setcounter{SideTOCDepth}{2}

% Renew commands

\renewcommand*\contentsname{Subjects}
\renewcommand\sidetocname{Subjects}

% HTML Directives
\HTMLTitle{Notes on Everything}
\HTMLAuthor{Anthony}
\HTMLLanguage{en-US}
\HTMLDescription{Personal notes on Mathematics and Computer Science}
\HTMLPageBottom{\LinkHome}

% Styling
\CSSFilename{lwarp.css}

\begin{document}

\maketitle % Or titlepage/titlingpage environment.
% An article abstract would go here.

Notes on Mathematics and Computer Science.

\tableofcontents % MUST BE BEFORE THE FIRST SECTION BREAK!

\listoffigures

\input{./sums/index}
\input{./algorithms/index}
\input{./combinatorics/index}
\input{./probability/index}
\input{./foundations/index}

\ForceHTMLPage 	% HTML index will be on its own page.
\ForceHTMLTOC 	% HTML index will have its own toc entry.
\printindex
\end{document}

% Save this as tutorial.tex for the lwarp package tutorial.
\documentclass{book}
\usepackage{iftex}

% --- LOAD FONT SELECTION AND ENCODING BEFORE LOADING LWARP ---
\ifPDFTeX
	\usepackage{lmodern} % pdflatex or dvi latex
	\usepackage[T1]{fontenc}
	\usepackage[utf8]{inputenc}
\else
	\usepackage{fontspec} % XeLaTeX or LuaLaTeX
\fi

% --- LWARP IS LOADED NEXT ---

\usepackage[
% HomeHTMLFilename=index, % Filename of the homepage.
% HTMLFilename={node-}, % Filename prefix of other pages.
% IndexLanguage=english, % Language for xindy index, glossary.
% latexmk, % Use latexmk to compile.
% OSWindows, % Force Windows. (Usually automatic.)
mathjax, % Use MathJax to display math.
]{lwarp}

% \boolfalse{FileSectionNames} % If false, numbers the files.

% --- LOAD PDFLATEX MATH FONTS HERE ---

% --- OTHER PACKAGES ARE LOADED AFTER LWARP ---

\usepackage{standalone}

\usepackage{tikz}
\usepackage{amsthm}
\usepackage{amsmath}
\usepackage{amssymb}
\usepackage{algorithm}
\usepackage{algpseudocode}

\usetikzlibrary{graphs, positioning, shapes.geometric}

\usepackage{makeidx} \makeindex
\usepackage{xcolor}               % (Demonstration purposes only.)
\usepackage{hyperref,cleveref}    % LOAD THESE LAST!

% Declare theorem environments for amsthm
\newtheorem{axiom}{axiom}
\newtheorem{lemma}{Lemma}
\newtheorem{theorem}{Theorem}
\newtheorem{example}{Example}
\newtheorem{definition}{Definition}

% --- LATEX AND HTML CUSTOMIZATION ---
\title{Notes on Everything}
\author{Anthony}
\date{\today}

\setcounter{tocdepth}{1}
\setcounter{secnumdepth}{-1}
\setcounter{FileDepth}{1}
\booltrue{CombineHigherDepths}
\setcounter{SideTOCDepth}{2}

% Renew commands

\renewcommand*\contentsname{Subjects}
\renewcommand\sidetocname{Subjects}

% HTML Directives
\HTMLTitle{Notes on Everything}
\HTMLAuthor{Anthony}
\HTMLLanguage{en-US}
\HTMLDescription{Personal notes on Mathematics and Computer Science}
\HTMLPageBottom{\LinkHome}

% Styling
\CSSFilename{lwarp.css}

\begin{document}

\maketitle % Or titlepage/titlingpage environment.
% An article abstract would go here.

Notes on Mathematics and Computer Science.

\tableofcontents % MUST BE BEFORE THE FIRST SECTION BREAK!

\listoffigures

\input{./sums/index}
\input{./algorithms/index}
\input{./combinatorics/index}
\input{./probability/index}
\input{./foundations/index}

\ForceHTMLPage 	% HTML index will be on its own page.
\ForceHTMLTOC 	% HTML index will have its own toc entry.
\printindex
\end{document}

% Save this as tutorial.tex for the lwarp package tutorial.
\documentclass{book}
\usepackage{iftex}

% --- LOAD FONT SELECTION AND ENCODING BEFORE LOADING LWARP ---
\ifPDFTeX
	\usepackage{lmodern} % pdflatex or dvi latex
	\usepackage[T1]{fontenc}
	\usepackage[utf8]{inputenc}
\else
	\usepackage{fontspec} % XeLaTeX or LuaLaTeX
\fi

% --- LWARP IS LOADED NEXT ---

\usepackage[
% HomeHTMLFilename=index, % Filename of the homepage.
% HTMLFilename={node-}, % Filename prefix of other pages.
% IndexLanguage=english, % Language for xindy index, glossary.
% latexmk, % Use latexmk to compile.
% OSWindows, % Force Windows. (Usually automatic.)
mathjax, % Use MathJax to display math.
]{lwarp}

% \boolfalse{FileSectionNames} % If false, numbers the files.

% --- LOAD PDFLATEX MATH FONTS HERE ---

% --- OTHER PACKAGES ARE LOADED AFTER LWARP ---

\usepackage{standalone}

\usepackage{tikz}
\usepackage{amsthm}
\usepackage{amsmath}
\usepackage{amssymb}
\usepackage{algorithm}
\usepackage{algpseudocode}

\usetikzlibrary{graphs, positioning, shapes.geometric}

\usepackage{makeidx} \makeindex
\usepackage{xcolor}               % (Demonstration purposes only.)
\usepackage{hyperref,cleveref}    % LOAD THESE LAST!

% Declare theorem environments for amsthm
\newtheorem{axiom}{axiom}
\newtheorem{lemma}{Lemma}
\newtheorem{theorem}{Theorem}
\newtheorem{example}{Example}
\newtheorem{definition}{Definition}

% --- LATEX AND HTML CUSTOMIZATION ---
\title{Notes on Everything}
\author{Anthony}
\date{\today}

\setcounter{tocdepth}{1}
\setcounter{secnumdepth}{-1}
\setcounter{FileDepth}{1}
\booltrue{CombineHigherDepths}
\setcounter{SideTOCDepth}{2}

% Renew commands

\renewcommand*\contentsname{Subjects}
\renewcommand\sidetocname{Subjects}

% HTML Directives
\HTMLTitle{Notes on Everything}
\HTMLAuthor{Anthony}
\HTMLLanguage{en-US}
\HTMLDescription{Personal notes on Mathematics and Computer Science}
\HTMLPageBottom{\LinkHome}

% Styling
\CSSFilename{lwarp.css}

\begin{document}

\maketitle % Or titlepage/titlingpage environment.
% An article abstract would go here.

Notes on Mathematics and Computer Science.

\tableofcontents % MUST BE BEFORE THE FIRST SECTION BREAK!

\listoffigures

\input{./sums/index}
\input{./algorithms/index}
\input{./combinatorics/index}
\input{./probability/index}
\input{./foundations/index}

\ForceHTMLPage 	% HTML index will be on its own page.
\ForceHTMLTOC 	% HTML index will have its own toc entry.
\printindex
\end{document}


\ForceHTMLPage 	% HTML index will be on its own page.
\ForceHTMLTOC 	% HTML index will have its own toc entry.
\printindex
\end{document}

% Save this as tutorial.tex for the lwarp package tutorial.
\documentclass{book}
\usepackage{iftex}

% --- LOAD FONT SELECTION AND ENCODING BEFORE LOADING LWARP ---
\ifPDFTeX
	\usepackage{lmodern} % pdflatex or dvi latex
	\usepackage[T1]{fontenc}
	\usepackage[utf8]{inputenc}
\else
	\usepackage{fontspec} % XeLaTeX or LuaLaTeX
\fi

% --- LWARP IS LOADED NEXT ---

\usepackage[
% HomeHTMLFilename=index, % Filename of the homepage.
% HTMLFilename={node-}, % Filename prefix of other pages.
% IndexLanguage=english, % Language for xindy index, glossary.
% latexmk, % Use latexmk to compile.
% OSWindows, % Force Windows. (Usually automatic.)
mathjax, % Use MathJax to display math.
]{lwarp}

% \boolfalse{FileSectionNames} % If false, numbers the files.

% --- LOAD PDFLATEX MATH FONTS HERE ---

% --- OTHER PACKAGES ARE LOADED AFTER LWARP ---

\usepackage{standalone}

\usepackage{tikz}
\usepackage{amsthm}
\usepackage{amsmath}
\usepackage{amssymb}
\usepackage{algorithm}
\usepackage{algpseudocode}

\usetikzlibrary{graphs, positioning, shapes.geometric}

\usepackage{makeidx} \makeindex
\usepackage{xcolor}               % (Demonstration purposes only.)
\usepackage{hyperref,cleveref}    % LOAD THESE LAST!

% Declare theorem environments for amsthm
\newtheorem{axiom}{axiom}
\newtheorem{lemma}{Lemma}
\newtheorem{theorem}{Theorem}
\newtheorem{example}{Example}
\newtheorem{definition}{Definition}

% --- LATEX AND HTML CUSTOMIZATION ---
\title{Notes on Everything}
\author{Anthony}
\date{\today}

\setcounter{tocdepth}{1}
\setcounter{secnumdepth}{-1}
\setcounter{FileDepth}{1}
\booltrue{CombineHigherDepths}
\setcounter{SideTOCDepth}{2}

% Renew commands

\renewcommand*\contentsname{Subjects}
\renewcommand\sidetocname{Subjects}

% HTML Directives
\HTMLTitle{Notes on Everything}
\HTMLAuthor{Anthony}
\HTMLLanguage{en-US}
\HTMLDescription{Personal notes on Mathematics and Computer Science}
\HTMLPageBottom{\LinkHome}

% Styling
\CSSFilename{lwarp.css}

\begin{document}

\maketitle % Or titlepage/titlingpage environment.
% An article abstract would go here.

Notes on Mathematics and Computer Science.

\tableofcontents % MUST BE BEFORE THE FIRST SECTION BREAK!

\listoffigures

% Save this as tutorial.tex for the lwarp package tutorial.
\documentclass{book}
\usepackage{iftex}

% --- LOAD FONT SELECTION AND ENCODING BEFORE LOADING LWARP ---
\ifPDFTeX
	\usepackage{lmodern} % pdflatex or dvi latex
	\usepackage[T1]{fontenc}
	\usepackage[utf8]{inputenc}
\else
	\usepackage{fontspec} % XeLaTeX or LuaLaTeX
\fi

% --- LWARP IS LOADED NEXT ---

\usepackage[
% HomeHTMLFilename=index, % Filename of the homepage.
% HTMLFilename={node-}, % Filename prefix of other pages.
% IndexLanguage=english, % Language for xindy index, glossary.
% latexmk, % Use latexmk to compile.
% OSWindows, % Force Windows. (Usually automatic.)
mathjax, % Use MathJax to display math.
]{lwarp}

% \boolfalse{FileSectionNames} % If false, numbers the files.

% --- LOAD PDFLATEX MATH FONTS HERE ---

% --- OTHER PACKAGES ARE LOADED AFTER LWARP ---

\usepackage{standalone}

\usepackage{tikz}
\usepackage{amsthm}
\usepackage{amsmath}
\usepackage{amssymb}
\usepackage{algorithm}
\usepackage{algpseudocode}

\usetikzlibrary{graphs, positioning, shapes.geometric}

\usepackage{makeidx} \makeindex
\usepackage{xcolor}               % (Demonstration purposes only.)
\usepackage{hyperref,cleveref}    % LOAD THESE LAST!

% Declare theorem environments for amsthm
\newtheorem{axiom}{axiom}
\newtheorem{lemma}{Lemma}
\newtheorem{theorem}{Theorem}
\newtheorem{example}{Example}
\newtheorem{definition}{Definition}

% --- LATEX AND HTML CUSTOMIZATION ---
\title{Notes on Everything}
\author{Anthony}
\date{\today}

\setcounter{tocdepth}{1}
\setcounter{secnumdepth}{-1}
\setcounter{FileDepth}{1}
\booltrue{CombineHigherDepths}
\setcounter{SideTOCDepth}{2}

% Renew commands

\renewcommand*\contentsname{Subjects}
\renewcommand\sidetocname{Subjects}

% HTML Directives
\HTMLTitle{Notes on Everything}
\HTMLAuthor{Anthony}
\HTMLLanguage{en-US}
\HTMLDescription{Personal notes on Mathematics and Computer Science}
\HTMLPageBottom{\LinkHome}

% Styling
\CSSFilename{lwarp.css}

\begin{document}

\maketitle % Or titlepage/titlingpage environment.
% An article abstract would go here.

Notes on Mathematics and Computer Science.

\tableofcontents % MUST BE BEFORE THE FIRST SECTION BREAK!

\listoffigures

\input{./sums/index}
\input{./algorithms/index}
\input{./combinatorics/index}
\input{./probability/index}
\input{./foundations/index}

\ForceHTMLPage 	% HTML index will be on its own page.
\ForceHTMLTOC 	% HTML index will have its own toc entry.
\printindex
\end{document}

% Save this as tutorial.tex for the lwarp package tutorial.
\documentclass{book}
\usepackage{iftex}

% --- LOAD FONT SELECTION AND ENCODING BEFORE LOADING LWARP ---
\ifPDFTeX
	\usepackage{lmodern} % pdflatex or dvi latex
	\usepackage[T1]{fontenc}
	\usepackage[utf8]{inputenc}
\else
	\usepackage{fontspec} % XeLaTeX or LuaLaTeX
\fi

% --- LWARP IS LOADED NEXT ---

\usepackage[
% HomeHTMLFilename=index, % Filename of the homepage.
% HTMLFilename={node-}, % Filename prefix of other pages.
% IndexLanguage=english, % Language for xindy index, glossary.
% latexmk, % Use latexmk to compile.
% OSWindows, % Force Windows. (Usually automatic.)
mathjax, % Use MathJax to display math.
]{lwarp}

% \boolfalse{FileSectionNames} % If false, numbers the files.

% --- LOAD PDFLATEX MATH FONTS HERE ---

% --- OTHER PACKAGES ARE LOADED AFTER LWARP ---

\usepackage{standalone}

\usepackage{tikz}
\usepackage{amsthm}
\usepackage{amsmath}
\usepackage{amssymb}
\usepackage{algorithm}
\usepackage{algpseudocode}

\usetikzlibrary{graphs, positioning, shapes.geometric}

\usepackage{makeidx} \makeindex
\usepackage{xcolor}               % (Demonstration purposes only.)
\usepackage{hyperref,cleveref}    % LOAD THESE LAST!

% Declare theorem environments for amsthm
\newtheorem{axiom}{axiom}
\newtheorem{lemma}{Lemma}
\newtheorem{theorem}{Theorem}
\newtheorem{example}{Example}
\newtheorem{definition}{Definition}

% --- LATEX AND HTML CUSTOMIZATION ---
\title{Notes on Everything}
\author{Anthony}
\date{\today}

\setcounter{tocdepth}{1}
\setcounter{secnumdepth}{-1}
\setcounter{FileDepth}{1}
\booltrue{CombineHigherDepths}
\setcounter{SideTOCDepth}{2}

% Renew commands

\renewcommand*\contentsname{Subjects}
\renewcommand\sidetocname{Subjects}

% HTML Directives
\HTMLTitle{Notes on Everything}
\HTMLAuthor{Anthony}
\HTMLLanguage{en-US}
\HTMLDescription{Personal notes on Mathematics and Computer Science}
\HTMLPageBottom{\LinkHome}

% Styling
\CSSFilename{lwarp.css}

\begin{document}

\maketitle % Or titlepage/titlingpage environment.
% An article abstract would go here.

Notes on Mathematics and Computer Science.

\tableofcontents % MUST BE BEFORE THE FIRST SECTION BREAK!

\listoffigures

\input{./sums/index}
\input{./algorithms/index}
\input{./combinatorics/index}
\input{./probability/index}
\input{./foundations/index}

\ForceHTMLPage 	% HTML index will be on its own page.
\ForceHTMLTOC 	% HTML index will have its own toc entry.
\printindex
\end{document}

% Save this as tutorial.tex for the lwarp package tutorial.
\documentclass{book}
\usepackage{iftex}

% --- LOAD FONT SELECTION AND ENCODING BEFORE LOADING LWARP ---
\ifPDFTeX
	\usepackage{lmodern} % pdflatex or dvi latex
	\usepackage[T1]{fontenc}
	\usepackage[utf8]{inputenc}
\else
	\usepackage{fontspec} % XeLaTeX or LuaLaTeX
\fi

% --- LWARP IS LOADED NEXT ---

\usepackage[
% HomeHTMLFilename=index, % Filename of the homepage.
% HTMLFilename={node-}, % Filename prefix of other pages.
% IndexLanguage=english, % Language for xindy index, glossary.
% latexmk, % Use latexmk to compile.
% OSWindows, % Force Windows. (Usually automatic.)
mathjax, % Use MathJax to display math.
]{lwarp}

% \boolfalse{FileSectionNames} % If false, numbers the files.

% --- LOAD PDFLATEX MATH FONTS HERE ---

% --- OTHER PACKAGES ARE LOADED AFTER LWARP ---

\usepackage{standalone}

\usepackage{tikz}
\usepackage{amsthm}
\usepackage{amsmath}
\usepackage{amssymb}
\usepackage{algorithm}
\usepackage{algpseudocode}

\usetikzlibrary{graphs, positioning, shapes.geometric}

\usepackage{makeidx} \makeindex
\usepackage{xcolor}               % (Demonstration purposes only.)
\usepackage{hyperref,cleveref}    % LOAD THESE LAST!

% Declare theorem environments for amsthm
\newtheorem{axiom}{axiom}
\newtheorem{lemma}{Lemma}
\newtheorem{theorem}{Theorem}
\newtheorem{example}{Example}
\newtheorem{definition}{Definition}

% --- LATEX AND HTML CUSTOMIZATION ---
\title{Notes on Everything}
\author{Anthony}
\date{\today}

\setcounter{tocdepth}{1}
\setcounter{secnumdepth}{-1}
\setcounter{FileDepth}{1}
\booltrue{CombineHigherDepths}
\setcounter{SideTOCDepth}{2}

% Renew commands

\renewcommand*\contentsname{Subjects}
\renewcommand\sidetocname{Subjects}

% HTML Directives
\HTMLTitle{Notes on Everything}
\HTMLAuthor{Anthony}
\HTMLLanguage{en-US}
\HTMLDescription{Personal notes on Mathematics and Computer Science}
\HTMLPageBottom{\LinkHome}

% Styling
\CSSFilename{lwarp.css}

\begin{document}

\maketitle % Or titlepage/titlingpage environment.
% An article abstract would go here.

Notes on Mathematics and Computer Science.

\tableofcontents % MUST BE BEFORE THE FIRST SECTION BREAK!

\listoffigures

\input{./sums/index}
\input{./algorithms/index}
\input{./combinatorics/index}
\input{./probability/index}
\input{./foundations/index}

\ForceHTMLPage 	% HTML index will be on its own page.
\ForceHTMLTOC 	% HTML index will have its own toc entry.
\printindex
\end{document}

% Save this as tutorial.tex for the lwarp package tutorial.
\documentclass{book}
\usepackage{iftex}

% --- LOAD FONT SELECTION AND ENCODING BEFORE LOADING LWARP ---
\ifPDFTeX
	\usepackage{lmodern} % pdflatex or dvi latex
	\usepackage[T1]{fontenc}
	\usepackage[utf8]{inputenc}
\else
	\usepackage{fontspec} % XeLaTeX or LuaLaTeX
\fi

% --- LWARP IS LOADED NEXT ---

\usepackage[
% HomeHTMLFilename=index, % Filename of the homepage.
% HTMLFilename={node-}, % Filename prefix of other pages.
% IndexLanguage=english, % Language for xindy index, glossary.
% latexmk, % Use latexmk to compile.
% OSWindows, % Force Windows. (Usually automatic.)
mathjax, % Use MathJax to display math.
]{lwarp}

% \boolfalse{FileSectionNames} % If false, numbers the files.

% --- LOAD PDFLATEX MATH FONTS HERE ---

% --- OTHER PACKAGES ARE LOADED AFTER LWARP ---

\usepackage{standalone}

\usepackage{tikz}
\usepackage{amsthm}
\usepackage{amsmath}
\usepackage{amssymb}
\usepackage{algorithm}
\usepackage{algpseudocode}

\usetikzlibrary{graphs, positioning, shapes.geometric}

\usepackage{makeidx} \makeindex
\usepackage{xcolor}               % (Demonstration purposes only.)
\usepackage{hyperref,cleveref}    % LOAD THESE LAST!

% Declare theorem environments for amsthm
\newtheorem{axiom}{axiom}
\newtheorem{lemma}{Lemma}
\newtheorem{theorem}{Theorem}
\newtheorem{example}{Example}
\newtheorem{definition}{Definition}

% --- LATEX AND HTML CUSTOMIZATION ---
\title{Notes on Everything}
\author{Anthony}
\date{\today}

\setcounter{tocdepth}{1}
\setcounter{secnumdepth}{-1}
\setcounter{FileDepth}{1}
\booltrue{CombineHigherDepths}
\setcounter{SideTOCDepth}{2}

% Renew commands

\renewcommand*\contentsname{Subjects}
\renewcommand\sidetocname{Subjects}

% HTML Directives
\HTMLTitle{Notes on Everything}
\HTMLAuthor{Anthony}
\HTMLLanguage{en-US}
\HTMLDescription{Personal notes on Mathematics and Computer Science}
\HTMLPageBottom{\LinkHome}

% Styling
\CSSFilename{lwarp.css}

\begin{document}

\maketitle % Or titlepage/titlingpage environment.
% An article abstract would go here.

Notes on Mathematics and Computer Science.

\tableofcontents % MUST BE BEFORE THE FIRST SECTION BREAK!

\listoffigures

\input{./sums/index}
\input{./algorithms/index}
\input{./combinatorics/index}
\input{./probability/index}
\input{./foundations/index}

\ForceHTMLPage 	% HTML index will be on its own page.
\ForceHTMLTOC 	% HTML index will have its own toc entry.
\printindex
\end{document}

% Save this as tutorial.tex for the lwarp package tutorial.
\documentclass{book}
\usepackage{iftex}

% --- LOAD FONT SELECTION AND ENCODING BEFORE LOADING LWARP ---
\ifPDFTeX
	\usepackage{lmodern} % pdflatex or dvi latex
	\usepackage[T1]{fontenc}
	\usepackage[utf8]{inputenc}
\else
	\usepackage{fontspec} % XeLaTeX or LuaLaTeX
\fi

% --- LWARP IS LOADED NEXT ---

\usepackage[
% HomeHTMLFilename=index, % Filename of the homepage.
% HTMLFilename={node-}, % Filename prefix of other pages.
% IndexLanguage=english, % Language for xindy index, glossary.
% latexmk, % Use latexmk to compile.
% OSWindows, % Force Windows. (Usually automatic.)
mathjax, % Use MathJax to display math.
]{lwarp}

% \boolfalse{FileSectionNames} % If false, numbers the files.

% --- LOAD PDFLATEX MATH FONTS HERE ---

% --- OTHER PACKAGES ARE LOADED AFTER LWARP ---

\usepackage{standalone}

\usepackage{tikz}
\usepackage{amsthm}
\usepackage{amsmath}
\usepackage{amssymb}
\usepackage{algorithm}
\usepackage{algpseudocode}

\usetikzlibrary{graphs, positioning, shapes.geometric}

\usepackage{makeidx} \makeindex
\usepackage{xcolor}               % (Demonstration purposes only.)
\usepackage{hyperref,cleveref}    % LOAD THESE LAST!

% Declare theorem environments for amsthm
\newtheorem{axiom}{axiom}
\newtheorem{lemma}{Lemma}
\newtheorem{theorem}{Theorem}
\newtheorem{example}{Example}
\newtheorem{definition}{Definition}

% --- LATEX AND HTML CUSTOMIZATION ---
\title{Notes on Everything}
\author{Anthony}
\date{\today}

\setcounter{tocdepth}{1}
\setcounter{secnumdepth}{-1}
\setcounter{FileDepth}{1}
\booltrue{CombineHigherDepths}
\setcounter{SideTOCDepth}{2}

% Renew commands

\renewcommand*\contentsname{Subjects}
\renewcommand\sidetocname{Subjects}

% HTML Directives
\HTMLTitle{Notes on Everything}
\HTMLAuthor{Anthony}
\HTMLLanguage{en-US}
\HTMLDescription{Personal notes on Mathematics and Computer Science}
\HTMLPageBottom{\LinkHome}

% Styling
\CSSFilename{lwarp.css}

\begin{document}

\maketitle % Or titlepage/titlingpage environment.
% An article abstract would go here.

Notes on Mathematics and Computer Science.

\tableofcontents % MUST BE BEFORE THE FIRST SECTION BREAK!

\listoffigures

\input{./sums/index}
\input{./algorithms/index}
\input{./combinatorics/index}
\input{./probability/index}
\input{./foundations/index}

\ForceHTMLPage 	% HTML index will be on its own page.
\ForceHTMLTOC 	% HTML index will have its own toc entry.
\printindex
\end{document}


\ForceHTMLPage 	% HTML index will be on its own page.
\ForceHTMLTOC 	% HTML index will have its own toc entry.
\printindex
\end{document}

% Save this as tutorial.tex for the lwarp package tutorial.
\documentclass{book}
\usepackage{iftex}

% --- LOAD FONT SELECTION AND ENCODING BEFORE LOADING LWARP ---
\ifPDFTeX
	\usepackage{lmodern} % pdflatex or dvi latex
	\usepackage[T1]{fontenc}
	\usepackage[utf8]{inputenc}
\else
	\usepackage{fontspec} % XeLaTeX or LuaLaTeX
\fi

% --- LWARP IS LOADED NEXT ---

\usepackage[
% HomeHTMLFilename=index, % Filename of the homepage.
% HTMLFilename={node-}, % Filename prefix of other pages.
% IndexLanguage=english, % Language for xindy index, glossary.
% latexmk, % Use latexmk to compile.
% OSWindows, % Force Windows. (Usually automatic.)
mathjax, % Use MathJax to display math.
]{lwarp}

% \boolfalse{FileSectionNames} % If false, numbers the files.

% --- LOAD PDFLATEX MATH FONTS HERE ---

% --- OTHER PACKAGES ARE LOADED AFTER LWARP ---

\usepackage{standalone}

\usepackage{tikz}
\usepackage{amsthm}
\usepackage{amsmath}
\usepackage{amssymb}
\usepackage{algorithm}
\usepackage{algpseudocode}

\usetikzlibrary{graphs, positioning, shapes.geometric}

\usepackage{makeidx} \makeindex
\usepackage{xcolor}               % (Demonstration purposes only.)
\usepackage{hyperref,cleveref}    % LOAD THESE LAST!

% Declare theorem environments for amsthm
\newtheorem{axiom}{axiom}
\newtheorem{lemma}{Lemma}
\newtheorem{theorem}{Theorem}
\newtheorem{example}{Example}
\newtheorem{definition}{Definition}

% --- LATEX AND HTML CUSTOMIZATION ---
\title{Notes on Everything}
\author{Anthony}
\date{\today}

\setcounter{tocdepth}{1}
\setcounter{secnumdepth}{-1}
\setcounter{FileDepth}{1}
\booltrue{CombineHigherDepths}
\setcounter{SideTOCDepth}{2}

% Renew commands

\renewcommand*\contentsname{Subjects}
\renewcommand\sidetocname{Subjects}

% HTML Directives
\HTMLTitle{Notes on Everything}
\HTMLAuthor{Anthony}
\HTMLLanguage{en-US}
\HTMLDescription{Personal notes on Mathematics and Computer Science}
\HTMLPageBottom{\LinkHome}

% Styling
\CSSFilename{lwarp.css}

\begin{document}

\maketitle % Or titlepage/titlingpage environment.
% An article abstract would go here.

Notes on Mathematics and Computer Science.

\tableofcontents % MUST BE BEFORE THE FIRST SECTION BREAK!

\listoffigures

% Save this as tutorial.tex for the lwarp package tutorial.
\documentclass{book}
\usepackage{iftex}

% --- LOAD FONT SELECTION AND ENCODING BEFORE LOADING LWARP ---
\ifPDFTeX
	\usepackage{lmodern} % pdflatex or dvi latex
	\usepackage[T1]{fontenc}
	\usepackage[utf8]{inputenc}
\else
	\usepackage{fontspec} % XeLaTeX or LuaLaTeX
\fi

% --- LWARP IS LOADED NEXT ---

\usepackage[
% HomeHTMLFilename=index, % Filename of the homepage.
% HTMLFilename={node-}, % Filename prefix of other pages.
% IndexLanguage=english, % Language for xindy index, glossary.
% latexmk, % Use latexmk to compile.
% OSWindows, % Force Windows. (Usually automatic.)
mathjax, % Use MathJax to display math.
]{lwarp}

% \boolfalse{FileSectionNames} % If false, numbers the files.

% --- LOAD PDFLATEX MATH FONTS HERE ---

% --- OTHER PACKAGES ARE LOADED AFTER LWARP ---

\usepackage{standalone}

\usepackage{tikz}
\usepackage{amsthm}
\usepackage{amsmath}
\usepackage{amssymb}
\usepackage{algorithm}
\usepackage{algpseudocode}

\usetikzlibrary{graphs, positioning, shapes.geometric}

\usepackage{makeidx} \makeindex
\usepackage{xcolor}               % (Demonstration purposes only.)
\usepackage{hyperref,cleveref}    % LOAD THESE LAST!

% Declare theorem environments for amsthm
\newtheorem{axiom}{axiom}
\newtheorem{lemma}{Lemma}
\newtheorem{theorem}{Theorem}
\newtheorem{example}{Example}
\newtheorem{definition}{Definition}

% --- LATEX AND HTML CUSTOMIZATION ---
\title{Notes on Everything}
\author{Anthony}
\date{\today}

\setcounter{tocdepth}{1}
\setcounter{secnumdepth}{-1}
\setcounter{FileDepth}{1}
\booltrue{CombineHigherDepths}
\setcounter{SideTOCDepth}{2}

% Renew commands

\renewcommand*\contentsname{Subjects}
\renewcommand\sidetocname{Subjects}

% HTML Directives
\HTMLTitle{Notes on Everything}
\HTMLAuthor{Anthony}
\HTMLLanguage{en-US}
\HTMLDescription{Personal notes on Mathematics and Computer Science}
\HTMLPageBottom{\LinkHome}

% Styling
\CSSFilename{lwarp.css}

\begin{document}

\maketitle % Or titlepage/titlingpage environment.
% An article abstract would go here.

Notes on Mathematics and Computer Science.

\tableofcontents % MUST BE BEFORE THE FIRST SECTION BREAK!

\listoffigures

\input{./sums/index}
\input{./algorithms/index}
\input{./combinatorics/index}
\input{./probability/index}
\input{./foundations/index}

\ForceHTMLPage 	% HTML index will be on its own page.
\ForceHTMLTOC 	% HTML index will have its own toc entry.
\printindex
\end{document}

% Save this as tutorial.tex for the lwarp package tutorial.
\documentclass{book}
\usepackage{iftex}

% --- LOAD FONT SELECTION AND ENCODING BEFORE LOADING LWARP ---
\ifPDFTeX
	\usepackage{lmodern} % pdflatex or dvi latex
	\usepackage[T1]{fontenc}
	\usepackage[utf8]{inputenc}
\else
	\usepackage{fontspec} % XeLaTeX or LuaLaTeX
\fi

% --- LWARP IS LOADED NEXT ---

\usepackage[
% HomeHTMLFilename=index, % Filename of the homepage.
% HTMLFilename={node-}, % Filename prefix of other pages.
% IndexLanguage=english, % Language for xindy index, glossary.
% latexmk, % Use latexmk to compile.
% OSWindows, % Force Windows. (Usually automatic.)
mathjax, % Use MathJax to display math.
]{lwarp}

% \boolfalse{FileSectionNames} % If false, numbers the files.

% --- LOAD PDFLATEX MATH FONTS HERE ---

% --- OTHER PACKAGES ARE LOADED AFTER LWARP ---

\usepackage{standalone}

\usepackage{tikz}
\usepackage{amsthm}
\usepackage{amsmath}
\usepackage{amssymb}
\usepackage{algorithm}
\usepackage{algpseudocode}

\usetikzlibrary{graphs, positioning, shapes.geometric}

\usepackage{makeidx} \makeindex
\usepackage{xcolor}               % (Demonstration purposes only.)
\usepackage{hyperref,cleveref}    % LOAD THESE LAST!

% Declare theorem environments for amsthm
\newtheorem{axiom}{axiom}
\newtheorem{lemma}{Lemma}
\newtheorem{theorem}{Theorem}
\newtheorem{example}{Example}
\newtheorem{definition}{Definition}

% --- LATEX AND HTML CUSTOMIZATION ---
\title{Notes on Everything}
\author{Anthony}
\date{\today}

\setcounter{tocdepth}{1}
\setcounter{secnumdepth}{-1}
\setcounter{FileDepth}{1}
\booltrue{CombineHigherDepths}
\setcounter{SideTOCDepth}{2}

% Renew commands

\renewcommand*\contentsname{Subjects}
\renewcommand\sidetocname{Subjects}

% HTML Directives
\HTMLTitle{Notes on Everything}
\HTMLAuthor{Anthony}
\HTMLLanguage{en-US}
\HTMLDescription{Personal notes on Mathematics and Computer Science}
\HTMLPageBottom{\LinkHome}

% Styling
\CSSFilename{lwarp.css}

\begin{document}

\maketitle % Or titlepage/titlingpage environment.
% An article abstract would go here.

Notes on Mathematics and Computer Science.

\tableofcontents % MUST BE BEFORE THE FIRST SECTION BREAK!

\listoffigures

\input{./sums/index}
\input{./algorithms/index}
\input{./combinatorics/index}
\input{./probability/index}
\input{./foundations/index}

\ForceHTMLPage 	% HTML index will be on its own page.
\ForceHTMLTOC 	% HTML index will have its own toc entry.
\printindex
\end{document}

% Save this as tutorial.tex for the lwarp package tutorial.
\documentclass{book}
\usepackage{iftex}

% --- LOAD FONT SELECTION AND ENCODING BEFORE LOADING LWARP ---
\ifPDFTeX
	\usepackage{lmodern} % pdflatex or dvi latex
	\usepackage[T1]{fontenc}
	\usepackage[utf8]{inputenc}
\else
	\usepackage{fontspec} % XeLaTeX or LuaLaTeX
\fi

% --- LWARP IS LOADED NEXT ---

\usepackage[
% HomeHTMLFilename=index, % Filename of the homepage.
% HTMLFilename={node-}, % Filename prefix of other pages.
% IndexLanguage=english, % Language for xindy index, glossary.
% latexmk, % Use latexmk to compile.
% OSWindows, % Force Windows. (Usually automatic.)
mathjax, % Use MathJax to display math.
]{lwarp}

% \boolfalse{FileSectionNames} % If false, numbers the files.

% --- LOAD PDFLATEX MATH FONTS HERE ---

% --- OTHER PACKAGES ARE LOADED AFTER LWARP ---

\usepackage{standalone}

\usepackage{tikz}
\usepackage{amsthm}
\usepackage{amsmath}
\usepackage{amssymb}
\usepackage{algorithm}
\usepackage{algpseudocode}

\usetikzlibrary{graphs, positioning, shapes.geometric}

\usepackage{makeidx} \makeindex
\usepackage{xcolor}               % (Demonstration purposes only.)
\usepackage{hyperref,cleveref}    % LOAD THESE LAST!

% Declare theorem environments for amsthm
\newtheorem{axiom}{axiom}
\newtheorem{lemma}{Lemma}
\newtheorem{theorem}{Theorem}
\newtheorem{example}{Example}
\newtheorem{definition}{Definition}

% --- LATEX AND HTML CUSTOMIZATION ---
\title{Notes on Everything}
\author{Anthony}
\date{\today}

\setcounter{tocdepth}{1}
\setcounter{secnumdepth}{-1}
\setcounter{FileDepth}{1}
\booltrue{CombineHigherDepths}
\setcounter{SideTOCDepth}{2}

% Renew commands

\renewcommand*\contentsname{Subjects}
\renewcommand\sidetocname{Subjects}

% HTML Directives
\HTMLTitle{Notes on Everything}
\HTMLAuthor{Anthony}
\HTMLLanguage{en-US}
\HTMLDescription{Personal notes on Mathematics and Computer Science}
\HTMLPageBottom{\LinkHome}

% Styling
\CSSFilename{lwarp.css}

\begin{document}

\maketitle % Or titlepage/titlingpage environment.
% An article abstract would go here.

Notes on Mathematics and Computer Science.

\tableofcontents % MUST BE BEFORE THE FIRST SECTION BREAK!

\listoffigures

\input{./sums/index}
\input{./algorithms/index}
\input{./combinatorics/index}
\input{./probability/index}
\input{./foundations/index}

\ForceHTMLPage 	% HTML index will be on its own page.
\ForceHTMLTOC 	% HTML index will have its own toc entry.
\printindex
\end{document}

% Save this as tutorial.tex for the lwarp package tutorial.
\documentclass{book}
\usepackage{iftex}

% --- LOAD FONT SELECTION AND ENCODING BEFORE LOADING LWARP ---
\ifPDFTeX
	\usepackage{lmodern} % pdflatex or dvi latex
	\usepackage[T1]{fontenc}
	\usepackage[utf8]{inputenc}
\else
	\usepackage{fontspec} % XeLaTeX or LuaLaTeX
\fi

% --- LWARP IS LOADED NEXT ---

\usepackage[
% HomeHTMLFilename=index, % Filename of the homepage.
% HTMLFilename={node-}, % Filename prefix of other pages.
% IndexLanguage=english, % Language for xindy index, glossary.
% latexmk, % Use latexmk to compile.
% OSWindows, % Force Windows. (Usually automatic.)
mathjax, % Use MathJax to display math.
]{lwarp}

% \boolfalse{FileSectionNames} % If false, numbers the files.

% --- LOAD PDFLATEX MATH FONTS HERE ---

% --- OTHER PACKAGES ARE LOADED AFTER LWARP ---

\usepackage{standalone}

\usepackage{tikz}
\usepackage{amsthm}
\usepackage{amsmath}
\usepackage{amssymb}
\usepackage{algorithm}
\usepackage{algpseudocode}

\usetikzlibrary{graphs, positioning, shapes.geometric}

\usepackage{makeidx} \makeindex
\usepackage{xcolor}               % (Demonstration purposes only.)
\usepackage{hyperref,cleveref}    % LOAD THESE LAST!

% Declare theorem environments for amsthm
\newtheorem{axiom}{axiom}
\newtheorem{lemma}{Lemma}
\newtheorem{theorem}{Theorem}
\newtheorem{example}{Example}
\newtheorem{definition}{Definition}

% --- LATEX AND HTML CUSTOMIZATION ---
\title{Notes on Everything}
\author{Anthony}
\date{\today}

\setcounter{tocdepth}{1}
\setcounter{secnumdepth}{-1}
\setcounter{FileDepth}{1}
\booltrue{CombineHigherDepths}
\setcounter{SideTOCDepth}{2}

% Renew commands

\renewcommand*\contentsname{Subjects}
\renewcommand\sidetocname{Subjects}

% HTML Directives
\HTMLTitle{Notes on Everything}
\HTMLAuthor{Anthony}
\HTMLLanguage{en-US}
\HTMLDescription{Personal notes on Mathematics and Computer Science}
\HTMLPageBottom{\LinkHome}

% Styling
\CSSFilename{lwarp.css}

\begin{document}

\maketitle % Or titlepage/titlingpage environment.
% An article abstract would go here.

Notes on Mathematics and Computer Science.

\tableofcontents % MUST BE BEFORE THE FIRST SECTION BREAK!

\listoffigures

\input{./sums/index}
\input{./algorithms/index}
\input{./combinatorics/index}
\input{./probability/index}
\input{./foundations/index}

\ForceHTMLPage 	% HTML index will be on its own page.
\ForceHTMLTOC 	% HTML index will have its own toc entry.
\printindex
\end{document}

% Save this as tutorial.tex for the lwarp package tutorial.
\documentclass{book}
\usepackage{iftex}

% --- LOAD FONT SELECTION AND ENCODING BEFORE LOADING LWARP ---
\ifPDFTeX
	\usepackage{lmodern} % pdflatex or dvi latex
	\usepackage[T1]{fontenc}
	\usepackage[utf8]{inputenc}
\else
	\usepackage{fontspec} % XeLaTeX or LuaLaTeX
\fi

% --- LWARP IS LOADED NEXT ---

\usepackage[
% HomeHTMLFilename=index, % Filename of the homepage.
% HTMLFilename={node-}, % Filename prefix of other pages.
% IndexLanguage=english, % Language for xindy index, glossary.
% latexmk, % Use latexmk to compile.
% OSWindows, % Force Windows. (Usually automatic.)
mathjax, % Use MathJax to display math.
]{lwarp}

% \boolfalse{FileSectionNames} % If false, numbers the files.

% --- LOAD PDFLATEX MATH FONTS HERE ---

% --- OTHER PACKAGES ARE LOADED AFTER LWARP ---

\usepackage{standalone}

\usepackage{tikz}
\usepackage{amsthm}
\usepackage{amsmath}
\usepackage{amssymb}
\usepackage{algorithm}
\usepackage{algpseudocode}

\usetikzlibrary{graphs, positioning, shapes.geometric}

\usepackage{makeidx} \makeindex
\usepackage{xcolor}               % (Demonstration purposes only.)
\usepackage{hyperref,cleveref}    % LOAD THESE LAST!

% Declare theorem environments for amsthm
\newtheorem{axiom}{axiom}
\newtheorem{lemma}{Lemma}
\newtheorem{theorem}{Theorem}
\newtheorem{example}{Example}
\newtheorem{definition}{Definition}

% --- LATEX AND HTML CUSTOMIZATION ---
\title{Notes on Everything}
\author{Anthony}
\date{\today}

\setcounter{tocdepth}{1}
\setcounter{secnumdepth}{-1}
\setcounter{FileDepth}{1}
\booltrue{CombineHigherDepths}
\setcounter{SideTOCDepth}{2}

% Renew commands

\renewcommand*\contentsname{Subjects}
\renewcommand\sidetocname{Subjects}

% HTML Directives
\HTMLTitle{Notes on Everything}
\HTMLAuthor{Anthony}
\HTMLLanguage{en-US}
\HTMLDescription{Personal notes on Mathematics and Computer Science}
\HTMLPageBottom{\LinkHome}

% Styling
\CSSFilename{lwarp.css}

\begin{document}

\maketitle % Or titlepage/titlingpage environment.
% An article abstract would go here.

Notes on Mathematics and Computer Science.

\tableofcontents % MUST BE BEFORE THE FIRST SECTION BREAK!

\listoffigures

\input{./sums/index}
\input{./algorithms/index}
\input{./combinatorics/index}
\input{./probability/index}
\input{./foundations/index}

\ForceHTMLPage 	% HTML index will be on its own page.
\ForceHTMLTOC 	% HTML index will have its own toc entry.
\printindex
\end{document}


\ForceHTMLPage 	% HTML index will be on its own page.
\ForceHTMLTOC 	% HTML index will have its own toc entry.
\printindex
\end{document}

% Save this as tutorial.tex for the lwarp package tutorial.
\documentclass{book}
\usepackage{iftex}

% --- LOAD FONT SELECTION AND ENCODING BEFORE LOADING LWARP ---
\ifPDFTeX
	\usepackage{lmodern} % pdflatex or dvi latex
	\usepackage[T1]{fontenc}
	\usepackage[utf8]{inputenc}
\else
	\usepackage{fontspec} % XeLaTeX or LuaLaTeX
\fi

% --- LWARP IS LOADED NEXT ---

\usepackage[
% HomeHTMLFilename=index, % Filename of the homepage.
% HTMLFilename={node-}, % Filename prefix of other pages.
% IndexLanguage=english, % Language for xindy index, glossary.
% latexmk, % Use latexmk to compile.
% OSWindows, % Force Windows. (Usually automatic.)
mathjax, % Use MathJax to display math.
]{lwarp}

% \boolfalse{FileSectionNames} % If false, numbers the files.

% --- LOAD PDFLATEX MATH FONTS HERE ---

% --- OTHER PACKAGES ARE LOADED AFTER LWARP ---

\usepackage{standalone}

\usepackage{tikz}
\usepackage{amsthm}
\usepackage{amsmath}
\usepackage{amssymb}
\usepackage{algorithm}
\usepackage{algpseudocode}

\usetikzlibrary{graphs, positioning, shapes.geometric}

\usepackage{makeidx} \makeindex
\usepackage{xcolor}               % (Demonstration purposes only.)
\usepackage{hyperref,cleveref}    % LOAD THESE LAST!

% Declare theorem environments for amsthm
\newtheorem{axiom}{axiom}
\newtheorem{lemma}{Lemma}
\newtheorem{theorem}{Theorem}
\newtheorem{example}{Example}
\newtheorem{definition}{Definition}

% --- LATEX AND HTML CUSTOMIZATION ---
\title{Notes on Everything}
\author{Anthony}
\date{\today}

\setcounter{tocdepth}{1}
\setcounter{secnumdepth}{-1}
\setcounter{FileDepth}{1}
\booltrue{CombineHigherDepths}
\setcounter{SideTOCDepth}{2}

% Renew commands

\renewcommand*\contentsname{Subjects}
\renewcommand\sidetocname{Subjects}

% HTML Directives
\HTMLTitle{Notes on Everything}
\HTMLAuthor{Anthony}
\HTMLLanguage{en-US}
\HTMLDescription{Personal notes on Mathematics and Computer Science}
\HTMLPageBottom{\LinkHome}

% Styling
\CSSFilename{lwarp.css}

\begin{document}

\maketitle % Or titlepage/titlingpage environment.
% An article abstract would go here.

Notes on Mathematics and Computer Science.

\tableofcontents % MUST BE BEFORE THE FIRST SECTION BREAK!

\listoffigures

% Save this as tutorial.tex for the lwarp package tutorial.
\documentclass{book}
\usepackage{iftex}

% --- LOAD FONT SELECTION AND ENCODING BEFORE LOADING LWARP ---
\ifPDFTeX
	\usepackage{lmodern} % pdflatex or dvi latex
	\usepackage[T1]{fontenc}
	\usepackage[utf8]{inputenc}
\else
	\usepackage{fontspec} % XeLaTeX or LuaLaTeX
\fi

% --- LWARP IS LOADED NEXT ---

\usepackage[
% HomeHTMLFilename=index, % Filename of the homepage.
% HTMLFilename={node-}, % Filename prefix of other pages.
% IndexLanguage=english, % Language for xindy index, glossary.
% latexmk, % Use latexmk to compile.
% OSWindows, % Force Windows. (Usually automatic.)
mathjax, % Use MathJax to display math.
]{lwarp}

% \boolfalse{FileSectionNames} % If false, numbers the files.

% --- LOAD PDFLATEX MATH FONTS HERE ---

% --- OTHER PACKAGES ARE LOADED AFTER LWARP ---

\usepackage{standalone}

\usepackage{tikz}
\usepackage{amsthm}
\usepackage{amsmath}
\usepackage{amssymb}
\usepackage{algorithm}
\usepackage{algpseudocode}

\usetikzlibrary{graphs, positioning, shapes.geometric}

\usepackage{makeidx} \makeindex
\usepackage{xcolor}               % (Demonstration purposes only.)
\usepackage{hyperref,cleveref}    % LOAD THESE LAST!

% Declare theorem environments for amsthm
\newtheorem{axiom}{axiom}
\newtheorem{lemma}{Lemma}
\newtheorem{theorem}{Theorem}
\newtheorem{example}{Example}
\newtheorem{definition}{Definition}

% --- LATEX AND HTML CUSTOMIZATION ---
\title{Notes on Everything}
\author{Anthony}
\date{\today}

\setcounter{tocdepth}{1}
\setcounter{secnumdepth}{-1}
\setcounter{FileDepth}{1}
\booltrue{CombineHigherDepths}
\setcounter{SideTOCDepth}{2}

% Renew commands

\renewcommand*\contentsname{Subjects}
\renewcommand\sidetocname{Subjects}

% HTML Directives
\HTMLTitle{Notes on Everything}
\HTMLAuthor{Anthony}
\HTMLLanguage{en-US}
\HTMLDescription{Personal notes on Mathematics and Computer Science}
\HTMLPageBottom{\LinkHome}

% Styling
\CSSFilename{lwarp.css}

\begin{document}

\maketitle % Or titlepage/titlingpage environment.
% An article abstract would go here.

Notes on Mathematics and Computer Science.

\tableofcontents % MUST BE BEFORE THE FIRST SECTION BREAK!

\listoffigures

\input{./sums/index}
\input{./algorithms/index}
\input{./combinatorics/index}
\input{./probability/index}
\input{./foundations/index}

\ForceHTMLPage 	% HTML index will be on its own page.
\ForceHTMLTOC 	% HTML index will have its own toc entry.
\printindex
\end{document}

% Save this as tutorial.tex for the lwarp package tutorial.
\documentclass{book}
\usepackage{iftex}

% --- LOAD FONT SELECTION AND ENCODING BEFORE LOADING LWARP ---
\ifPDFTeX
	\usepackage{lmodern} % pdflatex or dvi latex
	\usepackage[T1]{fontenc}
	\usepackage[utf8]{inputenc}
\else
	\usepackage{fontspec} % XeLaTeX or LuaLaTeX
\fi

% --- LWARP IS LOADED NEXT ---

\usepackage[
% HomeHTMLFilename=index, % Filename of the homepage.
% HTMLFilename={node-}, % Filename prefix of other pages.
% IndexLanguage=english, % Language for xindy index, glossary.
% latexmk, % Use latexmk to compile.
% OSWindows, % Force Windows. (Usually automatic.)
mathjax, % Use MathJax to display math.
]{lwarp}

% \boolfalse{FileSectionNames} % If false, numbers the files.

% --- LOAD PDFLATEX MATH FONTS HERE ---

% --- OTHER PACKAGES ARE LOADED AFTER LWARP ---

\usepackage{standalone}

\usepackage{tikz}
\usepackage{amsthm}
\usepackage{amsmath}
\usepackage{amssymb}
\usepackage{algorithm}
\usepackage{algpseudocode}

\usetikzlibrary{graphs, positioning, shapes.geometric}

\usepackage{makeidx} \makeindex
\usepackage{xcolor}               % (Demonstration purposes only.)
\usepackage{hyperref,cleveref}    % LOAD THESE LAST!

% Declare theorem environments for amsthm
\newtheorem{axiom}{axiom}
\newtheorem{lemma}{Lemma}
\newtheorem{theorem}{Theorem}
\newtheorem{example}{Example}
\newtheorem{definition}{Definition}

% --- LATEX AND HTML CUSTOMIZATION ---
\title{Notes on Everything}
\author{Anthony}
\date{\today}

\setcounter{tocdepth}{1}
\setcounter{secnumdepth}{-1}
\setcounter{FileDepth}{1}
\booltrue{CombineHigherDepths}
\setcounter{SideTOCDepth}{2}

% Renew commands

\renewcommand*\contentsname{Subjects}
\renewcommand\sidetocname{Subjects}

% HTML Directives
\HTMLTitle{Notes on Everything}
\HTMLAuthor{Anthony}
\HTMLLanguage{en-US}
\HTMLDescription{Personal notes on Mathematics and Computer Science}
\HTMLPageBottom{\LinkHome}

% Styling
\CSSFilename{lwarp.css}

\begin{document}

\maketitle % Or titlepage/titlingpage environment.
% An article abstract would go here.

Notes on Mathematics and Computer Science.

\tableofcontents % MUST BE BEFORE THE FIRST SECTION BREAK!

\listoffigures

\input{./sums/index}
\input{./algorithms/index}
\input{./combinatorics/index}
\input{./probability/index}
\input{./foundations/index}

\ForceHTMLPage 	% HTML index will be on its own page.
\ForceHTMLTOC 	% HTML index will have its own toc entry.
\printindex
\end{document}

% Save this as tutorial.tex for the lwarp package tutorial.
\documentclass{book}
\usepackage{iftex}

% --- LOAD FONT SELECTION AND ENCODING BEFORE LOADING LWARP ---
\ifPDFTeX
	\usepackage{lmodern} % pdflatex or dvi latex
	\usepackage[T1]{fontenc}
	\usepackage[utf8]{inputenc}
\else
	\usepackage{fontspec} % XeLaTeX or LuaLaTeX
\fi

% --- LWARP IS LOADED NEXT ---

\usepackage[
% HomeHTMLFilename=index, % Filename of the homepage.
% HTMLFilename={node-}, % Filename prefix of other pages.
% IndexLanguage=english, % Language for xindy index, glossary.
% latexmk, % Use latexmk to compile.
% OSWindows, % Force Windows. (Usually automatic.)
mathjax, % Use MathJax to display math.
]{lwarp}

% \boolfalse{FileSectionNames} % If false, numbers the files.

% --- LOAD PDFLATEX MATH FONTS HERE ---

% --- OTHER PACKAGES ARE LOADED AFTER LWARP ---

\usepackage{standalone}

\usepackage{tikz}
\usepackage{amsthm}
\usepackage{amsmath}
\usepackage{amssymb}
\usepackage{algorithm}
\usepackage{algpseudocode}

\usetikzlibrary{graphs, positioning, shapes.geometric}

\usepackage{makeidx} \makeindex
\usepackage{xcolor}               % (Demonstration purposes only.)
\usepackage{hyperref,cleveref}    % LOAD THESE LAST!

% Declare theorem environments for amsthm
\newtheorem{axiom}{axiom}
\newtheorem{lemma}{Lemma}
\newtheorem{theorem}{Theorem}
\newtheorem{example}{Example}
\newtheorem{definition}{Definition}

% --- LATEX AND HTML CUSTOMIZATION ---
\title{Notes on Everything}
\author{Anthony}
\date{\today}

\setcounter{tocdepth}{1}
\setcounter{secnumdepth}{-1}
\setcounter{FileDepth}{1}
\booltrue{CombineHigherDepths}
\setcounter{SideTOCDepth}{2}

% Renew commands

\renewcommand*\contentsname{Subjects}
\renewcommand\sidetocname{Subjects}

% HTML Directives
\HTMLTitle{Notes on Everything}
\HTMLAuthor{Anthony}
\HTMLLanguage{en-US}
\HTMLDescription{Personal notes on Mathematics and Computer Science}
\HTMLPageBottom{\LinkHome}

% Styling
\CSSFilename{lwarp.css}

\begin{document}

\maketitle % Or titlepage/titlingpage environment.
% An article abstract would go here.

Notes on Mathematics and Computer Science.

\tableofcontents % MUST BE BEFORE THE FIRST SECTION BREAK!

\listoffigures

\input{./sums/index}
\input{./algorithms/index}
\input{./combinatorics/index}
\input{./probability/index}
\input{./foundations/index}

\ForceHTMLPage 	% HTML index will be on its own page.
\ForceHTMLTOC 	% HTML index will have its own toc entry.
\printindex
\end{document}

% Save this as tutorial.tex for the lwarp package tutorial.
\documentclass{book}
\usepackage{iftex}

% --- LOAD FONT SELECTION AND ENCODING BEFORE LOADING LWARP ---
\ifPDFTeX
	\usepackage{lmodern} % pdflatex or dvi latex
	\usepackage[T1]{fontenc}
	\usepackage[utf8]{inputenc}
\else
	\usepackage{fontspec} % XeLaTeX or LuaLaTeX
\fi

% --- LWARP IS LOADED NEXT ---

\usepackage[
% HomeHTMLFilename=index, % Filename of the homepage.
% HTMLFilename={node-}, % Filename prefix of other pages.
% IndexLanguage=english, % Language for xindy index, glossary.
% latexmk, % Use latexmk to compile.
% OSWindows, % Force Windows. (Usually automatic.)
mathjax, % Use MathJax to display math.
]{lwarp}

% \boolfalse{FileSectionNames} % If false, numbers the files.

% --- LOAD PDFLATEX MATH FONTS HERE ---

% --- OTHER PACKAGES ARE LOADED AFTER LWARP ---

\usepackage{standalone}

\usepackage{tikz}
\usepackage{amsthm}
\usepackage{amsmath}
\usepackage{amssymb}
\usepackage{algorithm}
\usepackage{algpseudocode}

\usetikzlibrary{graphs, positioning, shapes.geometric}

\usepackage{makeidx} \makeindex
\usepackage{xcolor}               % (Demonstration purposes only.)
\usepackage{hyperref,cleveref}    % LOAD THESE LAST!

% Declare theorem environments for amsthm
\newtheorem{axiom}{axiom}
\newtheorem{lemma}{Lemma}
\newtheorem{theorem}{Theorem}
\newtheorem{example}{Example}
\newtheorem{definition}{Definition}

% --- LATEX AND HTML CUSTOMIZATION ---
\title{Notes on Everything}
\author{Anthony}
\date{\today}

\setcounter{tocdepth}{1}
\setcounter{secnumdepth}{-1}
\setcounter{FileDepth}{1}
\booltrue{CombineHigherDepths}
\setcounter{SideTOCDepth}{2}

% Renew commands

\renewcommand*\contentsname{Subjects}
\renewcommand\sidetocname{Subjects}

% HTML Directives
\HTMLTitle{Notes on Everything}
\HTMLAuthor{Anthony}
\HTMLLanguage{en-US}
\HTMLDescription{Personal notes on Mathematics and Computer Science}
\HTMLPageBottom{\LinkHome}

% Styling
\CSSFilename{lwarp.css}

\begin{document}

\maketitle % Or titlepage/titlingpage environment.
% An article abstract would go here.

Notes on Mathematics and Computer Science.

\tableofcontents % MUST BE BEFORE THE FIRST SECTION BREAK!

\listoffigures

\input{./sums/index}
\input{./algorithms/index}
\input{./combinatorics/index}
\input{./probability/index}
\input{./foundations/index}

\ForceHTMLPage 	% HTML index will be on its own page.
\ForceHTMLTOC 	% HTML index will have its own toc entry.
\printindex
\end{document}

% Save this as tutorial.tex for the lwarp package tutorial.
\documentclass{book}
\usepackage{iftex}

% --- LOAD FONT SELECTION AND ENCODING BEFORE LOADING LWARP ---
\ifPDFTeX
	\usepackage{lmodern} % pdflatex or dvi latex
	\usepackage[T1]{fontenc}
	\usepackage[utf8]{inputenc}
\else
	\usepackage{fontspec} % XeLaTeX or LuaLaTeX
\fi

% --- LWARP IS LOADED NEXT ---

\usepackage[
% HomeHTMLFilename=index, % Filename of the homepage.
% HTMLFilename={node-}, % Filename prefix of other pages.
% IndexLanguage=english, % Language for xindy index, glossary.
% latexmk, % Use latexmk to compile.
% OSWindows, % Force Windows. (Usually automatic.)
mathjax, % Use MathJax to display math.
]{lwarp}

% \boolfalse{FileSectionNames} % If false, numbers the files.

% --- LOAD PDFLATEX MATH FONTS HERE ---

% --- OTHER PACKAGES ARE LOADED AFTER LWARP ---

\usepackage{standalone}

\usepackage{tikz}
\usepackage{amsthm}
\usepackage{amsmath}
\usepackage{amssymb}
\usepackage{algorithm}
\usepackage{algpseudocode}

\usetikzlibrary{graphs, positioning, shapes.geometric}

\usepackage{makeidx} \makeindex
\usepackage{xcolor}               % (Demonstration purposes only.)
\usepackage{hyperref,cleveref}    % LOAD THESE LAST!

% Declare theorem environments for amsthm
\newtheorem{axiom}{axiom}
\newtheorem{lemma}{Lemma}
\newtheorem{theorem}{Theorem}
\newtheorem{example}{Example}
\newtheorem{definition}{Definition}

% --- LATEX AND HTML CUSTOMIZATION ---
\title{Notes on Everything}
\author{Anthony}
\date{\today}

\setcounter{tocdepth}{1}
\setcounter{secnumdepth}{-1}
\setcounter{FileDepth}{1}
\booltrue{CombineHigherDepths}
\setcounter{SideTOCDepth}{2}

% Renew commands

\renewcommand*\contentsname{Subjects}
\renewcommand\sidetocname{Subjects}

% HTML Directives
\HTMLTitle{Notes on Everything}
\HTMLAuthor{Anthony}
\HTMLLanguage{en-US}
\HTMLDescription{Personal notes on Mathematics and Computer Science}
\HTMLPageBottom{\LinkHome}

% Styling
\CSSFilename{lwarp.css}

\begin{document}

\maketitle % Or titlepage/titlingpage environment.
% An article abstract would go here.

Notes on Mathematics and Computer Science.

\tableofcontents % MUST BE BEFORE THE FIRST SECTION BREAK!

\listoffigures

\input{./sums/index}
\input{./algorithms/index}
\input{./combinatorics/index}
\input{./probability/index}
\input{./foundations/index}

\ForceHTMLPage 	% HTML index will be on its own page.
\ForceHTMLTOC 	% HTML index will have its own toc entry.
\printindex
\end{document}


\ForceHTMLPage 	% HTML index will be on its own page.
\ForceHTMLTOC 	% HTML index will have its own toc entry.
\printindex
\end{document}

% Save this as tutorial.tex for the lwarp package tutorial.
\documentclass{book}
\usepackage{iftex}

% --- LOAD FONT SELECTION AND ENCODING BEFORE LOADING LWARP ---
\ifPDFTeX
	\usepackage{lmodern} % pdflatex or dvi latex
	\usepackage[T1]{fontenc}
	\usepackage[utf8]{inputenc}
\else
	\usepackage{fontspec} % XeLaTeX or LuaLaTeX
\fi

% --- LWARP IS LOADED NEXT ---

\usepackage[
% HomeHTMLFilename=index, % Filename of the homepage.
% HTMLFilename={node-}, % Filename prefix of other pages.
% IndexLanguage=english, % Language for xindy index, glossary.
% latexmk, % Use latexmk to compile.
% OSWindows, % Force Windows. (Usually automatic.)
mathjax, % Use MathJax to display math.
]{lwarp}

% \boolfalse{FileSectionNames} % If false, numbers the files.

% --- LOAD PDFLATEX MATH FONTS HERE ---

% --- OTHER PACKAGES ARE LOADED AFTER LWARP ---

\usepackage{standalone}

\usepackage{tikz}
\usepackage{amsthm}
\usepackage{amsmath}
\usepackage{amssymb}
\usepackage{algorithm}
\usepackage{algpseudocode}

\usetikzlibrary{graphs, positioning, shapes.geometric}

\usepackage{makeidx} \makeindex
\usepackage{xcolor}               % (Demonstration purposes only.)
\usepackage{hyperref,cleveref}    % LOAD THESE LAST!

% Declare theorem environments for amsthm
\newtheorem{axiom}{axiom}
\newtheorem{lemma}{Lemma}
\newtheorem{theorem}{Theorem}
\newtheorem{example}{Example}
\newtheorem{definition}{Definition}

% --- LATEX AND HTML CUSTOMIZATION ---
\title{Notes on Everything}
\author{Anthony}
\date{\today}

\setcounter{tocdepth}{1}
\setcounter{secnumdepth}{-1}
\setcounter{FileDepth}{1}
\booltrue{CombineHigherDepths}
\setcounter{SideTOCDepth}{2}

% Renew commands

\renewcommand*\contentsname{Subjects}
\renewcommand\sidetocname{Subjects}

% HTML Directives
\HTMLTitle{Notes on Everything}
\HTMLAuthor{Anthony}
\HTMLLanguage{en-US}
\HTMLDescription{Personal notes on Mathematics and Computer Science}
\HTMLPageBottom{\LinkHome}

% Styling
\CSSFilename{lwarp.css}

\begin{document}

\maketitle % Or titlepage/titlingpage environment.
% An article abstract would go here.

Notes on Mathematics and Computer Science.

\tableofcontents % MUST BE BEFORE THE FIRST SECTION BREAK!

\listoffigures

% Save this as tutorial.tex for the lwarp package tutorial.
\documentclass{book}
\usepackage{iftex}

% --- LOAD FONT SELECTION AND ENCODING BEFORE LOADING LWARP ---
\ifPDFTeX
	\usepackage{lmodern} % pdflatex or dvi latex
	\usepackage[T1]{fontenc}
	\usepackage[utf8]{inputenc}
\else
	\usepackage{fontspec} % XeLaTeX or LuaLaTeX
\fi

% --- LWARP IS LOADED NEXT ---

\usepackage[
% HomeHTMLFilename=index, % Filename of the homepage.
% HTMLFilename={node-}, % Filename prefix of other pages.
% IndexLanguage=english, % Language for xindy index, glossary.
% latexmk, % Use latexmk to compile.
% OSWindows, % Force Windows. (Usually automatic.)
mathjax, % Use MathJax to display math.
]{lwarp}

% \boolfalse{FileSectionNames} % If false, numbers the files.

% --- LOAD PDFLATEX MATH FONTS HERE ---

% --- OTHER PACKAGES ARE LOADED AFTER LWARP ---

\usepackage{standalone}

\usepackage{tikz}
\usepackage{amsthm}
\usepackage{amsmath}
\usepackage{amssymb}
\usepackage{algorithm}
\usepackage{algpseudocode}

\usetikzlibrary{graphs, positioning, shapes.geometric}

\usepackage{makeidx} \makeindex
\usepackage{xcolor}               % (Demonstration purposes only.)
\usepackage{hyperref,cleveref}    % LOAD THESE LAST!

% Declare theorem environments for amsthm
\newtheorem{axiom}{axiom}
\newtheorem{lemma}{Lemma}
\newtheorem{theorem}{Theorem}
\newtheorem{example}{Example}
\newtheorem{definition}{Definition}

% --- LATEX AND HTML CUSTOMIZATION ---
\title{Notes on Everything}
\author{Anthony}
\date{\today}

\setcounter{tocdepth}{1}
\setcounter{secnumdepth}{-1}
\setcounter{FileDepth}{1}
\booltrue{CombineHigherDepths}
\setcounter{SideTOCDepth}{2}

% Renew commands

\renewcommand*\contentsname{Subjects}
\renewcommand\sidetocname{Subjects}

% HTML Directives
\HTMLTitle{Notes on Everything}
\HTMLAuthor{Anthony}
\HTMLLanguage{en-US}
\HTMLDescription{Personal notes on Mathematics and Computer Science}
\HTMLPageBottom{\LinkHome}

% Styling
\CSSFilename{lwarp.css}

\begin{document}

\maketitle % Or titlepage/titlingpage environment.
% An article abstract would go here.

Notes on Mathematics and Computer Science.

\tableofcontents % MUST BE BEFORE THE FIRST SECTION BREAK!

\listoffigures

\input{./sums/index}
\input{./algorithms/index}
\input{./combinatorics/index}
\input{./probability/index}
\input{./foundations/index}

\ForceHTMLPage 	% HTML index will be on its own page.
\ForceHTMLTOC 	% HTML index will have its own toc entry.
\printindex
\end{document}

% Save this as tutorial.tex for the lwarp package tutorial.
\documentclass{book}
\usepackage{iftex}

% --- LOAD FONT SELECTION AND ENCODING BEFORE LOADING LWARP ---
\ifPDFTeX
	\usepackage{lmodern} % pdflatex or dvi latex
	\usepackage[T1]{fontenc}
	\usepackage[utf8]{inputenc}
\else
	\usepackage{fontspec} % XeLaTeX or LuaLaTeX
\fi

% --- LWARP IS LOADED NEXT ---

\usepackage[
% HomeHTMLFilename=index, % Filename of the homepage.
% HTMLFilename={node-}, % Filename prefix of other pages.
% IndexLanguage=english, % Language for xindy index, glossary.
% latexmk, % Use latexmk to compile.
% OSWindows, % Force Windows. (Usually automatic.)
mathjax, % Use MathJax to display math.
]{lwarp}

% \boolfalse{FileSectionNames} % If false, numbers the files.

% --- LOAD PDFLATEX MATH FONTS HERE ---

% --- OTHER PACKAGES ARE LOADED AFTER LWARP ---

\usepackage{standalone}

\usepackage{tikz}
\usepackage{amsthm}
\usepackage{amsmath}
\usepackage{amssymb}
\usepackage{algorithm}
\usepackage{algpseudocode}

\usetikzlibrary{graphs, positioning, shapes.geometric}

\usepackage{makeidx} \makeindex
\usepackage{xcolor}               % (Demonstration purposes only.)
\usepackage{hyperref,cleveref}    % LOAD THESE LAST!

% Declare theorem environments for amsthm
\newtheorem{axiom}{axiom}
\newtheorem{lemma}{Lemma}
\newtheorem{theorem}{Theorem}
\newtheorem{example}{Example}
\newtheorem{definition}{Definition}

% --- LATEX AND HTML CUSTOMIZATION ---
\title{Notes on Everything}
\author{Anthony}
\date{\today}

\setcounter{tocdepth}{1}
\setcounter{secnumdepth}{-1}
\setcounter{FileDepth}{1}
\booltrue{CombineHigherDepths}
\setcounter{SideTOCDepth}{2}

% Renew commands

\renewcommand*\contentsname{Subjects}
\renewcommand\sidetocname{Subjects}

% HTML Directives
\HTMLTitle{Notes on Everything}
\HTMLAuthor{Anthony}
\HTMLLanguage{en-US}
\HTMLDescription{Personal notes on Mathematics and Computer Science}
\HTMLPageBottom{\LinkHome}

% Styling
\CSSFilename{lwarp.css}

\begin{document}

\maketitle % Or titlepage/titlingpage environment.
% An article abstract would go here.

Notes on Mathematics and Computer Science.

\tableofcontents % MUST BE BEFORE THE FIRST SECTION BREAK!

\listoffigures

\input{./sums/index}
\input{./algorithms/index}
\input{./combinatorics/index}
\input{./probability/index}
\input{./foundations/index}

\ForceHTMLPage 	% HTML index will be on its own page.
\ForceHTMLTOC 	% HTML index will have its own toc entry.
\printindex
\end{document}

% Save this as tutorial.tex for the lwarp package tutorial.
\documentclass{book}
\usepackage{iftex}

% --- LOAD FONT SELECTION AND ENCODING BEFORE LOADING LWARP ---
\ifPDFTeX
	\usepackage{lmodern} % pdflatex or dvi latex
	\usepackage[T1]{fontenc}
	\usepackage[utf8]{inputenc}
\else
	\usepackage{fontspec} % XeLaTeX or LuaLaTeX
\fi

% --- LWARP IS LOADED NEXT ---

\usepackage[
% HomeHTMLFilename=index, % Filename of the homepage.
% HTMLFilename={node-}, % Filename prefix of other pages.
% IndexLanguage=english, % Language for xindy index, glossary.
% latexmk, % Use latexmk to compile.
% OSWindows, % Force Windows. (Usually automatic.)
mathjax, % Use MathJax to display math.
]{lwarp}

% \boolfalse{FileSectionNames} % If false, numbers the files.

% --- LOAD PDFLATEX MATH FONTS HERE ---

% --- OTHER PACKAGES ARE LOADED AFTER LWARP ---

\usepackage{standalone}

\usepackage{tikz}
\usepackage{amsthm}
\usepackage{amsmath}
\usepackage{amssymb}
\usepackage{algorithm}
\usepackage{algpseudocode}

\usetikzlibrary{graphs, positioning, shapes.geometric}

\usepackage{makeidx} \makeindex
\usepackage{xcolor}               % (Demonstration purposes only.)
\usepackage{hyperref,cleveref}    % LOAD THESE LAST!

% Declare theorem environments for amsthm
\newtheorem{axiom}{axiom}
\newtheorem{lemma}{Lemma}
\newtheorem{theorem}{Theorem}
\newtheorem{example}{Example}
\newtheorem{definition}{Definition}

% --- LATEX AND HTML CUSTOMIZATION ---
\title{Notes on Everything}
\author{Anthony}
\date{\today}

\setcounter{tocdepth}{1}
\setcounter{secnumdepth}{-1}
\setcounter{FileDepth}{1}
\booltrue{CombineHigherDepths}
\setcounter{SideTOCDepth}{2}

% Renew commands

\renewcommand*\contentsname{Subjects}
\renewcommand\sidetocname{Subjects}

% HTML Directives
\HTMLTitle{Notes on Everything}
\HTMLAuthor{Anthony}
\HTMLLanguage{en-US}
\HTMLDescription{Personal notes on Mathematics and Computer Science}
\HTMLPageBottom{\LinkHome}

% Styling
\CSSFilename{lwarp.css}

\begin{document}

\maketitle % Or titlepage/titlingpage environment.
% An article abstract would go here.

Notes on Mathematics and Computer Science.

\tableofcontents % MUST BE BEFORE THE FIRST SECTION BREAK!

\listoffigures

\input{./sums/index}
\input{./algorithms/index}
\input{./combinatorics/index}
\input{./probability/index}
\input{./foundations/index}

\ForceHTMLPage 	% HTML index will be on its own page.
\ForceHTMLTOC 	% HTML index will have its own toc entry.
\printindex
\end{document}

% Save this as tutorial.tex for the lwarp package tutorial.
\documentclass{book}
\usepackage{iftex}

% --- LOAD FONT SELECTION AND ENCODING BEFORE LOADING LWARP ---
\ifPDFTeX
	\usepackage{lmodern} % pdflatex or dvi latex
	\usepackage[T1]{fontenc}
	\usepackage[utf8]{inputenc}
\else
	\usepackage{fontspec} % XeLaTeX or LuaLaTeX
\fi

% --- LWARP IS LOADED NEXT ---

\usepackage[
% HomeHTMLFilename=index, % Filename of the homepage.
% HTMLFilename={node-}, % Filename prefix of other pages.
% IndexLanguage=english, % Language for xindy index, glossary.
% latexmk, % Use latexmk to compile.
% OSWindows, % Force Windows. (Usually automatic.)
mathjax, % Use MathJax to display math.
]{lwarp}

% \boolfalse{FileSectionNames} % If false, numbers the files.

% --- LOAD PDFLATEX MATH FONTS HERE ---

% --- OTHER PACKAGES ARE LOADED AFTER LWARP ---

\usepackage{standalone}

\usepackage{tikz}
\usepackage{amsthm}
\usepackage{amsmath}
\usepackage{amssymb}
\usepackage{algorithm}
\usepackage{algpseudocode}

\usetikzlibrary{graphs, positioning, shapes.geometric}

\usepackage{makeidx} \makeindex
\usepackage{xcolor}               % (Demonstration purposes only.)
\usepackage{hyperref,cleveref}    % LOAD THESE LAST!

% Declare theorem environments for amsthm
\newtheorem{axiom}{axiom}
\newtheorem{lemma}{Lemma}
\newtheorem{theorem}{Theorem}
\newtheorem{example}{Example}
\newtheorem{definition}{Definition}

% --- LATEX AND HTML CUSTOMIZATION ---
\title{Notes on Everything}
\author{Anthony}
\date{\today}

\setcounter{tocdepth}{1}
\setcounter{secnumdepth}{-1}
\setcounter{FileDepth}{1}
\booltrue{CombineHigherDepths}
\setcounter{SideTOCDepth}{2}

% Renew commands

\renewcommand*\contentsname{Subjects}
\renewcommand\sidetocname{Subjects}

% HTML Directives
\HTMLTitle{Notes on Everything}
\HTMLAuthor{Anthony}
\HTMLLanguage{en-US}
\HTMLDescription{Personal notes on Mathematics and Computer Science}
\HTMLPageBottom{\LinkHome}

% Styling
\CSSFilename{lwarp.css}

\begin{document}

\maketitle % Or titlepage/titlingpage environment.
% An article abstract would go here.

Notes on Mathematics and Computer Science.

\tableofcontents % MUST BE BEFORE THE FIRST SECTION BREAK!

\listoffigures

\input{./sums/index}
\input{./algorithms/index}
\input{./combinatorics/index}
\input{./probability/index}
\input{./foundations/index}

\ForceHTMLPage 	% HTML index will be on its own page.
\ForceHTMLTOC 	% HTML index will have its own toc entry.
\printindex
\end{document}

% Save this as tutorial.tex for the lwarp package tutorial.
\documentclass{book}
\usepackage{iftex}

% --- LOAD FONT SELECTION AND ENCODING BEFORE LOADING LWARP ---
\ifPDFTeX
	\usepackage{lmodern} % pdflatex or dvi latex
	\usepackage[T1]{fontenc}
	\usepackage[utf8]{inputenc}
\else
	\usepackage{fontspec} % XeLaTeX or LuaLaTeX
\fi

% --- LWARP IS LOADED NEXT ---

\usepackage[
% HomeHTMLFilename=index, % Filename of the homepage.
% HTMLFilename={node-}, % Filename prefix of other pages.
% IndexLanguage=english, % Language for xindy index, glossary.
% latexmk, % Use latexmk to compile.
% OSWindows, % Force Windows. (Usually automatic.)
mathjax, % Use MathJax to display math.
]{lwarp}

% \boolfalse{FileSectionNames} % If false, numbers the files.

% --- LOAD PDFLATEX MATH FONTS HERE ---

% --- OTHER PACKAGES ARE LOADED AFTER LWARP ---

\usepackage{standalone}

\usepackage{tikz}
\usepackage{amsthm}
\usepackage{amsmath}
\usepackage{amssymb}
\usepackage{algorithm}
\usepackage{algpseudocode}

\usetikzlibrary{graphs, positioning, shapes.geometric}

\usepackage{makeidx} \makeindex
\usepackage{xcolor}               % (Demonstration purposes only.)
\usepackage{hyperref,cleveref}    % LOAD THESE LAST!

% Declare theorem environments for amsthm
\newtheorem{axiom}{axiom}
\newtheorem{lemma}{Lemma}
\newtheorem{theorem}{Theorem}
\newtheorem{example}{Example}
\newtheorem{definition}{Definition}

% --- LATEX AND HTML CUSTOMIZATION ---
\title{Notes on Everything}
\author{Anthony}
\date{\today}

\setcounter{tocdepth}{1}
\setcounter{secnumdepth}{-1}
\setcounter{FileDepth}{1}
\booltrue{CombineHigherDepths}
\setcounter{SideTOCDepth}{2}

% Renew commands

\renewcommand*\contentsname{Subjects}
\renewcommand\sidetocname{Subjects}

% HTML Directives
\HTMLTitle{Notes on Everything}
\HTMLAuthor{Anthony}
\HTMLLanguage{en-US}
\HTMLDescription{Personal notes on Mathematics and Computer Science}
\HTMLPageBottom{\LinkHome}

% Styling
\CSSFilename{lwarp.css}

\begin{document}

\maketitle % Or titlepage/titlingpage environment.
% An article abstract would go here.

Notes on Mathematics and Computer Science.

\tableofcontents % MUST BE BEFORE THE FIRST SECTION BREAK!

\listoffigures

\input{./sums/index}
\input{./algorithms/index}
\input{./combinatorics/index}
\input{./probability/index}
\input{./foundations/index}

\ForceHTMLPage 	% HTML index will be on its own page.
\ForceHTMLTOC 	% HTML index will have its own toc entry.
\printindex
\end{document}


\ForceHTMLPage 	% HTML index will be on its own page.
\ForceHTMLTOC 	% HTML index will have its own toc entry.
\printindex
\end{document}


\ForceHTMLPage 	% HTML index will be on its own page.
\ForceHTMLTOC 	% HTML index will have its own toc entry.
\printindex
\end{document}

% Save this as tutorial.tex for the lwarp package tutorial.
\documentclass{book}
\usepackage{iftex}

% --- LOAD FONT SELECTION AND ENCODING BEFORE LOADING LWARP ---
\ifPDFTeX
	\usepackage{lmodern} % pdflatex or dvi latex
	\usepackage[T1]{fontenc}
	\usepackage[utf8]{inputenc}
\else
	\usepackage{fontspec} % XeLaTeX or LuaLaTeX
\fi

% --- LWARP IS LOADED NEXT ---

\usepackage[
% HomeHTMLFilename=index, % Filename of the homepage.
% HTMLFilename={node-}, % Filename prefix of other pages.
% IndexLanguage=english, % Language for xindy index, glossary.
% latexmk, % Use latexmk to compile.
% OSWindows, % Force Windows. (Usually automatic.)
mathjax, % Use MathJax to display math.
]{lwarp}

% \boolfalse{FileSectionNames} % If false, numbers the files.

% --- LOAD PDFLATEX MATH FONTS HERE ---

% --- OTHER PACKAGES ARE LOADED AFTER LWARP ---

\usepackage{standalone}

\usepackage{tikz}
\usepackage{amsthm}
\usepackage{amsmath}
\usepackage{amssymb}
\usepackage{algorithm}
\usepackage{algpseudocode}

\usetikzlibrary{graphs, positioning, shapes.geometric}

\usepackage{makeidx} \makeindex
\usepackage{xcolor}               % (Demonstration purposes only.)
\usepackage{hyperref,cleveref}    % LOAD THESE LAST!

% Declare theorem environments for amsthm
\newtheorem{axiom}{axiom}
\newtheorem{lemma}{Lemma}
\newtheorem{theorem}{Theorem}
\newtheorem{example}{Example}
\newtheorem{definition}{Definition}

% --- LATEX AND HTML CUSTOMIZATION ---
\title{Notes on Everything}
\author{Anthony}
\date{\today}

\setcounter{tocdepth}{1}
\setcounter{secnumdepth}{-1}
\setcounter{FileDepth}{1}
\booltrue{CombineHigherDepths}
\setcounter{SideTOCDepth}{2}

% Renew commands

\renewcommand*\contentsname{Subjects}
\renewcommand\sidetocname{Subjects}

% HTML Directives
\HTMLTitle{Notes on Everything}
\HTMLAuthor{Anthony}
\HTMLLanguage{en-US}
\HTMLDescription{Personal notes on Mathematics and Computer Science}
\HTMLPageBottom{\LinkHome}

% Styling
\CSSFilename{lwarp.css}

\begin{document}

\maketitle % Or titlepage/titlingpage environment.
% An article abstract would go here.

Notes on Mathematics and Computer Science.

\tableofcontents % MUST BE BEFORE THE FIRST SECTION BREAK!

\listoffigures

% Save this as tutorial.tex for the lwarp package tutorial.
\documentclass{book}
\usepackage{iftex}

% --- LOAD FONT SELECTION AND ENCODING BEFORE LOADING LWARP ---
\ifPDFTeX
	\usepackage{lmodern} % pdflatex or dvi latex
	\usepackage[T1]{fontenc}
	\usepackage[utf8]{inputenc}
\else
	\usepackage{fontspec} % XeLaTeX or LuaLaTeX
\fi

% --- LWARP IS LOADED NEXT ---

\usepackage[
% HomeHTMLFilename=index, % Filename of the homepage.
% HTMLFilename={node-}, % Filename prefix of other pages.
% IndexLanguage=english, % Language for xindy index, glossary.
% latexmk, % Use latexmk to compile.
% OSWindows, % Force Windows. (Usually automatic.)
mathjax, % Use MathJax to display math.
]{lwarp}

% \boolfalse{FileSectionNames} % If false, numbers the files.

% --- LOAD PDFLATEX MATH FONTS HERE ---

% --- OTHER PACKAGES ARE LOADED AFTER LWARP ---

\usepackage{standalone}

\usepackage{tikz}
\usepackage{amsthm}
\usepackage{amsmath}
\usepackage{amssymb}
\usepackage{algorithm}
\usepackage{algpseudocode}

\usetikzlibrary{graphs, positioning, shapes.geometric}

\usepackage{makeidx} \makeindex
\usepackage{xcolor}               % (Demonstration purposes only.)
\usepackage{hyperref,cleveref}    % LOAD THESE LAST!

% Declare theorem environments for amsthm
\newtheorem{axiom}{axiom}
\newtheorem{lemma}{Lemma}
\newtheorem{theorem}{Theorem}
\newtheorem{example}{Example}
\newtheorem{definition}{Definition}

% --- LATEX AND HTML CUSTOMIZATION ---
\title{Notes on Everything}
\author{Anthony}
\date{\today}

\setcounter{tocdepth}{1}
\setcounter{secnumdepth}{-1}
\setcounter{FileDepth}{1}
\booltrue{CombineHigherDepths}
\setcounter{SideTOCDepth}{2}

% Renew commands

\renewcommand*\contentsname{Subjects}
\renewcommand\sidetocname{Subjects}

% HTML Directives
\HTMLTitle{Notes on Everything}
\HTMLAuthor{Anthony}
\HTMLLanguage{en-US}
\HTMLDescription{Personal notes on Mathematics and Computer Science}
\HTMLPageBottom{\LinkHome}

% Styling
\CSSFilename{lwarp.css}

\begin{document}

\maketitle % Or titlepage/titlingpage environment.
% An article abstract would go here.

Notes on Mathematics and Computer Science.

\tableofcontents % MUST BE BEFORE THE FIRST SECTION BREAK!

\listoffigures

% Save this as tutorial.tex for the lwarp package tutorial.
\documentclass{book}
\usepackage{iftex}

% --- LOAD FONT SELECTION AND ENCODING BEFORE LOADING LWARP ---
\ifPDFTeX
	\usepackage{lmodern} % pdflatex or dvi latex
	\usepackage[T1]{fontenc}
	\usepackage[utf8]{inputenc}
\else
	\usepackage{fontspec} % XeLaTeX or LuaLaTeX
\fi

% --- LWARP IS LOADED NEXT ---

\usepackage[
% HomeHTMLFilename=index, % Filename of the homepage.
% HTMLFilename={node-}, % Filename prefix of other pages.
% IndexLanguage=english, % Language for xindy index, glossary.
% latexmk, % Use latexmk to compile.
% OSWindows, % Force Windows. (Usually automatic.)
mathjax, % Use MathJax to display math.
]{lwarp}

% \boolfalse{FileSectionNames} % If false, numbers the files.

% --- LOAD PDFLATEX MATH FONTS HERE ---

% --- OTHER PACKAGES ARE LOADED AFTER LWARP ---

\usepackage{standalone}

\usepackage{tikz}
\usepackage{amsthm}
\usepackage{amsmath}
\usepackage{amssymb}
\usepackage{algorithm}
\usepackage{algpseudocode}

\usetikzlibrary{graphs, positioning, shapes.geometric}

\usepackage{makeidx} \makeindex
\usepackage{xcolor}               % (Demonstration purposes only.)
\usepackage{hyperref,cleveref}    % LOAD THESE LAST!

% Declare theorem environments for amsthm
\newtheorem{axiom}{axiom}
\newtheorem{lemma}{Lemma}
\newtheorem{theorem}{Theorem}
\newtheorem{example}{Example}
\newtheorem{definition}{Definition}

% --- LATEX AND HTML CUSTOMIZATION ---
\title{Notes on Everything}
\author{Anthony}
\date{\today}

\setcounter{tocdepth}{1}
\setcounter{secnumdepth}{-1}
\setcounter{FileDepth}{1}
\booltrue{CombineHigherDepths}
\setcounter{SideTOCDepth}{2}

% Renew commands

\renewcommand*\contentsname{Subjects}
\renewcommand\sidetocname{Subjects}

% HTML Directives
\HTMLTitle{Notes on Everything}
\HTMLAuthor{Anthony}
\HTMLLanguage{en-US}
\HTMLDescription{Personal notes on Mathematics and Computer Science}
\HTMLPageBottom{\LinkHome}

% Styling
\CSSFilename{lwarp.css}

\begin{document}

\maketitle % Or titlepage/titlingpage environment.
% An article abstract would go here.

Notes on Mathematics and Computer Science.

\tableofcontents % MUST BE BEFORE THE FIRST SECTION BREAK!

\listoffigures

\input{./sums/index}
\input{./algorithms/index}
\input{./combinatorics/index}
\input{./probability/index}
\input{./foundations/index}

\ForceHTMLPage 	% HTML index will be on its own page.
\ForceHTMLTOC 	% HTML index will have its own toc entry.
\printindex
\end{document}

% Save this as tutorial.tex for the lwarp package tutorial.
\documentclass{book}
\usepackage{iftex}

% --- LOAD FONT SELECTION AND ENCODING BEFORE LOADING LWARP ---
\ifPDFTeX
	\usepackage{lmodern} % pdflatex or dvi latex
	\usepackage[T1]{fontenc}
	\usepackage[utf8]{inputenc}
\else
	\usepackage{fontspec} % XeLaTeX or LuaLaTeX
\fi

% --- LWARP IS LOADED NEXT ---

\usepackage[
% HomeHTMLFilename=index, % Filename of the homepage.
% HTMLFilename={node-}, % Filename prefix of other pages.
% IndexLanguage=english, % Language for xindy index, glossary.
% latexmk, % Use latexmk to compile.
% OSWindows, % Force Windows. (Usually automatic.)
mathjax, % Use MathJax to display math.
]{lwarp}

% \boolfalse{FileSectionNames} % If false, numbers the files.

% --- LOAD PDFLATEX MATH FONTS HERE ---

% --- OTHER PACKAGES ARE LOADED AFTER LWARP ---

\usepackage{standalone}

\usepackage{tikz}
\usepackage{amsthm}
\usepackage{amsmath}
\usepackage{amssymb}
\usepackage{algorithm}
\usepackage{algpseudocode}

\usetikzlibrary{graphs, positioning, shapes.geometric}

\usepackage{makeidx} \makeindex
\usepackage{xcolor}               % (Demonstration purposes only.)
\usepackage{hyperref,cleveref}    % LOAD THESE LAST!

% Declare theorem environments for amsthm
\newtheorem{axiom}{axiom}
\newtheorem{lemma}{Lemma}
\newtheorem{theorem}{Theorem}
\newtheorem{example}{Example}
\newtheorem{definition}{Definition}

% --- LATEX AND HTML CUSTOMIZATION ---
\title{Notes on Everything}
\author{Anthony}
\date{\today}

\setcounter{tocdepth}{1}
\setcounter{secnumdepth}{-1}
\setcounter{FileDepth}{1}
\booltrue{CombineHigherDepths}
\setcounter{SideTOCDepth}{2}

% Renew commands

\renewcommand*\contentsname{Subjects}
\renewcommand\sidetocname{Subjects}

% HTML Directives
\HTMLTitle{Notes on Everything}
\HTMLAuthor{Anthony}
\HTMLLanguage{en-US}
\HTMLDescription{Personal notes on Mathematics and Computer Science}
\HTMLPageBottom{\LinkHome}

% Styling
\CSSFilename{lwarp.css}

\begin{document}

\maketitle % Or titlepage/titlingpage environment.
% An article abstract would go here.

Notes on Mathematics and Computer Science.

\tableofcontents % MUST BE BEFORE THE FIRST SECTION BREAK!

\listoffigures

\input{./sums/index}
\input{./algorithms/index}
\input{./combinatorics/index}
\input{./probability/index}
\input{./foundations/index}

\ForceHTMLPage 	% HTML index will be on its own page.
\ForceHTMLTOC 	% HTML index will have its own toc entry.
\printindex
\end{document}

% Save this as tutorial.tex for the lwarp package tutorial.
\documentclass{book}
\usepackage{iftex}

% --- LOAD FONT SELECTION AND ENCODING BEFORE LOADING LWARP ---
\ifPDFTeX
	\usepackage{lmodern} % pdflatex or dvi latex
	\usepackage[T1]{fontenc}
	\usepackage[utf8]{inputenc}
\else
	\usepackage{fontspec} % XeLaTeX or LuaLaTeX
\fi

% --- LWARP IS LOADED NEXT ---

\usepackage[
% HomeHTMLFilename=index, % Filename of the homepage.
% HTMLFilename={node-}, % Filename prefix of other pages.
% IndexLanguage=english, % Language for xindy index, glossary.
% latexmk, % Use latexmk to compile.
% OSWindows, % Force Windows. (Usually automatic.)
mathjax, % Use MathJax to display math.
]{lwarp}

% \boolfalse{FileSectionNames} % If false, numbers the files.

% --- LOAD PDFLATEX MATH FONTS HERE ---

% --- OTHER PACKAGES ARE LOADED AFTER LWARP ---

\usepackage{standalone}

\usepackage{tikz}
\usepackage{amsthm}
\usepackage{amsmath}
\usepackage{amssymb}
\usepackage{algorithm}
\usepackage{algpseudocode}

\usetikzlibrary{graphs, positioning, shapes.geometric}

\usepackage{makeidx} \makeindex
\usepackage{xcolor}               % (Demonstration purposes only.)
\usepackage{hyperref,cleveref}    % LOAD THESE LAST!

% Declare theorem environments for amsthm
\newtheorem{axiom}{axiom}
\newtheorem{lemma}{Lemma}
\newtheorem{theorem}{Theorem}
\newtheorem{example}{Example}
\newtheorem{definition}{Definition}

% --- LATEX AND HTML CUSTOMIZATION ---
\title{Notes on Everything}
\author{Anthony}
\date{\today}

\setcounter{tocdepth}{1}
\setcounter{secnumdepth}{-1}
\setcounter{FileDepth}{1}
\booltrue{CombineHigherDepths}
\setcounter{SideTOCDepth}{2}

% Renew commands

\renewcommand*\contentsname{Subjects}
\renewcommand\sidetocname{Subjects}

% HTML Directives
\HTMLTitle{Notes on Everything}
\HTMLAuthor{Anthony}
\HTMLLanguage{en-US}
\HTMLDescription{Personal notes on Mathematics and Computer Science}
\HTMLPageBottom{\LinkHome}

% Styling
\CSSFilename{lwarp.css}

\begin{document}

\maketitle % Or titlepage/titlingpage environment.
% An article abstract would go here.

Notes on Mathematics and Computer Science.

\tableofcontents % MUST BE BEFORE THE FIRST SECTION BREAK!

\listoffigures

\input{./sums/index}
\input{./algorithms/index}
\input{./combinatorics/index}
\input{./probability/index}
\input{./foundations/index}

\ForceHTMLPage 	% HTML index will be on its own page.
\ForceHTMLTOC 	% HTML index will have its own toc entry.
\printindex
\end{document}

% Save this as tutorial.tex for the lwarp package tutorial.
\documentclass{book}
\usepackage{iftex}

% --- LOAD FONT SELECTION AND ENCODING BEFORE LOADING LWARP ---
\ifPDFTeX
	\usepackage{lmodern} % pdflatex or dvi latex
	\usepackage[T1]{fontenc}
	\usepackage[utf8]{inputenc}
\else
	\usepackage{fontspec} % XeLaTeX or LuaLaTeX
\fi

% --- LWARP IS LOADED NEXT ---

\usepackage[
% HomeHTMLFilename=index, % Filename of the homepage.
% HTMLFilename={node-}, % Filename prefix of other pages.
% IndexLanguage=english, % Language for xindy index, glossary.
% latexmk, % Use latexmk to compile.
% OSWindows, % Force Windows. (Usually automatic.)
mathjax, % Use MathJax to display math.
]{lwarp}

% \boolfalse{FileSectionNames} % If false, numbers the files.

% --- LOAD PDFLATEX MATH FONTS HERE ---

% --- OTHER PACKAGES ARE LOADED AFTER LWARP ---

\usepackage{standalone}

\usepackage{tikz}
\usepackage{amsthm}
\usepackage{amsmath}
\usepackage{amssymb}
\usepackage{algorithm}
\usepackage{algpseudocode}

\usetikzlibrary{graphs, positioning, shapes.geometric}

\usepackage{makeidx} \makeindex
\usepackage{xcolor}               % (Demonstration purposes only.)
\usepackage{hyperref,cleveref}    % LOAD THESE LAST!

% Declare theorem environments for amsthm
\newtheorem{axiom}{axiom}
\newtheorem{lemma}{Lemma}
\newtheorem{theorem}{Theorem}
\newtheorem{example}{Example}
\newtheorem{definition}{Definition}

% --- LATEX AND HTML CUSTOMIZATION ---
\title{Notes on Everything}
\author{Anthony}
\date{\today}

\setcounter{tocdepth}{1}
\setcounter{secnumdepth}{-1}
\setcounter{FileDepth}{1}
\booltrue{CombineHigherDepths}
\setcounter{SideTOCDepth}{2}

% Renew commands

\renewcommand*\contentsname{Subjects}
\renewcommand\sidetocname{Subjects}

% HTML Directives
\HTMLTitle{Notes on Everything}
\HTMLAuthor{Anthony}
\HTMLLanguage{en-US}
\HTMLDescription{Personal notes on Mathematics and Computer Science}
\HTMLPageBottom{\LinkHome}

% Styling
\CSSFilename{lwarp.css}

\begin{document}

\maketitle % Or titlepage/titlingpage environment.
% An article abstract would go here.

Notes on Mathematics and Computer Science.

\tableofcontents % MUST BE BEFORE THE FIRST SECTION BREAK!

\listoffigures

\input{./sums/index}
\input{./algorithms/index}
\input{./combinatorics/index}
\input{./probability/index}
\input{./foundations/index}

\ForceHTMLPage 	% HTML index will be on its own page.
\ForceHTMLTOC 	% HTML index will have its own toc entry.
\printindex
\end{document}

% Save this as tutorial.tex for the lwarp package tutorial.
\documentclass{book}
\usepackage{iftex}

% --- LOAD FONT SELECTION AND ENCODING BEFORE LOADING LWARP ---
\ifPDFTeX
	\usepackage{lmodern} % pdflatex or dvi latex
	\usepackage[T1]{fontenc}
	\usepackage[utf8]{inputenc}
\else
	\usepackage{fontspec} % XeLaTeX or LuaLaTeX
\fi

% --- LWARP IS LOADED NEXT ---

\usepackage[
% HomeHTMLFilename=index, % Filename of the homepage.
% HTMLFilename={node-}, % Filename prefix of other pages.
% IndexLanguage=english, % Language for xindy index, glossary.
% latexmk, % Use latexmk to compile.
% OSWindows, % Force Windows. (Usually automatic.)
mathjax, % Use MathJax to display math.
]{lwarp}

% \boolfalse{FileSectionNames} % If false, numbers the files.

% --- LOAD PDFLATEX MATH FONTS HERE ---

% --- OTHER PACKAGES ARE LOADED AFTER LWARP ---

\usepackage{standalone}

\usepackage{tikz}
\usepackage{amsthm}
\usepackage{amsmath}
\usepackage{amssymb}
\usepackage{algorithm}
\usepackage{algpseudocode}

\usetikzlibrary{graphs, positioning, shapes.geometric}

\usepackage{makeidx} \makeindex
\usepackage{xcolor}               % (Demonstration purposes only.)
\usepackage{hyperref,cleveref}    % LOAD THESE LAST!

% Declare theorem environments for amsthm
\newtheorem{axiom}{axiom}
\newtheorem{lemma}{Lemma}
\newtheorem{theorem}{Theorem}
\newtheorem{example}{Example}
\newtheorem{definition}{Definition}

% --- LATEX AND HTML CUSTOMIZATION ---
\title{Notes on Everything}
\author{Anthony}
\date{\today}

\setcounter{tocdepth}{1}
\setcounter{secnumdepth}{-1}
\setcounter{FileDepth}{1}
\booltrue{CombineHigherDepths}
\setcounter{SideTOCDepth}{2}

% Renew commands

\renewcommand*\contentsname{Subjects}
\renewcommand\sidetocname{Subjects}

% HTML Directives
\HTMLTitle{Notes on Everything}
\HTMLAuthor{Anthony}
\HTMLLanguage{en-US}
\HTMLDescription{Personal notes on Mathematics and Computer Science}
\HTMLPageBottom{\LinkHome}

% Styling
\CSSFilename{lwarp.css}

\begin{document}

\maketitle % Or titlepage/titlingpage environment.
% An article abstract would go here.

Notes on Mathematics and Computer Science.

\tableofcontents % MUST BE BEFORE THE FIRST SECTION BREAK!

\listoffigures

\input{./sums/index}
\input{./algorithms/index}
\input{./combinatorics/index}
\input{./probability/index}
\input{./foundations/index}

\ForceHTMLPage 	% HTML index will be on its own page.
\ForceHTMLTOC 	% HTML index will have its own toc entry.
\printindex
\end{document}


\ForceHTMLPage 	% HTML index will be on its own page.
\ForceHTMLTOC 	% HTML index will have its own toc entry.
\printindex
\end{document}

% Save this as tutorial.tex for the lwarp package tutorial.
\documentclass{book}
\usepackage{iftex}

% --- LOAD FONT SELECTION AND ENCODING BEFORE LOADING LWARP ---
\ifPDFTeX
	\usepackage{lmodern} % pdflatex or dvi latex
	\usepackage[T1]{fontenc}
	\usepackage[utf8]{inputenc}
\else
	\usepackage{fontspec} % XeLaTeX or LuaLaTeX
\fi

% --- LWARP IS LOADED NEXT ---

\usepackage[
% HomeHTMLFilename=index, % Filename of the homepage.
% HTMLFilename={node-}, % Filename prefix of other pages.
% IndexLanguage=english, % Language for xindy index, glossary.
% latexmk, % Use latexmk to compile.
% OSWindows, % Force Windows. (Usually automatic.)
mathjax, % Use MathJax to display math.
]{lwarp}

% \boolfalse{FileSectionNames} % If false, numbers the files.

% --- LOAD PDFLATEX MATH FONTS HERE ---

% --- OTHER PACKAGES ARE LOADED AFTER LWARP ---

\usepackage{standalone}

\usepackage{tikz}
\usepackage{amsthm}
\usepackage{amsmath}
\usepackage{amssymb}
\usepackage{algorithm}
\usepackage{algpseudocode}

\usetikzlibrary{graphs, positioning, shapes.geometric}

\usepackage{makeidx} \makeindex
\usepackage{xcolor}               % (Demonstration purposes only.)
\usepackage{hyperref,cleveref}    % LOAD THESE LAST!

% Declare theorem environments for amsthm
\newtheorem{axiom}{axiom}
\newtheorem{lemma}{Lemma}
\newtheorem{theorem}{Theorem}
\newtheorem{example}{Example}
\newtheorem{definition}{Definition}

% --- LATEX AND HTML CUSTOMIZATION ---
\title{Notes on Everything}
\author{Anthony}
\date{\today}

\setcounter{tocdepth}{1}
\setcounter{secnumdepth}{-1}
\setcounter{FileDepth}{1}
\booltrue{CombineHigherDepths}
\setcounter{SideTOCDepth}{2}

% Renew commands

\renewcommand*\contentsname{Subjects}
\renewcommand\sidetocname{Subjects}

% HTML Directives
\HTMLTitle{Notes on Everything}
\HTMLAuthor{Anthony}
\HTMLLanguage{en-US}
\HTMLDescription{Personal notes on Mathematics and Computer Science}
\HTMLPageBottom{\LinkHome}

% Styling
\CSSFilename{lwarp.css}

\begin{document}

\maketitle % Or titlepage/titlingpage environment.
% An article abstract would go here.

Notes on Mathematics and Computer Science.

\tableofcontents % MUST BE BEFORE THE FIRST SECTION BREAK!

\listoffigures

% Save this as tutorial.tex for the lwarp package tutorial.
\documentclass{book}
\usepackage{iftex}

% --- LOAD FONT SELECTION AND ENCODING BEFORE LOADING LWARP ---
\ifPDFTeX
	\usepackage{lmodern} % pdflatex or dvi latex
	\usepackage[T1]{fontenc}
	\usepackage[utf8]{inputenc}
\else
	\usepackage{fontspec} % XeLaTeX or LuaLaTeX
\fi

% --- LWARP IS LOADED NEXT ---

\usepackage[
% HomeHTMLFilename=index, % Filename of the homepage.
% HTMLFilename={node-}, % Filename prefix of other pages.
% IndexLanguage=english, % Language for xindy index, glossary.
% latexmk, % Use latexmk to compile.
% OSWindows, % Force Windows. (Usually automatic.)
mathjax, % Use MathJax to display math.
]{lwarp}

% \boolfalse{FileSectionNames} % If false, numbers the files.

% --- LOAD PDFLATEX MATH FONTS HERE ---

% --- OTHER PACKAGES ARE LOADED AFTER LWARP ---

\usepackage{standalone}

\usepackage{tikz}
\usepackage{amsthm}
\usepackage{amsmath}
\usepackage{amssymb}
\usepackage{algorithm}
\usepackage{algpseudocode}

\usetikzlibrary{graphs, positioning, shapes.geometric}

\usepackage{makeidx} \makeindex
\usepackage{xcolor}               % (Demonstration purposes only.)
\usepackage{hyperref,cleveref}    % LOAD THESE LAST!

% Declare theorem environments for amsthm
\newtheorem{axiom}{axiom}
\newtheorem{lemma}{Lemma}
\newtheorem{theorem}{Theorem}
\newtheorem{example}{Example}
\newtheorem{definition}{Definition}

% --- LATEX AND HTML CUSTOMIZATION ---
\title{Notes on Everything}
\author{Anthony}
\date{\today}

\setcounter{tocdepth}{1}
\setcounter{secnumdepth}{-1}
\setcounter{FileDepth}{1}
\booltrue{CombineHigherDepths}
\setcounter{SideTOCDepth}{2}

% Renew commands

\renewcommand*\contentsname{Subjects}
\renewcommand\sidetocname{Subjects}

% HTML Directives
\HTMLTitle{Notes on Everything}
\HTMLAuthor{Anthony}
\HTMLLanguage{en-US}
\HTMLDescription{Personal notes on Mathematics and Computer Science}
\HTMLPageBottom{\LinkHome}

% Styling
\CSSFilename{lwarp.css}

\begin{document}

\maketitle % Or titlepage/titlingpage environment.
% An article abstract would go here.

Notes on Mathematics and Computer Science.

\tableofcontents % MUST BE BEFORE THE FIRST SECTION BREAK!

\listoffigures

\input{./sums/index}
\input{./algorithms/index}
\input{./combinatorics/index}
\input{./probability/index}
\input{./foundations/index}

\ForceHTMLPage 	% HTML index will be on its own page.
\ForceHTMLTOC 	% HTML index will have its own toc entry.
\printindex
\end{document}

% Save this as tutorial.tex for the lwarp package tutorial.
\documentclass{book}
\usepackage{iftex}

% --- LOAD FONT SELECTION AND ENCODING BEFORE LOADING LWARP ---
\ifPDFTeX
	\usepackage{lmodern} % pdflatex or dvi latex
	\usepackage[T1]{fontenc}
	\usepackage[utf8]{inputenc}
\else
	\usepackage{fontspec} % XeLaTeX or LuaLaTeX
\fi

% --- LWARP IS LOADED NEXT ---

\usepackage[
% HomeHTMLFilename=index, % Filename of the homepage.
% HTMLFilename={node-}, % Filename prefix of other pages.
% IndexLanguage=english, % Language for xindy index, glossary.
% latexmk, % Use latexmk to compile.
% OSWindows, % Force Windows. (Usually automatic.)
mathjax, % Use MathJax to display math.
]{lwarp}

% \boolfalse{FileSectionNames} % If false, numbers the files.

% --- LOAD PDFLATEX MATH FONTS HERE ---

% --- OTHER PACKAGES ARE LOADED AFTER LWARP ---

\usepackage{standalone}

\usepackage{tikz}
\usepackage{amsthm}
\usepackage{amsmath}
\usepackage{amssymb}
\usepackage{algorithm}
\usepackage{algpseudocode}

\usetikzlibrary{graphs, positioning, shapes.geometric}

\usepackage{makeidx} \makeindex
\usepackage{xcolor}               % (Demonstration purposes only.)
\usepackage{hyperref,cleveref}    % LOAD THESE LAST!

% Declare theorem environments for amsthm
\newtheorem{axiom}{axiom}
\newtheorem{lemma}{Lemma}
\newtheorem{theorem}{Theorem}
\newtheorem{example}{Example}
\newtheorem{definition}{Definition}

% --- LATEX AND HTML CUSTOMIZATION ---
\title{Notes on Everything}
\author{Anthony}
\date{\today}

\setcounter{tocdepth}{1}
\setcounter{secnumdepth}{-1}
\setcounter{FileDepth}{1}
\booltrue{CombineHigherDepths}
\setcounter{SideTOCDepth}{2}

% Renew commands

\renewcommand*\contentsname{Subjects}
\renewcommand\sidetocname{Subjects}

% HTML Directives
\HTMLTitle{Notes on Everything}
\HTMLAuthor{Anthony}
\HTMLLanguage{en-US}
\HTMLDescription{Personal notes on Mathematics and Computer Science}
\HTMLPageBottom{\LinkHome}

% Styling
\CSSFilename{lwarp.css}

\begin{document}

\maketitle % Or titlepage/titlingpage environment.
% An article abstract would go here.

Notes on Mathematics and Computer Science.

\tableofcontents % MUST BE BEFORE THE FIRST SECTION BREAK!

\listoffigures

\input{./sums/index}
\input{./algorithms/index}
\input{./combinatorics/index}
\input{./probability/index}
\input{./foundations/index}

\ForceHTMLPage 	% HTML index will be on its own page.
\ForceHTMLTOC 	% HTML index will have its own toc entry.
\printindex
\end{document}

% Save this as tutorial.tex for the lwarp package tutorial.
\documentclass{book}
\usepackage{iftex}

% --- LOAD FONT SELECTION AND ENCODING BEFORE LOADING LWARP ---
\ifPDFTeX
	\usepackage{lmodern} % pdflatex or dvi latex
	\usepackage[T1]{fontenc}
	\usepackage[utf8]{inputenc}
\else
	\usepackage{fontspec} % XeLaTeX or LuaLaTeX
\fi

% --- LWARP IS LOADED NEXT ---

\usepackage[
% HomeHTMLFilename=index, % Filename of the homepage.
% HTMLFilename={node-}, % Filename prefix of other pages.
% IndexLanguage=english, % Language for xindy index, glossary.
% latexmk, % Use latexmk to compile.
% OSWindows, % Force Windows. (Usually automatic.)
mathjax, % Use MathJax to display math.
]{lwarp}

% \boolfalse{FileSectionNames} % If false, numbers the files.

% --- LOAD PDFLATEX MATH FONTS HERE ---

% --- OTHER PACKAGES ARE LOADED AFTER LWARP ---

\usepackage{standalone}

\usepackage{tikz}
\usepackage{amsthm}
\usepackage{amsmath}
\usepackage{amssymb}
\usepackage{algorithm}
\usepackage{algpseudocode}

\usetikzlibrary{graphs, positioning, shapes.geometric}

\usepackage{makeidx} \makeindex
\usepackage{xcolor}               % (Demonstration purposes only.)
\usepackage{hyperref,cleveref}    % LOAD THESE LAST!

% Declare theorem environments for amsthm
\newtheorem{axiom}{axiom}
\newtheorem{lemma}{Lemma}
\newtheorem{theorem}{Theorem}
\newtheorem{example}{Example}
\newtheorem{definition}{Definition}

% --- LATEX AND HTML CUSTOMIZATION ---
\title{Notes on Everything}
\author{Anthony}
\date{\today}

\setcounter{tocdepth}{1}
\setcounter{secnumdepth}{-1}
\setcounter{FileDepth}{1}
\booltrue{CombineHigherDepths}
\setcounter{SideTOCDepth}{2}

% Renew commands

\renewcommand*\contentsname{Subjects}
\renewcommand\sidetocname{Subjects}

% HTML Directives
\HTMLTitle{Notes on Everything}
\HTMLAuthor{Anthony}
\HTMLLanguage{en-US}
\HTMLDescription{Personal notes on Mathematics and Computer Science}
\HTMLPageBottom{\LinkHome}

% Styling
\CSSFilename{lwarp.css}

\begin{document}

\maketitle % Or titlepage/titlingpage environment.
% An article abstract would go here.

Notes on Mathematics and Computer Science.

\tableofcontents % MUST BE BEFORE THE FIRST SECTION BREAK!

\listoffigures

\input{./sums/index}
\input{./algorithms/index}
\input{./combinatorics/index}
\input{./probability/index}
\input{./foundations/index}

\ForceHTMLPage 	% HTML index will be on its own page.
\ForceHTMLTOC 	% HTML index will have its own toc entry.
\printindex
\end{document}

% Save this as tutorial.tex for the lwarp package tutorial.
\documentclass{book}
\usepackage{iftex}

% --- LOAD FONT SELECTION AND ENCODING BEFORE LOADING LWARP ---
\ifPDFTeX
	\usepackage{lmodern} % pdflatex or dvi latex
	\usepackage[T1]{fontenc}
	\usepackage[utf8]{inputenc}
\else
	\usepackage{fontspec} % XeLaTeX or LuaLaTeX
\fi

% --- LWARP IS LOADED NEXT ---

\usepackage[
% HomeHTMLFilename=index, % Filename of the homepage.
% HTMLFilename={node-}, % Filename prefix of other pages.
% IndexLanguage=english, % Language for xindy index, glossary.
% latexmk, % Use latexmk to compile.
% OSWindows, % Force Windows. (Usually automatic.)
mathjax, % Use MathJax to display math.
]{lwarp}

% \boolfalse{FileSectionNames} % If false, numbers the files.

% --- LOAD PDFLATEX MATH FONTS HERE ---

% --- OTHER PACKAGES ARE LOADED AFTER LWARP ---

\usepackage{standalone}

\usepackage{tikz}
\usepackage{amsthm}
\usepackage{amsmath}
\usepackage{amssymb}
\usepackage{algorithm}
\usepackage{algpseudocode}

\usetikzlibrary{graphs, positioning, shapes.geometric}

\usepackage{makeidx} \makeindex
\usepackage{xcolor}               % (Demonstration purposes only.)
\usepackage{hyperref,cleveref}    % LOAD THESE LAST!

% Declare theorem environments for amsthm
\newtheorem{axiom}{axiom}
\newtheorem{lemma}{Lemma}
\newtheorem{theorem}{Theorem}
\newtheorem{example}{Example}
\newtheorem{definition}{Definition}

% --- LATEX AND HTML CUSTOMIZATION ---
\title{Notes on Everything}
\author{Anthony}
\date{\today}

\setcounter{tocdepth}{1}
\setcounter{secnumdepth}{-1}
\setcounter{FileDepth}{1}
\booltrue{CombineHigherDepths}
\setcounter{SideTOCDepth}{2}

% Renew commands

\renewcommand*\contentsname{Subjects}
\renewcommand\sidetocname{Subjects}

% HTML Directives
\HTMLTitle{Notes on Everything}
\HTMLAuthor{Anthony}
\HTMLLanguage{en-US}
\HTMLDescription{Personal notes on Mathematics and Computer Science}
\HTMLPageBottom{\LinkHome}

% Styling
\CSSFilename{lwarp.css}

\begin{document}

\maketitle % Or titlepage/titlingpage environment.
% An article abstract would go here.

Notes on Mathematics and Computer Science.

\tableofcontents % MUST BE BEFORE THE FIRST SECTION BREAK!

\listoffigures

\input{./sums/index}
\input{./algorithms/index}
\input{./combinatorics/index}
\input{./probability/index}
\input{./foundations/index}

\ForceHTMLPage 	% HTML index will be on its own page.
\ForceHTMLTOC 	% HTML index will have its own toc entry.
\printindex
\end{document}

% Save this as tutorial.tex for the lwarp package tutorial.
\documentclass{book}
\usepackage{iftex}

% --- LOAD FONT SELECTION AND ENCODING BEFORE LOADING LWARP ---
\ifPDFTeX
	\usepackage{lmodern} % pdflatex or dvi latex
	\usepackage[T1]{fontenc}
	\usepackage[utf8]{inputenc}
\else
	\usepackage{fontspec} % XeLaTeX or LuaLaTeX
\fi

% --- LWARP IS LOADED NEXT ---

\usepackage[
% HomeHTMLFilename=index, % Filename of the homepage.
% HTMLFilename={node-}, % Filename prefix of other pages.
% IndexLanguage=english, % Language for xindy index, glossary.
% latexmk, % Use latexmk to compile.
% OSWindows, % Force Windows. (Usually automatic.)
mathjax, % Use MathJax to display math.
]{lwarp}

% \boolfalse{FileSectionNames} % If false, numbers the files.

% --- LOAD PDFLATEX MATH FONTS HERE ---

% --- OTHER PACKAGES ARE LOADED AFTER LWARP ---

\usepackage{standalone}

\usepackage{tikz}
\usepackage{amsthm}
\usepackage{amsmath}
\usepackage{amssymb}
\usepackage{algorithm}
\usepackage{algpseudocode}

\usetikzlibrary{graphs, positioning, shapes.geometric}

\usepackage{makeidx} \makeindex
\usepackage{xcolor}               % (Demonstration purposes only.)
\usepackage{hyperref,cleveref}    % LOAD THESE LAST!

% Declare theorem environments for amsthm
\newtheorem{axiom}{axiom}
\newtheorem{lemma}{Lemma}
\newtheorem{theorem}{Theorem}
\newtheorem{example}{Example}
\newtheorem{definition}{Definition}

% --- LATEX AND HTML CUSTOMIZATION ---
\title{Notes on Everything}
\author{Anthony}
\date{\today}

\setcounter{tocdepth}{1}
\setcounter{secnumdepth}{-1}
\setcounter{FileDepth}{1}
\booltrue{CombineHigherDepths}
\setcounter{SideTOCDepth}{2}

% Renew commands

\renewcommand*\contentsname{Subjects}
\renewcommand\sidetocname{Subjects}

% HTML Directives
\HTMLTitle{Notes on Everything}
\HTMLAuthor{Anthony}
\HTMLLanguage{en-US}
\HTMLDescription{Personal notes on Mathematics and Computer Science}
\HTMLPageBottom{\LinkHome}

% Styling
\CSSFilename{lwarp.css}

\begin{document}

\maketitle % Or titlepage/titlingpage environment.
% An article abstract would go here.

Notes on Mathematics and Computer Science.

\tableofcontents % MUST BE BEFORE THE FIRST SECTION BREAK!

\listoffigures

\input{./sums/index}
\input{./algorithms/index}
\input{./combinatorics/index}
\input{./probability/index}
\input{./foundations/index}

\ForceHTMLPage 	% HTML index will be on its own page.
\ForceHTMLTOC 	% HTML index will have its own toc entry.
\printindex
\end{document}


\ForceHTMLPage 	% HTML index will be on its own page.
\ForceHTMLTOC 	% HTML index will have its own toc entry.
\printindex
\end{document}

% Save this as tutorial.tex for the lwarp package tutorial.
\documentclass{book}
\usepackage{iftex}

% --- LOAD FONT SELECTION AND ENCODING BEFORE LOADING LWARP ---
\ifPDFTeX
	\usepackage{lmodern} % pdflatex or dvi latex
	\usepackage[T1]{fontenc}
	\usepackage[utf8]{inputenc}
\else
	\usepackage{fontspec} % XeLaTeX or LuaLaTeX
\fi

% --- LWARP IS LOADED NEXT ---

\usepackage[
% HomeHTMLFilename=index, % Filename of the homepage.
% HTMLFilename={node-}, % Filename prefix of other pages.
% IndexLanguage=english, % Language for xindy index, glossary.
% latexmk, % Use latexmk to compile.
% OSWindows, % Force Windows. (Usually automatic.)
mathjax, % Use MathJax to display math.
]{lwarp}

% \boolfalse{FileSectionNames} % If false, numbers the files.

% --- LOAD PDFLATEX MATH FONTS HERE ---

% --- OTHER PACKAGES ARE LOADED AFTER LWARP ---

\usepackage{standalone}

\usepackage{tikz}
\usepackage{amsthm}
\usepackage{amsmath}
\usepackage{amssymb}
\usepackage{algorithm}
\usepackage{algpseudocode}

\usetikzlibrary{graphs, positioning, shapes.geometric}

\usepackage{makeidx} \makeindex
\usepackage{xcolor}               % (Demonstration purposes only.)
\usepackage{hyperref,cleveref}    % LOAD THESE LAST!

% Declare theorem environments for amsthm
\newtheorem{axiom}{axiom}
\newtheorem{lemma}{Lemma}
\newtheorem{theorem}{Theorem}
\newtheorem{example}{Example}
\newtheorem{definition}{Definition}

% --- LATEX AND HTML CUSTOMIZATION ---
\title{Notes on Everything}
\author{Anthony}
\date{\today}

\setcounter{tocdepth}{1}
\setcounter{secnumdepth}{-1}
\setcounter{FileDepth}{1}
\booltrue{CombineHigherDepths}
\setcounter{SideTOCDepth}{2}

% Renew commands

\renewcommand*\contentsname{Subjects}
\renewcommand\sidetocname{Subjects}

% HTML Directives
\HTMLTitle{Notes on Everything}
\HTMLAuthor{Anthony}
\HTMLLanguage{en-US}
\HTMLDescription{Personal notes on Mathematics and Computer Science}
\HTMLPageBottom{\LinkHome}

% Styling
\CSSFilename{lwarp.css}

\begin{document}

\maketitle % Or titlepage/titlingpage environment.
% An article abstract would go here.

Notes on Mathematics and Computer Science.

\tableofcontents % MUST BE BEFORE THE FIRST SECTION BREAK!

\listoffigures

% Save this as tutorial.tex for the lwarp package tutorial.
\documentclass{book}
\usepackage{iftex}

% --- LOAD FONT SELECTION AND ENCODING BEFORE LOADING LWARP ---
\ifPDFTeX
	\usepackage{lmodern} % pdflatex or dvi latex
	\usepackage[T1]{fontenc}
	\usepackage[utf8]{inputenc}
\else
	\usepackage{fontspec} % XeLaTeX or LuaLaTeX
\fi

% --- LWARP IS LOADED NEXT ---

\usepackage[
% HomeHTMLFilename=index, % Filename of the homepage.
% HTMLFilename={node-}, % Filename prefix of other pages.
% IndexLanguage=english, % Language for xindy index, glossary.
% latexmk, % Use latexmk to compile.
% OSWindows, % Force Windows. (Usually automatic.)
mathjax, % Use MathJax to display math.
]{lwarp}

% \boolfalse{FileSectionNames} % If false, numbers the files.

% --- LOAD PDFLATEX MATH FONTS HERE ---

% --- OTHER PACKAGES ARE LOADED AFTER LWARP ---

\usepackage{standalone}

\usepackage{tikz}
\usepackage{amsthm}
\usepackage{amsmath}
\usepackage{amssymb}
\usepackage{algorithm}
\usepackage{algpseudocode}

\usetikzlibrary{graphs, positioning, shapes.geometric}

\usepackage{makeidx} \makeindex
\usepackage{xcolor}               % (Demonstration purposes only.)
\usepackage{hyperref,cleveref}    % LOAD THESE LAST!

% Declare theorem environments for amsthm
\newtheorem{axiom}{axiom}
\newtheorem{lemma}{Lemma}
\newtheorem{theorem}{Theorem}
\newtheorem{example}{Example}
\newtheorem{definition}{Definition}

% --- LATEX AND HTML CUSTOMIZATION ---
\title{Notes on Everything}
\author{Anthony}
\date{\today}

\setcounter{tocdepth}{1}
\setcounter{secnumdepth}{-1}
\setcounter{FileDepth}{1}
\booltrue{CombineHigherDepths}
\setcounter{SideTOCDepth}{2}

% Renew commands

\renewcommand*\contentsname{Subjects}
\renewcommand\sidetocname{Subjects}

% HTML Directives
\HTMLTitle{Notes on Everything}
\HTMLAuthor{Anthony}
\HTMLLanguage{en-US}
\HTMLDescription{Personal notes on Mathematics and Computer Science}
\HTMLPageBottom{\LinkHome}

% Styling
\CSSFilename{lwarp.css}

\begin{document}

\maketitle % Or titlepage/titlingpage environment.
% An article abstract would go here.

Notes on Mathematics and Computer Science.

\tableofcontents % MUST BE BEFORE THE FIRST SECTION BREAK!

\listoffigures

\input{./sums/index}
\input{./algorithms/index}
\input{./combinatorics/index}
\input{./probability/index}
\input{./foundations/index}

\ForceHTMLPage 	% HTML index will be on its own page.
\ForceHTMLTOC 	% HTML index will have its own toc entry.
\printindex
\end{document}

% Save this as tutorial.tex for the lwarp package tutorial.
\documentclass{book}
\usepackage{iftex}

% --- LOAD FONT SELECTION AND ENCODING BEFORE LOADING LWARP ---
\ifPDFTeX
	\usepackage{lmodern} % pdflatex or dvi latex
	\usepackage[T1]{fontenc}
	\usepackage[utf8]{inputenc}
\else
	\usepackage{fontspec} % XeLaTeX or LuaLaTeX
\fi

% --- LWARP IS LOADED NEXT ---

\usepackage[
% HomeHTMLFilename=index, % Filename of the homepage.
% HTMLFilename={node-}, % Filename prefix of other pages.
% IndexLanguage=english, % Language for xindy index, glossary.
% latexmk, % Use latexmk to compile.
% OSWindows, % Force Windows. (Usually automatic.)
mathjax, % Use MathJax to display math.
]{lwarp}

% \boolfalse{FileSectionNames} % If false, numbers the files.

% --- LOAD PDFLATEX MATH FONTS HERE ---

% --- OTHER PACKAGES ARE LOADED AFTER LWARP ---

\usepackage{standalone}

\usepackage{tikz}
\usepackage{amsthm}
\usepackage{amsmath}
\usepackage{amssymb}
\usepackage{algorithm}
\usepackage{algpseudocode}

\usetikzlibrary{graphs, positioning, shapes.geometric}

\usepackage{makeidx} \makeindex
\usepackage{xcolor}               % (Demonstration purposes only.)
\usepackage{hyperref,cleveref}    % LOAD THESE LAST!

% Declare theorem environments for amsthm
\newtheorem{axiom}{axiom}
\newtheorem{lemma}{Lemma}
\newtheorem{theorem}{Theorem}
\newtheorem{example}{Example}
\newtheorem{definition}{Definition}

% --- LATEX AND HTML CUSTOMIZATION ---
\title{Notes on Everything}
\author{Anthony}
\date{\today}

\setcounter{tocdepth}{1}
\setcounter{secnumdepth}{-1}
\setcounter{FileDepth}{1}
\booltrue{CombineHigherDepths}
\setcounter{SideTOCDepth}{2}

% Renew commands

\renewcommand*\contentsname{Subjects}
\renewcommand\sidetocname{Subjects}

% HTML Directives
\HTMLTitle{Notes on Everything}
\HTMLAuthor{Anthony}
\HTMLLanguage{en-US}
\HTMLDescription{Personal notes on Mathematics and Computer Science}
\HTMLPageBottom{\LinkHome}

% Styling
\CSSFilename{lwarp.css}

\begin{document}

\maketitle % Or titlepage/titlingpage environment.
% An article abstract would go here.

Notes on Mathematics and Computer Science.

\tableofcontents % MUST BE BEFORE THE FIRST SECTION BREAK!

\listoffigures

\input{./sums/index}
\input{./algorithms/index}
\input{./combinatorics/index}
\input{./probability/index}
\input{./foundations/index}

\ForceHTMLPage 	% HTML index will be on its own page.
\ForceHTMLTOC 	% HTML index will have its own toc entry.
\printindex
\end{document}

% Save this as tutorial.tex for the lwarp package tutorial.
\documentclass{book}
\usepackage{iftex}

% --- LOAD FONT SELECTION AND ENCODING BEFORE LOADING LWARP ---
\ifPDFTeX
	\usepackage{lmodern} % pdflatex or dvi latex
	\usepackage[T1]{fontenc}
	\usepackage[utf8]{inputenc}
\else
	\usepackage{fontspec} % XeLaTeX or LuaLaTeX
\fi

% --- LWARP IS LOADED NEXT ---

\usepackage[
% HomeHTMLFilename=index, % Filename of the homepage.
% HTMLFilename={node-}, % Filename prefix of other pages.
% IndexLanguage=english, % Language for xindy index, glossary.
% latexmk, % Use latexmk to compile.
% OSWindows, % Force Windows. (Usually automatic.)
mathjax, % Use MathJax to display math.
]{lwarp}

% \boolfalse{FileSectionNames} % If false, numbers the files.

% --- LOAD PDFLATEX MATH FONTS HERE ---

% --- OTHER PACKAGES ARE LOADED AFTER LWARP ---

\usepackage{standalone}

\usepackage{tikz}
\usepackage{amsthm}
\usepackage{amsmath}
\usepackage{amssymb}
\usepackage{algorithm}
\usepackage{algpseudocode}

\usetikzlibrary{graphs, positioning, shapes.geometric}

\usepackage{makeidx} \makeindex
\usepackage{xcolor}               % (Demonstration purposes only.)
\usepackage{hyperref,cleveref}    % LOAD THESE LAST!

% Declare theorem environments for amsthm
\newtheorem{axiom}{axiom}
\newtheorem{lemma}{Lemma}
\newtheorem{theorem}{Theorem}
\newtheorem{example}{Example}
\newtheorem{definition}{Definition}

% --- LATEX AND HTML CUSTOMIZATION ---
\title{Notes on Everything}
\author{Anthony}
\date{\today}

\setcounter{tocdepth}{1}
\setcounter{secnumdepth}{-1}
\setcounter{FileDepth}{1}
\booltrue{CombineHigherDepths}
\setcounter{SideTOCDepth}{2}

% Renew commands

\renewcommand*\contentsname{Subjects}
\renewcommand\sidetocname{Subjects}

% HTML Directives
\HTMLTitle{Notes on Everything}
\HTMLAuthor{Anthony}
\HTMLLanguage{en-US}
\HTMLDescription{Personal notes on Mathematics and Computer Science}
\HTMLPageBottom{\LinkHome}

% Styling
\CSSFilename{lwarp.css}

\begin{document}

\maketitle % Or titlepage/titlingpage environment.
% An article abstract would go here.

Notes on Mathematics and Computer Science.

\tableofcontents % MUST BE BEFORE THE FIRST SECTION BREAK!

\listoffigures

\input{./sums/index}
\input{./algorithms/index}
\input{./combinatorics/index}
\input{./probability/index}
\input{./foundations/index}

\ForceHTMLPage 	% HTML index will be on its own page.
\ForceHTMLTOC 	% HTML index will have its own toc entry.
\printindex
\end{document}

% Save this as tutorial.tex for the lwarp package tutorial.
\documentclass{book}
\usepackage{iftex}

% --- LOAD FONT SELECTION AND ENCODING BEFORE LOADING LWARP ---
\ifPDFTeX
	\usepackage{lmodern} % pdflatex or dvi latex
	\usepackage[T1]{fontenc}
	\usepackage[utf8]{inputenc}
\else
	\usepackage{fontspec} % XeLaTeX or LuaLaTeX
\fi

% --- LWARP IS LOADED NEXT ---

\usepackage[
% HomeHTMLFilename=index, % Filename of the homepage.
% HTMLFilename={node-}, % Filename prefix of other pages.
% IndexLanguage=english, % Language for xindy index, glossary.
% latexmk, % Use latexmk to compile.
% OSWindows, % Force Windows. (Usually automatic.)
mathjax, % Use MathJax to display math.
]{lwarp}

% \boolfalse{FileSectionNames} % If false, numbers the files.

% --- LOAD PDFLATEX MATH FONTS HERE ---

% --- OTHER PACKAGES ARE LOADED AFTER LWARP ---

\usepackage{standalone}

\usepackage{tikz}
\usepackage{amsthm}
\usepackage{amsmath}
\usepackage{amssymb}
\usepackage{algorithm}
\usepackage{algpseudocode}

\usetikzlibrary{graphs, positioning, shapes.geometric}

\usepackage{makeidx} \makeindex
\usepackage{xcolor}               % (Demonstration purposes only.)
\usepackage{hyperref,cleveref}    % LOAD THESE LAST!

% Declare theorem environments for amsthm
\newtheorem{axiom}{axiom}
\newtheorem{lemma}{Lemma}
\newtheorem{theorem}{Theorem}
\newtheorem{example}{Example}
\newtheorem{definition}{Definition}

% --- LATEX AND HTML CUSTOMIZATION ---
\title{Notes on Everything}
\author{Anthony}
\date{\today}

\setcounter{tocdepth}{1}
\setcounter{secnumdepth}{-1}
\setcounter{FileDepth}{1}
\booltrue{CombineHigherDepths}
\setcounter{SideTOCDepth}{2}

% Renew commands

\renewcommand*\contentsname{Subjects}
\renewcommand\sidetocname{Subjects}

% HTML Directives
\HTMLTitle{Notes on Everything}
\HTMLAuthor{Anthony}
\HTMLLanguage{en-US}
\HTMLDescription{Personal notes on Mathematics and Computer Science}
\HTMLPageBottom{\LinkHome}

% Styling
\CSSFilename{lwarp.css}

\begin{document}

\maketitle % Or titlepage/titlingpage environment.
% An article abstract would go here.

Notes on Mathematics and Computer Science.

\tableofcontents % MUST BE BEFORE THE FIRST SECTION BREAK!

\listoffigures

\input{./sums/index}
\input{./algorithms/index}
\input{./combinatorics/index}
\input{./probability/index}
\input{./foundations/index}

\ForceHTMLPage 	% HTML index will be on its own page.
\ForceHTMLTOC 	% HTML index will have its own toc entry.
\printindex
\end{document}

% Save this as tutorial.tex for the lwarp package tutorial.
\documentclass{book}
\usepackage{iftex}

% --- LOAD FONT SELECTION AND ENCODING BEFORE LOADING LWARP ---
\ifPDFTeX
	\usepackage{lmodern} % pdflatex or dvi latex
	\usepackage[T1]{fontenc}
	\usepackage[utf8]{inputenc}
\else
	\usepackage{fontspec} % XeLaTeX or LuaLaTeX
\fi

% --- LWARP IS LOADED NEXT ---

\usepackage[
% HomeHTMLFilename=index, % Filename of the homepage.
% HTMLFilename={node-}, % Filename prefix of other pages.
% IndexLanguage=english, % Language for xindy index, glossary.
% latexmk, % Use latexmk to compile.
% OSWindows, % Force Windows. (Usually automatic.)
mathjax, % Use MathJax to display math.
]{lwarp}

% \boolfalse{FileSectionNames} % If false, numbers the files.

% --- LOAD PDFLATEX MATH FONTS HERE ---

% --- OTHER PACKAGES ARE LOADED AFTER LWARP ---

\usepackage{standalone}

\usepackage{tikz}
\usepackage{amsthm}
\usepackage{amsmath}
\usepackage{amssymb}
\usepackage{algorithm}
\usepackage{algpseudocode}

\usetikzlibrary{graphs, positioning, shapes.geometric}

\usepackage{makeidx} \makeindex
\usepackage{xcolor}               % (Demonstration purposes only.)
\usepackage{hyperref,cleveref}    % LOAD THESE LAST!

% Declare theorem environments for amsthm
\newtheorem{axiom}{axiom}
\newtheorem{lemma}{Lemma}
\newtheorem{theorem}{Theorem}
\newtheorem{example}{Example}
\newtheorem{definition}{Definition}

% --- LATEX AND HTML CUSTOMIZATION ---
\title{Notes on Everything}
\author{Anthony}
\date{\today}

\setcounter{tocdepth}{1}
\setcounter{secnumdepth}{-1}
\setcounter{FileDepth}{1}
\booltrue{CombineHigherDepths}
\setcounter{SideTOCDepth}{2}

% Renew commands

\renewcommand*\contentsname{Subjects}
\renewcommand\sidetocname{Subjects}

% HTML Directives
\HTMLTitle{Notes on Everything}
\HTMLAuthor{Anthony}
\HTMLLanguage{en-US}
\HTMLDescription{Personal notes on Mathematics and Computer Science}
\HTMLPageBottom{\LinkHome}

% Styling
\CSSFilename{lwarp.css}

\begin{document}

\maketitle % Or titlepage/titlingpage environment.
% An article abstract would go here.

Notes on Mathematics and Computer Science.

\tableofcontents % MUST BE BEFORE THE FIRST SECTION BREAK!

\listoffigures

\input{./sums/index}
\input{./algorithms/index}
\input{./combinatorics/index}
\input{./probability/index}
\input{./foundations/index}

\ForceHTMLPage 	% HTML index will be on its own page.
\ForceHTMLTOC 	% HTML index will have its own toc entry.
\printindex
\end{document}


\ForceHTMLPage 	% HTML index will be on its own page.
\ForceHTMLTOC 	% HTML index will have its own toc entry.
\printindex
\end{document}

% Save this as tutorial.tex for the lwarp package tutorial.
\documentclass{book}
\usepackage{iftex}

% --- LOAD FONT SELECTION AND ENCODING BEFORE LOADING LWARP ---
\ifPDFTeX
	\usepackage{lmodern} % pdflatex or dvi latex
	\usepackage[T1]{fontenc}
	\usepackage[utf8]{inputenc}
\else
	\usepackage{fontspec} % XeLaTeX or LuaLaTeX
\fi

% --- LWARP IS LOADED NEXT ---

\usepackage[
% HomeHTMLFilename=index, % Filename of the homepage.
% HTMLFilename={node-}, % Filename prefix of other pages.
% IndexLanguage=english, % Language for xindy index, glossary.
% latexmk, % Use latexmk to compile.
% OSWindows, % Force Windows. (Usually automatic.)
mathjax, % Use MathJax to display math.
]{lwarp}

% \boolfalse{FileSectionNames} % If false, numbers the files.

% --- LOAD PDFLATEX MATH FONTS HERE ---

% --- OTHER PACKAGES ARE LOADED AFTER LWARP ---

\usepackage{standalone}

\usepackage{tikz}
\usepackage{amsthm}
\usepackage{amsmath}
\usepackage{amssymb}
\usepackage{algorithm}
\usepackage{algpseudocode}

\usetikzlibrary{graphs, positioning, shapes.geometric}

\usepackage{makeidx} \makeindex
\usepackage{xcolor}               % (Demonstration purposes only.)
\usepackage{hyperref,cleveref}    % LOAD THESE LAST!

% Declare theorem environments for amsthm
\newtheorem{axiom}{axiom}
\newtheorem{lemma}{Lemma}
\newtheorem{theorem}{Theorem}
\newtheorem{example}{Example}
\newtheorem{definition}{Definition}

% --- LATEX AND HTML CUSTOMIZATION ---
\title{Notes on Everything}
\author{Anthony}
\date{\today}

\setcounter{tocdepth}{1}
\setcounter{secnumdepth}{-1}
\setcounter{FileDepth}{1}
\booltrue{CombineHigherDepths}
\setcounter{SideTOCDepth}{2}

% Renew commands

\renewcommand*\contentsname{Subjects}
\renewcommand\sidetocname{Subjects}

% HTML Directives
\HTMLTitle{Notes on Everything}
\HTMLAuthor{Anthony}
\HTMLLanguage{en-US}
\HTMLDescription{Personal notes on Mathematics and Computer Science}
\HTMLPageBottom{\LinkHome}

% Styling
\CSSFilename{lwarp.css}

\begin{document}

\maketitle % Or titlepage/titlingpage environment.
% An article abstract would go here.

Notes on Mathematics and Computer Science.

\tableofcontents % MUST BE BEFORE THE FIRST SECTION BREAK!

\listoffigures

% Save this as tutorial.tex for the lwarp package tutorial.
\documentclass{book}
\usepackage{iftex}

% --- LOAD FONT SELECTION AND ENCODING BEFORE LOADING LWARP ---
\ifPDFTeX
	\usepackage{lmodern} % pdflatex or dvi latex
	\usepackage[T1]{fontenc}
	\usepackage[utf8]{inputenc}
\else
	\usepackage{fontspec} % XeLaTeX or LuaLaTeX
\fi

% --- LWARP IS LOADED NEXT ---

\usepackage[
% HomeHTMLFilename=index, % Filename of the homepage.
% HTMLFilename={node-}, % Filename prefix of other pages.
% IndexLanguage=english, % Language for xindy index, glossary.
% latexmk, % Use latexmk to compile.
% OSWindows, % Force Windows. (Usually automatic.)
mathjax, % Use MathJax to display math.
]{lwarp}

% \boolfalse{FileSectionNames} % If false, numbers the files.

% --- LOAD PDFLATEX MATH FONTS HERE ---

% --- OTHER PACKAGES ARE LOADED AFTER LWARP ---

\usepackage{standalone}

\usepackage{tikz}
\usepackage{amsthm}
\usepackage{amsmath}
\usepackage{amssymb}
\usepackage{algorithm}
\usepackage{algpseudocode}

\usetikzlibrary{graphs, positioning, shapes.geometric}

\usepackage{makeidx} \makeindex
\usepackage{xcolor}               % (Demonstration purposes only.)
\usepackage{hyperref,cleveref}    % LOAD THESE LAST!

% Declare theorem environments for amsthm
\newtheorem{axiom}{axiom}
\newtheorem{lemma}{Lemma}
\newtheorem{theorem}{Theorem}
\newtheorem{example}{Example}
\newtheorem{definition}{Definition}

% --- LATEX AND HTML CUSTOMIZATION ---
\title{Notes on Everything}
\author{Anthony}
\date{\today}

\setcounter{tocdepth}{1}
\setcounter{secnumdepth}{-1}
\setcounter{FileDepth}{1}
\booltrue{CombineHigherDepths}
\setcounter{SideTOCDepth}{2}

% Renew commands

\renewcommand*\contentsname{Subjects}
\renewcommand\sidetocname{Subjects}

% HTML Directives
\HTMLTitle{Notes on Everything}
\HTMLAuthor{Anthony}
\HTMLLanguage{en-US}
\HTMLDescription{Personal notes on Mathematics and Computer Science}
\HTMLPageBottom{\LinkHome}

% Styling
\CSSFilename{lwarp.css}

\begin{document}

\maketitle % Or titlepage/titlingpage environment.
% An article abstract would go here.

Notes on Mathematics and Computer Science.

\tableofcontents % MUST BE BEFORE THE FIRST SECTION BREAK!

\listoffigures

\input{./sums/index}
\input{./algorithms/index}
\input{./combinatorics/index}
\input{./probability/index}
\input{./foundations/index}

\ForceHTMLPage 	% HTML index will be on its own page.
\ForceHTMLTOC 	% HTML index will have its own toc entry.
\printindex
\end{document}

% Save this as tutorial.tex for the lwarp package tutorial.
\documentclass{book}
\usepackage{iftex}

% --- LOAD FONT SELECTION AND ENCODING BEFORE LOADING LWARP ---
\ifPDFTeX
	\usepackage{lmodern} % pdflatex or dvi latex
	\usepackage[T1]{fontenc}
	\usepackage[utf8]{inputenc}
\else
	\usepackage{fontspec} % XeLaTeX or LuaLaTeX
\fi

% --- LWARP IS LOADED NEXT ---

\usepackage[
% HomeHTMLFilename=index, % Filename of the homepage.
% HTMLFilename={node-}, % Filename prefix of other pages.
% IndexLanguage=english, % Language for xindy index, glossary.
% latexmk, % Use latexmk to compile.
% OSWindows, % Force Windows. (Usually automatic.)
mathjax, % Use MathJax to display math.
]{lwarp}

% \boolfalse{FileSectionNames} % If false, numbers the files.

% --- LOAD PDFLATEX MATH FONTS HERE ---

% --- OTHER PACKAGES ARE LOADED AFTER LWARP ---

\usepackage{standalone}

\usepackage{tikz}
\usepackage{amsthm}
\usepackage{amsmath}
\usepackage{amssymb}
\usepackage{algorithm}
\usepackage{algpseudocode}

\usetikzlibrary{graphs, positioning, shapes.geometric}

\usepackage{makeidx} \makeindex
\usepackage{xcolor}               % (Demonstration purposes only.)
\usepackage{hyperref,cleveref}    % LOAD THESE LAST!

% Declare theorem environments for amsthm
\newtheorem{axiom}{axiom}
\newtheorem{lemma}{Lemma}
\newtheorem{theorem}{Theorem}
\newtheorem{example}{Example}
\newtheorem{definition}{Definition}

% --- LATEX AND HTML CUSTOMIZATION ---
\title{Notes on Everything}
\author{Anthony}
\date{\today}

\setcounter{tocdepth}{1}
\setcounter{secnumdepth}{-1}
\setcounter{FileDepth}{1}
\booltrue{CombineHigherDepths}
\setcounter{SideTOCDepth}{2}

% Renew commands

\renewcommand*\contentsname{Subjects}
\renewcommand\sidetocname{Subjects}

% HTML Directives
\HTMLTitle{Notes on Everything}
\HTMLAuthor{Anthony}
\HTMLLanguage{en-US}
\HTMLDescription{Personal notes on Mathematics and Computer Science}
\HTMLPageBottom{\LinkHome}

% Styling
\CSSFilename{lwarp.css}

\begin{document}

\maketitle % Or titlepage/titlingpage environment.
% An article abstract would go here.

Notes on Mathematics and Computer Science.

\tableofcontents % MUST BE BEFORE THE FIRST SECTION BREAK!

\listoffigures

\input{./sums/index}
\input{./algorithms/index}
\input{./combinatorics/index}
\input{./probability/index}
\input{./foundations/index}

\ForceHTMLPage 	% HTML index will be on its own page.
\ForceHTMLTOC 	% HTML index will have its own toc entry.
\printindex
\end{document}

% Save this as tutorial.tex for the lwarp package tutorial.
\documentclass{book}
\usepackage{iftex}

% --- LOAD FONT SELECTION AND ENCODING BEFORE LOADING LWARP ---
\ifPDFTeX
	\usepackage{lmodern} % pdflatex or dvi latex
	\usepackage[T1]{fontenc}
	\usepackage[utf8]{inputenc}
\else
	\usepackage{fontspec} % XeLaTeX or LuaLaTeX
\fi

% --- LWARP IS LOADED NEXT ---

\usepackage[
% HomeHTMLFilename=index, % Filename of the homepage.
% HTMLFilename={node-}, % Filename prefix of other pages.
% IndexLanguage=english, % Language for xindy index, glossary.
% latexmk, % Use latexmk to compile.
% OSWindows, % Force Windows. (Usually automatic.)
mathjax, % Use MathJax to display math.
]{lwarp}

% \boolfalse{FileSectionNames} % If false, numbers the files.

% --- LOAD PDFLATEX MATH FONTS HERE ---

% --- OTHER PACKAGES ARE LOADED AFTER LWARP ---

\usepackage{standalone}

\usepackage{tikz}
\usepackage{amsthm}
\usepackage{amsmath}
\usepackage{amssymb}
\usepackage{algorithm}
\usepackage{algpseudocode}

\usetikzlibrary{graphs, positioning, shapes.geometric}

\usepackage{makeidx} \makeindex
\usepackage{xcolor}               % (Demonstration purposes only.)
\usepackage{hyperref,cleveref}    % LOAD THESE LAST!

% Declare theorem environments for amsthm
\newtheorem{axiom}{axiom}
\newtheorem{lemma}{Lemma}
\newtheorem{theorem}{Theorem}
\newtheorem{example}{Example}
\newtheorem{definition}{Definition}

% --- LATEX AND HTML CUSTOMIZATION ---
\title{Notes on Everything}
\author{Anthony}
\date{\today}

\setcounter{tocdepth}{1}
\setcounter{secnumdepth}{-1}
\setcounter{FileDepth}{1}
\booltrue{CombineHigherDepths}
\setcounter{SideTOCDepth}{2}

% Renew commands

\renewcommand*\contentsname{Subjects}
\renewcommand\sidetocname{Subjects}

% HTML Directives
\HTMLTitle{Notes on Everything}
\HTMLAuthor{Anthony}
\HTMLLanguage{en-US}
\HTMLDescription{Personal notes on Mathematics and Computer Science}
\HTMLPageBottom{\LinkHome}

% Styling
\CSSFilename{lwarp.css}

\begin{document}

\maketitle % Or titlepage/titlingpage environment.
% An article abstract would go here.

Notes on Mathematics and Computer Science.

\tableofcontents % MUST BE BEFORE THE FIRST SECTION BREAK!

\listoffigures

\input{./sums/index}
\input{./algorithms/index}
\input{./combinatorics/index}
\input{./probability/index}
\input{./foundations/index}

\ForceHTMLPage 	% HTML index will be on its own page.
\ForceHTMLTOC 	% HTML index will have its own toc entry.
\printindex
\end{document}

% Save this as tutorial.tex for the lwarp package tutorial.
\documentclass{book}
\usepackage{iftex}

% --- LOAD FONT SELECTION AND ENCODING BEFORE LOADING LWARP ---
\ifPDFTeX
	\usepackage{lmodern} % pdflatex or dvi latex
	\usepackage[T1]{fontenc}
	\usepackage[utf8]{inputenc}
\else
	\usepackage{fontspec} % XeLaTeX or LuaLaTeX
\fi

% --- LWARP IS LOADED NEXT ---

\usepackage[
% HomeHTMLFilename=index, % Filename of the homepage.
% HTMLFilename={node-}, % Filename prefix of other pages.
% IndexLanguage=english, % Language for xindy index, glossary.
% latexmk, % Use latexmk to compile.
% OSWindows, % Force Windows. (Usually automatic.)
mathjax, % Use MathJax to display math.
]{lwarp}

% \boolfalse{FileSectionNames} % If false, numbers the files.

% --- LOAD PDFLATEX MATH FONTS HERE ---

% --- OTHER PACKAGES ARE LOADED AFTER LWARP ---

\usepackage{standalone}

\usepackage{tikz}
\usepackage{amsthm}
\usepackage{amsmath}
\usepackage{amssymb}
\usepackage{algorithm}
\usepackage{algpseudocode}

\usetikzlibrary{graphs, positioning, shapes.geometric}

\usepackage{makeidx} \makeindex
\usepackage{xcolor}               % (Demonstration purposes only.)
\usepackage{hyperref,cleveref}    % LOAD THESE LAST!

% Declare theorem environments for amsthm
\newtheorem{axiom}{axiom}
\newtheorem{lemma}{Lemma}
\newtheorem{theorem}{Theorem}
\newtheorem{example}{Example}
\newtheorem{definition}{Definition}

% --- LATEX AND HTML CUSTOMIZATION ---
\title{Notes on Everything}
\author{Anthony}
\date{\today}

\setcounter{tocdepth}{1}
\setcounter{secnumdepth}{-1}
\setcounter{FileDepth}{1}
\booltrue{CombineHigherDepths}
\setcounter{SideTOCDepth}{2}

% Renew commands

\renewcommand*\contentsname{Subjects}
\renewcommand\sidetocname{Subjects}

% HTML Directives
\HTMLTitle{Notes on Everything}
\HTMLAuthor{Anthony}
\HTMLLanguage{en-US}
\HTMLDescription{Personal notes on Mathematics and Computer Science}
\HTMLPageBottom{\LinkHome}

% Styling
\CSSFilename{lwarp.css}

\begin{document}

\maketitle % Or titlepage/titlingpage environment.
% An article abstract would go here.

Notes on Mathematics and Computer Science.

\tableofcontents % MUST BE BEFORE THE FIRST SECTION BREAK!

\listoffigures

\input{./sums/index}
\input{./algorithms/index}
\input{./combinatorics/index}
\input{./probability/index}
\input{./foundations/index}

\ForceHTMLPage 	% HTML index will be on its own page.
\ForceHTMLTOC 	% HTML index will have its own toc entry.
\printindex
\end{document}

% Save this as tutorial.tex for the lwarp package tutorial.
\documentclass{book}
\usepackage{iftex}

% --- LOAD FONT SELECTION AND ENCODING BEFORE LOADING LWARP ---
\ifPDFTeX
	\usepackage{lmodern} % pdflatex or dvi latex
	\usepackage[T1]{fontenc}
	\usepackage[utf8]{inputenc}
\else
	\usepackage{fontspec} % XeLaTeX or LuaLaTeX
\fi

% --- LWARP IS LOADED NEXT ---

\usepackage[
% HomeHTMLFilename=index, % Filename of the homepage.
% HTMLFilename={node-}, % Filename prefix of other pages.
% IndexLanguage=english, % Language for xindy index, glossary.
% latexmk, % Use latexmk to compile.
% OSWindows, % Force Windows. (Usually automatic.)
mathjax, % Use MathJax to display math.
]{lwarp}

% \boolfalse{FileSectionNames} % If false, numbers the files.

% --- LOAD PDFLATEX MATH FONTS HERE ---

% --- OTHER PACKAGES ARE LOADED AFTER LWARP ---

\usepackage{standalone}

\usepackage{tikz}
\usepackage{amsthm}
\usepackage{amsmath}
\usepackage{amssymb}
\usepackage{algorithm}
\usepackage{algpseudocode}

\usetikzlibrary{graphs, positioning, shapes.geometric}

\usepackage{makeidx} \makeindex
\usepackage{xcolor}               % (Demonstration purposes only.)
\usepackage{hyperref,cleveref}    % LOAD THESE LAST!

% Declare theorem environments for amsthm
\newtheorem{axiom}{axiom}
\newtheorem{lemma}{Lemma}
\newtheorem{theorem}{Theorem}
\newtheorem{example}{Example}
\newtheorem{definition}{Definition}

% --- LATEX AND HTML CUSTOMIZATION ---
\title{Notes on Everything}
\author{Anthony}
\date{\today}

\setcounter{tocdepth}{1}
\setcounter{secnumdepth}{-1}
\setcounter{FileDepth}{1}
\booltrue{CombineHigherDepths}
\setcounter{SideTOCDepth}{2}

% Renew commands

\renewcommand*\contentsname{Subjects}
\renewcommand\sidetocname{Subjects}

% HTML Directives
\HTMLTitle{Notes on Everything}
\HTMLAuthor{Anthony}
\HTMLLanguage{en-US}
\HTMLDescription{Personal notes on Mathematics and Computer Science}
\HTMLPageBottom{\LinkHome}

% Styling
\CSSFilename{lwarp.css}

\begin{document}

\maketitle % Or titlepage/titlingpage environment.
% An article abstract would go here.

Notes on Mathematics and Computer Science.

\tableofcontents % MUST BE BEFORE THE FIRST SECTION BREAK!

\listoffigures

\input{./sums/index}
\input{./algorithms/index}
\input{./combinatorics/index}
\input{./probability/index}
\input{./foundations/index}

\ForceHTMLPage 	% HTML index will be on its own page.
\ForceHTMLTOC 	% HTML index will have its own toc entry.
\printindex
\end{document}


\ForceHTMLPage 	% HTML index will be on its own page.
\ForceHTMLTOC 	% HTML index will have its own toc entry.
\printindex
\end{document}

% Save this as tutorial.tex for the lwarp package tutorial.
\documentclass{book}
\usepackage{iftex}

% --- LOAD FONT SELECTION AND ENCODING BEFORE LOADING LWARP ---
\ifPDFTeX
	\usepackage{lmodern} % pdflatex or dvi latex
	\usepackage[T1]{fontenc}
	\usepackage[utf8]{inputenc}
\else
	\usepackage{fontspec} % XeLaTeX or LuaLaTeX
\fi

% --- LWARP IS LOADED NEXT ---

\usepackage[
% HomeHTMLFilename=index, % Filename of the homepage.
% HTMLFilename={node-}, % Filename prefix of other pages.
% IndexLanguage=english, % Language for xindy index, glossary.
% latexmk, % Use latexmk to compile.
% OSWindows, % Force Windows. (Usually automatic.)
mathjax, % Use MathJax to display math.
]{lwarp}

% \boolfalse{FileSectionNames} % If false, numbers the files.

% --- LOAD PDFLATEX MATH FONTS HERE ---

% --- OTHER PACKAGES ARE LOADED AFTER LWARP ---

\usepackage{standalone}

\usepackage{tikz}
\usepackage{amsthm}
\usepackage{amsmath}
\usepackage{amssymb}
\usepackage{algorithm}
\usepackage{algpseudocode}

\usetikzlibrary{graphs, positioning, shapes.geometric}

\usepackage{makeidx} \makeindex
\usepackage{xcolor}               % (Demonstration purposes only.)
\usepackage{hyperref,cleveref}    % LOAD THESE LAST!

% Declare theorem environments for amsthm
\newtheorem{axiom}{axiom}
\newtheorem{lemma}{Lemma}
\newtheorem{theorem}{Theorem}
\newtheorem{example}{Example}
\newtheorem{definition}{Definition}

% --- LATEX AND HTML CUSTOMIZATION ---
\title{Notes on Everything}
\author{Anthony}
\date{\today}

\setcounter{tocdepth}{1}
\setcounter{secnumdepth}{-1}
\setcounter{FileDepth}{1}
\booltrue{CombineHigherDepths}
\setcounter{SideTOCDepth}{2}

% Renew commands

\renewcommand*\contentsname{Subjects}
\renewcommand\sidetocname{Subjects}

% HTML Directives
\HTMLTitle{Notes on Everything}
\HTMLAuthor{Anthony}
\HTMLLanguage{en-US}
\HTMLDescription{Personal notes on Mathematics and Computer Science}
\HTMLPageBottom{\LinkHome}

% Styling
\CSSFilename{lwarp.css}

\begin{document}

\maketitle % Or titlepage/titlingpage environment.
% An article abstract would go here.

Notes on Mathematics and Computer Science.

\tableofcontents % MUST BE BEFORE THE FIRST SECTION BREAK!

\listoffigures

% Save this as tutorial.tex for the lwarp package tutorial.
\documentclass{book}
\usepackage{iftex}

% --- LOAD FONT SELECTION AND ENCODING BEFORE LOADING LWARP ---
\ifPDFTeX
	\usepackage{lmodern} % pdflatex or dvi latex
	\usepackage[T1]{fontenc}
	\usepackage[utf8]{inputenc}
\else
	\usepackage{fontspec} % XeLaTeX or LuaLaTeX
\fi

% --- LWARP IS LOADED NEXT ---

\usepackage[
% HomeHTMLFilename=index, % Filename of the homepage.
% HTMLFilename={node-}, % Filename prefix of other pages.
% IndexLanguage=english, % Language for xindy index, glossary.
% latexmk, % Use latexmk to compile.
% OSWindows, % Force Windows. (Usually automatic.)
mathjax, % Use MathJax to display math.
]{lwarp}

% \boolfalse{FileSectionNames} % If false, numbers the files.

% --- LOAD PDFLATEX MATH FONTS HERE ---

% --- OTHER PACKAGES ARE LOADED AFTER LWARP ---

\usepackage{standalone}

\usepackage{tikz}
\usepackage{amsthm}
\usepackage{amsmath}
\usepackage{amssymb}
\usepackage{algorithm}
\usepackage{algpseudocode}

\usetikzlibrary{graphs, positioning, shapes.geometric}

\usepackage{makeidx} \makeindex
\usepackage{xcolor}               % (Demonstration purposes only.)
\usepackage{hyperref,cleveref}    % LOAD THESE LAST!

% Declare theorem environments for amsthm
\newtheorem{axiom}{axiom}
\newtheorem{lemma}{Lemma}
\newtheorem{theorem}{Theorem}
\newtheorem{example}{Example}
\newtheorem{definition}{Definition}

% --- LATEX AND HTML CUSTOMIZATION ---
\title{Notes on Everything}
\author{Anthony}
\date{\today}

\setcounter{tocdepth}{1}
\setcounter{secnumdepth}{-1}
\setcounter{FileDepth}{1}
\booltrue{CombineHigherDepths}
\setcounter{SideTOCDepth}{2}

% Renew commands

\renewcommand*\contentsname{Subjects}
\renewcommand\sidetocname{Subjects}

% HTML Directives
\HTMLTitle{Notes on Everything}
\HTMLAuthor{Anthony}
\HTMLLanguage{en-US}
\HTMLDescription{Personal notes on Mathematics and Computer Science}
\HTMLPageBottom{\LinkHome}

% Styling
\CSSFilename{lwarp.css}

\begin{document}

\maketitle % Or titlepage/titlingpage environment.
% An article abstract would go here.

Notes on Mathematics and Computer Science.

\tableofcontents % MUST BE BEFORE THE FIRST SECTION BREAK!

\listoffigures

\input{./sums/index}
\input{./algorithms/index}
\input{./combinatorics/index}
\input{./probability/index}
\input{./foundations/index}

\ForceHTMLPage 	% HTML index will be on its own page.
\ForceHTMLTOC 	% HTML index will have its own toc entry.
\printindex
\end{document}

% Save this as tutorial.tex for the lwarp package tutorial.
\documentclass{book}
\usepackage{iftex}

% --- LOAD FONT SELECTION AND ENCODING BEFORE LOADING LWARP ---
\ifPDFTeX
	\usepackage{lmodern} % pdflatex or dvi latex
	\usepackage[T1]{fontenc}
	\usepackage[utf8]{inputenc}
\else
	\usepackage{fontspec} % XeLaTeX or LuaLaTeX
\fi

% --- LWARP IS LOADED NEXT ---

\usepackage[
% HomeHTMLFilename=index, % Filename of the homepage.
% HTMLFilename={node-}, % Filename prefix of other pages.
% IndexLanguage=english, % Language for xindy index, glossary.
% latexmk, % Use latexmk to compile.
% OSWindows, % Force Windows. (Usually automatic.)
mathjax, % Use MathJax to display math.
]{lwarp}

% \boolfalse{FileSectionNames} % If false, numbers the files.

% --- LOAD PDFLATEX MATH FONTS HERE ---

% --- OTHER PACKAGES ARE LOADED AFTER LWARP ---

\usepackage{standalone}

\usepackage{tikz}
\usepackage{amsthm}
\usepackage{amsmath}
\usepackage{amssymb}
\usepackage{algorithm}
\usepackage{algpseudocode}

\usetikzlibrary{graphs, positioning, shapes.geometric}

\usepackage{makeidx} \makeindex
\usepackage{xcolor}               % (Demonstration purposes only.)
\usepackage{hyperref,cleveref}    % LOAD THESE LAST!

% Declare theorem environments for amsthm
\newtheorem{axiom}{axiom}
\newtheorem{lemma}{Lemma}
\newtheorem{theorem}{Theorem}
\newtheorem{example}{Example}
\newtheorem{definition}{Definition}

% --- LATEX AND HTML CUSTOMIZATION ---
\title{Notes on Everything}
\author{Anthony}
\date{\today}

\setcounter{tocdepth}{1}
\setcounter{secnumdepth}{-1}
\setcounter{FileDepth}{1}
\booltrue{CombineHigherDepths}
\setcounter{SideTOCDepth}{2}

% Renew commands

\renewcommand*\contentsname{Subjects}
\renewcommand\sidetocname{Subjects}

% HTML Directives
\HTMLTitle{Notes on Everything}
\HTMLAuthor{Anthony}
\HTMLLanguage{en-US}
\HTMLDescription{Personal notes on Mathematics and Computer Science}
\HTMLPageBottom{\LinkHome}

% Styling
\CSSFilename{lwarp.css}

\begin{document}

\maketitle % Or titlepage/titlingpage environment.
% An article abstract would go here.

Notes on Mathematics and Computer Science.

\tableofcontents % MUST BE BEFORE THE FIRST SECTION BREAK!

\listoffigures

\input{./sums/index}
\input{./algorithms/index}
\input{./combinatorics/index}
\input{./probability/index}
\input{./foundations/index}

\ForceHTMLPage 	% HTML index will be on its own page.
\ForceHTMLTOC 	% HTML index will have its own toc entry.
\printindex
\end{document}

% Save this as tutorial.tex for the lwarp package tutorial.
\documentclass{book}
\usepackage{iftex}

% --- LOAD FONT SELECTION AND ENCODING BEFORE LOADING LWARP ---
\ifPDFTeX
	\usepackage{lmodern} % pdflatex or dvi latex
	\usepackage[T1]{fontenc}
	\usepackage[utf8]{inputenc}
\else
	\usepackage{fontspec} % XeLaTeX or LuaLaTeX
\fi

% --- LWARP IS LOADED NEXT ---

\usepackage[
% HomeHTMLFilename=index, % Filename of the homepage.
% HTMLFilename={node-}, % Filename prefix of other pages.
% IndexLanguage=english, % Language for xindy index, glossary.
% latexmk, % Use latexmk to compile.
% OSWindows, % Force Windows. (Usually automatic.)
mathjax, % Use MathJax to display math.
]{lwarp}

% \boolfalse{FileSectionNames} % If false, numbers the files.

% --- LOAD PDFLATEX MATH FONTS HERE ---

% --- OTHER PACKAGES ARE LOADED AFTER LWARP ---

\usepackage{standalone}

\usepackage{tikz}
\usepackage{amsthm}
\usepackage{amsmath}
\usepackage{amssymb}
\usepackage{algorithm}
\usepackage{algpseudocode}

\usetikzlibrary{graphs, positioning, shapes.geometric}

\usepackage{makeidx} \makeindex
\usepackage{xcolor}               % (Demonstration purposes only.)
\usepackage{hyperref,cleveref}    % LOAD THESE LAST!

% Declare theorem environments for amsthm
\newtheorem{axiom}{axiom}
\newtheorem{lemma}{Lemma}
\newtheorem{theorem}{Theorem}
\newtheorem{example}{Example}
\newtheorem{definition}{Definition}

% --- LATEX AND HTML CUSTOMIZATION ---
\title{Notes on Everything}
\author{Anthony}
\date{\today}

\setcounter{tocdepth}{1}
\setcounter{secnumdepth}{-1}
\setcounter{FileDepth}{1}
\booltrue{CombineHigherDepths}
\setcounter{SideTOCDepth}{2}

% Renew commands

\renewcommand*\contentsname{Subjects}
\renewcommand\sidetocname{Subjects}

% HTML Directives
\HTMLTitle{Notes on Everything}
\HTMLAuthor{Anthony}
\HTMLLanguage{en-US}
\HTMLDescription{Personal notes on Mathematics and Computer Science}
\HTMLPageBottom{\LinkHome}

% Styling
\CSSFilename{lwarp.css}

\begin{document}

\maketitle % Or titlepage/titlingpage environment.
% An article abstract would go here.

Notes on Mathematics and Computer Science.

\tableofcontents % MUST BE BEFORE THE FIRST SECTION BREAK!

\listoffigures

\input{./sums/index}
\input{./algorithms/index}
\input{./combinatorics/index}
\input{./probability/index}
\input{./foundations/index}

\ForceHTMLPage 	% HTML index will be on its own page.
\ForceHTMLTOC 	% HTML index will have its own toc entry.
\printindex
\end{document}

% Save this as tutorial.tex for the lwarp package tutorial.
\documentclass{book}
\usepackage{iftex}

% --- LOAD FONT SELECTION AND ENCODING BEFORE LOADING LWARP ---
\ifPDFTeX
	\usepackage{lmodern} % pdflatex or dvi latex
	\usepackage[T1]{fontenc}
	\usepackage[utf8]{inputenc}
\else
	\usepackage{fontspec} % XeLaTeX or LuaLaTeX
\fi

% --- LWARP IS LOADED NEXT ---

\usepackage[
% HomeHTMLFilename=index, % Filename of the homepage.
% HTMLFilename={node-}, % Filename prefix of other pages.
% IndexLanguage=english, % Language for xindy index, glossary.
% latexmk, % Use latexmk to compile.
% OSWindows, % Force Windows. (Usually automatic.)
mathjax, % Use MathJax to display math.
]{lwarp}

% \boolfalse{FileSectionNames} % If false, numbers the files.

% --- LOAD PDFLATEX MATH FONTS HERE ---

% --- OTHER PACKAGES ARE LOADED AFTER LWARP ---

\usepackage{standalone}

\usepackage{tikz}
\usepackage{amsthm}
\usepackage{amsmath}
\usepackage{amssymb}
\usepackage{algorithm}
\usepackage{algpseudocode}

\usetikzlibrary{graphs, positioning, shapes.geometric}

\usepackage{makeidx} \makeindex
\usepackage{xcolor}               % (Demonstration purposes only.)
\usepackage{hyperref,cleveref}    % LOAD THESE LAST!

% Declare theorem environments for amsthm
\newtheorem{axiom}{axiom}
\newtheorem{lemma}{Lemma}
\newtheorem{theorem}{Theorem}
\newtheorem{example}{Example}
\newtheorem{definition}{Definition}

% --- LATEX AND HTML CUSTOMIZATION ---
\title{Notes on Everything}
\author{Anthony}
\date{\today}

\setcounter{tocdepth}{1}
\setcounter{secnumdepth}{-1}
\setcounter{FileDepth}{1}
\booltrue{CombineHigherDepths}
\setcounter{SideTOCDepth}{2}

% Renew commands

\renewcommand*\contentsname{Subjects}
\renewcommand\sidetocname{Subjects}

% HTML Directives
\HTMLTitle{Notes on Everything}
\HTMLAuthor{Anthony}
\HTMLLanguage{en-US}
\HTMLDescription{Personal notes on Mathematics and Computer Science}
\HTMLPageBottom{\LinkHome}

% Styling
\CSSFilename{lwarp.css}

\begin{document}

\maketitle % Or titlepage/titlingpage environment.
% An article abstract would go here.

Notes on Mathematics and Computer Science.

\tableofcontents % MUST BE BEFORE THE FIRST SECTION BREAK!

\listoffigures

\input{./sums/index}
\input{./algorithms/index}
\input{./combinatorics/index}
\input{./probability/index}
\input{./foundations/index}

\ForceHTMLPage 	% HTML index will be on its own page.
\ForceHTMLTOC 	% HTML index will have its own toc entry.
\printindex
\end{document}

% Save this as tutorial.tex for the lwarp package tutorial.
\documentclass{book}
\usepackage{iftex}

% --- LOAD FONT SELECTION AND ENCODING BEFORE LOADING LWARP ---
\ifPDFTeX
	\usepackage{lmodern} % pdflatex or dvi latex
	\usepackage[T1]{fontenc}
	\usepackage[utf8]{inputenc}
\else
	\usepackage{fontspec} % XeLaTeX or LuaLaTeX
\fi

% --- LWARP IS LOADED NEXT ---

\usepackage[
% HomeHTMLFilename=index, % Filename of the homepage.
% HTMLFilename={node-}, % Filename prefix of other pages.
% IndexLanguage=english, % Language for xindy index, glossary.
% latexmk, % Use latexmk to compile.
% OSWindows, % Force Windows. (Usually automatic.)
mathjax, % Use MathJax to display math.
]{lwarp}

% \boolfalse{FileSectionNames} % If false, numbers the files.

% --- LOAD PDFLATEX MATH FONTS HERE ---

% --- OTHER PACKAGES ARE LOADED AFTER LWARP ---

\usepackage{standalone}

\usepackage{tikz}
\usepackage{amsthm}
\usepackage{amsmath}
\usepackage{amssymb}
\usepackage{algorithm}
\usepackage{algpseudocode}

\usetikzlibrary{graphs, positioning, shapes.geometric}

\usepackage{makeidx} \makeindex
\usepackage{xcolor}               % (Demonstration purposes only.)
\usepackage{hyperref,cleveref}    % LOAD THESE LAST!

% Declare theorem environments for amsthm
\newtheorem{axiom}{axiom}
\newtheorem{lemma}{Lemma}
\newtheorem{theorem}{Theorem}
\newtheorem{example}{Example}
\newtheorem{definition}{Definition}

% --- LATEX AND HTML CUSTOMIZATION ---
\title{Notes on Everything}
\author{Anthony}
\date{\today}

\setcounter{tocdepth}{1}
\setcounter{secnumdepth}{-1}
\setcounter{FileDepth}{1}
\booltrue{CombineHigherDepths}
\setcounter{SideTOCDepth}{2}

% Renew commands

\renewcommand*\contentsname{Subjects}
\renewcommand\sidetocname{Subjects}

% HTML Directives
\HTMLTitle{Notes on Everything}
\HTMLAuthor{Anthony}
\HTMLLanguage{en-US}
\HTMLDescription{Personal notes on Mathematics and Computer Science}
\HTMLPageBottom{\LinkHome}

% Styling
\CSSFilename{lwarp.css}

\begin{document}

\maketitle % Or titlepage/titlingpage environment.
% An article abstract would go here.

Notes on Mathematics and Computer Science.

\tableofcontents % MUST BE BEFORE THE FIRST SECTION BREAK!

\listoffigures

\input{./sums/index}
\input{./algorithms/index}
\input{./combinatorics/index}
\input{./probability/index}
\input{./foundations/index}

\ForceHTMLPage 	% HTML index will be on its own page.
\ForceHTMLTOC 	% HTML index will have its own toc entry.
\printindex
\end{document}


\ForceHTMLPage 	% HTML index will be on its own page.
\ForceHTMLTOC 	% HTML index will have its own toc entry.
\printindex
\end{document}


\ForceHTMLPage 	% HTML index will be on its own page.
\ForceHTMLTOC 	% HTML index will have its own toc entry.
\printindex
\end{document}


\ForceHTMLPage 	% HTML index will be on its own page.
\ForceHTMLTOC 	% HTML index will have its own toc entry.
\printindex
\end{document}
