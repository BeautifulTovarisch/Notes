\documentclass{standalone}
\begin{document}

\section{Manipulating Sums}

Techniques for manipulating summations can often lead to a closed form solution or simpler summation.
This is especially helpful when dealing with multiple summations and loops in which the inner index depends on the outer.

\subsection{Basic Rules}

Often times a complex summation can be simplified using a few simple rules:

\begin{align*}
  &\sum_{k \in K} c a_k = c \sum_{k \in K} a_k &\text{Distribution} \\
  &\sum_{k \in K} (a_k + b_k) = \sum_{k \in K} a_k + \sum_{k \in K} b_k &\text{Association} \\
  &\sum_{k \in K} (a_k + b_k) = \sum_{p(k) \in K'} a_{p(k)} &\text{Commutativity}
\end{align*}

where $p(k)$ is a \emph{permutation} of the elements of $k$ to some set with image $p(k)$

For example, suppose $K  = \{-1, 0, 1\}$ and $p(k) = -k$, then $p : K \mapsto K'$ where $K' = \{1, 0, -1\}$.
The end result is the terms of $K$ are rearranged, and by \textbf{commutativity}, the summation is equal to the original ordering.

\subsubsection{Example}

Using the rules above, we can simplify the following summation:

\begin{align*} \label{eq:sum_manip_example}
  S &= \sum_{0 \leqslant k \leqslant n} (a + bk) \\
  &= a(n+1) + \sum_{0 \leqslant k \leqslant n} bk &\text{$a$ is constant} \\
  &= a(n+1) + \sum_{0 \leqslant n-k \leqslant n} bn - bk &\text{$k \to n-k$} \\
  &= a(n+1) + \sum_{0 \leqslant k \leqslant n} bn - bk &\text{$n-k \to k$} \\
\end{align*}

We employ another technique of adding $S$ to itself and solving a simplified expression

\begin{align*}
  2S &= \sum_{0 \leqslant k \leqslant n} (a + bk) + a(n+1) + \sum_{0 \leqslant k \leqslant n} bn - bk \\
  &= a(n+1) \sum_{0 \leqslant k \leqslant n} (an + bn) &\text{combining summations and simplifying} \\
  &= (n+1)(2a + bn) &\text{$an$, $bn$ constant}
\end{align*}

Finally, dividing by 2 we get $S = (a + \frac 1 2 bn)(n+1)$

\subsection{Iverson Notation}

In order to make manipulation of indexes easier, we can use notation invented
by Kenneth E. Iverson as an indicator function:

\begin{equation*} \label{eqn:iverson}
[P] =
\begin{cases}
  1 & \text{if $P$ is true} \\
  0 & \text{otherwise}
\end{cases}
\end{equation*}

This lets the terms of a summation be written as a linear combination with the
values of the Iverson predicate as coefficients. The "double-counting" rule of
summations follows immediately from manipulating Iverson notation:

\begin{equation*}
  [k \in K] + [k \in K'] = [k \in K \cap K'] + [k \in K \cup K']
\end{equation*}

which translates to summations with indices:

\begin{equation*}
  \sum_{k \in K} a_k + \sum_{k \in K'} a_k = \sum_{k \in k \cap K'} a_k + \sum_{k \in k \cup K'} a_k
\end{equation*}

\subsection{Perturbing the Sum}

Another technique is to manipulate the first and last terms. This reveal a way
to solve the summation via the above manipulations.

\subsubsection{Example}

For example, take the following summation representing a geometric series. A
good approach is to write $S_{n+1}$ in terms of $S_n$ and simplify the right
side of the equation.

Consider the following summation:

\[
  S_n = \sum_{0 \leqslant k \leqslant n} ax^k
\]

Adding the $n+1$st term and solving for $S_n$:

\begin{align*}
  S_n + ax^{n+1} &= a + \sum_{1 \leqslant k \leqslant n+1} ax^k \\
  &\iff S_n + ax^{n+1} = a + \sum_{1 \leqslant k+1 \leqslant n+1} ax^{k+1} \\
  &\iff S_n + ax^{n+1} = a + \sum_{0 \leqslant k \leqslant n} ax^{k+1} \\
  &\iff S_n + ax^{n+1} = a + xS_n \\
  &\iff S_n = \frac {a - ax^{n+1}} {1 - x}
\end{align*}

\subsubsection{Example}

Taking another example and using the results from the previous derivation, the
following sum is solved almost in the exact same way:

\[
  S_n = \sum_{0 \leqslant k \leqslant n} k2^k
\]

Once again perturbing the sum:

\begin{align*}
  S_n + (n+1)2^{n+1} &= 0 + \sum_{1 \leqslant k \leqslant n+1} k2^{k+1} \\
  &= \sum_{1 \leqslant k+1 \leqslant n+1} k2^{k+1} + \sum_{1 \leqslant k+1 \leqslant n+1} 2^{k+1} \\
  &= 2S_n + \sum_{0 \leqslant k \leqslant n} 2^{k+1} \\
\end{align*}

The second sum is a geometric series, and so after evaluating and simplifying:

\[
  S_n = \frac {2 - 2^{n+2}} {1 - 2} - (n+1)2^{n+1} = (n-1)2^{n+1} + 2
\]

\end{document}
