\documentclass{standalone}

\begin{document}

\section{Manipulating Sums}

Techniques for manipulating summations can often lead to a closed form solution or simpler summation.
This is especially helpful when dealing with multiple summations and loops in which the inner index depends on the outer.

\subsection{Basic Rules}

Often times a complex summation can be simplified using a few simple rules:

\begin{align}
  &\sum_{k \in K} c a_k = c \sum_{k \in K} a_k &\text{Distribution} \\
  &\sum_{k \in K} (a_k + b_k) = \sum_{k \in K} a_k + \sum_{k \in K} b_k &\text{Association} \\
  &\sum_{k \in K} (a_k + b_k) = \sum_{p(k) \in K'} a_{p(k)} &\text{Commutativity}
\end{align}

where $p(k)$ is a \emph{permutation} of the elements of $k$ to some set with image $p(k)$

For example, suppose $K  = \{-1, 0, 1\}$ and $p(k) = -k$, then $p : K \mapsto K'$ where $K' = \{1, 0, -1\}$.
The end result is the terms of $K$ are rearranged, and by \textbf{commutativity}, the summation is equal to the original ordering.

\subsubsection{Example}

Using the rules above, we can simplify the following summation:

\begin{align}
  S &= \sum_{0 \leqslant k \leqslant n} (a + bk) \\
  &= a(n+1) + \sum_{0 \leqslant k \leqslant n} bk &\text{$a$ is constant} \\
  &= a(n+1) + \sum_{0 \leqslant n-k \leqslant n} bn - bk &\text{$k \to n-k$} \\
  &= a(n+1) + \sum_{0 \leqslant k \leqslant n} bn - bk &\text{$n-k \to k$} \\
\end{align}

We employ another technique of adding $S$ to itself and solving a simplified expression

\begin{align}
  2S &= \sum_{0 \leqslant k \leqslant n} (a + bk) + a(n+1) + \sum_{0 \leqslant k \leqslant n} bn - bk \\
  &= a(n+1) \sum_{0 \leqslant k \leqslant n} (an + bn) &\text{combining summations and simplifying} \\
  &= (n+1)(2a + bn) &\text{$an$, $bn$ constant}
\end{align}

Finally, dividing by 2 we get $S = (a + \frac 1 2 bn)(n+1)$

\end{document}
