\documentclass{standalone}
\begin{document}

\section{Finite Calculus}

Finite Calculus establishes analogs between the techniques of infinite Calculus
(Analysis etc.) in order to develop sophisticated ways of dealing with sums.
Specifically, we're after methods that remind us of dealing with integrals and
derivatives since there are nice formulas and tricks for handling those most of
the time.

Another important theme of Finite Calculus is discovering functions that act as
discrete versions of their continuous counterparts. If we can find a finite
analog of a function, we'll have an easier time reasoning about a closed form
solution.

\subsection{The Difference Operator}

One of the first facts established in Calculus is the limit definition of the
derivative of a function:

\[
  \frac d {dx} f(x) = \lim_{h \to 0} \frac {f(x + h) - f(x)} h
\]

The righthand side is the limit of the \textbf{difference quotient} of $f(x)$
as $h$ approaches 0. In the finite world, there is a similar operation:

\begin{equation*}
  \Delta f(x) = f(x + 1) - f(x)
\end{equation*}

This is known as the \textbf{difference operator}. Since we're dealing with a
discrete function and not a continous one, we have to restrict $h$ to being an
integer. The closest integer $h$ can be without being 0 is 1. This gives a nice
tie-in to the limit definition.

Crucially, $\Delta f(x)$ is called an \emph{operator}, because it operates on
$f(x)$ and produces another function, similar to differentiation. Introducing
this operator allows for the development of more sophisticated tools a little
later.

\subsection{Rising and Falling Factorials}

Unfortunately, the convenient power rule of differential Calculus:

\[
  \frac d {dx} x^m = mx^{m-1}
\]

fails miserably in the finite world:

\[
  \Delta(x^3) = (x + 1)^3 - x^3 = 3x^2 + 3x + 1 \neq 3x^2
\]

At least for polynomials. There are, however, interesting functions for which
the difference operator is a near-perfect match-up to differentiation, although
this fact is not obvious right away. Some definitions first:

\begin{definition}
  Falling Factorial

  Let $m$ be an integer and $x$ be any real number. Define the falling factorial
  of $x$ as:

  \[
    x^{\underline{m}} = x(x-1)(x-2)\dots(x-m+1)
  \]
\end{definition}

\begin{definition}
  Rising Factorial

  Let $m$ be an integer and $x$ be any real number. Define the rising factorial
  of $x$ as:

  \[
    x^{\overline{m}} = x(x+1)(x+2)\dots(x+m-1)
  \]
\end{definition}

$m$ denotes the number of terms in the factorial product, and the line above or
below $m$ determines whether the terms are increasing or decreasing. There's a
hint here that the difference operator might be important. Taking a peek at the
first two consecutive terms in the product, they have the form:

\[
  (x)(x+1) \; \text{or} \; x(x-1)
\]

It's still too early to tell what's going on, but this deserves some suspicion.

This strange notation takes some getting used to, but is ultimately a very neat
way of compressing a lot of information into notation that can be manipulated.

Here's an explanation for why there are $m$ terms:

\begin{align*}
  x^{\underline{m}} = \underbrace{x(x-1)(x-2)\dots(x-m+1)}_{x - (x - m + 1) + 1 = m \; \text{terms}}
\end{align*}

\begin{align*}
  x^{\overline{m}} = \underbrace{x(x+1)(x+2)\dots(x+m-1)}_{(x + m - 1) - x + 1 = m \; \text{terms}}
\end{align*}

This comes from counting how many integers there are in a given range. As an
example, there are $(5 - 2) + 1 = 4$ integers between 2 and 5.

Here's a visual aid as a quick refresher:

% TODO: number line

But what does this have to do with the difference operator? The answer may lie
in blindly applying the operator and seeing what happens. Nothing ventured!

\begin{align*}
  \Delta(x^{\underline{m}})
  &= (x + 1)^{\underline{m}} - x^{\underline{m}} \\ \\
  &= (x+1)(x)(x-1)\dots(x-m+2) - x(x-1)(x-2)\dots(x-m+2)(x-m+1) \\ \\
  &= x(x-1)(x-2)\dots(x-m+2)(x + 1 - (x - m + 1)) \\ \\
  &= m[x(x-1)(x-2)\dots(x-m+2)]
\end{align*}

Some clever factoring got us an $m$ times something vaguely familiar. Let's try
counting how many terms there are:

\[
  x - (x - m + 2) + 1 = m - 1 \; \text{terms}
\]

This rings a bell! We have a product of $m - 1$ decreasing terms. Don't we have
a way of cramming that into a compact expression? Putting it all together:

\[
  \Delta(x^{\underline{m}}) = mx^{\underline{m-1}}
\]

which bears a striking similarity to the power rule.

\end{document}
