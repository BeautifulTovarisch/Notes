\documentclass{standalone}
\begin{document}

\section{Basics}

Probability is intimately related to combinatorics and set theory, and theorems
from both routinely show up when trying to precisely describe the total number
of outcomes of something happening.

\subsection{Terminology}

Generally, outcomes in a \textbf{random experiment} belong to an universal set
called the \textbf{sample space}. Groups of outcomes (subsets) are referred to
as \textbf{events}, and an event is said to occurred if one of its outcomes was
the result of the experiment.

\begin{definition} \label{def:samplespace}
  The sample space of a random experiment is the set of all possible outcomes
  of that experiment. Denote the sample space by $\Omega$ and define $A$ to be
  a subset of $\Omega$. We call $A$ an \emph{event} of sample space $\Omega$.
  We say $A$ \emph{occurred} if one of its outcomes was produced by the random
  experiment.
\end{definition}

The comparatively droll theorems and identities found in set theory make their
presence known throughout all of probability theory. Likewise, the basics of
counting and combinatorial proofs are very prominient in determining the number
of outcomes of a certain experiment.

\subsection{Naive Definition of Probability}

A simple yet easily abused model of Probability is called the \textbf{naive}
definition of probability. This applies when:

\begin{itemize}
  \item There is \emph{symmetry} in events, (e.g a Deck of Cards)
  \item The experiment is defined to have equally likely outcomes (e.g Dice)
  \item As a thought experiment, to be compared against a more realistic model
\end{itemize}

Formally:

\begin{definition}
  Let $A \subseteq \Omega$ be an event in the sample space of an experiment. We
  define the \emph{Naive Probability} of $A$ as:

  \[
    P_{naive}(A) = \frac {|A|} {|\Omega|}
  \]

  In other words, the ratio of outcomes in $A$ and total outcomes in $\Omega$.
\end{definition}

\subsection{Non-Naive Definition of Probability}

As the name implies, this definition is used when the outcomes in an experiment
are not equally likely. Predictably, this requires a bit more thought and care
to set up as a workable model of probability. Fortunately, there are only two
axioms to worry about (for now).

We also want to define a \textbf{probability space} as a mathematical object
consisting of a sample space and a \textbf{probability function}. This function
maps events in a sample space to real numbers, letting us define the likelihood
that an event in the sample space occurs.

\begin{definition}
  A probability space consists of a sample space $\Omega$ and a probability
  function $P$ defined as:

  \[
    P : A \subseteq \Omega \to \mathbb{R}, \; P(A) = \; \text{the probability
    of $A$}
  \]

  and governed by the following axioms:

  \begin{axiom}
    $P(\emptyset) = 0$ and $P(\Omega) = 1$
  \end{axiom}

  \begin{axiom}
    If $A_1, A_2, \dots, A_n$ are disjoint events of $\Omega$, then

    \[
      P(\bigcup_{k=1}^n A_k) = \sum_{k=1}^n P(A_k)
    \]
  \end{axiom}
\end{definition}

\end{document}
