\documentclass{standlone}
\begin{document}

\section{Basics of Counting}

These are the basic "rules" of counting. In general, counting using these rules
involves modeling the problem as a correct sequence of choices and applying the
appropriate strategy to the sequence.

\subsection{Product Rule}

The Product rule is applied to combinatorial problems that can be modeled as a
"process" of making mutually exclusive choices. For such problems the total
number of ways to perform the process is given by multiplying the number of
options at each step.

\begin{equation}
  \text{number of ways} = N_1 \cdot N_2 \cdot \dots \cdot N_k
\end{equation}

where $N_i$ is the number of ways of performing the $i$th step of the process.

\subsection{Complement Rule}

The Complement rule is helpful when expressing the size of a subset. Suppose
$A \subseteq U$, then

\[
  |A| = |U| - |U - A|
\]

In other words, the size of $A$ is the same as the size of $U$ when taking away
the elements of $U$ which are also in $A$ (leaving behind only those elements
which belong to $A$).

(I remember this "algebraically": $U - (U - A) = U - U + A = A$)

\subsection{Sum Rule}

This is a specicial form of the Inclusion/Exclusion rule applicable for
sets of elements who are mutually disjoint.

\subsection{Inclusion/Exclusion}

The Inclusion/Exclusion rule is a generalization of the Sum rule, and can be
used when sets are not-disjoint. The intuition here is to subtract the "double-
counted" elements from the union of two sets.

\[
  |A \cup B| = |A| + |B| - |A \cap B|
\]

For example, let $A = \{1, 2, 3\}, \; B = \{2, 3\}$. Since $A$ and $B$ are not
disjoint, we only want to count the elements in their union once:

\begin{align*}
  &A = \{1, 2, 3\} \\
  &B = \{2, 3\} \\
  &A \cup B = \{1, 2, 3, 2, 3\} \\
  &A \cap B = \{2, 3\} \\
  &(A \cup B) - (A \cap B) = \{1, 2, 3\}
\end{align*}

When $A$ and $B$ are disjoint, their intersection is $0$, and so the formula
holds.

\subsection{Permutations}

We can think of a permutation of a sequence $n$ as a particular ordering of
its elements. We consider each ordering to be a unique object. Another way to
think of a permutation is a function that sends an item located at index $i$ to
index $j$ in the output.

for example, consider the string $S = \text{abc}$. Then the permutations of $S$
are:

\[
  \{ abc, acb, bac, bca, cab, cba \}
\]

The process of arranging the elements of $S$ is called \textbf{permuting}.

The number of permutations a given string has can be modeled as a sequence of
choices for the positions of each of its characters. For a string of length $n$
there are initially $n$ choices. After the first character is chosen, there are
$n-1$ choices, and so on. Using the Product Rule:

\[
  n(n-1)(n-2) \dots 1 = n!
\]

\subsubsection{K-Permutations}

A $K$-Permutation follows the same basic idea of a permutation. The goal here
is to count the number of subsets of size $K$ a given set.

(Permutations can be thought of as $k$-permutations where $k=n$)

Let $S = \text{abc}, \; k = 2$. Then the $2$-permutations of $S$ are:

\[
  \{ ab, ba, ac, ca, bc, cb \}
\]

The formula for a $k$-permutation uses the product rule again, this time making
choices on a sequence of length $n-k+1$:

\[
  (n)(n-1)(n-2)\dots(n-k+1) = \frac {n!} {(n-k)!}
\]

\end{document}
