\documentclass{standlone}
\begin{document}

\section{Basics of Counting}

These are the basic "rules" of counting. In general, counting using these rules
involves modeling the problem as a correct sequence of choices and applying the
appropriate strategy to the sequence.

Proofs of the basic counting principles are usually extremely heavy on notation
and formalism. As such, they tend to be cryptic and not offer very much in the
way of an intuitive understanding of combinatorics. Actually, as it turns out,
not much can serve as a silver bullet for imbuing the contrived way of thinking
involved in solving difficult combinatorics problems.

Solving an absurd number of problems is just about the only way to approach the
subject unfortunately.

\subsection{Product Rule}

The Product rule is used when a counting problem is modeled after a sequence of
choices. A heuristic for understanding when to use the product rule comes from
comparing a situation to a logical and $\land$ operator. In other words, if we
need to pick among some choices \emph{and} some other choices, the total number
of outcomes is given by multiplying them together.

\begin{theorem}
  Let $A$ be a set of $n$ outcomes and $B$ be a set of $m$ outcomes. The number
  of ways to choose an outcome of $A$ and an outcome of $B$ is given by:

  \begin{equation}
    \text{number of ways} = n \cdot m
  \end{equation}

  generalizing over sets $A_1, A_2, \dots, A_n$, the total number of outcomes
  is given by\footnote{Unlike most of the other rules, the proof of the product
  rule is actually insightful, but sadly just as convoluted}:

  \begin{equation}
    |A_1| \cdot |A_2| \cdot \dots \cdot |A_n|
  \end{equation}
\end{theorem}

This rule along with the Sum rule are the basis for virtually every counting
problem. A good way to visualize the product rule is by considering the number
of branches of a decision tree:

% TODO: Decision tree

\subsection{Complement Rule}

The Complement rule is helpful when expressing the size of a subset. The main
idea here is to express the elements of $A$ as a disjoint combination of $U$
and $A$. This comes in handy every once in a while in proofs.

\begin{theorem}
  Suppose $A \subseteq B$, then

  \[
    |A| = |U| - |U - A|
  \]
\end{theorem}

\begin{proof}
  Since this just involves some straightforward thinking about sets, we might
  as well go ahead and prove it. Thinking through the notation slowly, all we
  do in the righthand expression is remove any element not in $A$ from $U$.
  This is because the elements we take away from $U$ are those leftover from
  deleting the ones in $A$.

  % TODO: Picture

  Admittedly, this still takes a moment to sink in. Another way of expressing
  this identity is to realize that we're just talking about integers, since the
  equation is in terms of the \emph{sizes} of sets. We can use a little algebra
  to help out:

  \begin{align*}
    a
    &= u - (u - a) \\
    &= u - u + a \\
    &= a
  \end{align*}

  Since $A \subseteq U$, all of $A$ is already inside of $U$. Taking away any
  elements not in $A$ leaves us with exactly $A$. Here's a formal way of saying
  that\footnote{Actually, a subtle fact that's glossed over here is that the
  theorem claims the sets are the same size. This demands that we establish a
  bijection between the sets, however, proving that they're exactly the same
  set is (probably) good enough here}:

  Suppose $A \subseteq U$. We will show both sides of the provided equation are
  subsets of one another, and therefore equal by set equality. Let $x \in A$.
  Then $x \in U$ by definition of subset, and likewise $x \not in U - A$ by set
  subtraction. But then $x \in U - (U - A)$ since it will not be removed from
  $U$ during the subtraction, and so $A \subseteq U - (U - A)$.

  Now suppose $x \in U - (U - A)$. Then $x \in U$ and not $U - A$ by definition
  of set subtraction. But if $x \not \in U - A$, then $x \in A$. Therefore the
  righthand side is also a subset of $A$, and the proof is complete.
\end{proof}

\subsection{Inclusion/Exclusion}

The Inclusion/Exclusion rule adjusts for overcounting the union of sets whose
intersection is non-empty. The intuition is fairly straightforward; when taking
the union of two non-disjoint sets, any element found in \emph{both} sets will
be double counted. To correct for this misake, the inclusion/exclusion rule
subtracts the intersection of the two elements, ensuring a given element is only
ever counted once.

\begin{theorem}
  Suppose $A$ and $B$ are sets, not necessarily disjoint. The total number of
  elements in the union of $A$ and $B$ is given by:

  \[
    |A \cup B| = |A| + |B| - |A \cap B|
  \]
\end{theorem}

A special case arises when $A$ and $B$ are disjoint. This reduces the equation
to $|A \cup B| = |A| + |B|$ since $|A \cap B|$ is empty, which clearly has no
elements. This identity is called the \textbf{Sum Rule}. Most of the time, it's
a good idea to think in terms of the inclusion/exclusion rule until confirming
whether $A$ and $B$ have any shared elements.

\begin{example}
  For example, let $A = \{1, 2, 3\}, \; B = \{2, 3\}$. Since $A$ and $B$ are not
  disjoint, we only want to count the elements in their union once:

  \begin{align*}
    A &= \{1, 2, 3\} \\
    B &= \{2, 3\} \\
    A \cup B &= \{1, 2, 3, 2, 3\} \\
    A \cap B &= \{2, 3\} \\
    (A \cup B) - (A \cap B) &= \{1, 2, 3\}
  \end{align*}
\end{example}

\subsection{Permutations}

We can think of a permutation of a sequence $n$ as a particular ordering of
its elements. We consider each ordering to be a unique object. Another way to
think of a permutation is a function that sends an item located at index $i$ to
index $j$ in the output.

for example, consider the string $S = \text{abc}$. Then the permutations of $S$
are:

\[
  \{ abc, acb, bac, bca, cab, cba \}
\]

The process of arranging the elements of $S$ is called \textbf{permuting}.

The number of permutations a given string has can be modeled as a sequence of
choices for the positions of each of its characters. For a string of length $n$
there are initially $n$ choices. After the first character is chosen, there are
$n-1$ choices, and so on. Using the Product Rule:

\[
  n(n-1)(n-2) \dots 1 = n!
\]

\subsubsection{K-Permutations}

A $K$-Permutation follows the same basic idea of a permutation. The goal here
is to count the number of subsets of size $K$ a given set.

(Permutations can be thought of as $k$-permutations where $k=n$)

Let $S = \text{abc}, \; k = 2$. Then the $2$-permutations of $S$ are:

\[
  \{ ab, ba, ac, ca, bc, cb \}
\]

The formula for a $k$-permutation uses the product rule again, this time making
choices on a sequence of length $n-k+1$:

\[
  (n)(n-1)(n-2)\dots(n-k+1) = \frac {n!} {(n-k)!}
\]

\end{document}
