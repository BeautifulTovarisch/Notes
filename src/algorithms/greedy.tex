\documentclass{standalone}
\begin{document}

\section{Greedy}

The greedy algorithm design paradigm produces straightforward and fast solutions to
certain problems. Usually, however, greedy algorithms do not produce correct
results, and great care must be taken to prove their correctness.

In general, the strategy is to choose a locally optimal solution in the hopes
that it produces a globally optimal output. Proofs of correctness and optimality
usually involve an exchange argument and/or induction.

\subsection{Dijkstra's Shortest Path Algorithm}

The canonical greedy algorithm. Dijkstra's algorithm computes the shortest paths
from a starting vertex by choose the least costly edge spanning a graph "cut"
or partition incident to the visited nodes. In other words, greedily choose the
edge which minimizes the current distance traveled.

\begin{figure}[h]
  \centering
  \begin{tikzpicture}[
  vertex/.style={circle, draw, fill=white, thick, minimum size=15, scale=1.5},
  ]

  % Vertices
  \node[vertex] (S) {$S$};
  \node[vertex] (V) [above right=of S] {$V$};
  \node[vertex] (W) [below right=of S] {$W$};
  \node[vertex] (T) [below right=of V] {$T$};

  % Edges
  \draw[->, thick] (S) to node[above] {1} (V);
  \draw[->, thick] (S) to node[below] {4} (W);
  \draw[->, thick] (V) to node[right] {2} (W);
  \draw[->, thick] (V) to node[above] {6} (T);
  \draw[->, thick] (W) to node[below] {3} (T);
  \end{tikzpicture}
  \caption{Weighted, directed graph}
  \label{fig:dijkstra1}
\end{figure}

The shortest path algorithm outputs the following when starting from $S$:

\begin{tabular}{ |c|c| }
  \hline
  Vertex & Shortest Path \\
  \hline
  S & 0 \\
  \hline
  V & 1 \\
  \hline
  W & 3 \\
  \hline
  T & 6 \\
  \hline
\end{tabular}

\begin{algorithm}
\caption{Dijkstra's Shortest-Path}
\begin{algorithmic}
  \Require $G = (V, E)$ has nonnegative edge weights
  \State $X = \{s\}$
  \State len(s) = 0, len(v) = $+\infty$ for every $v \neq s$ \\
  \While{There exists an edge $(v, w) \; v \in X, w \not \in X$ }
    \State $(v', w')$ = edge minimizing $len(v) + l_{vw}$
    \State Add $w'$ to $X$
    \State len(w') = len(v') + $l_{v'w'}$ \\
  \Return len
  \EndWhile
\end{algorithmic}
\end{algorithm}

The proof of Dijkstra's shortest-path algorithm differs from that of Prim's and
Kruskal's in that it proceeds by basic induction rather than presenting a proof
by contradiction and exchange argument\footnote{Consequently, I find this proof
to be slightly more difficult}.

First, a precise statement of the theorem:

\begin{theorem} \label{thm:dijkstra}
  For every directed graph $G = (V, E)$, for every starting vertex $s \in V$,
  Dijkstra's shortest-path algorithm outputs the distance of the shortest paths
  from $s$ to every (reachable) $v \in V$.
\end{theorem}

% TODO: Rewrite to not sound like a dick.
\begin{proof}[Proof of Dijkstra's Shortest-Path Algorithm] \label{prf:dijkstra}
  Let $k$ be the current iteration of the while loop in Dijkstra's algorithm,
  that is, the $k$th choice of vertex, say, $v$ to be added to the solution.
  During this iteration, the shortest distance $dist[v]$ from $s$ is computed.
  The goal of this proof is to show that every iteration of the algorithm gives
  the correct distance from $s$ for all vertices.

  Proceeding by induction, for $k = 1$ the computed distance from the starting
  vertex $s$ to itself is 0, which is clearly correct. Now assume the correct
  distance is computed for all $k = 1, 2, \dots n-1$, and consider the moment
  the algorithm must choose the $n$th vertex, call it $u$, to add to the
  solution and let the edge chosen by the algorithm be labeled $(v, u)$.

  By the algorithm, $u$'s distance is computed to be:

  \[
    dist[u] = dist[v] + length_{vu}
  \]

  To show that this is indeed the shortest path from $s$ to $u$, consider an
  arbitrary path $P$ from $s$ to $u$. We will show the length of this path is
  at least the value produced by Dijkstra's algorithm. We can deduce that $P$
  must be comprised of three segments:

  \begin{itemize}
    \item A prefix of vertices already processed
    \item At least one edge crossing the cut of visited and unvisited nodes
    \item A path consisting of unvisited nodes that reaches $u$
  \end{itemize}

  \begin{figure}
    \centering
    \begin{tikzpicture}[outer sep=auto]
      \graph[nodes={draw, circle, fill=white, thick, minimum size=15, scale=1.5}] {
        s -> a -> b -> u
      };
    \end{tikzpicture}
    \caption{Arbitrary shortest path}
    \label{fig:shortest_path}
  \end{figure}

  Let $(a, b)$ be the edge in $P$ bridging the aforementioned cut. It suffices
  to compute a lower bound for the length of $P$. The first segment consisting
  of a path from $s$ to $a$ has length $dist[a]$. By the inductive hypothesis,
  each vertex in the first segment has its correct shortest distance recorded.
  The segment consisting of only the edge $(a, b)$ has length $length_{ab}$.
  Finally, the final segment must be non-negative due to the restrictions the
  algorithm imposes on the input graph.

  \[
    len(P) \geqslant dist[a] + length_{ab}
  \]

  The final step in the proof makes use of the fact that Dijkstra's algorithm
  always chooses the edge which minimizes the sum of a candidate path's prefix
  and the edge cross the cut. In other words, we have shown:

  \[
    dist[u] + length_{uw} \leqslant dist[a] + length_{ab} \leqslant len(P)
  \]

  therefore, the path chosen by the algorithm is always the shortest such path.
\end{proof}

\subsubsection{Analysis}

% TODO: git gud and perform a proper analysis.

The cost of repeatedly selecting the minimum edge via brute force dominates the
runtime of the algorithm. For $G = (V, E)$, we traverse $|V|$ nodes, each time
performing $O(|E|)$ work to select the minimum edge crossing the cut. Therefore,
the naive version of Dijkstra's runs in $O(|E||V|)$.

Using a [[Heap]] this algorithm can be sped up significantly, achieving
$O((|E| + |V|)\log|V|)$ runtime.

\subsection{Prim's MST Algorithm}

"Robert Prim's" algorithm, which was actually discovered a few decades prior,
computes a minimum spanning tree of an undirected graph. The general strategy
is very close to how Dijkstra's chooses which edge weight will be used in order
to compute the current distance from the starting vertex. The main idea here is
simply to choose the minimum edge out of all the edges crossing the cut formed
by visited and unvisited nodes in $V$.

The key difference is subtle, but important. Dijkstra's algorithm chooses the
edge which minimizes the total distance from the starting vertex, while Prim's
chooses the edge with minimum individual cost. This has the overall effect of
\emph{minimizing} the \emph{maximum} edge cost along every path. This key fact
is critical in proving Prim's algorithm correct.

\begin{center}
  \begin{tikzpicture}[
  vertex/.style={circle, draw, fill=white, thick, minimum size=15, scale=1.5},
  ]

  % Vertices
  \node[vertex] (a) {$a$};
  \node[vertex] (b) [right=of a] {$b$};
  \node[vertex] (c) [below=of a] {$c$};
  \node[vertex] (d) [below=of b] {$d$};

  % Edges
  \draw[-, thick] (a) to node[above] {1} (b);
  \draw[-, thick] (a) to node[above] {3} (d);
  \draw[-, thick] (a) to node[left] {4} (c);
  \draw[-, thick] (b) to node[right] {2} (d);
  \draw[-, thick] (c) to node[below] {5} (d);
  \end{tikzpicture}
\end{center}

\begin{algorithm}
\caption{Prim's MST Algorithm}
\begin{algorithmic}
  \Require $G = (V, E)$ is connected
  \State $X = \{s\}$
  \State $T = \emptyset$
  \While{There is an edge $(v, w)$ s.t $v \in X$, $w \not \in X$}
    \State $(v', w')$ = minimum cost edge \\
    \State Add $w'$ to $X$
    \State Add $(v', w')$ to $T$
  \EndWhile
  \Return $T$
\end{algorithmic}
\end{algorithm}

\subsubsection{Correctness}

We need to prove that Prim's algorithm not only computes a spanning tree, but
that this tree is the best possible one that could have been produced. The main
ideas of this proof rely on properties of graph \ref{sec:graphs:cuts}, and the
crucial fact that Prim's only ever chooses the minimal edge spanning the cut.

Proving each part as separate lemmas keeps the arguments nice and tidy:

\begin{lemma}
  Prim's algorithm outputs a spanning tree.
\end{lemma}

The first order of business is to show that Prim's algorithm maintains the
invariant that $T$ spans the visited nodes in $X$. In other words, $T$ is a
graph that contains all of the vertices in $X$ at every iteration. This can
be achieved via a straightforward inductive proof. All that really needs to
be shown here is that whenever the algorithm adds a vertex to $X$, it always
adds the edge that led to that vertex. This is clear from reading the code,
but a proof is good for building character.

\begin{proof}
  Consider Prim's algorithm given an input graph of exactly one edge. Clearly
  this edge will immediately be added to $T$ during the first iteration, and
  since this edge is incident to the only two vertices in the graph, $T$ spans
  $X$ trivially. Now assume the $T$ spans $X$ for graphs having $n \geqslant 2$
  vertices. Notice when the $n+1$st vertex $w'$ is added to $X$, then so is the
  edge incident to it. By hypothesis, $T$ spans all $n$ vertices already added
  to $X$, and by virtue of adding an edge incident to $w'$, must also span $w'$
  as well. Thus, $T$ spans $X$ at every iteration of Prim's algorithm.
\end{proof}

Onto more interesting business. We know that $T$ spans $X$, but still need to
show the algorithm terminates eventually, and that $T$ is in fact a spanning
tree by the time the loop is finished. This is where a clever application of
the \ref{lem:emptycut} lemma helps out in a proof by contradiction.

Since the algorithm adds a vertex to $X$ every iteration, we'll eventually
run out of edges that cross the cut, and therefore exit the loop. What if the
algorithm terminates early, however? Can $T$ fail to span all of the nodes in
$G$? From the pseudocode, the while loop stops as soon as there are no further
edges to be processed. If there are nodes left in $V-X$ when this happens, we
have ourselves a graph cut. By the empty cut lemma, this means that $G$ must
have been disconnected, violating the constraints of the algorithm! The proof
sketch is solid, now to get to work.

% TODO: Picture

\begin{proof}
  To see that Prim's algorithm terminates, notice that every iteration of the
  while loop some vertex is added to $X$. Because $G$ is finite, there can only
  be a finite number of edges to cross the cut. Therefore, since each iteration
  of the loop reduces this number by exactly one, the loop will terminate by
  induction.

  Now to show that the output $T$ spans all vertices of $G$, assume by way of
  contradiction that there are vertices in $G$ which $T$ does not span. By the
  previous lemma, $T$ spans all vertices in $X$, and so those $T$ does not span
  must be in $V-X$. This exhibits a graph cut which must have no crossing edges
  by way of the termination condition. But then by the empty cut lemma, $G$ is
  disconnected, contradicting the constraint on the input to the algorithm.

  Thus, the output graph $T$ spans all vertices in $G$.
\end{proof}

\subsubsection{Runtime Analysis}

Once again the cost of repeatedly selecting the minimum edge dominates the
runtime of the algorithm leading to $O(|V||E|)$ worst-case runtime.

We can employ the same technique of using a Heap as in Dijkstra's algorithm to
achieve a runtime of $O(|E|\log|V|)$.

\subsection{Kruskal's MST Algorithm}

Kruskal's algorithm adopts a different approach, instead opting to choose the
minimum edge that would not introduce a cycle.

\begin{tikzpicture}[
vertex/.style={circle, draw, fill=white, thick, scale=2},
]

% Vertices
\node[vertex] (a) {$a$};
\node[vertex] (b) [right=of a] {$b$};
\node[vertex] (c) [below=of a] {$c$};
\node[vertex] (d) [below=of b] {$d$};

% Edges
\draw[-, thick] (a) to node[above] {1} (b);
\draw[-, thick] (a) to node[above] {3} (d);
\draw[-, thick] (a) to node[left] {4} (c);
\draw[-, thick] (b) to node[right] {2} (d);
\draw[-, thick] (c) to node[below] {5} (d);
\end{tikzpicture}

Kruskal's algorithm would execute the following steps:

\begin{enumerate}
  \item Choose $(a, b)$, since it is the minimum cost edge
  \item Choose $(b, d)$, since no cycle is produced
  \item Choose $(d, e)$, since $(a, c)$ would produce a cycle
  \item The chosen edges form a spanning tree of minimum cost!
\end{enumerate}

\begin{figure}[h]
  \centering
  \begin{tikzpicture}[
  vertex/.style={circle, draw, fill=white, thick, minimum size=1.5, scale=1.5},
  ]

  % Vertices
  \node[vertex] (a) {$a$};
  \node[vertex] (b) [right=of a] {$b$};
  \node[vertex] (c) [below=of a] {$c$};
  \node[vertex] (d) [below=of b] {$d$};

  % Edges
  \draw[-, thick] (a) to node[above] {1} (b);
  \draw[-, thick] (a) to node[left] {4} (c);
  \draw[-, thick] (b) to node[right] {2} (d);
  \end{tikzpicture}
  \caption{Minimum spanning tree}
  \label{fig:kruskal-mst}
\end{figure}

In a real program, we sort the edges of the input graph $G$ by weight as a
pre-processing step to avoid quadratic searches for successive minima.

\begin{algorithm}
\caption{Kruskal's Algorithm}
\begin{algorithmic}
  \State $T = \emptyset$
  \State sort $E$ by edge weight \\
  \For{$(v, w) \in E$}
    \If{$(v, w)$ does not produce a cycle in $T$}
      \State add $(v, w)$ to $T$ \\
    \EndIf
  \EndFor
  \Return $T$
\end{algorithmic}
\end{algorithm}

\subsubsection{Analysis}

Sorting the edges takes $O(n\log n)$ time. Cycle detection in the inner loop
dominates the runtime of naive Kruskal's and therefore the overall runtime is
subject to the implementation details. For a brute-force cycle detection
approach, the inner loop runs $O(|E||E + V|) = O(|E||V|)$ time.

By using a Union-Find data structure, we can dramatically improve the runtime.
In particular by implementing optimizations such as Path Compression and
Union-by-Rank

> TODO: Do the detailed analysis later, (Inverse Ackermann)

\subsection{Huffman Codes}

Invented by David Huffman in the 50s as a way to compute the optimal prefix-free
variable length encoding for a (mathematical) language $\sum$. The algorithm
constructs a tree from the "bottom up", repeatedly merging the least frequently
occurring codes in order to ensure the most frequently occurring have minimum
possible depth.

\begin{algorithm}
\caption{Huffman Encoding}
\begin{algorithmic}
  \State $H = \emptyset$
  \For{symbol $\sigma \in \sum$}
    \State ${T_\sigma} = (\sigma, P_\sigma)$
    \State $H = H \cup T_\sigma$
  \EndFor
  \While{There is more than one $T_\sigma \in H$}
    \State $T_1$ = tree with minimum frequency
    \State $T_2$ = tree with 2nd smallest frequency
    \State $T_3$ = \Call{MergeTrees}{$T_1, T_2$}
    \State $H = H \cup T_3$ \\\\
  \EndWhile
\end{algorithmic}
\end{algorithm}

For example, given the frequencies:

\begin{tabular}{ |c|r| }
  \hline
  Symbol & Frequency \\
  \hline
  a & 0.60 \\
  \hline
  b & 0.25 \\
  \hline
  c & 0.10 \\
  \hline
  d & 0.05 \\
  \hline
\end{tabular}

Huffman's greedy algorithm will produce the following encoding tree:

\begin{figure}
  \centering
  \begin{tikzpicture}[
  vertex/.style={circle, draw, fill=white, thick, minimum size=15, scale=1.5},
  ]

  % Vertices
  \node[vertex] (r1) {};
  \node[vertex] (a) [below left=of r1] {$a$};
  \node[vertex] (r2) [below right=of r1] {};

  \node[vertex] (b) [below left=of r2] {$b$};
  \node[vertex] (r3) [below right=of r2] {};

  \node[vertex] (c) [below left=of r3] {$c$};
  \node[vertex] (d) [below right=of r3] {$d$};

  % Edges
  \draw[-, thick] (r1) to node[above left] {0} (a);
  \draw[-, thick] (r1) to node[above right] {1} (r2);

  \draw[-, thick] (r2) to node[above left] {0} (b);
  \draw[-, thick] (r2) to node[above right] {1} (r3);

  \draw[-, thick] (r3) to node[above left] {0} (c);
  \draw[-, thick] (r3) to node[above right] {1} (d);
  \end{tikzpicture}
  \caption{The output of Huffman's greedy algorithm}
  \label{fig:huffman_tree}
\end{figure}

which ensures that symbol $a$, the most frequently encountered will be the
quickest to encode and decode since its depth in minimized in the output.

\subsubsection{Analysis}

Preprocessing the nodes can be done quickly in $O(n)$ time. The inner loop of
the algorithm is bound by the time it takes to select a minimum, therefore,
repeated brute-force searching for minima each iteration leads to a runtime of
$O(n^2)$.

Once again, the Heap data structure is a perfect fit for this kind of problem.
A heap is initialized with the frequencies provided to the algorithm, and keeps
track of the current minimum by re-balancing in $O(\log n)$ time. This speeds up
Huffman's algorithm tremendously, bringing the runtime down to $O(n \log n)$.

\end{document}
