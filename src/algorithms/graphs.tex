\documentclass{standalone}
\begin{document}

\section{Graphs} \label{sec:graphs}

An extraordinary number of problems can be modeled by graphs. Despite being
relatively simple, graph algorithms form the basis for more advanced solutions
to problems concering connectivity, precedence, optimality, and much more. The
solution to a graph problem is usually built on top of the methodology used to
explore the nodes in a graph, or \textbf{\emph{graph traversal}} algorithm.

\subsection{Cuts} \label{sec:graphs:cuts}

A graph cut is a way of partitioning the nodes of a graph into two non-empty
sets. Cuts and their properties are very useful to model various kinds of graph
problems, and algorithms frequently exploit properties of a cut to obtain some
desired solution. For example, certain computer vision problems can be solved
using graph-cut optimization algorithms.

Considering the edges "crossing" the cut also reveals information about a graph
itself, and is the basis for a large number of graph algorithms.

% TODO: tikz figure, edges crossing the cut

Fun fact:

\begin{theorem}
  A graph $G = (V,E)$ with $n \geqslant 2$ vertices has exactly $2^n - 2$ graph
  cuts.
\end{theorem}

\begin{proof}
  This proof is rather short, but hints at a strong connection between graphs
  and combinatorics. The main idea behind the proof is to recognize that there
  are two choices for a given node to be placed into a set. Call the two sets
  a node can be placed into $A$ and $B$. We'll start by counting the total ways
  to assign nodes to either set. Invoking the product rule $n$ times, with two
  choices per invocation:

  \[
    \underbrace{2 \cdot 2 \cdot \dots \cdot 2}_\text{$n$ choices} = 2^n
  \]

  However, the definition of a cut forbids either set from being empty. To make
  up for this, we need to subtract the invalid cases from the total. Since this
  problem only makes senses for graphs with at least two nodes, the only way a
  set can be empty is if we shove all the nodes into one or the other. This can
  only be done two ways; either they all go into $A$, leaving $B$ empty, or the
  reverse. Therefore, the grand total comes out to:

  \[
    2^n - 2
  \]
\end{proof}

\subsubsection{Properties of a Graph Cut}

Some basic properties of a graph cut serve an important role in proving graph
algorithms to be correct. A notable example is Prim's algorithm, which relies
on the \textbf{Cut Property} for its proof. Here's three important properties
and their proofs

\paragraph{Empty Cut}

This lemma is an interesting way to rephrase what it means for a graph to be
connected. Actually, the empty cut lemma is usually given as the contrapositive
statement, that a graph is not connected if there is even a single cut where no
edges are crossing. In other words, if there is some partitioning of nodes such
that the sets can't reach other (since there are no crossing edges), then this
means we've found (at least) two subgraphs that have no connection between them
which is the definition of a disconnected graph. A picture greatly clarifies the
situation:

% TODO: Disconnected graph + empty cut diagram

\begin{lemma} \label{lem:emptycut}
  Empty Cut

  A graph $G = (V, E)$ is disconnected if and only if there is some cut $C$ in
  which there are no crossing edges.
\end{lemma}

\begin{proof}
  Starting with the righthand side, if there are no edges crossing the cut, all
  we have to do is select a vertex from either of the sets and see that there's
  no path from one to the other. But if there are \emph{any} two vertices that
  can't reach one another in a graph, it's disconnected by definition.

  If $G$ is disconnected, then there must be at least two groups of nodes that
  \emph{are} connected. This holds even in the extreme case of a graph with no
  edges. Finding a cut with no crossing edges is a matter of identifying two of
  these disconnected groups and placing them in opposite sets in the cut. Seen
  another way, since $G$ is disconnected, it must have at least two connected
  components. We can compute those and create the necessary partition. Now for
  the formalities:

  Suppose $G$ is disconnected. Then there exist vertices $u, v \in V$ such that
  there is no path from $u$ to $v$. Construct a graph cut as follows, in one of
  the sets in the partition, place all of those nodes reachable from $u$. Let
  this set be called $A$ and its construction be given by

  \[
    A = \{u' \in V \mid \exists \; \text{a path from $u$ to $u'$ in $G$} \}
  \]

  Similarly, construct the other set in the partition, $B$ as those nodes that
  can be reached by $v$

  \[
    B = \{v' \in V \mid \exists \; \text{a path from $v$ to $v'$ in $G$} \}
  \]

  These sets are disjoint, and there cannot be any edge from a vertex in $A$ to
  one in $B$. Therefore, we have exhibited a cut with no crossing edges and the
  forward direction is shown.

  To see the reverse implication, assume that there exists a cut with no edges
  between the elements of either set forming the partition of nodes. Select an
  arbitrary node from either, say $u, v$ and notice by construction there is no
  path from $u$ to $v$. But then by definition there exist vertices in $G$ such
  that no path exists between them, and so $G$ cannot be connected.
\end{proof}

\paragraph{Double-Crossing Lemma}

This lemma claims that if some edge in a cycle crosses a graph cut, then there
must be at least two edges of that cycle crossing it. If this isn't intuitive,
another way to think about it is to imagine that some nodes involved in a cycle
are in one set of the partition and the rest are in the other. Since we have a
cycle, we know we can start at one vertex and travel the edges until returning
to the start. This means that each set has one "way out" and one "way in". If
this were not the case, then we wouldn't be able to get back to the beginning
of the cycle, which would mean we didn't actually have one to begin with!

This seemingly random bit of knowledge presents itself in a very elegant way in
trying to prove the output of certain algorithms is a bona fide tree. A cycle
is not allowed in a tree, and so if we can be certain that only one edge among
all of the ones crossing the cut is ever chosen, we eliminate any possibility
of ever forming a cycle. This is similar to some arguments which systematically
remove the edges of a cycle in a connected graph until none exists, just from
the opposite perspective.

\begin{lemma} \label{lem:doublecross}
  Suppose a cycle $C \in E$ has an edge crossing the cut $(A, B)$. Then there
  must be at least two edges of $C$ crossing the cut.
\end{lemma}

\begin{proof}
  To prove the Double-Crossing Lemma, all we have to show is that it is not
  possible for a cycle which has an edge crossing a cut to have fewer than 2
  such edges. To see this imagine if there were only one edge, and attempt to
  navigate the cycle back to the starting point. Clearly it's impossible, since
  there's no way back to the beginning.

  This reasoning is straightforward, and a more formal proof barely serves any
  purpose in providing additional clarity. Nevertheless:

  Suppose $C \in E$ has an edge crossing the cut labeled $(A, B)$ and further
  suppose by way of contradiction that $C$ has fewer than 2 edges crossing the
  cut. By assumption, $C$ has at least one edge, so it will suffice to consider
  the case in which $C$ has exactly one crossing edge.

  Choose an arbitrary vertex, say $u$, involved in $C$. By definition, there is
  a path:

  \[
    u-e_1-v_1-e_2-\dots-v_{n-1}-e_{n-1}-u
  \]

  such that $e_1 \neq e_{n-1}$. Consider the edge $e_j$ in $C$ crossing the cut
  which has endpoints $v_i, v_j \; i \neq j$. Then $v_j$ is in the opposite set
  of $u$. By definition of a cycle, there must be some path from $v_j$ to $u$.
  This is impossible, since such an edge would need contain vertices $v_k, v_l$
  with $v_k$ in the same set as $v_j$ and $v_l$ in the same set as $u$, meaning
  it would have to cross the cut. This contradicts the claim that $C$ had only
  single edge crossing the cut, and thus the proof is complete.
\end{proof}

% TODO: Picture

\subsection{Depth-First Search}

Depth-first search is arguably the simpler of the two main traversal algorithms.
The key idea is to explore a graph as "deeply" as possible, following the trail
of nodes as far as possible before backtracking out to the starting point. DFS
ignores neighbors until after there are no additional nodes to explore or after
reaching an already visited node.

% TODO: Picture (animation???)

\begin{algorithm}
  \caption{Depth-First Search}
  \begin{algorithmic}
  \State visited = $\emptyset$
  \Procedure{DFS}{$v$}
  \State add $v$ to visited
  \For{$u \in V$ such that $(v, u) \in E$}
  \If{$u \not \in$ visited}
  \State DFS($u$)
    \EndIf
    \EndFor
    \EndProcedure
  \end{algorithmic}
\end{algorithm}

% \subsubsection{Cycle Detection}

% TODO

\subsubsection{Analysis}

Depth-first search explores each reachable node from starting vertex $v$ and traverses edge incident on $v$. Additionally, the check for whether a node has been visited before (using a hash) runs in expected $O(1)$ time.

Due to visiting every node, the runtime of DFS is $O(|V| + |E|)$.

\subsection{Breadth-First Search}

Breadth-first search (BFS) proceeds to process the "levels" of a graph, proceeding only when every neighbor of the current node has been explored. Typical implementations enqueue the neighbors of a vertex in order to enforce a BFS discipline.

\begin{algorithm}
  \caption{Breadth-First Search}
  \begin{algorithmic}
  \State $Q = \{v\}$
  \State visited = $\emptyset$
  \While{$Q$ is not empty}
    \State $u =$ dequeue($Q$)
    \For{$w \in V$ such that $(u, w) \in E$}
      \If{$w \not \in$ visited}
        \State add $w$ to visited
        \State enqueue($Q, w$)
      \EndIf
    \EndFor
  \EndWhile
  \end{algorithmic}
\end{algorithm}

\subsubsection{Analysis}

Similar to DFS, the runtime of BFS is $O(|V| + |E|)$ assuming an implementation of a queue which allows constant time add and remove operations.

\subsection{Topological Ordering}

A Topological ordering in a directed graph is the order in which DFS explores vertices (formally called reverse post-order).  The topological sort algorithm computes a valid topological ordering of a directed graph $G$, if one exists.

Such orderings are typically used to determine whether a precedence graph is logically consistent, and if so, compute a correct sequence of operations.

\begin{algorithm}
  \caption{TopoSort}
  \begin{algorithmic}
  \Require{$G$ is a directed acyclic graph}
  \State visited = $\emptyset$
  \State processing = $\emptyset$
  \State ordering = Stack \\
  \Procedure{DFS}{$v$}
    \State mark $v$ as visited
    \State processing(v) = True
    \For{$u$ such that $(v, u) \in E$}
      \If{$u \not \in visited$}
        \State DFS($u$)
        \ElsIf{$u$ is a node currently being processed}
        \Return	"no topological ordering"
      \EndIf
    \EndFor
    \State processing(v) = False
    \State ordering.push(v)
  \EndProcedure
  \For{$v \in V$}
    \If {$v \not \in visited$}
      \State DFS($v$)
    \EndIf
  \EndFor
  \Return ordering
  \end{algorithmic}
\end{algorithm}

\subsection{Strongly Connected Components}

We define "strongly-connected" as a relation between two vertices $u, v \in V$ such that $(u, v) \in R$ if there exists a path from $u$ to $v$ and a path from $v$ to $u$. Strongly connected components represent an equivalence class of the vertices of that component.

Kosaraju's algorithm computes the strongly connected components of $G = (V, E)$ via two passes. The first past computes a "topological ordering" of the reverse of $G$, $G_{\text{rev}}$, which provides the ordering necessary to provide to an additional pass of DFS.

> Note that there cannot be a valid topological ordering in the strictest sense due to the fact that there must be a cycle in a directed graph in order for two vertices to be strongly connected.

\begin{algorithm}
  \caption{Kosaraju-Sharir}
  \begin{algorithmic}
  \State scc = $\emptyset$
  \State count = 0
  \State visited = $\emptyset$
  \State ordering = \Call{TopoSort}{$G_{\text{rev}}$}
    \State \\
  \Function{DFS-SCC}{$G, v$}
  \State mark $v$ as visited
  \State scc($v$) = count
  \For{$u \in V$ such that $(v, u) \in E$}
  \If{$u \not \in$ visited}
  \State DFS-SCC($G, u$)
    \EndIf
    \EndFor
    \EndFunction
    \For{$v$ in ordering}
    \State DFS-SCC($G, v$)
    \State count = count + 1
    \EndFor
    \State \\
    \Return scc
  \end{algorithmic}
\end{algorithm}

\end{document}
