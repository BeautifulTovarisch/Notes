\documentclass{standalone}
\begin{document}

\section{Quadratics}

Polynomials of the form $ax^2 + bx + c$ are called \textbf{quadratic}. They can
also be described as polynomials of degree 2, or \textbf{second order}. A huge
number of problems in physics and engineering can be modeled after a quadratic
equation, and they occur frequently enough to warrant some special techniques.

\[
  \boxed{ax^2 + bx + c}
\]

where $a \neq 0$

\subsection{Finding Roots of a Quadratic Function}

A \textbf{root} of a polynomial function $f$ is a number, say, $\alpha$, such
that $f(\alpha) = 0$. Finding roots of a quadratic function is equivalent to
solving $f(x) = ax^2 + bx + c = 0$.

Several techniques for finding roots exist:

\subsubsection{Factoring}

Factoring here is used in the same context as listing the factors of an integer,
such as in \emph{prime factorization}. Factoring allows the polynomial to be
written in such as a way as to make the values for which it takes on the value
zero to be obvious.

Every polynomial can be written as a product of its factors. Quadratics have a
simple form:

\[
  P(x) = (x - \alpha)(x - \beta)
\]

where $\alpha$ and $\beta$ are roots of polynomial $P$.

For example,

\[
  x^2 - 4 = (x - 2)(x + 2) = 0
\]

This can easily be factored by noticing the expression is the difference of two
squares. This equation has solutions at $x \pm 2$.

% \ref{sec:factoring_tricks} are incredibly helpful here.

\subsubsection{Completing the Square}

A good indication to try this technique is when a quadratic function has a 1 as
its first coefficient.

\[
  x^2 + bx + c
\]

This is referred to as a \textbf{reduced quadratic} form.

Completing the square is often taught as a formula to be memorized without any
context or justification as to why it works. It helps to consider both what we
mean geometrically and then derive the formula with algebra.

\paragraph{Geometric}

TODO

\paragraph{Algebraic}

The clever idea behind coming up with the formula lies in thinking about what
term would need to be added to $x^2 + bx$ in order to make it a perfect square.
Another way of thinking about this is how we could start with the square of a
binomial, $(x + k)^2$ and figure out what $k$ and $c$ need to be in order to be
equivalent to the lefthand side\footnote{Working backwards like this is the way
that clicked for me when nothing else did}:

\[
  x^2 + bx = (x + k)^2 - c
\]

Naively computing $(x + b)^2$ might reveal a hint, since we'd end up with the
square of a binomial by design. let $k = b$:

\[
  (x + b)^2 = x^2 + 2bx + b^2
\]

This is closer, but we got $2bx$ instead of $bx$ like we wanted. To get $bx$,
we could let $k = \frac b 2$. Now see what happens:

\[
  (x + \frac b 2)^2 = x^2 + 2 (\frac b 2) x + (\frac b 2)^2
\]

Nearly there! All that's left to do is subtract $c = (\frac b 2)^2$ from the
equation. This gives us:

\[
  \boxed{x^2 + bx = (x + \frac b 2)^2 - (\frac b 2)^2}
\]

A typical follow-up to completing the square is solving for $x$ via a shortcut
used when an expression is the \emph{difference of two squares}.

\subsection{Vieta's Theorem}

Vieta's theorem describes the roots of a quadratic polynomial in terms of its
coefficients. This is especially helpful in eliminating candidates for roots if
a more straightforward approach fails.

If a quadratic expression $x^2 + px + q$ has distinct roots $\alpha, \beta$,
then

\begin{align}
  &\alpha + \beta &= -p \\
  &\alpha \beta &= -q
\end{align}

\begin{proof}
  (Theorem):
\end{proof}

Vieta's theorem also has a corollary:

If a quadratic expression $x^2 + px + q$ has distinct roots $\alpha, \beta$,

\[
  (x - \alpha)(x - \beta) = x^2 - (\alpha + \beta) + \alpha \beta
\]

\begin{proof}
\end{proof}

\subsection{The Quadratic Formula}

The famous quadratic formula has a surprisingly simple derivation based on the
formula for completing the square. Simply solving a "completed square" for $x$
yields the formula.

Begin by factoring out the leading coefficient $a$:

\begin{align*} \label{eqn:quadratic_formula}
  ax^2 + bx + c
  &= x^2 + (\frac b a)x + \frac c a \\\\
  &= (x + \frac b {2a})^2 - (\frac b {2a})x - \frac c a &\text{completing the square} \\\\
  &= (x + \frac b {2a})^2 - \frac {b^2 - 4ac} {4a^2} &\text{by algebra} \\\\
  &= (x + \frac b {2a} + \sqrt{\frac {b^2 - 4ac} {4a^2}})(x + \frac b {2a} - \sqrt{\frac {b^2 - 4ac} {4a^2}}) &\text{difference of squares} \\\\
  &\implies x = -\frac b {2a} \pm \frac{\sqrt{b^2 - 4ac}} {2a} \\\\
  &\implies x = \frac {-b \pm \sqrt{b^2 - 4ac}} {2a}
\end{align*}

\end{document}
