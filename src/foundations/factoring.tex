\documentclass{standalone}
\begin{document}

\section{Factoring Tricks}

Factoring a polynomial is a ubquitous technique and is especially helpful with
certain types of integrals (or rather in avoiding their tedious computation).

It is important to show the derivations of these "tricks", since they develop
critical problem-solving ability. Many of the clever ideas used in obtaining a
factorization (especially "adding zero") are helpful regardless of context.

\subsection{Basic Identities}

These are worth "memorizing" or rather, it's a very good idea to work enough
problems involving these identities that they become muscle memory. Being able
to recognize when a problem reduces to one of these expressions is incredibly
satisfying.

\begin{align*}
  &(a + b)^2 = a^2 + 2ab + b^2 \\\\
  &(a + b)^3 = a^3 + 3a^2b + 3ab^2 + b^3 \\\\
  &(a^2 - b^2) = (a + b)(a - b)           &\text{Difference of Squares} \\\\
  &(a^3 - b^3) = (a - b)(a^2 + ab + b^2)  &\text{Difference of Cubes} \\\\
\end{align*}

\subsection{Splitting Apart a Monomial}

This strategy is mostly a matter of trial and error. The overall objective here
varies, but usually what we want is to be able to decompose a single term into
multiple in order to be able to apply a more straightforward technique.

Splitting out terms like this doesn't always work, but it's simple and cheap to
try.

\begin{align*}
  a^4 + 2a^2b^2 + b^4
  &= a^4 + a^2b^2 + a^2b^2 + b^4 \\
  &= a^4 + a^2b^2 + b^4 + a^2b^2 \\
  &= a^2(a^2 + b^2) + b^2(a^2 + b^2) \\
  &= (a^2 + b^2)^2
\end{align*}

Intuition for when this works is probably a matter of experience. Here we can
see somewhat of a hint due to the fact that $2a^2b^2$ is suspiciously "close"
to either $a^4$ or $b^4$. If a problem doesn't seem to work out after shuffling
the terms around, it's likely a good idea to move on to another strategy rather
than spend a lot of time fiddling with this one.

\subsection{Adding Zero}

A fan favorite. This is used quite frequently in clever proofs as well as in
simplifying nasty computations. The intuition for when to try this comes with
experience and time. Until then, it'll seem like a dirty trick reserved for the
truly enlightened.

Patience is key here.

The idea is similar in spirit to splitting part a monomial with the exception
that the term we're breaking up is "invisible". We're free to add and subtract
the same quantity of course, since we'd just be \emph{adding zero}. This can be
a good way to break up a rational expression or introduce a "missing" term that
would support a factorization.

\begin{align*}
  x^5 + x + 1
  &= x^5 + (x^4 + x^3 + x^2 - x^4 - x^3 - x^2) + x + 1 \\
  &= x^3(x^2 + x + 1) - x^2(x^2 + x + 1) + x^2 + x + 1 \\
  &= (x^3 - x^2 + 1)(x^2 + x + 1)
\end{align*}

The particular usage of adding zero above is sometimes called adding back the
"annihilated" term(s). The reasoning behind this is that we imagine the term we
want to add back as being cancelled out (annihiliated) at some point during the
expansion of the polynomial.

As with other techniques, there isn't a universal indicator as to whether this
one will work. Even in the event that adding zero doesn't work, however, some
information is usually gained from the attempt.

\subsection{\( (a^n - b^n) \)}

Here is a good example of where adding zero works really well. We know at some
point in any polynomial of degree $n$ there must have been terms raised to the
power $1, 2, \dots, n-1$, so the key idea is to enumerate them and see how the
problem changes.

Adding this many terms back ought to be done systematically so none is left out.
If we ignore the coefficients (they get cancelled out anyway), we can make use
of the binomial theorem as a completely mechanical way of writing down these
terms in the same way every time.

\[
  \sum_{k=1}^n a^{n-k} b^k
\]

This has the somewhat unfortunate drawback of itemizing each term in a specific
order, which may make it more difficult to see how the terms can be rearranged
to work their magic. Some books that show off this strategy appear to magically
place the terms in precisely the correct location that reveals the solution.

Good for them.

It's worth spending some time playing around with the ordering and making some
educated guesses, but spending a lot of time trying dozens of permutations is
ill-advised\footnote{Perhaps there is a rhyme or reason to figuring out the way
to group the terms and unlock the secret of the factorization, but most of the
time it feels like luck. Thankfully, this particular family of factorizations
has a general form we can follow}.

These derivations are worth working out as practice for manipulating terms and
the satisfaction of solving the puzzle, however, after a few cases (maybe two),
a general pattern starts to emerge. It's probably a good idea to conserve brain
power and jot down the general identities somewhere (like GitHub) for posterity
rather than working them out each time from scratch.

After fooling around for awhile, the simplicity of having an $(a - b)$ factor in
the solution seems plausible. It's good to stop and think about this hypothesis.
If we're after $(a - b)$, it wouldn't make sense to group $(-ab^2 - b^3)$, since
the $-1$ would factor out into $-(ab^2 + b^3)$. This at least tells us that we
should think about grouping terms with alternate signs. Such clues are subtle,
but can save some wasted effort.

All that remains is to keep working at the problem until all the pieces fit or
it's time to take a break and grab another coffee.

\begin{align*}
  a^3 - b^3
  &= a^3 + (a^2b + ab^2 - a^2b - ab^2) - b^3 \\
  &= (a^3 - a^2b) + (a^2b - ab^2) + (ab^3 - b^3) \\
  &= a^2(a - b) + ab(a - b) + b^2(a-b) \\
  &= (a - b)(a^2 + ab + b^2)
\end{align*}

Here's another example:

\begin{align*}
  a^4 - b^4
  &= a^4 + a^3b + a^2b^2 + ab^3 -a^3b - a^2b^2 - ab^3 - b^4 \\
  &= (a^4 - a^3b) + (a^2b^2 - ab^3) + (a^3b - a^2b^2) + (ab^3 - b^4) \\
  &= a^3(a - b) + ab^2(a - b) + a^2b(a - b) + b^3(a - b) \\
  &= (a - b)(a^3 + a^2b + ab^2 + b^3)
\end{align*}

Suspiciously similar to the cubic example. A mathematician may be inspired to
pose a conjecture and provide a proof. For now, we'll be satisfied that one is
likely to exist and move on.

\[
  a^n - b^n = (a - b)(a^{n-1} + a^{n-2}b + \dots + ab^{n-2} + b^{n-1})
\]

The best thing to do when discovering a new tool is play around with it. Here's
a general formula, so try some concrete values. An interesting identity arises
when $b = 1$:

\[
  (a^n - 1) = (a - 1)(a^{n-1} + a^{n-2} + \dots + a + 1)
\]

\subsection{\( (a^n + b^n) \)}

Approaching this kind of problem directly is too difficult. Breaking it into
subproblems may reveal a way to re-use some of the techniques previously worked
out. Since $n$ is a positive integer, a good guess is to consider the cases in
which $n$ is even and odd\footnote{As it turns out, this is almost always worth
a shot when dealing with integers.}

\subsubsection{\(n\) is odd}

What is special about numbers being raised to an odd power? There are a few
ways to think about this, but scribbling down some examples is usually handy:

\begin{gather*}
  2^3 = 8 \\
  4^3 = 64
  5^3 = 125 \\
  3^5 = 243
\end{gather*}

Nothing terribly insightful, however, this is due to our poor choices of $a$ and
$b$. Variety is key here, since the forumla should hold for \emph{all} numbers
$a$ and $b$. Trying out some negative numbers reveals a big clue:

\begin{gather*}
  -1^3 = -1 \\
  -2^3 = -8 \\
  -3^5 = -243 \\
  -2^5 = -32
\end{gather*}

This shows that any number $b^n$ has the same sign as $b$ when $n$ is odd. This
property combined with the fact that $a + b = a - (-b)$ allows the technique we
showed previously to be used as-is:

\begin{align*}
  (a^n + b^n)
  &= (a^n - (-b^n)) \\
  &= (a - (-b))(a^{n-1} + a^{n-2}(-b) + \dots + a(-b)^{n-2} + (-b)^{n-1}) \\
  &= (a + b)(a^{n-1} - a^{n-2}b + \dots - ab^{n-2} + b^{n-1})
\end{align*}

Thinking through whether $b$ is raised to an even power or odd power in each of
the above terms explains why some terms are subtracted. Concrete examples help.

\end{document}
