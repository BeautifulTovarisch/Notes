\documentclass{standalone}
\begin{document}

\section{Set Theory}

Sets represent unordered, arbitrary collections. Most of modern mathematics is
founded on the basics of set theory, and they show up in any serious treatment
of a mathematical subject. These notes contain some simple proofs about sets. I
hope I have provided an unbearable amount of detail in the explanations.

Despite usually being among the first proofs students learn, some of the longer
set proofs can be somewhat tricky, and involve multiple proof techniques.

\begin{definition}
  These two axioms form the basis for nearly all set identities and proofs.

  \begin{enumerate}
    \item There exists a set with no elements called the null set ($\emptyset$)
    \item If every element $x \in X$ is also in $Y$ and vice-versa, then $X = Y$
  \end{enumerate}
\end{definition}

A clever proof by contradiction establishes the fact that $\emptyset$ is unique
(in other words, there is exactly one $\emptyset$). The main idea behind this
proof is to use axiom two in order to prove that any two sets with no elements
are actually subsets of one another and by definition the same set.

\begin{theorem}
  There is exactly one empty set.
\end{theorem}

\begin{proof}
  Suppose not, there there are at least two sets, $\emptyset_1$ and $\emptyset_2$
  such that $\emptyset_1 \neq \emptyset_2$. By definition of empty, there is no
  element $x$ in either of the sets. However, this means that exactly the same
  elements are in $\emptyset_1$ and $\emptyset_2$, and axiom 2 establishes they
  must be equal. Therefore we have arrived at a contradiction, and so there can
  only be a single, unique empty set.
\end{proof}

\subsection{Basic Definitions}

The following set operations are ubiquitous in set proofs.

\begin{align*}
  \overline{X} &= \{ x \mid x \not \in X \}              &\text{Complement} \\\\
  X - Y &= \{ x \mid x \in X, x \not \in Y \}       &\text{Difference} \\\\
  \mathcal{P}(A) &= \{ A' \mid A' \subseteq A \}    &\text{Power Set} \\\\
  X \times Y &= \{ (x, y) \mid x \in X, y \in Y \}  &\text{Cartesian Product}
\end{align*}

\subsection{Laws}

Set laws are identities proven by thinking through the membership of arbitrary
elements in a set and applying some basic logical operators. These come up all
the time, so it's worth having a small list jotted down somewhere.

\begin{theorem}
  The set union operator $\cup$ is associative, that is

  \[
    A \cup (B \cup C) = (A \cup B) \cup C
  \]
\end{theorem}

\begin{proof}
  Suppose $x \in A \cup (B \cup C)$. Then by definition of set union, $x$ is
  either in $A$ or $B \cup C$ (or both). Breaking down these cases:

  If $x \in A$. Then $x$ is in $A \cup B$, and so it is in the right-hand side.

  Alternatively, if $x \in (B \cup C$, this breaks down into another two cases,
  $x \in B$ or $x \in C$. If $x \in C$, then it's automatically in the union of
  $C$ and anything else, and if $x \in B$, then $x \in (A \cup B)$ and so must
  be in $(A \cup B) \cup C$.

  Since we've proven the two expressions are subsets of one another, they must
  be the same set by axiom two, and the proof is complete.
\end{proof}

\begin{theorem}
  \[
    A \cap B \subseteq B
  \]
\end{theorem}

\begin{proof}
  Intuitively, this is saying that if an element is in $A$ \emph{and} $B$, then
  it must be in $B$. This is just the definition of set intersection. Formally:

  Suppose $x \in A \cap B$, then $x \in A$ and $x \in B$. But then $x$ must be
  an element in $B$ by definition. Because $x$ was arbitrarily chosen, every
  element in $A \cap B$ must be in $B$, and therefore the definition of subset
  is satisfied.
\end{proof}

\begin{theorem}
  The union operation is distributive over a set expression.

  \[
    X \cup (Y \cap Z) = (X \cup Y) \cap (X \cup Z)
  \]
\end{theorem}

\begin{proof}
  Starting with the right-hand side, we see that $X$ is involved in two unions,
  meaning that every element in $X$ automatically shows up in the intersection.
  This is consistent with the left-hand side, since an element being in $X$ can
  be thought of as "short-circuting" the union expression; we don't even need to
  consider the other operations to know an element is present in whatever comes
  next.

  On the other hand, if an element is not in $X$, then it had better be in both
  $Y$ \emph{and} $Z$. This is similar to the rules of logic, in which a known
  false combined with a proposition reduces to just the proposition. In other
  words, since we know the element is not in $X$, we can imagine the form on
  the right as:

  \begin{align*}
    &(x \in X \lor x \in Y) \land (x \in X \lor x \in Z) \\
    &\iff (F \lor x \in Y) \land (F \lor x \in Z) \\
    &\iff (x \in Y) \land (x \in Z)
  \end{align*}

  which is precisely what the left-hand side is saying. More formally:

  Suppose $x \in X$. Then $x \in (X \cup Y)$ and $x \in (X \cup Z)$ by union of
  two sets, and so $x$ is in the right-hand expression. By similar argument, if
  $x \in (Y \cap Z)$, then $x \in (X \cup Y)$ and $x \in (X \cup Z)$, and so
  the forward direction holds.

  Now suppose $x \in (X \cup Y) \cap (X \cup Z)$. Then $x \in X$ or $x$ is in
  both $Y$ and $Z$ by definition of set union. But then $x \in X \cup (Y \cap Z)$
  and since both expressions are subsets of one another, the identity holds.
\end{proof}

\begin{theorem}
  A Cartesian Product distributes over a set union.

  \[
    A \times (B \cup C) = (A \times B) \cup (A \times C)
  \]
\end{theorem}

\begin{proof}
  The main idea here is to think about all the ordered pairs that would end up
  on the left-hand side. Taking every element in $A$ and forming a tuple with
  all the elements in either $B$ or $C$, we get

  \[
    (a_1, b_1), (a_1, c_1), (a_2, b_1), \dots (a_n, b_m), (a_n, c_k)
  \]

  but these are the same ordered pairs as if we had taken each product by itself
  and combined the results together:

  \[
    (a_1, b_1), (a_1, b_2), \dots (a_n, b_m) \cup (a_1, c_1) \dots (a_n, c_k)
  \]

  For the actual proof, we consider an arbitrary ordered pair in each of the
  expressions and once again prove both sides are subsets of one another.

  Take $(x, y) \in A \times (B \cup C)$, then by definition of set union and
  Cartesian product, $x \in A$ and either $y \in B$ or $y \in C$.

  Suppose $y \in B$, then $(x, y) \in (A \times B)$ and so it is in the union
  on the right. Now let $y \in C$, and see by similar argument it is also in
  the union. Now assume $(x, y) \in (A \times B) \cup (A \times C)$. Then once
  againt, $x \in A$ and $y \in B$ or $y \in C$. This completes the proof as in
  either case, $(x, y) \in A \times (B \cup C)$ and so the sets are equal.
\end{proof}

\begin{theorem}
  \[
    A \setminus B \subseteq \overline{B}
  \]
\end{theorem}

\begin{proof}
  Any element left over after $A \setminus B$ couldn't have been in $B$ by the
  definition of set subtraction. In other words, every such element would have
  to be in $\overline{B}$, otherwise it wouldn't have survived the operation. This
  yields a fairly straightforward proof.

  Let $x \in A \setminus B$. Then by definition of the set difference operator,
  $x \in \overline{B}$ and so because $x$ was arbitrarily chosen, the identity must
  hold for any element of $A \setminus B$ and the subset relation is shown.
\end{proof}

\begin{theorem}
  \[
    A \cap \overline{B} = A \setminus B
  \]
\end{theorem}

\begin{proof}
  As in the previous proof, we know that any element that is left over after a
  set subtraction must have been in the first set \emph{and} not the second.
  But this is verbatim what the left side of the equation claims, so all that's
  left to do is show the bi-directional relation once more.

  Let $x$ be an element in $A \cap \overline{B}$. Then by definition, $x \in A$ and
  $x \not \in B$ which means that $x \in \overline{B}$. Then by set difference, $x$
  must be in $A \setminus B$.

  Now assume $x \in A \setminus B$. Once again, $x \in A$ and $x \not \in B$,
  and set equality is shown in both directions.
\end{proof}

\begin{theorem}
  \[
    A \cup (A \cap B) = A
  \]
\end{theorem}

\begin{proof}
  There are a couple of decent ways of thinking about this identity. The most
  straightforward way might be to say that this boils down to declaring all the
  elements in $A$, do in fact, belong to $A$. This line of thought comes from
  the fact that any element in $A \cap B$ must be in $A$, and so the expression
  reduces to $A \cup A$, which of course is $A$ itself.

  Another clever way of thinking about this is to consider the problem in terms
  of boolean logic, in which a similar identity is presented:

  \[
    A \cup (A \cap B) \iff T \lor (T \land P) \iff T
  \]

  here, the value of $P$ is irrelevant, because the $T$ immediately renders the
  entire expression true regardless. Analogously, if we know $x \in A$, then we
  don't have to consider again whether $x \in A$, since the intersection with
  $B$ could only ever produce elements that were in $A$ anyway.

  Let $x \in A \cup (A \cap B)$. Then $x \in A$ or $x \in (A \cap B$. In either
  case, $x \in A$. Now suppose $x \in A$. It immediately follows that $x$ is in
  the union by virtue of being in $A$, and the proof is complete.

  Something interesting about this theorem is that we could instead try to apply
  the previously shown distributive property as a first step:

  \[
    A \cup (A \cap B) = (A \cup A) \cap (A \cup B) = A \cap (A \cup B)
  \]

  Since we've proven the original identity and that this one is equivalent, we
  get another identity (basically) for free!
\end{proof}

\begin{theorem}
  DeMorgan's Law for Sets

  \[
    \overline{A \cup B} = \overline{A} \cap \overline{B}
  \]
\end{theorem}

\begin{proof}
  Another direct analog to boolean logic. Essentially this works in exactly the
  same way as applying DeMorgan's law to logical operators. In plain English,
  this can be described as simply saying if an element is not in $A$ or $B$, it
  can't be in $A$ and it also can't be in $B$.

  Let $x \in \overline{A \cup B}$, then by definition of set complement, $x$ is not
  in the union of $A$ and $B$, which means $x \not \in A$ and $x \not \in B$.
  This implies that $x \in \overline{A}$ and $x \in \overline{B}$ and is therefore in the
  intersection of the two.

  Conversely, if $x \in \overline{A} \cap \overline{B}$, it is neither in $A$ nor $B$, so
  it does not exist in their union and must therefore be in the complement of
  the union, $\overline{A \cup B}$, and the proof is complete.
\end{proof}

\begin{theorem}
  \[
    (A \setminus B) \cap B = \emptyset
  \]
\end{theorem}

\begin{proof}
  This theorem can be thought of as the logical negation of $A \cup (A \cap B)$
  if we treat the $\emptyset$ as a contradiction (logically always false). This
  makes sense, since if we delete all the elements found in $B$ from $A$, and
  afterwards ask the question how many elements are now in both $B$ and $A$, we
  will always find exactly zero (we just got rid of any!).

  For "fun" and variety, this proof can proceed by contradiction. It also makes
  subtle use of the very first proof in that once we show the left-hand side is
  empty, we have proven that is must be equal to the unique empty set, and so
  the proof of the converse statement can be skipped straight away.

  Suppose not, that there is some element $x$ in $A \setminus B \cap B$. Then
  it must be the case that $x \in A \setminus B$ and $x \in B$ by definition of
  intersection. However, if $x \in A \setminus B$, then $x$ cannot be in $B$.
  This contradicts the original assumption that there was some element in the
  set, and so there can be no such element.
\end{proof}

\begin{theorem}
  This theorem is actually \textbf{false}, but sets up the next theorem nicely
  and demonstrates a disproof.

  \[
    (A \setminus B) \cup B = A
  \]
\end{theorem}

\begin{proof}[Counterexample]
  As the next theorem shows, this is only true of $B$ is a subset of $A$. This
  can be shown with a simple counterexample:

  Let $A = \{1\}, \; B = \{2\}$. Then $A \setminus B = \{1\}$, but this set and
  $B$ have a union of $\{1, 2\} \neq A$.

  The reason this fails when $B$ is not a subset of $A$ is because when the set
  difference occurs, only those elements in $B$ are taken away from $A$. If any
  elements of $B$ are not in $A$, they'll be tacked on as "extra" in the union.

  The next theorem provides the corrected claim.
\end{proof}

\begin{theorem}
  \[
    B \subseteq A \iff (A \setminus B) \cup B = A
  \]
\end{theorem}

\begin{proof}
  In contrast to the previous false claim, the condition that $B$ is a subset
  of $A$ ensures that the elements we take away during the set difference are
  exactly the same as the ones added back during the set union. This follows
  from the definition of subset, which says that \emph{every} element in $B$
  must also be in $A$. This guarantees the operations cancel each other out.

  The first part of this proof shows the forward direction in the usual way,
  while the second proceeds by contradiction since it ties in nicely with the
  reasoning of the previous disproof.

  To see the forward claim, let $B \subseteq A$ and let $x$ be an element in
  $(A \setminus B) \cup B$. By definition of union and set difference, either
  $x \in A$ and $x \not \in B$ or $x \in B$. In the first case, $x \in A$ shows
  the identity immediately. Now assume $x \in B$. Once again, we know $x \in A$
  by definition of subset and so the first part of the claim is proven.

  Now assume $(A \setminus B) \cup B = A$ and assume for the sake of obtaining
  a contradiction that $B$ is not a subset of $A$. Then there is some element,
  say, $y \in B$ that is not in $A$. But then $y \in (A \setminus B) \cup B$,
  contradicting the assumption that this set was equal to $A$. Therefore, $B$
  must be a subset of $A$, and the claim holds in both directions.
\end{proof}

\begin{theorem}
  \[
    A \cap (B \setminus (A \cap B)) = \emptyset
  \]
\end{theorem}

\begin{proof}
  There's a lot of notation here, but translating to English helps. This claims
  that there are no elements both in $A$ and in $B$ after taking away those in
  $A$ and $B$ from $B$. This gets us a little closer to what's going on, but we
  can break the problem down slowly and focus on the "inner" expression first.

  \[
    B \setminus (A \cap B)
  \]

  This expression remove anything in both $B$ and $A$ from $B$. Crucially, this
  means any trace of $A$ vanishes from $B$, and so we're left with exactly no
  elements in common between the two sets by the time we take the intersection.
  Of course, this shows that there can never be any element in this set!

  Doing this proof directly isn't terribly difficult, but it is pretty tedious
  given the slighty subtle cases. Instead, we can try to find a contradiction.

  Suppose by way of contradiction there there is some element in the left-hand
  expression, call it $x$. Then $x \in A$ and $x \in B \setminus (A \cap B)$.
  Because $x \in B \setminus (A \cap B)$, it must also be in $B$, and not in
  $A \cap B$. However, this contradicts the fact that $x$ must be in both $A$
  and $B$ in order to be present in the intersection. Therefore, we have shown
  there can be no such element, and the proof is finished.
\end{proof}
\end{document}
